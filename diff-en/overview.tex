3c3
< % $Id: overview.tex,v 1.6 1998/09/29 08:05:09 oetiker Exp oetiker $
---
> % $Id: overview.tex,v 1.2 2003/03/19 20:57:46 oetiker Exp $
4a5,12
> 
> % Because this introduction is the reader's first impression, I have
> % edited very heavily to try to clarify and economize the language.
> % I hope you do not mind! I always try to ask "is this word needed?"
> % in my own writing but I don't want to impose my style on you... 
> % but here I think it may be more important than the rest of the book.
> % --baron
> 
7c15
< \LaTeX{} \cite{manual} is a typesetting system which is very 
---
> \LaTeX{} \cite{manual} is a typesetting system that is very 
9c17
< typographical quality. The system is also suitable for producing all
---
> typographical quality. It is also suitable for producing all
17,27c25,26
< \LaTeX{} is available for most computers, from the PC and Mac to large
< UNIX and VMS systems. On many university computer clusters, you will
< find that a \LaTeX{} installation is available, ready to use.
< Information on how to access
< the local \LaTeX{} installation should be provided in the \guide. If
< you have problems getting started, ask the person who gave you this
< booklet. The scope of this document is \emph{not} to tell you how to
< install and set up a \LaTeX{} system, but to teach you how to write
< your documents so that they can be processed by~\LaTeX{}.
< 
< \noindent This Introduction is split into 5 chapters:
---
> \bigskip
> \noindent This introduction is split into 6 chapters:
31,34c30,31
<   After reading this chapter, you should have a rough picture of
<   \LaTeX{}. The picture will only be a framework, but it will enable
<   you to integrate the information provided in the other chapters into
<   the big picture.
---
>   After reading this chapter, you should have a rough understanding how
>   \LaTeX{} works.
39,46c36,48
< \item[Chapter 3] explains how to typeset formulae with \LaTeX. Again, a
<   lot of examples help you to understand how to use one of \LaTeX{}'s
<   main strengths. At the end of this chapter, you will find tables, listing
<   all the mathematical symbols available in \LaTeX{}.
< \item[Chapter 4] explains index and  bibliography generation,
<   inclusion of EPS graphics, and some other useful extensions.
< \item[Chapter 5] contains some potentially dangerous information about
<   how to make alterations to the
---
> \item[Chapter 3] explains how to typeset formulae with \LaTeX. Many
>   examples demonstrate how to use one of \LaTeX{}'s
>   main strengths. At the end of the chapter are tables listing
>   all mathematical symbols available in \LaTeX{}.
> \item[Chapter 4] explains indexes,  bibliography generation and
>   inclusion of EPS graphics. It introduces creation of PDF documents with pdf\LaTeX{}
>   and presents some handy extension packages.
> \item[Chapter 5] shows how to use \LaTeX{} for creating graphics. Instead
>   of drawing a picture with some graphics program, saving it to a file and
>   then including it into \LaTeX{} you describe the picture and have \LaTeX{}
>   draw it for you.
> \item[Chapter 6] contains some potentially dangerous information about
>   how to alter the
49c51
<   begins looking quite bad.
---
>   turns ugly or stunning, depending on your abilities.
52,55c54,68
< It is important to read the chapters in sequential order. The book is
< not that big after all. Make sure to carefully read the examples,
< because a great part of the information is contained in the various
< examples you will find all throughout the book.
---
> \noindent It is important to read the chapters in order---the book is
> not that big, after all. Be sure to carefully read the examples,
> because a lot of the information is in the
> examples placed throughout the book.
> 
> \bigskip
> \noindent \LaTeX{} is available for most computers, from the PC and Mac to large
> UNIX and VMS systems. On many university computer clusters you will
> find that a \LaTeX{} installation is available, ready to use.
> Information on how to access
> the local \LaTeX{} installation should be provided in the \guide. If
> you have problems getting started, ask the person who gave you this
> booklet. The scope of this document is \emph{not} to tell you how to
> install and set up a \LaTeX{} system, but to teach you how to write
> your documents so that they can be processed by~\LaTeX{}.
60c73
< (\texttt{CTAN}) sites. The homesite is at
---
> (\texttt{CTAN}) sites. The homepage is at
62,66c75,76
< the ftp archive \texttt{ftp://www.ctan.org} and it's various mirror
< sites all over the world.  They can be found e.g.{} at
< \texttt{ftp://ctan.tug.org} (US), \texttt{ftp://ftp.dante.de}
< (Germany), \texttt{ftp://ftp.tex.ac.uk} (UK).  If you are not in one
< of these countries, choose the archive closest to you.
---
> the ftp archive \texttt{ftp://www.ctan.org} and its mirror
> sites all over the world.
68c78
< You will find other references to CTAN throughout the book. Especially
---
> You will find other references to CTAN throughout the book, especially
74c84
< is available from \texttt{CTAN:/tex-archive/systems}.
---
> is available from \CTAN|systems|.
85,87c95,96
< \contrib{Tobias Oetiker}{oetiker@ee.ethz.ch}%
< \noindent{Department of Electrical Engineering,\\
< Swiss Federal Institute of Technology}
---
> \contrib{Tobias Oetiker}{tobi@oetiker.ch}%
> \noindent{OETIKER+PARTNER AG\\Aarweg 15\\4600 Olten\\Switzerland}
91c100
< \texttt{CTAN:/tex-archive/info/lshort}
---
> \CTAN|info/lshort|
97,100c106,112
< %%% Local Variables: 
< %%% mode: latex
< %%% TeX-master: "lshort2e"
< %%% End: 
---
> %
> 
> % Local Variables:
> % TeX-master: "lshort2e"
> % mode: latex
> % mode: flyspell
> % End:
