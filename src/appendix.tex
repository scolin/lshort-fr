\appendix
\chapter{Installation de \LaTeX}
\begin{intro}
Knuth a diffusé les sources de \TeX{} à une époque où personne ne
savait ce qu'étaient les concepts de logiciel libre ou à source
ouvert. La licence fournie avec \TeX{} vous laisse faire tout ce que
vous voulez avec les sources, mais vous ne pouvez appeler le résultat
\TeX{} que si le programme passe une série de tests fournis aussi par
Knuth. Ceci a mené à une situation où il existe une implantation libre
de \TeX{} pour presque tous les systèmes d'exploitation
existants. Ce chapitre vous donnera quelques astuces sur ce
dont vous avez besoin sous Linux,
macOS et Windows pour avoir une installation \TeX{} fonctionnelle.
\end{intro}

\section{Ce qu'il faut installer}

Pour utiliser \LaTeX{} sur un ordinateur, les logiciels suivants sont
essentiels~:

\begin{enumerate}

\item Le programme \TeX{}/\LaTeX{} pour compiler vos fichiers sources
  \LaTeX{} en documents PDF ou DVI ;

\item Un éditeur de texte pour éditer vos fichiers sources
  \LaTeX{}. Certains programmes permettent même le lancement du
  programme \LaTeX{} depuis l'éditeur ;

\item Un visualiseur PDF ou DVI pour afficher et imprimer vos documents ;

\item Un programme pour gérer les fichiers \PSi{} et les images
  pour inclusion dans vos documents.

\end{enumerate}

Pour chaque plateforme il existe plusieurs programmes qui rentrent dans les critères définis
ci-dessus\ldots{} Nous discuterons ici seulement de ceux que nous
connaissons, apprécions et pour lesquels nous avons quelque
expérience.

\section{Éditeur multi-plateformes}
\label{sec:texmaker}

Bien que \TeX{} soit disponible sur de nombreuses plateformes
différentes, les éditeurs \LaTeX{} ont longtemps été chacun restreints
à une plateforme donnée.

Ces dernières années j'en suis venu à apprécier grandement Texmaker.
En plus d'être un éditeur très utile avec une prévisualisation pdf
intégrée et la colorisation syntaxique, il a l'avantage de fonctionner
aussi bien sous Windows, Mac et Unix/Linux. Voyez
\url{http://www.xm1math.net/texmaker} pour plus d'informations.
Il en existe un \emph{fork}\footnote{Ou embranchement
  logiciel, ou \emph{fourche} pour nos amis québecois} nommé TeXstudio
sur \url{http://texstudio.sourceforge.net/} qui semble bien maintenu
et disponible sur les trois plateformes majeures.

Vous trouverez quelques suggestions d'éditeurs plus spécifiques à
certaines plateformes dans les sections sur les systèmes
d'exploitation ci-après.

\section{\TeX{} sous macOS}

\subsection{Distribution \TeX{}}

Téléchargez tout simplement \wi{MacTeX}. Il s'agit d'une distribution
précompilée de \LaTeX{} pour macOS. \wi{MacTeX} met à disposition
une installation complète de \LaTeX{} fournie avec des outils
supplémentaires. Obtenez Mac\TeX{} sur \url{http://www.tug.org/mactex/}.

\subsection{Éditeur \TeX{} macOS}

Si la suggestion d'utiliser l'éditeur multi-plateformes Texmaker
(section \ref{sec:texmaker}) ne vous convient pas...

L'éditeur ouvert le plus populaire pour \LaTeX{} sous Mac semble être
\TeX{}shop. Téléchargez-en une copie sur
\url{http://www.uoregon.edu/~koch/texshop}. Il est aussi fourni avec
la distribution \wi{MacTeX}.

Les distributions récentes \TeX Live contiennent l'éditeur \TeX{}works
\url{http://texworks.org/}, un éditeur multi-plateformes basé sur les
concepts de \TeX{}Shop. Puisque \TeX{}works utilise la bibliothèque
Qt, il est disponible sur toute plateforme sur laquelle cette
bibliothèque fonctionne (macOS, Windows, Linux).

\subsection{Faites-vous plaisir avec \wi{PDFView}}

Utilisez PDFView pour afficher des fichiers PDF générés par \LaTeX{},
il s'intègre parfaitement avec votre éditeur. PDFView est une
application libre disponible au téléchargement sur le site de PDFView
\url{http://pdfview.sourceforge.net/}. Une fois téléchargé et
installé, ouvrez les préférences de PDFView et assurez-vous que
l'option \emph{recharger/rafraîchir automatiquement les documents} est
activée et que le support PDFSync est réglé correctement.

\section{\TeX{} sous Windows}

\subsection{Obtenir \TeX{}}

En premier lieu, obtenez une copie de l'excellente distribution
MiK\TeX\index{MiKTeX@MiK\TeX} sur \url{http://www.miktex.org/}. Elle contient tous les
programmes et fichiers de base pour compiler des documents
\LaTeX{}. La fonctionnalité la plus sympathique, à mes yeux, est que
MiK\TeX{} téléchargera les extensions \LaTeX{} manquantes à la volée et
les installera lors de la compilation d'un document. Vous pouvez aussi
utiliser la distribution TeXlive pour Windows, Unix et Mac~OS pour
obtenir un bon environnement \LaTeX{} de base
\url{http://www.tug.org/texlive/}.

\subsection{Un éditeur \LaTeX{}}

Si la suggestion d'utiliser l'éditeur multi-plateformes Texmaker
(section \ref{sec:texmaker}) ne vous convient pas...

\wi{TeXnicCenter} fait appel à plusieurs concepts du monde
de la programmation pour mettre à disposition un environnement
\LaTeX{} agréable et efficace pour la saisie sous Windows. Obtenez une
copie sur \url{http://http://www.texniccenter.org/}. TeXnicCenter s'intègre
bien avec MiKTeX.

Un autre très bon choix est l'éditeur fourni par le projet LEd sur
\url{http://www.latexeditor.org}.

Voyez la note sur Texmaker dans la section Mac plus haut pour un
troisième choix.

Les distributions \TeX Live récentes contiennent l'éditeur \TeX{}works
\url{http://texworks.org/}. Celui-ci prend en compte l'Unicode et
nécessite Windows XP au minimum.

\subsection{Prévisualisation}

Vous utiliserez probablement Yap pour visualiser votre document DVI,
puisqu'il est installé avec MikTeX. Pour un document PDF vous pouvez
regarder du côté de Sumatra PDF
\url{http://blog.kowalczyk.info/software/sumatrapdf/}. Nous le mentionnons
ici parce qu'il permet après réglages d'aller directement d'une
position dans le document PDF à la position correspondante dans le
document source.

\subsection{Travailler avec des images}

Travailler avec des images de haute qualité dans \LaTeX{} est synonyme
d'utilisation de \EPSi{} (eps) ou de PDF comme format d'image. Le
programme qui vous aidera dans cette tâche s'appelle
\wi{GhostScript}. Vous pouvez l'obtenir avec son interface
graphique propre \wi{GhostView} sur \url{http://www.cs.wisc.edu/~ghost/}.

Si vous utilisez plutôt des formats bitmap (photos et images
numérisées), vous pouvez jeter un oeil à \wi{Gimp}, une alternative
libre à PhotoShop, téléchargeable sur
\url{http://gimp-win.sourceforge.net/}.

\section{\TeX{} sous Linux}

Si vous travaillez sous Linux, il est probable que \LaTeX{} soit déjà
installé sur votre système, ou au moins disponible sur les dépôts que
vous avez utilisés lors de son installation. Utilisez votre
gestionnaire de paquets pour installer les applications suivantes :

\begin{itemize}
\item texlive -- l'installation \TeX{}/\LaTeX{} de base ;
\item emacs (avec AUCTeX) -- un éditeur qui s'intègre
  étroitement avec \LaTeX{} via une extension appelée AucTeX ;
\item ghostscript -- un visualiseur \PSi{} ;
\item xpdf et acrobat -- des visualiseurs PDF ;
\item imagemagick -- un programme libre pour convertir des images
  bitmap ;
\item gimp -- un clone libre de PhotoShop ;
\item inkscape -- un clone libre d'Illustrator/Corel Draw (dessin
  vectoriel).
\end{itemize}

Si vous cherchez un environnement graphique d'édition plus proche de
ce qui se fait sous Windows, essayez Texmaker (cf section
\ref{sec:texmaker}).

Attention, la plupart des distributions découpent \TeX\,Live en
plusieurs paquets. Si vous souhaitez une installation complète,
cherchez un paquet appelé \texttt{texlive-full} ou bien installez tous
les paquets dont le nom contient \texttt{texlive} ou \texttt{latex},
ainsi que les paquets \texttt{cm-super} et \texttt{lmodern} s'ils
existent.
