\appendix
\chapter{Installation de \LaTeX}
\begin{intro}
Knuth a diffusé les sources de \TeX{} à une époque où personne ne
savait ce qu'étaient les concepts de logiciel libre ou à source
ouvert. La licence fournie avec \TeX{} vous laisse faire tout ce que
vous voulez avec les sources, mais vous ne pouvez appeler le résultat
\TeX{} que si le programme passe une série de tests fournis aussi par
Knuth. Ceci a mené à une situation où il existe une implantation libre
de \TeX{} pour presque tous les systèmes d'exploitation
existants. Ce chapitre vous donnera quelques astuces sur ce
dont vous avez besoin sous Linux,
macOS et Windows pour avoir une installation \TeX{} fonctionnelle.
\end{intro}

\section{Ce qu'il faut installer}

Pour utiliser \LaTeX{} sur un ordinateur, les logiciels suivants sont
essentiels~:

\begin{enumerate}

\item Le programme \TeX{}/\LaTeX{} pour compiler vos fichiers sources
  \LaTeX{} en documents PDF ou DVI ;

\item Un éditeur de texte pour éditer vos fichiers sources
  \LaTeX{}. Certains programmes permettent même le lancement du
  programme \LaTeX{} depuis l'éditeur ;

\item Un visualiseur PDF ou DVI pour afficher et imprimer vos documents ;

\item Un programme pour gérer les fichiers \PSi{} et les images
  pour inclusion dans vos documents.

\end{enumerate}

Pour chaque plateforme il existe plusieurs programmes qui rentrent dans les critères définis
ci-dessus\ldots{} Nous discuterons ici seulement de ceux que nous
connaissons, apprécions et pour lesquels nous avons quelque
expérience.

\section{Éditeur multi-plateformes}
\label{sec:texmaker}

Bien que \TeX{} soit disponible sur de nombreuses plateformes
différentes, les éditeurs \LaTeX{} ont longtemps été chacun restreints
à une plateforme donnée.

Ces dernières années j'en suis venu à apprécier grandement Texmaker.
En plus d'être un éditeur très utile avec une prévisualisation pdf
intégrée et la colorisation syntaxique, il a l'avantage de fonctionner
aussi bien sous Windows, Mac et Unix/Linux. Voyez
\url{http://www.xm1math.net/texmaker} pour plus d'informations.
Il en existe un \emph{fork}\footnote{Ou embranchement
  logiciel, ou \emph{fourche} pour nos amis québecois} nommé TeXstudio
sur \url{http://texstudio.sourceforge.net/} qui semble bien maintenu
et disponible sur les trois plateformes majeures.

Vous trouverez quelques suggestions d'éditeurs plus spécifiques à
certaines plateformes dans les sections sur les systèmes
d'exploitation ci-après.

\section{\TeX{} sous macOS}

\subsection{Distribution \TeX{}}

Téléchargez tout simplement \wi{MacTeX}. Il s'agit d'une distribution
précompilée de \LaTeX{} pour macOS. \wi{MacTeX} met à disposition
une installation complète de \LaTeX{} fournie avec des outils
supplémentaires. Obtenez Mac\TeX{} sur \url{http://www.tug.org/mactex/}.

\subsection{Éditeur \TeX{} macOS}

Si la suggestion d'utiliser l'éditeur multi-plateformes Texmaker
(section \ref{sec:texmaker}) ne vous convient pas...

L'éditeur ouvert le plus populaire pour \LaTeX{} sous Mac semble être
\TeX{}shop. Téléchargez-en une copie sur
\url{http://www.uoregon.edu/~koch/texshop}. Il est aussi fourni avec
la distribution \wi{MacTeX}.

Les distributions récentes \TeX Live contiennent l'éditeur \TeX{}works
\url{http://texworks.org/}, un éditeur multi-plateformes basé sur les
concepts de \TeX{}Shop. Puisque \TeX{}works utilise la bibliothèque
Qt, il est disponible sur toute plateforme sur laquelle cette
bibliothèque fonctionne (macOS, Windows, Linux).

\subsection{Faites-vous plaisir avec \wi{PDFView}}

Utilisez PDFView pour afficher des fichiers PDF générés par \LaTeX{},
il s'intègre parfaitement avec votre éditeur. PDFView est une
application libre disponible au téléchargement sur le site de PDFView
\url{http://pdfview.sourceforge.net/}. Une fois téléchargé et
installé, ouvrez les préférences de PDFView et assurez-vous que
l'option \emph{recharger/rafraîchir automatiquement les documents} est
activée et que le support PDFSync est réglé correctement.

\section{\TeX{} sous Windows}

\subsection{Obtenir \TeX{}}

En premier lieu, obtenez une copie de l'excellente distribution
MiK\TeX\index{MiKTeX@MiK\TeX} sur \url{http://www.miktex.org/}. Elle contient tous les
programmes et fichiers de base pour compiler des documents
\LaTeX{}. La fonctionnalité la plus sympathique, à mes yeux, est que
MiK\TeX{} téléchargera les extensions \LaTeX{} manquantes à la volée et
les installera lors de la compilation d'un document. Vous pouvez aussi
utiliser la distribution TeXlive pour Windows, Unix et Mac~OS pour
obtenir un bon environnement \LaTeX{} de base
\url{http://www.tug.org/texlive/}.

\subsection{Un éditeur \LaTeX{}}

Si la suggestion d'utiliser l'éditeur multi-plateformes Texmaker
(section \ref{sec:texmaker}) ne vous convient pas...

\wi{TeXnicCenter} fait appel à plusieurs concepts du monde
de la programmation pour mettre à disposition un environnement
\LaTeX{} agréable et efficace pour la saisie sous Windows. Obtenez une
copie sur \url{http://http://www.texniccenter.org/}. TeXnicCenter s'intègre
bien avec MiKTeX.

Un autre très bon choix est l'éditeur fourni par le projet LEd sur
\url{http://www.latexeditor.org}.

Voyez la note sur Texmaker dans la section Mac plus haut pour un
troisième choix.

Les distributions \TeX Live récentes contiennent l'éditeur \TeX{}works
\url{http://texworks.org/}. Celui-ci prend en compte l'Unicode et
nécessite Windows XP au minimum.

\subsection{Prévisualisation}

Vous utiliserez probablement Yap pour visualiser votre document DVI,
puisqu'il est installé avec MikTeX. Pour un document PDF vous pouvez
regarder du côté de Sumatra PDF
\url{http://blog.kowalczyk.info/software/sumatrapdf/}. Nous le mentionnons
ici parce qu'il permet après réglages d'aller directement d'une
position dans le document PDF à la position correspondante dans le
document source.

\subsection{Travailler avec des images}

Travailler avec des images de haute qualité dans \LaTeX{} est synonyme
d'utilisation de \EPSi{} (eps) ou de PDF comme format d'image. Le
programme qui vous aidera dans cette tâche s'appelle
\wi{GhostScript}. Vous pouvez l'obtenir avec son interface
graphique propre \wi{GhostView} sur \url{http://www.cs.wisc.edu/~ghost/}.

Si vous utilisez plutôt des formats bitmap (photos et images
numérisées), vous pouvez jeter un oeil à \wi{Gimp}, une alternative
libre à PhotoShop, téléchargeable sur
\url{http://gimp-win.sourceforge.net/}.

\section{\TeX{} sous Linux}

Si vous travaillez sous Linux, il est probable que \LaTeX{} soit déjà
installé sur votre système, ou au moins disponible sur les dépôts que
vous avez utilisés lors de son installation. Utilisez votre
gestionnaire de paquets pour installer les applications suivantes :

\begin{itemize}
\item texlive -- l'installation \TeX{}/\LaTeX{} de base ;
\item emacs (avec AUCTeX) -- un éditeur qui s'intègre
  étroitement avec \LaTeX{} via une extension appelée AucTeX ;
\item ghostscript -- un visualiseur \PSi{} ;
\item xpdf et acrobat -- des visualiseurs PDF ;
\item imagemagick -- un programme libre pour convertir des images
  bitmap ;
\item gimp -- un clone libre de PhotoShop ;
\item inkscape -- un clone libre d'Illustrator/Corel Draw (dessin
  vectoriel).
\end{itemize}

Si vous cherchez un environnement graphique d'édition plus proche de
ce qui se fait sous Windows, essayez Texmaker (cf section
\ref{sec:texmaker}).

Attention, la plupart des distributions découpent \TeX\,Live en
plusieurs paquets. Si vous souhaitez une installation complète,
cherchez un paquet appelé \texttt{texlive-full} ou bien installez tous
les paquets dont le nom contient \texttt{texlive} ou \texttt{latex},
ainsi que les paquets \texttt{cm-super} et \texttt{lmodern} s'ils
existent.

\chapter{Notes sur le support d'autres langues avec Babel}

\begin{intro}
Lors de la rédaction de texte en langues autres que l'anglais, vous
avez pu rencontrer le paquet \pai{babel}. De nos jours il a été rendu
obsolète par le paquet \pai{polyglossia}, mais si vous souhaitez lire
un pan d'histoire, vous trouverez ci-après quelques notes sur
l'ancienne façon de mettre en forme dans différentes langues.
\end{intro}

\subsection{Support de la langue française} \label{ss-frenchb}

% Pour le français, l'extension \pai{french} a été
% développée par Bernard Gaulle~\cite{french}.
% Mais le traducteur boycotte les versions récentes (frenchpro)

\secby{Daniel Flipo}{daniel.flipo@univ-lille1.fr}
Voici quelques conseils pour créer des documents en
\wi{français}\index{français} à l'aide de \LaTeX{}. Le support de la
langue française est activé par la commande suivante :

\begin{lscommand}
\verb|\usepackage[francais]{babel}|
\end{lscommand}

Vous pouvez aussi utiliser l'option \texttt{frenchb} qui est un synonyme.
L'option \texttt{french} a pu être différente à une époque et sur certains
systèmes, mais est équivalente aux précédentes sur tous les systèmes
depuis~2003. C'est désormais le nom recommandé.

Cette commande active les règles de césure spécifiques du français et
adaptent \LaTeX{} à la plupart des règles spécifiques de la
typographie française~\cite{ftypo} : présentation des listes,
insertion automatique de l'espacement avant les signes de ponctuation
doubles, etc.  Les mots générés automatiquement par \LaTeX{} sont
traduits en français et certaines commandes supplémentaires (cf.
table~\ref{cmd-french}) sont disponibles.

\begin{table}[!htbp]
\caption{Commandes de saisie en français} \label{cmd-french}
\begin{lined}{9cm}
\begin{tabular}{ll}
\verb+\og guillemets \fg{}+         \quad &\og guillemets \fg \\[1ex]
\verb+M\up{me}, D\up{r}+            \quad &M\up{me}, D\up{r}  \\[1ex]
\verb+1\ier{}, 1\iere{}, 1\ieres{}+ \quad &1\ier{}, 1\iere{}, 1\ieres{}\\[1ex]
\verb+2\ieme{} 4\iemes{}+           \quad &2\ieme{} 4\iemes{}\\[1ex]
\verb+\No 1, \no 2+                 \quad &\No 1, \no 2   \\[1ex]
\verb+20~\degres C, 45\degres+      \quad &20~\degres C, 45\degres \\[1ex]
\verb+\bsc{M. Durand}+              \quad &\bsc{M.~Durand} \\[1ex]
\verb+\nombre{1234,56789}+          \quad &\nombre{1234,56789}
\end{tabular}
\par\bigskip
\end{lined}
\end{table}


Vous remarquerez également que la mise en page des listes est changée
lors du passage à la langue française. Pour obtenir toutes les
informations sur l'option \pai{francais} de \pai{babel} et comment
modifier son comportement, consultez la partie~29 de
\texttt{babel.pdf}\footnote{Ou la documentation en français sur la page de son
  auteur : \url{http://daniel.flipo.free.fr/frenchb/frenchb2-doc.pdf}. \NdT}

Notez également que les versions récentes de \pai{francais}
nécessitent l'extension \pai{numprint} pour implanter la commande
\ci{nombre}.

Dans cette traduction, un certain nombre d'ajouts présentent les
spécificités de la typographie française tout au long du texte.

\subsection{Support de la langue allemande}

Voici quelques conseils pour créer des documents en
\wi{allemand}\index{deutsch} à l'aide de \LaTeX{}. Le support de la
langue allemande est activé par la commande suivante :

\begin{lscommand}
\verb|\usepackage[german]{babel}|
\end{lscommand}

La césure allemande est alors activée, si votre système a été
configuré pour cela. Le texte produit automatiquement par \LaTeX{} est
traduit en allemand (par ex. \og Kapitel \fg pour un chapitre). De
nouvelles commandes (listées dans la table~\ref{german}) permettent la
saisie simplifiée des caractères spéciaux même sans utiliser
l'extension inputenc. Avec inputenc cette capacité devient un peu vaine
mais votre texte est alors un peu enfermé dans un type d'encodage particulier.

\begin{table}[!htbp]
\caption{Caractères spéciaux en allemand} \label{german}
\begin{lined}{8cm}
\begin{tabular}{*2{ll}}
\verb|"a| & "a \hspace*{1ex} & \verb|"s| & "s \\[1ex]
\verb|"`| & "` & \verb|"'| & "' \\[1ex]
\verb|"<| ou \ci{flqq} & "<  & \verb|">| ou \ci{frqq} & "> \\[1ex]
\ci{flq} & \flq & \ci{frq} & \frq \\[1ex]
\ci{dq} & " \\
\end{tabular}
\bigskip
\end{lined}
\end{table}

Les livres allemands contiennent souvent des marques de citation
françaises (\og guil\-le\-mets \fg). Les typographes allemands les
utilisent différemment, cependant. Une citation dans un livre allemand
ressemblerait plutôt à \frqq ceci\flqq. En Suisse allemande, les
typographes utilisent les \og guillemets \fg~comme les français le font.

Un problème majeur découle de l'utilisation de commandes comme
\verb+\flq+ : si vous utilisez la police OT1 (la police par défaut) les
guillemets ressembleront au symbole mathématique \og $\ll$ \fg{}, de
quoi causer des maux d'estomac à un typographe. Les polices codées T1,
par contre, contiennent les symboles requis. Ainsi, si vous
utilisez ce type de marque de citation, assurez-vous d'utiliser le
codage T1 (\verb|\usepackage[T1]{fontenc}|).

\subsection[Support de la langue coréenne]
           {Support de la langue coréenne\footnotemark}
           \label{support_korean}%
\footnotetext{Écrit par Karnes Kim $<$\href{mailto:karnes@ktug.org}{karnes@ktug.org}$>$ et Kihwang Lee $<$\href{mailto:leekh@ktug.org}{leekh@ktug.org}$>$ au nom du groupe coréen d'utilisateurs \TeX{} et la Société Coréenne \TeX{}.}

Pour traiter les caractères Hangeul\footnote{Hangeul est le nom du
  système d'écriture coréen, cf
  \url{https://fr.wikipedia.org/wiki/Hangeul}} ou préparer un document
rédigé en coréen avec \LaTeX, insérer la commande suivante dans le
préambule du document.

\begin{lscommand}
\verb|\usepackage{kotex}|
\end{lscommand}

Un document contenant cette déclaration devra être interprété par
pdf\LaTeX, \hologo{XeLaTeX} ou Lua\LaTeX{}. Assurez-vous que le
fichier d'entrée écrit en Hangeul est codé en Unicode UTF-8. Le paquet
appelé ko.\TeX\footnote{Lire ``Korean \TeX{}''. ko.\TeX{} est le nom
  d'une collection de paquets incluant \texttt{cjk-ko},
  \texttt{kotex-utf}, \texttt{xetexko} and \texttt{luatexko}.} est
en développement constant pour le groupe coréen d'utilisateurs
\TeX{}\footnote{\url{http://ktug.org}} and la Société Coréenne
\TeX{}\footnote{\url{http://ktug.kr}}. De nombreuses personnes
utilisent ce paquet pour créer des documents coréens pour leurs
besoins de tous les jours. ko.\TeX\ est disponible sur CTAN depuis
2014. Il est inclus dans \TeX\,Live, MiK\TeX{} ainsi que d'autres
distributions \TeX{} modernes. Il y a fort à parier que vous pouvez
dès à présent commencer à travailler directement sans avoir à
installer de paquets supplémentaires.

ko.\TeX\ n'utilise pas le paquet \texttt{babel}. De nombreuses
fonctions liées au coréen peuvent être activées via les options et
commandes de configuration fournies par le paquet \pai{kotex}. Si vous
souhaitez composer un document coréen réaliste, vous êtes invités à
consulter la documentation du paquet (ces documents sont rédigés en
coréen).

Avec ko.\TeX, vous obtenez également \pai{oblivoir}, une classe de
document basée sur \pai{memoir}, adaptée à la préparation de documents
coréens. Ainsi votre document en coréen commencerait par ceci:

\begin{lscommand}
\verb|\documentclass{oblivoir}|
\end{lscommand}

Pour générer l'index d'un document en coréen, exécutez
\texttt{komkindex} au lieu de \texttt{makeindex}. C'est une version
modifiée de \texttt{makeindex} en vue de traiter la langue
coréenne. Pour le tri lexicographique des entrées d'index en coréen,
vous pouvez utiliser le style d'index \texttt{kotex.ist} fourni par
ko.\TeX{} comme suit:


\begin{lscommand}
\verb|komkindex -s kotex foo.idx|
\end{lscommand}

Vous pouvez aussi utiliser \texttt{xindy} pour la génération d'index
puisque le module coréen pour \texttt{xindy} est inclus dans
\TeX\,Live.

Il existe un autre paquet de support du Coréen/Hangeul appelé
CJK. Comme son nom le suggère, il facilite le support des caractères
chinois, coréens et japonais. Il prend en compte plusieurs types
d'encodage des caractères CJK. L'exemple suivant illustre comment
gérer le Hangeul encodé en UTF-8 avec le paquet CJK. Cela peut être
utile lorsque vous soumettez un manuscript à certains journaux
académiques qui autorisent les noms d'auteurs écrits dans leur langue
native.

\begin{verbatim}
\usepackage{CJK}

\begin{CJK}{UTF8}{}
\CJKfamily{nanummj}
...
\end{CJK}
\end{verbatim}


\subsection{Support du grec}
\secby{Nikolaos Pothitos}{pothitos@di.uoa.gr}
Les commandes à insérer en préambule pour écrire en \wi{grec}
\index{grec} se trouvent dans le tableau~\ref{preamble-greek}. Ce
préambule active la césure et change le texte automatique en grec.%
\footnote{Si vous ajoutez l'option \texttt{utf8x} à
  \texttt{inputenc}, vous pourrez saisir du grec et des caractères
  unicodes grecs polytoniques.}

\begin{table}[btp]
\caption{Préambule pour les documents grecs} \label{preamble-greek}
\begin{lined}{7cm}
\begin{verbatim}
\usepackage[english,greek]{babel}
\end{verbatim}
\end{lined}
\end{table}

De nouvelles commandes pour une saisie simplifiée du grec deviennent
aussi disponibles. Pour passer temporairement en alphabet latin et
vice-versa, vous pouvez utiliser les commandes
\verb|\textlatin{|\emph{texte latin}\verb|}| and
\verb|\textgreek{|\emph{texte grec}\verb|}|. Elles prennent un
argument qui sera formaté avec la police la plus pertinente.
Sinon vous pouvez aussi utiliser la commande
\verb|\selectlanguage{...}| présentée précédemment. Le
tableau~\ref{sym-greek} présente quelques caractères de ponctuation
grecs. Vous pouvez utiliser \verb|\euro| pour obtenir le symbole de
l'euro.

\begin{table}[!htbp]
\caption{Caractères spéciaux grecs} \label{sym-greek}
\begin{lined}{4cm}
%SC: Nope, already in french :-P
%\selectlanguage{french}
\begin{tabular}{*2{ll}}
\verb|;| \hspace*{1ex}  &  $\cdot$ \hspace*{1ex}  &  \verb|?| \hspace*{1ex}&  ;   \\[1ex]
\verb|((|               &  \og                    &  \verb|))|&  \fg \\[1ex]
\verb|``|               &  `                      &  \verb|''| &  '   \\
\end{tabular}
%\selectlanguage{english}
\bigskip
\end{lined}
\end{table}


\subsection{Support du cyrillique}

\secby{Maksym Polyakov}{polyama@myrealbox.com}
La version~3.7h de \pai{babel} comprend un support pour les codages
\pai{T2*} et pour le formatage des textes bulgares, russes et
ukrainiens à base de lettres cyrilliques.

Le support du cyrillique se base sur des mécanismes \LaTeX{} standards
en utilisant les extensions \pai{fontenc} et \pai{inputenc}. Si cependant vous
voulez utiliser du cyrillique en mode mathématique, vous devrez de plus
charger \pai{mathtext} avant \pai{fontenc}
\footnote{
Si vous utilisez les extensions \AmS-\LaTeX{}, chargez-les aussi avant
\pai{fontenc} et \pai{babel} pour éviter des conflits.
} :
\begin{lscommand}
\verb+\usepackage{mathtext}+\\
\verb+%\usepackage{amsmath}+\\
\verb+\usepackage[+\pai{T1}\verb+,+\pai{T2A}\verb+]{fontenc}+\\
\verb+\usepackage[english,bulgarian,russian,ukranian]{babel}+
\end{lscommand}

En général, \pai{babel} choisira lui-même le codage de police par
défaut : pour les trois langues précitées il s'agit de \pai{T2A}. Les
documents ne sont cependant pas limités à un seule codage de
police. Pour les documents multilingues avec des alphabets cyrillique
et latin, il est raisonnable d'inclure les codages de polices latines
explicitement. \pai{babel} prendra en charge le changement vers un
codage de fonte approprié lorsqu'un autre langage est sélectionné dans
le document.

En plus d'activer les césures, de traduire les textes automatiques et
d'activer certaines règles typographiques (comme \ci{frenchspacing}),
\pai{babel} fournit des commandes additionnelles pour permettre un
formatage conforme aux conventions bulgare, russe ou ukrainienne.


Pour ces trois langues, une ponctuation spéciale est fournie. Le
tiret cyrillique pour le texte (plus étroit que le tiret latin et
entouré d'espaces fines), un tiret pour le dialogue, des marques de
citation et des commandes pour faciliter les césures, voir le
tableau~\ref{Cyrillic}.


% Table borrowed from Ukrainian.dtx
%SC: please, some french specialist of cyrillic typography could help ?
\begin{table}[htb]
  \begin{center}
  \index{""-@\texttt{""}\texttt{-}}
  \index{""---@\texttt{""}\texttt{-}\texttt{-}\texttt{-}}
  \index{""=@\texttt{""}\texttt{=}}
  \index{""`@\texttt{""}\texttt{`}}
  \index{""'@\texttt{""}\texttt{'}}
  \index{"">@\texttt{""}\texttt{>}}
  \index{""<@\texttt{""}\texttt{<}}
  \caption[Bulgare, russe et ukrainien]{Les définitions additionnelles
    de \pai{babel} pour le bulgare, le russe et l'ukrainien}\label{Cyrillic}
  \begin{tabular}{@{}p{.1\hsize}@{}p{.9\hsize}@{}}
   \hline
   \verb="|= & Désactiver la ligature ici.               \\
   \verb|"-| & Un tiret explicite autorisant la césure dans
               le reste du mot.                             \\
   \verb|"---| & Tiret long cyrillique pour le texte.        \\
   \verb|"--~| & Tiret long cyrillique pour les noms composés.       \\
   \verb|"--*| & Tiret long cyrillique pour le dialogue.         \\
   \verb|""| & Comme \verb|"-|, mais n'affiche pas le tiret
               (pour les mots composés avec tiret, p.e.\ \verb|x-""y|
                ou d'autres signes comme \og disable/enable \fg{}). \\
   \verb|"~| & Pour une marque de mot composé sans rupture.        \\
   \verb|"=| & Pour une marque de mot composé avec rupture, pour
               autoriser les césures dans les mots le composant. \\
   \verb|",| & Espace fine pour les initiales avec rupture possible
               pour le nom qui suit.                            \\
   \verb|"`| & Pour les marques doubles allemandes de citation
               (ressemble à ,\kern-0.08em,).                     \\
   \verb|"'| & Pour les marques doubles allemandes de citation,
               à droite (ressemble à ``).                        \\%''
   \verb|"<| & Pour les guillemets à gauche (comme \flqq).  \\
   \verb|">| & Pour les guillemets à droite (comme \frqq). \\
   \hline
  \end{tabular}
  \end{center}
\end{table}


Les options \pai{babel} pour le russe et l'ukrainien définissent les
commandes \ci{Asbuk} et \ci{asbuk} qui agissent comme \ci{Alph} et
\ci{alph}\footnote{les commandes pour transformer les compteurs en a,
  b, c, \dots}, mais produisent des majuscules et des minuscules des
alphabets russe et ukrainien (en fonction duquel est la langue active
du document). L'option bulgare de \pai{babel} fournit les commandes
\ci{enumBul} et \ci{enumLat} (\ci{enumEng}) qui font produire à
\ci{Alph} et \ci{alph} des lettres des alphabets bulgare ou latin
(anglais, resp.). Le comportement par défaut de \ci{Alph} et \ci{alph}
pour le bulgare est de produire des lettres de l'alphabet bulgare.

%Finally, math alphabets are redefined and  as well as the commands for math
%operators according to Cyrillic typesetting traditions.

\subsection{Support du mongol}

Pour composer du mongol avec \LaTeX, vous avec le choix entre deux
extensions : \pai{babel} (multilingue) et Mon\TeX{} d'Oliver Corff.

Mon\TeX{} prend en compte aussi bien le cyrillique que l'écriture mongole
traditionnelle. Pour avoir accès aux commandes de Mon\TeX, ajoutez
\begin{lscommand}
\ci{usepackage}\verb|[|\emph{langue},\emph{encodage}\verb|]{mls}|
\end{lscommand}
\noindent à votre préambule. Pour obtenir les titres et dates en mongol
moderne, utilisez l'option \pai{xalx} comme \emph{langue}. Pour écrire un
document complet en mongol traditionnel, choisissez \pai{bicig}. Cette
dernière active la méthode d'entrée de caractères par translittération
simplifiée.

Le mode de translittération latine peut être activé ou désactivé avec les
commandes suivantes.
\begin{lscommand}
\verb|\SetDocumentEncodingLMC|
\verb|\SetDocumentEncodingNeutral|
\end{lscommand}

Vous trouverez plus d'information sur Mon\TeX{} sur
\CTAN|language/mongolian/montex/doc|.

L'écriture cyrillique du mongol est supportée par \pai{babel}. Les commandes
suivantes activent ce support.

\begin{lscommand}
\verb|\usepackage[mongolian]{babel}|
\end{lscommand}
