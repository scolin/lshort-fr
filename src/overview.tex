%%%%%%%%%%%%%%%%%%%%%%%%%%%%%%%%%%%%%%%%%%%%%%%%%%%%%%%%%%%%%%%%%
% Contents: Who contributed to this Document
% $Id$
%%%%%%%%%%%%%%%%%%%%%%%%%%%%%%%%%%%%%%%%%%%%%%%%%%%%%%%%%%%%%%%%%

% Pour les informations de licence, voir contrib.tex.
% See contrib.tex for license information.


% Because this introduction is the reader's first impression, I have
% edited very heavily to try to clarify and economize the language.
% I hope you do not mind! I always try to ask "is this word needed?"
% in my own writing but I don't want to impose my style on you... 
% but here I think it may be more important than the rest of the book.
% --baron

\chapter{Préface}
\thispagestyle{plain}

\LaTeX{} \cite{manual} est un logiciel de composition typographique
particulièrement adapté à la production de documents scientifiques et mathématiques de
grande qualité typographique. Il permet également de produire
toutes sortes d'autres documents, qu'il s'agisse de simples lettres ou
de livres entiers. \LaTeX{} utilise \TeX{} \cite{texbook} comme outil de
mise en page. 

Cette introduction décrit \LaTeXe{} et devrait se montrer suffisante
pour la plupart des applications de \LaTeX. Pour une description
complète du système \LaTeX{}, reportez-vous
à~\cite{manual,companion}. 

\bigskip
\noindent Cette introduction est composée de six chapitres :
\begin{description}

\item[Le chapitre 1] présente la structure élémentaire d'un document
  \LaTeXe{}. Il vous apprendra également quelques éléments sur
  l'histoire de \LaTeX{}. Après avoir lu ce chapitre, vous devriez
  avoir une vue générale de ce qu'est \LaTeX{} et de son
  fonctionnement.

\item[Le chapitre 2] entre dans les détails de la mise en page d'un
  document. Il explique les commandes et les environnements
  essentiels de \LaTeX{}. Après avoir lu ce chapitre, vous serez
  capables de rédiger vos premiers documents.

\item[Le chapitre 3] explique comment produire des formules
  mathématiques en \LaTeX{}. De nombreux exemples
  montrent comment utiliser cet atout majeur de \LaTeX{}. À la fin de ce
  chapitre, vous trouverez des tableaux qui listent tous les symboles
  mathématiques disponibles sous \LaTeX{}.

\item[Le chapitre 4] traite des index, listes de
  références bibliographiques et de l'insertion de figures
  PostScript. Il présente aussi la création de documents PDF avec
  pdf\LaTeX{} ainsi que quelques autres extensions utiles.

\item[Le chapitre 5] montre comment utiliser \LaTeX{} pour créer des
  images. Au lieu de dessiner une image à l'aide d'un programme
  d'infographie donné, la sauvegarder et l'inclure dans \LaTeX{}, vous
  décrirez l'image et laisserez \LaTeX{} la dessiner pour vous.

\item[Le chapitre 6] contient des informations potentiellement
  dangeureuses. Il vous apprend à modifier la mise en page standard
  produite par \LaTeX{} et vous  permet de transformer 
  les présentations plutôt réussies de \LaTeX{} en quelque chose
  de laid ou magnifique, selon votre habileté.
\end{description}

\bigskip
\noindent Il est important de lire ces chapitres dans l'ordre --- après
tout, ce livre n'est pas si long.  L'étude attentive des exemples
est indispensable à la compréhension de l'ensemble car ils contiennent
une bonne partie de l'information que vous pourrez trouver ici.

\bigskip
\noindent \LaTeX{} est disponible pour une vaste gamme de systèmes
informatiques, des PCs et Macs aux systèmes UNIX
\footnote{UNIX est une marque déposée de The Open Group.}
 et VMS. Dans de
nombreuses universités, il est installé sur le réseau informatique,
prêt à être utilisé. L'information nécessaire pour y accéder devrait
être fournie dans le \guide. Si vous avez des difficultés pour
commencer, demandez de l'aide à la personne qui vous a donné cette
brochure.  Ce document \emph{n'est pas} un guide d'installation du
système \LaTeX{}. Son but est de vous montrer comment écrire vos
documents afin qu'ils puissent être traités par \LaTeX{}.

\bigskip
\index{CTAN}
Si vous avez besoin de récupérer des fichiers relatifs à \LaTeX{},
utilisez les sites CTAN (\emph{Comprehensive
  \TeX{} Archive Network})\label{CTAN}.
Le site principal est sur \url{http://www.ctan.org}.

Vous verrez plusieurs références au CTAN au long de ce document, en
particulier des pointeurs vers des logiciels ou des documents. Au lieu
d'écrire des URL complètes, nous avons simplement écrit \texttt{CTAN:}
suivi du chemin dans l'arborescence du CTAN.

Si vous souhaitez installer \LaTeX{} sur votre ordinateur, vous
trouverez sans doute une version adaptée à votre système sur
\CTAN|systems|.

\vspace{\stretch{1}}
\noindent Si vous avez des suggestions de choses à
ajouter, supprimer ou modifier dans ce document, contactez soit
directement l'auteur de la version originale, soit moi-même, le
traducteur.  Nous sommes particulièrement intéressés par des retours
d'utilisateurs débutants en \LaTeX{} au sujet des passages de ce livre
qui devraient être mieux expliqués.


\bigskip
\begin{verse}
\contrib{Tobias Oetiker}{tobi@oetiker.ch}%
{OETIKER+PARTNER AG\\Aarweg 15\\4600 Olten\\Switzerland}

\contrib{Matthieu Herrb}{matthieu.herrb@laas.fr}%
{(jusqu'à la version 3.20, cité ici en guise d'hommage)}

\contrib{Manuel Pégourié-Gonnard}{mpg@elzevir.fr}{%
Institut de mathématiques de Jussieu, France.\\
(avait débuté un travail de traduction, désormais repris ici)}

\contrib{Samuel Colin}{scolin@hivernal.org}%
{(à partir de la version 3.21fr)}

\end{verse}
\vspace{\stretch{1}}
\noindent La version courante de ce document est disponible sur
\CTAN|info/lshort|.

\endinput

%

% Local Variables:
% TeX-master: "lshort2e"
% mode: latex
% mode: flyspell
% End:
