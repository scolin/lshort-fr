%%%%%%%%%%%%%%%%%%%%%%%%%%%%%%%%%%%%%%%%%%%%%%%%%%%%%%%%%%%%%%%%%
% Contents: Main Input File of the LaTeX2e Introduction
% $Id: lshort.tex,v 1.2 2003/03/19 20:57:46 oetiker Exp $
%%%%%%%%%%%%%%%%%%%%%%%%%%%%%%%%%%%%%%%%%%%%%%%%%%%%%%%%%%%%%%%%%
% lshort.tex - Une courte (?) introduction a LaTeX2e
%                                                      by Tobias Oetiker
%                                                     oetiker@ee.ethz.ch
%
%                           based on LKURTZ.TEX Uni Graz & TU Wien,1987
%                           traduit en fran�ais par Matthieu Herrb,1996
%-----------------------------------------------------------------------
%
% To compile lshort, you need TeX 3.x, LaTeX2e with the babel frenchb
% extension  and makeindex
%
% The sources files of the Intro are:
%      lshort.tex (this file),
%      title.tex, contrib.tex, biblio.tex
%      things.tes, typeset.tex, math.tex, lssym.tex, spec.tex,
%      lshort.sty
%
% Further the  verbatim.sty and the layout.sty 
% from the LaTeX Tools distribution is
% required.
%
% To print the AMS symbols you need the AMS fonts and the packages
% amsfonts, eufrak and eucal from (AMS LaTeX 1.2)
%
% ---------------------------------------------------------------------
\newif\ifpdf 
\ifx\pdfoutput\undefined 
  \pdffalse 
\else \ifx \pdfoutput\relax 
    \pdffalse 
  \else 
    \pdftrue 
  \fi 
\fi 

\ifpdf
\documentclass[pdftex,11pt,a4paper,twoside]{book}
\usepackage{thumbpdf}
\pdfcompresslevel=9
\else
\documentclass[dvips,11pt,a4paper,twoside]{book}
\fi
\usepackage[T1]{fontenc}
\usepackage[latin1]{inputenc}
\usepackage{amsmath}
\usepackage{lshort}
\usepackage{makeidx,shortvrb,latexsym}
\usepackage[french]{mylayout}
\usepackage{lettrine}

\ifpdf
\RequirePackage[colorlinks,hyperindex,plainpages=false]{hyperref}
\def\pdfBorderAttrs{/Border [0 0 0] } % No border arround Links
\else
\RequirePackage[plainpages=true]{hyperref}
\usepackage{color}
\fi

\usepackage[frenchb,english]{babel}
%
% Copyright \copyright{} 1998 Tobias Oetiker and all the Contributers
% to LShort.  All rights reserved.
% 
% This document is free; you can redistribute it and/or modify it
% under the terms of the GNU General Public License as published by
% the Free Software Foundation; either version 2 of the License, or
% (at your option) any later version.
% 
% This document is distributed in the hope that it will be useful, but
% WITHOUT ANY WARRANTY; without even the implied warranty of
% MERCHANTABILITY or FITNESS FOR A PARTICULAR PURPOSE.  See the GNU
% General Public License for more details.
% 
% You should have received a copy of the GNU General Public License
% along with this program; if not, write to the Free Software
% Foundation, Inc., 59 Temple Place, Suite 330, Boston, MA 02111-1307,
% USA.
%
% Original Copyright H.Partl, E.Schlegl, and I.Hyna (1987).
% English Version Copyright by Tobias Oetiker (1994,1995),
% 
% ---------------------------------------------------------------------
%
% Formats also with\textt{letterpaper} option, but the pagebreaks might not
% fall as nicely.
%
% To produce a A5 booklet, use a tool like  pstops or dvitodvi
% to  past them together in the right order. Most dvi printer drivers
% can shrink the resulting output to fit on a A4 sheet.
%
\makeindex
\typeout{Copyright T.Oetiker, H.Partl, E.Schlegl, I.Hyna}
%\includeonly{typeset,math} 
\begin{document}
\selectlanguage{frenchb}
\NoAutoSpaceBeforeFDP
\frontmatter
%%%%%%%%%%%%%%%%%%%%%%%%%%%%%%%%%%%%%%%%%%%%%%%%%%%%%%%%%%%%%%%%%
% Contents: The title page
% $Id: title.tex 174 2008-09-25 05:37:39Z oetiker $
%%%%%%%%%%%%%%%%%%%%%%%%%%%%%%%%%%%%%%%%%%%%%%%%%%%%%%%%%%%%%%%%%

% Voir la fin de ce fichier pour les informations de licence
% See the end of this file for license information

\ifpdf
  \pdfbookmark{Page de titre}{title}
\fi
\newlength{\centeroffset}
\setlength{\centeroffset}{-0.5\oddsidemargin}
\addtolength{\centeroffset}{0.5\evensidemargin}
\addtolength{\textwidth}{-\centeroffset}
\thispagestyle{empty}
\vspace*{\stretch{1}}
\noindent\hspace*{\centeroffset}%
\begin{minipage}{\textwidth}
\parindent=0pt
\flushright
{\Huge\bfseries Une courte (?)\\ 
introduction � \LaTeXe}\\
\rule[-1ex]{\textwidth}{5pt}\\[2.5ex]
\emph{\Large ou \LaTeX2e en \pageref{LastPage} minutes}\\[2ex]
\end{minipage}

\vspace{\stretch{1}}
\noindent\hspace*{\centeroffset}\begin{minipage}{\textwidth}
\flushright
{\bfseries 
par Tobias Oetiker\\[1.5ex]
Hubert Partl, Irene Hyna et  Elisabeth Schlegl\\[1.5ex]
traduit en fran�ais par Samuel Colin\\
(� partir de la version 3.21)\\[3ex]} 
Version~4.26fr-1, 7 ao�t 2009
\end{minipage}
\addtolength{\textwidth}{\centeroffset}
\vspace{\stretch{2}}


\pagebreak
\begin{small} 

  Copyright \copyright{} 1995-2009 Tobias Oetiker et les contributeurs.

  Copyright \copyright{} 1998-2001 LAAS/CNRS pour la traduction
  jusqu'� la version 3.20 incluse.

  Copyright \copyright{} 2009 Samuel Colin pour la traduction
  � partir de la version 3.21 incluse.

  Ce document est libre ; vous pouvez le redistribuer et/ou le
  modifier selon les termes de la licence publique g�n�rale de GNU
  publi�e par la Free Software Foundation (version 2 ou tout autre
  version ult�rieure choisie par vous)

  Ce document est diffus� en esp�rant qu'il sera utile, mais \emph{sans
    aucune garantie}, ni explicite ni implicite, sans m�me la garantie
  implicite d'�tre \emph{commercialisable} ou \emph{adapt� � un but
    sp�cifique}.  Reportez-vous � la licence publique g�n�rale de GNU pour
  plus de d�tails.

  Vous devez avoir re�u une copie de la licence publique g�n�rale de
  GNU en m�me temps que ce document. Si ce n'est pas le cas, �crivez �
  la Free Software Fundation, Inc., 675 Mass Ave, Cambridge, MA 02139,
  �tats-Unis.

  Les fichiers suivants, qui ont servi � la production de ce document,
  sont couverts par cette notice de licence :
  biblio.tex contrib.tex custom.tex graphic.tex lshort-a5.tex
  lshort-base.tex lshort.tex lssym.tex math.tex overview.tex
  spec.tex things.tex title.tex typeset.tex


\vspace{4ex}

  Copyright \copyright 1995-2005 Tobias Oetiker and Contributers.  All rights reserved.
 
  This document is free; you can redistribute it and/or modify it
  under the terms of the GNU General Public License as published by
  the Free Software Foundation; either version 2 of the License, or
  (at your option) any later version.
  
  This document is distributed in the hope that it will be useful, but
  \emph{without any warranty}; without even the implied warranty of
  \emph{merchantability} or \emph{fitness for a particular purpose}.
  See the GNU General Public License for more details.
  
  You should have received a copy of the GNU General Public License
  along with this document; if not, write to the Free Software
  Foundation, Inc., 675 Mass Ave, Cambridge, MA 02139, USA.

  This license notice covers the following files, that were used for
  producing this document:
  biblio.tex contrib.tex custom.tex graphic.tex lshort-a5.tex
  lshort-base.tex lshort.tex lssym.tex math.tex overview.tex
  spec.tex things.tex title.tex typeset.tex

\end{small}


\endinput


%

% Local Variables:
% TeX-master: "lshort2e"
% mode: latex
% mode: flyspell
% End:

%%%%%%%%%%%%%%%%%%%%%%%%%%%%%%%%%%%%%%%%%%%%%%%%%%%%%%%%%%%%%%%%%
% Contents: Who contributed to this Document
% $Id: contrib.tex,v 1.1.1.1 2002/02/26 10:04:20 oetiker Exp $
%%%%%%%%%%%%%%%%%%%%%%%%%%%%%%%%%%%%%%%%%%%%%%%%%%%%%%%%%%%%%%%%%
\chapter{Merci !}
\thispagestyle{plain}

\label{lettrine}
\lettrine{C}{e document}
est une traduction en fran�ais de �\,\emph{The
not so short introduction to LaTeX2e}\,� par Tobias Oetiker.

\noindent 
Une grande partie de ce document provient d'une introduction
autrichienne � \LaTeX\ 2.09, �crite en allemand par :
\begin{verse}
\contrib{Hubert Partl}{partl@mail.boku.ac.at}%
{Zentraler Informatikdienst der Universit\"at f\"ur Bodenkultur, Wien}
\contrib{Irene Hyna}{Irene.Hyna@bmwf.ac.at}%
   {Bundesministerium f\"ur Wissenschaft und Forschung, Wien}
\contrib{Elisabeth Schlegl}{no email}%
   {in Graz}
\end{verse}

La version courante en fran�ais est disponible sur :\\
\texttt{CTAN:/info/lshort/french/}% 
\,\footnote{Voir page~\pageref{CTAN} la liste des sites \texttt{CTAN}.}

Vous trouverez la version anglaise de Tobias Oetiker sur :\\
\texttt{CTAN:/info/lshort/english/}

Si vous �tes int�ress�s par la version allemande, vous trouverez une
version adapt�e � \LaTeXe{} par J\"org Knappen sur :\\
\texttt{CTAN:/info/lshort/german/}

%\vspace{\stretch{1}}

\newpage
Pour la pr�paration de ce document, l'aide des lecteurs du forum 
\mbox{Usenet}
\texttt{comp.text.tex} a �t� sollicit�e. De nombreuses personnes ont
r�pondu et ont fourni des corrections, des suggestions et du texte pour
am�liorer ce document. Qu'ils en soient ici remerci�s
sinc�rement. Ajoutons que je suis responsable de toutes les erreurs
que vous pourriez trouver dans ce document.

Merci en particulier � :

\begin{quote}
\flushleft
Rosemary~Bailey,        %r.a.bailey@qmw.ac.uk 0.2
David~Carlisle,         %carlisle@cs.man.ac.uk 1.0
Christopher~Chin,       %chris.chin@rmit.edu.au 3.1
Chris~McCormack,        %chrismc@eecs.umich.edu 0.1
Wim~van~Dam,            %wimvdam@cs.kun.nl 2.2
David~Dureisseix,       %dureisse@lmt.ens-cachan.fr 1.1
Elliot,                 %enh-a@minster.york.ac.uk 1.1
David~Frey,             %david@eos.lugs.ch 2.2
Robin~Fairbairns,       %Robin.Fairbairns@cl.cam.ac.uk 0.2 1.0
Alexandre~Guimond,      %guimond@IRO.UMontreal.CA 0.9
Cyril~Goutte,           %goutte@ei.dtu.dk 2.1 2.2
Greg~Gamble,            %gregg@maths.uwa.edu.au 2.2
Neil~Hammond,           %nfh@dmu.ac.uk 0.3
Rasmus~Borup~Hansen,    %rbhfamos@math.ku.dk 0.2 0.9 0.91 0.92 1.9.9
Martien~Hulsen,         %M.A.Hulsen@WbMt.TUDelft.NL 1.0 1.1
Werner~Icking,          %<Werner.Icking@gmd.de> 3.1
Jakob,                  %diness@get2net.dk
Eric~Jacoboni,          %jacoboni@enseeiht.fr 0.1 0.9
Alan~Jeffrey,           %alanje@cogs.sussex.ac.uk 0.2
Byron~Jones,            %bj@dmu.ac.uk 1.1
David~Jones,            %djones@CA.McMaster.dcss.insight 1.1
Johannes-Maria~Kaltenbach, %<kaltenbach@zeiss.de> 3.01
Andrzej~Kawalec,        %akawalec@prz.rzeszow.pl 1.9.9
Alain~Kessi,            %alain.kessi@psi.ch 2.2
Christian Kern,         %ck@unixen.hrz.uni-oldenburg.de 2.1
J\"org~Knappen,         %knappen@vkpmzd.kph.uni-mainz.de 0.1
Kjetil~Kjernsmo,        %<kjetil.kjernsmo@astro.uio.no> 3.2
Maik~Lehradt,           %greek@uni-paderborn.de 0.1
Martin~Maechler,        %<maechler@stat.math.ethz.ch> 2.2
Claus~Malten,           %<ASI138%BITNET.DJUKFA11@BITNET.CEARN> 1.1
Hubert~Partl,           %partl@mail.boku.ac.at 0.2 1.1
John~Refling,           %refling@gov.lbl.sierra 0.1 0.9
Mike~Ressler,           %ressler@cougar.jpl.nasa.gov 0.1 0.2 0.9 1.0 1.9.9
Brian~Ripley,           %ripley@stats.ox.ac.uk 2.1
Young~U.~Ryu,           %ryoung@utdallas.edu 2.1
Chris~Rowley,           %C.A.Rowley@open.ac.uk 0.91
Hanspeter~Schmid,       %schmid@isi.ee.ethz.ch
Craig~Schlenter,        %cschle@lucy.ee.und.ac.za 0.1 0.2 0.9
Josef~Tkadlec,          %tkadlec@math.feld.cvut.cz 2.0 2.2
Didier~Verna,           %verna@inf.enst.fr 2.2
Fabian~Wernli,          %wernli@iap.fr 3.2
Fritz~Zaucker,          %zaucker@ee.ethz.ch 3.0
Rick~Zaccone            %zaccone@bucknell.edu 2.2
et Mikhail~Zotov        % <zotov@eas.npi.msu.su> 3.1

\end{quote}

\vspace{2ex}
La version fran�aise a b�n�fici� des contributions des lecteurs du 
forum \texttt{fr.comp.text.tex} et en particulier de :
\begin{quote}
\flushleft
 Sebastien Blondeel,            % blondeel@clipper.ens.fr 
 Marie-Dominique Cabanne,       % Marie-Dominique.Cabanne@laas.fr
 Christophe Dousson,            % christophe.dousson@rd.francetelecom.com
 Olivier Dupuis,		% olivier.dupuis@incotec.fr
 Daniel Flipo,                  % Daniel.Flipo@univ-lille1.fr
 Paul Gaborit,                  % gaborit@enstimac.fr
 Thomas Ribo,			% thomas.ribo@free.fr
 Philippe Spiesser              % Philippe.Spiesser@laas.fr
et Vincent Zoonekynd.           % zoonek@math.jussieu.fr
\end{quote}

\vspace*{\stretch{1}}

\emph{Note du traducteur :} je tiens �galement � remercier
chaleureusement les auteurs de ce document de le rendre publiquement
utilisable et d'avoir ainsi rendu possible cette version fran�aise.

\pagebreak
\endinput
%%% Local Variables: 
%%% mode: latex
%%% TeX-master: "lshort"
%%% End: 
% Local Variables:
% mode: flyspell
% End:

%%%%%%%%%%%%%%%%%%%%%%%%%%%%%%%%%%%%%%%%%%%%%%%%%%%%%%%%%%%%%%%%%
% Contents: Who contributed to this Document
% $Id: overview.tex,v 1.1.1.1 2002/02/26 10:04:21 oetiker Exp $
%%%%%%%%%%%%%%%%%%%%%%%%%%%%%%%%%%%%%%%%%%%%%%%%%%%%%%%%%%%%%%%%%

% Because this introduction is the reader's first impression, I have
% edited very heavily to try to clarify and economize the language.
% I hope you do not mind! I always try to ask "is this word needed?"
% in my own writing but I don't want to impose my style on you... 
% but here I think it may be more important than the rest of the book.
% --baron

\chapter{Pr�face}
\thispagestyle{plain}

\LaTeX{}\cite{manual} est un logiciel de composition typographique
adapt� � la production de documents scientifiques et math�matiques de
grande qualit� typographique. Il permet �galement de produire
toutes sortes d'autres documents, qu'il s'agisse de simples lettres ou
de livres entiers. \LaTeX{} utilise \TeX\cite{texbook} comme outil de
mise en page. 

Cette introduction d�crit \LaTeXe{} et devrait se montrer suffisante
pour la plupart des applications de \LaTeX. Pour une description
compl�te du syst�me \LaTeX{}, reportez-vous
�~\cite{manual,companion}. 

\bigskip
\noindent Cette introduction est compos�e de cinq chapitres :
\begin{description}

\item[Le chapitre 1] pr�sente la structure �l�mentaire d'un document
  \LaTeXe{}. Il vous apprendra �galement quelques �l�ments sur
  l'histoire de \LaTeX{}. Apr�s avoir lu ce chapitre, vous devriez
  avoir une vue g�n�rale de ce qu'est \LaTeX{}.

\item[Le chapitre 2] entre dans les d�tails de la mise en page d'un
  document. Il explique les commandes et les environnements
  essentiels de \LaTeX{}. Apr�s avoir lu ce chapitre, vous serez
  capables de r�diger vos premiers documents.

\item[Le chapitre 3] explique comment coder des formules
  math�matiques avec \LaTeX{}. De nombreux exemples sont donn�s pour
  montrer comment utiliser cet atout majeur de \LaTeX{}. � la fin de ce
  chapitre, vous trouverez des tableaux qui listent tous les symboles
  math�matiques disponibles.

\item[Le chapitre 4] explique comment r�aliser un index, une liste de
  r�f�rences bibliographiques ou l'insertion de figures en PostScript
  encapsul�.  Il pr�sente la cr�ation de documents PDF avec
  pdf\LaTeX{} ainsi que quelques autres extensions utiles.

\item[Le chapitre 5] montre comment utiliser \LaTeX{} pour cr�er des
  figures. Au lieu de dessiner une image � l'aide d'un programme
  d'infographie donn�, la sauvegarder et l'inclure dans \LaTeX{}, vous
  d�crirez l'image et laisserez \LaTeX{} la dessiner pour vous.

\item[Le chapitre 6] contient des informations potentiellement
  dangeureuses. Il vous apprend � modifier la mise en page standard
  produite par \LaTeX{} et vous  permet de transformer 
  les pr�sentations plut�t r�ussies de \LaTeX{} en quelque chose
  de laid ou magnifique, selon votre habilet�.
  d'assez laid.
\end{description}

\bigskip
\noindent Il est important de lire ces chapitres dans l'ordre. Apr�s
tout, ce livre n'est pas si long.  L'�tude attentive des exemples
est indispensable � la compr�hension de l'ensemble car ils contiennent
une bonne partie de l'information que vous pourrez trouver ici.

\bigskip
\noindent \LaTeX{} est disponible pour une vaste gamme de syst�mes
informatiques, des PCs et Macs aux syst�mes UNIX
\footnote{UNIX est une marque d�pos�e de The Open Group}
 et VMS. Dans de
nombreuses universit�s, il est install� sur le r�seau informatique,
pr�t � �tre utilis�. L'information n�cessaire pour y acc�der devrait
�tre fournie dans le \guide. Si vous avez des difficult�s pour
commencer, demandez de l'aide � la personne qui vous a donn� cette
brochure.  Ce document \emph{n'est pas} un guide d'installation du
syst�me \LaTeX{}. Son but est de vous montrer comment �crire vos
documents afin qu'ils puissent �tre trait�s par \LaTeX{}.

\bigskip
\index{CTAN@\texttt{CTAN}}
Si vous avez besoin de r�cup�rer des fichiers relatifs � \LaTeX{},
utilisez les sites \texttt{CTAN}(\emph{Comprehensive
  \TeX{} Archive Network})\label{CTAN}.
Le site principal est sur \url{http://www.ctan.org} et toutes les
extensions peuvent �tre obtenues sur l'archive ftp
\url{ftp://www.ctan.org} ou l'un de ses nombreux miroirs. En France un
miroir se trouve sur \url{ftp\string://ftp.lip6.fr/pub/TeX/CTAN/}. Aux
�tats-Unis, il s'agit de \url{ftp\string://ctan.tug.org/}, en Allemagne de
\url{ftp\string://ftp.dante.de/} et au Royaume-Uni de
\url{ftp\string://ftp.tex.ac.uk/}. 
Si vous n'�tes pas dans l'un de ces pays, choisissez le site le plus
proche de chez vous.

Vous verrez plusieurs r�f�rences � CTAN au long de ce document, en
particulier des pointeurs vers des logiciels ou des documents. Au lieu
d'�crire des URL complets, nous avons simplement �crit \texttt{CTAN:}
suivi du chemin d'acc�s � partir de l'un des sites CTAN ci-dessus.

Si vous souhaitez installer \LaTeX{} sur votre ordinateur, vous
trouverez sans doute une version adapt�e � votre syst�me sur
sur \CTAN|systems|.

\vspace{\stretch{1}}
\noindent Si vous avez des suggestions concernant ce qui pourrait �tre
ajout�, supprim� ou modifi� dans ce document, contactez soit
directement l'auteur de la version originale, soit moi-m�me, le
traducteur.  Nous sommes particuli�rement int�ress�s par des retours
d'utilisateurs d�butants en \LaTeX{} au sujet des passages de ce livre
qui devraient �tre mieux expliqu�s.


\bigskip
\begin{verse}
\contrib{Tobias Oetiker}{oetiker@ee.ethz.ch}%
{Department of Information Technology and\\ Electrical Engineering,
Swiss Federal Institute of Technology, Z�rich.}

\contrib{Matthieu Herrb}{matthieu.herrb@laas.fr}%
{Laboratoire d'Analyse et d'Architecture des Syst�mes,\\
Centre National de la Recherche Scientifique, Toulouse.}

\contrib{Samuel Colin}{scolin@hivernal.org}%
{(� partir de la version 3.21fr)}
\end{verse}
\vspace{\stretch{1}}
\noindent La version courante de ce document est disponible sur
\CTAN|info/lshort|

\endinput

%

% Local Variables:
% TeX-master: "lshort2e"
% mode: latex
% mode: flyspell
% End:

\tableofcontents
\listoffigures
\listoftables
\mainmatter
% chapitre ��ce qu'il faut savoir�� pour flshort (encod� en latin1)
%
% Pour les informations de licence, voir flshort.tex.
% See flshort.tex for licence information.
%
\svnid{$Id$}

\chapter{Ce qu'il faut savoir}

\begin{intro}
Dans la premi�re partie de ce chapitre vous trouverez une rapide
pr�sentation de la philosophie et de l'histoire de \LaTeXe. La
deuxi�me partie met l'accent sur les structures fondamentales d'un
document \LaTeX. Apr�s avoir lu ce chapitre, vous devriez avoir une
id�e d'ensemble du fonctionnement de \LaTeX qui vous aidera � mieux
comprendre les chapitres suivants.
\end{intro}

\section{Le nom de la b�te}

\subsection{\TeX}
 
\TeX est un programme �crit par \index{Knuth, Donald E.}Donald
E.~Knuth~\cite{texbook}.  Il est con�u pour la composition de textes et de
formules math�matiques. Knuth a commenc� � �crire \TeX en 1977 pour tenter
d'exploiter les possibilit�s du mat�riel d'impression num�rique qui commen�ait
� s'introduire dans le milieu de l'�dition de l'�poque. En particulier, il
souhaitait contrecarrer la baisse de qualit� typographique qui touchait ses
propres livres et articles. Le \TeX que nous connaissons aujourd'hui a �t�
publi� en 1982, avec de l�g�res am�liorations en 1989 visant � mieux g�rer les
caract�res 8 bits ainsi que plusieurs langages. \TeX est connu pour sa tr�s
grande stabilit�, sa capacit� � fonctionner sur toutes sortes d'ordinateurs,
et son absence quasi-totale de bogues. Son num�ro de version converge vers
$\pi$ et vaut actuellement\footnote{Au moment de la traduction\dots \NdT}
$3.1415926$.

\TeX se prononce ��Tech��, avec un ��ch�� comme dans le mot
allemand\footnote{Il y a en fait en Allemand deux fa�ons de prononcer le
  ��ch�� et peut-�tre que le ��ch�� doux de ��Pech�� serait plus adapt�.
  Interrog� sur ce point, Knuth a �crit dans la Wikipedia allemande�: \emph{�a
    ne me d�range pas que les gens prononcent \TeX de la fa�on qui leur
    pla�t\dots et en Allemagne beaucoup utilisent le ch doux parce que le X
    suit la voyelle e, et pas le ch dur qui suit la voyelle a. En Russie,
    ``tex'' est un mot tr�s courant, prononc� ``tyekh''. Mais je pense que
    c'est en Gr�ce qu'on entend la prononciation la plus correcte, avec le ch
    plus rude de ach et de Loch.}} ��Ach�� ou l'�cossais ��Loch��. Le ��ch��
provient de l'alphabet grec o� X est la lettre ��chi��. \TeX est aussi la
premi�re syllabe du mot grec texnologia (technologie). Dans un environnement
\texttt{ASCII}, \TeX devient \texttt{TeX}.

\subsection{\LaTeX}
 
\LaTeX est un ensemble de macros qui permettent � un auteur de
mettre en page son travail avec la meilleure qualit� typographique en
utilisant un format professionnel pr�-d�fini. \LaTeX a �t� �crit par
\index{Lamport, Leslie}Leslie Lamport~\cite{manual}. Il utilise \TeX
comme moteur typographique. De nos jours, \LaTeX est maintenu par
\index{Mittelbach, Frank}Frank Mittelbach.

% En 1994, \LaTeX a �t� mis � jour par
% l'�quipe \index{LaTeX3@\LaTeX 3}\LaTeX 3, men�e par \index{Mittelbach,
% Frank}Frank Mittelbach, afin de r�aliser certaines am�liorations
% demand�es depuis longtemps et de fusionner toutes les variantes qui
% s'�taient d�velopp�es depuis la sortie de \index{LaTeX 2.09@\LaTeX
% 2.09}\LaTeX 2.09 quelques ann�es auparavent. Pour distinguer cette
% nouvelle version des pr�c�dentes, elle est appel�e \index{LaTeX
% 2e@\LaTeXe}\LaTeXe. Ce document est relatif � \LaTeXe.

\LaTeX se prononce \textbf{[latex]}. Si vous voulez faire r�f�rence � \LaTeX
dans un environnement \texttt{ASCII}, utilisez \texttt{LaTeX}. \LaTeXe se
prononce \textbf{[latex d\o{}z\o{}]} et s'�crit \texttt{LaTeX2e}.

En anglais, cela donne \textbf{[la\i{}tex]} et \textbf{[la\i{}tex tu: i:]}. 

% La figure~\ref{components}, page~\pageref{components} montre
% l'interaction entre les diff�rents �l�ments d'un syst�me \TeX. Cette
% figure est extraite de \texttt{wots.tex} de Kees van der Laan.
% 
% \begin{figure}[btp]
% \begin{lined}{0.8\textwidth}
% \begin{center}
% \input{kees.fig}
% \end{center}
% \end{lined}
% \caption{�l�ments d'un syst�me \TeX} \label{components}
% \end{figure}

\section{Les bases}
 
\subsection{Auteur, �diteur et typographe}

Pour publier un texte, un auteur confie son manuscrit  � une maison
d'�dition. L'�diteur d�cide alors de la mise en page du document
(largeur des colonnes, polices de caract�res, pr�sentation des
en-t�tes,\dots). L'�diteur note ses instructions sur le manuscrit et
le passe � un technicien typographe qui r�alise la mise en page
en suivant ces instructions.

Un �diteur humain essaye de comprendre ce que l'auteur avait en t�te en
�crivant le manuscrit. Il d�cide de la pr�sentation des en-t�tes de chapitres,
citation, exemples, formules, etc. en fonction de son exp�rience
professionnelle et du contenu du manuscrit. 

Dans un environnement \LaTeX, celui-ci joue le r�le de l'�diteur et utilise
\TeX comme typographe pour la composition. Mais \LaTeX n'est qu'un programme
et a donc besoin de plus de directives. L'auteur doit en particulier lui
fournir la structure logique de son document. Cette information est ins�r�e
dans le texte sous la forme de ��commandes \LaTeX��.

Cette approche est totalement diff�rente de l'approche
\wi{WYSIWYG}\footnote{What you see is what you get -- Ce que vous voyez est ce
  qui sera imprim�.} utilis�e par des traitements de texte comme par exemple
\emph{\mbox{Microsoft} \mbox{Word}} ou \emph{\mbox{Corel} \mbox{WordPerfect}}.
Avec ces programmes, l'auteur d�finit la mise en page du document de mani�re
interactive pendant la saisie du texte. Tout au long de cette op�ration, il
voit � l'�cran � quoi ressemblera le document final une fois imprim�.

Avec \LaTeX, il n'est normalement pas possible de voir le r�sultat final
durant la saisie du texte. Mais celui-ci peut �tre pr�-visualis� apr�s
traitement du fichier par \LaTeX. Des corrections peuvent alors �tre apport�es
avant d'envoyer la version d�finitive � l'impression.

\subsection{Choix de la mise en page}

La typographie est un m�tier (un art ?). Les auteurs inexp�riment�s font
souvent de graves erreurs en consid�rant que la mise en page est avant tout
une question d'esth�tique�: ��si un document est beau, il est bien con�u��.
Mais un document doit �tre lu et non accroch� dans une galerie d'art. La
lisibilit� et la compr�hensibilit� sont bien
plus importantes que l'apparence. Par exemple�:
\begin{itemize}
  \item la taille de la police et la num�rotation des en-t�tes doivent �tre
    choisies afin de mettre en �vidence la structure des chapitres et des
    sections�;
  \item les lignes ne doivent pas �tre trop longues pour ne pas fatiguer la
    vue du lecteur, tout en remplissant la page de mani�re harmonieuse. 
\end{itemize}

Avec un logiciel \wi{WYSIWYG}, l'auteur produit g�n�ralement des documents
esth�tiquement plaisants (quoi que\dots) mais tr�s peu ou mal structur�s.
\LaTeX emp�che de telles erreurs de formatage en for�ant l'auteur � d�crire la
structure logique de son document et en choisissant lui-m�me la mise en page
la plus\footnote{Le traducteur n'est pas aussi optimiste et pense que \LaTeX
  choisit seulement \emph{une} mise en pages appropri�e, ce qui n'est d�j� pas
  mal. \NdT} appropri�e.

\subsection{Avantages et inconv�nients}

Un sujet de discussion qui  revient souvent quand des gens du monde
\wi{WYSIWYG} rencontrent des utilisateurs de \LaTeX est le suivant�: ��les
\wi{avantages de \LaTeX} par rapport � un traitement de texte classique�� ou
bien le contraire.  La meilleure chose � faire quand une telle discussion
d�marre, est de garder son calme, car souvent cela d�g�n�re. Mais parfois on
ne peut y �chapper\dots

\medskip
Voici donc quelques arguments. Les principaux avantages de \LaTeX par rapport
� un traitement de texte traditionnel sont�:

\begin{itemize}
  \item mise en page professionnelle qui donne aux documents l'air de sortir
    de l'atelier d'un imprimeur�;
  \item la composition des formules math�matiques se fait de mani�re
    pratique�;
  \item il suffit de conna�tre quelques commandes de base pour d�crire la
    structure logique du document. Il n'est pas n�cessaire de se pr�occuper
    de la mise en page�;
  \item m�me les structures complexes telles que des notes de bas de page, des
    renvois, la table des mati�res ou les r�f�rences bibliographiques sont
    faciles � produire�;
  \item pour la plupart des t�ches de la typographie qui ne sont pas
    directement g�r�es par \LaTeX, il existe des extensions gratuites : par
    exemple pour inclure des figures \textsc{PostScript} ou pour formater une
    bibliographie selon un standard pr�cis. La majorit� de ces extensions sont
    d�crites dans \companion�;
  \item \LaTeX encourage les auteurs � �crire des documents bien structur�s,
    parce que c'est ainsi qu'il fonctionne (en d�crivant la structure)�;
  \item \TeX, l'outil de formatage de \LaTeXe, est r�ellement
    portable et gratuit. Le syst�me peut ainsi tourner sur presque
    tous les types de machines existants.
\end{itemize}

\medskip
\LaTeX a �galement quelques inconv�nients�; il est difficile pour moi d'en
trouver de s�rieux, mais d'autres vous en citeront des centaines�:

\begin{itemize}
  \item \LaTeX ne fonctionne pas bien pour ceux qui ont vendu leur �me�;
  \item bien que quelques param�tres des mises en page pr�-d�finies puissent
    �tre personnalis�s, la mise au point d'une pr�sentation enti�rement
    nouvelle est difficile et demande beaucoup de temps\footnote{La rumeur dit
      que c'est un des points qui devrait �tre am�lior�s dans la future
      version \LaTeX 3}\index{LaTeX3@\LaTeX 3.}�;
  \item �crire des documents mal organis�s et mal structur�s est tr�s
    difficile ;
  \item il est possible que votre hamster, malgr� des d�buts encourageants, ne
    parvienne jamais � bien comprendre la notion de balisage logique.
\end{itemize}

\section{Fichiers source \LaTeX}

L'entr�e de \LaTeX est un fichier en texte brut. Vous pouvez le cr�er avec
l'�diteur de texte de votre choix. Il contient le texte de votre document
ainsi que les commandes qui vont dire � \LaTeX comment mettre en page le
texte. On appelle ce fichier le \emph{\wi{fichier source}}\footnote{Couramment
  �lid� en � le source � dans la conversation. \NdT}.

\subsection{Espaces}

Les caract�res d'espacement, tels que les blancs ou les tabulations
sont trait�s de mani�re unique comme ��\wi{espace}�� par
\LaTeX. Plusieurs \wi{blancs} \emph{cons�cutifs} sont consid�r�s
comme \emph{une seule} espace\footnote{En langage typographique,
    \emph{espace} est un mot f�minin. \NdT}. Les espaces en d�but
de ligne sont en g�n�ral ignor�es et un seul retour � la ligne est
trait� comme une espace. \index{espace!en d�but de ligne}

Une ligne vide entre deux lignes de texte marque la fin d'un paragraphe.
\emph{Plusieurs} lignes vides sont consid�r�es comme \emph{une seule} ligne
vide. Le texte ci-dessous est un exemple. On voit � gauche le contenu du
fichier source et � droite le r�sultat format�.

\begin{example}
Saisir un ou      plusieurs
espaces entre  les     mots
n'a pas d'importance.

Une ligne vide commence 
un nouveau paragraphe.
\end{example}
 
\subsection{Caract�res sp�ciaux}

Les symboles suivants sont des \wi{caract�res r�serv�s} qui, soit ont 
une signification sp�ciale dans \LaTeX, soit ne sont pas disponibles
dans toutes les polices. Si vous les saisissez directement dans votre
texte, ils ne seront pas imprim�s mais forceront \LaTeX � faire des
choses que vous n'avez pas voulues.
\begin{code}
\verb.#  $  %  ^  &  _  {  }  ~  \ .
\end{code}

\index{Backslash|see{Contre-oblique}}
\index{Antislash|see{Contre-oblique}}
\index{Barre oblique inverse|see{Contre-oblique}}
Comme vous le voyez, certains de ces caract�res peuvent �tre utilis�s dans vos
documents en les pr�fixant par une \wi{contre-oblique}\footnote{Aussi nomm�e
  ��barre oblique inverse��, ou parfois \nolang{antislash} d'apr�s l'anglais
  \eng{backslash}. \NdT } :

\begin{example}
\# \$ \% \^{} \& \_ \{ \} \~{} 
\end{example} 

Les autres et bien d'autres encore peuvent �tre obtenus avec des commandes
sp�ciales � l'int�rieur de formules math�matiques ou comme accents. La
contre-oblique \textbackslash ne peut pas �tre saisi en ajoutant une
contre-oblique devant (�\\�). Cette s�quence est utilis�e pour indiquer les
coupures de ligne\footnote{Pour produire une contre-oblique dans le texte,
  utilisez la commande \ci{textbackslash}, fournie par l'extension
  \pai{textcomp}. En mode maths, utilisez \ci{backslash}. [La version
  originale ne fait pas cette distinction. \NdT]}.

\subsection{Commandes \LaTeX}

Les commandes \LaTeX sont sensibles � la casse des caract�res (majuscules ou
minuscules) et utilisent l'une des deux formes suivantes :

\begin{itemize}
  \item soit elles commencent par une \wi{contre-oblique} �\� et ont un nom
    compos� uniquement de lettres. Un nom de commande est termin� par une
    espace, un chiffre ou tout autre caract�re qui n'est pas une lettre�;
  \item soit elles sont compos�es d'une contre-oblique et d'exactement un
    caract�re sp�cial (non-lettre).
\end{itemize}

\label{whitespace}
\LaTeX ignore les espaces apr�s les commandes. Si vous souhaitez obtenir un
blanc apr�s une commande\index{espace!apr�s une commande}, il faut ou bien
ins�rer \verb|{}| suivi d'un blanc ou bien utiliser une commande d'espacement
sp�cifique de \LaTeX. La s�quence \verb|{}| emp�che \LaTeX d'ignorer les
blancs apr�s une commande.

\begin{example}
J'ai lu que Knuth classe les
gens qui utilisent \TeX en
\TeX{}niciens et en \TeX perts.\\
Nous sommes le \today.
\end{example}

Certaines commandes sont suivies d'un \wi{param�tre} qui est mis entre
\wi{accolades} \verb|{ }|. Certaines commandes supportent des
\wi{param�tres optionnels} qui suivent le nom de la commande entre
\wi{crochets}~\verb|[ ]|. L'exemple suivant montre quelques commandes
\LaTeX. Ne vous tracassez pas pour les comprendre, elles seront
expliqu�es plus loin.

\begin{example}
\textsl{Penchez}-vous !
\end{example}
\begin{example}
S'il vous pla�t, passez � la 
ligne ici.\newline
Merci !
\end{example}

\subsection{Commentaires}
\index{commentaires}

Quand \LaTeX rencontre un caract�re \verb|%| dans le fichier source, il ignore
le reste de la ligne en cours, le changement de ligne, et tous les espaces au
d�but de la ligne\footnote{Ceci est habituel et n'est pas propre �
  \texttt{\%}. \NdT} suivante. C'est utile pour ajouter des notes qui
  n'appara�tront  pas dans la version imprim�e.

\begin{example}
% D�monstration�:
Ceci est un % mauvais
exemple: anticonstitu%
       tionnellement
\end{example}

Le caract�re \verb|%| peut �galement �tre utilis� pour couper des lignes trop
longues dans le fichier d'entr�e, lorsqu'aucun espace ou coupure n'est
autoris�. 

Pour cr�er des commentaires plus longs, on peut utiliser
l'environnement \ei{comment} fourni par l'extension \pai{verbatim}. 
Ceci signifie que vous devez ajouter la ligne �\usepackage{verbatim}� au
pr�ambule de votre document, comme nous le verrons plus loin, pour utiliser
cette commande.

\begin{example}
Voici un autre exemple
\begin{comment}
Limit� mais d�monstratif
\end{comment}
de commentaires. 
\end{example}

Cet environnement n'est pas utilisable � l'int�rieur d'autres
environnements complexes, tels que le mode math�matique par exemple. 

\section{Structure du fichier source}

Quand \LaTeX traite un fichier source, il s'attend � y trouver une certaine
structure. C'est pourquoi chaque fichier source doit commencer par la
commande�:
\begin{code}
\verb|\documentclass{...}|
\end{code}
Elle indique quel type de document vous voulez �crire. Apr�s cela vous
pouvez ins�rer des commandes qui vont influencer le style du document
ou vous pouvez charger des \wi{extension}s qui ajoutent de nouvelles
fonctions au syst�me \LaTeX. Pour charger une extension, utilisez la
commande�:
\begin{code}
\verb|\usepackage{...}|
\end{code}

Quand tout le travail de pr�paration est fait\footnote{La partie entre
  �documentclass� et �\begin{document}� est appel�e le
  \emph{\wi{pr�ambule}}.}, vous pouvez commencer le corps du texte avec la
commande�:
\begin{code}
\verb|\begin{document}|
\end{code}

Maintenant vous pouvez saisir votre texte et y ins�rer des commandes \LaTeX. �
la fin de votre document, utilisez la commande
\begin{code}
\verb|\end{document}|
\end{code}
pour dire � \LaTeX qu'il en a fini. Tout ce qui suivra dans le fichier source
sera ignor�.

La figure~\ref{mini} montre le contenu d'un document \LaTeXe
minimum. Un fichier source plus complet est pr�sent� sur la
figure~\ref{document}. 

\begin{figure}[hbp]
\begin{lined}{6cm}
\begin{verbatim}
\documentclass{article}
\begin{document}
Small is beautiful.
\end{document}
\end{verbatim}
\end{lined}
\caption{Un fichier \LaTeX minimal} \label{mini}
\end{figure}
 
\begin{figure}[htbp]
\begin{lined}{10cm}
\begin{verbatim}
\documentclass[a4paper,11pt]{article}
\usepackage[T1]{fontenc}
\usepackage[english,francais]{babel}
\author{P.~Tar}
\title{Le Minimalisme}
\begin{document}
\maketitle
\tableofcontents
\section{D\'ebut}
\`A \'ecrire\dots
\section{Suite et fin}
On verra plus tard.
\end{document}
\end{verbatim}
\end{lined}
\caption[Exemple d'un article de revue plus r�aliste]{Exemple d'un article de
  revue plus r�aliste. Toutes les commandes de cet exemple seront expliqu�es
  plus tard dans cette introduction.}
\label{document}
\end{figure}
 
\section{Utilisation typique en ligne de commande}

Vous br�lez probablement d'envie d'essayer l'exemple pr�sent�
page~\pageref{mini}. Voici quelques informations�: \LaTeX lui-m�me ne propose
pas d'interface graphique ni de jolis boutons � cliquer. Il s'agit simplement
d'un programme qui ��dig�re�� votre fichier source. Certaines installations de
\LaTeX ajoutent une telle interface graphique permettant de cliquer pour
lancer la � compilation � de votre document. Sur d'autres syst�mes, il faut
taper quelques commandes, donc voyons comment convaincre \LaTeX de compiler
votre fichier source sur un syst�me orient� texte. Attention : ceci suppose
qu'une installation \LaTeX fonctionnelle est d�j� pr�sente sur votre
ordinateur\footnote{C'est le cas sur la plupart des syst�mes Unix bien
  �lev�s\dots et les Vrais Hommes utilisent Unix, donc\dots \texttt{;-)}}.

%%%%%%%%%%%%%%%%%%%%%%%%%%%%%%%%%%%%%%%%%%%%%%%%%%%%%%%%%%%%%%%%%%%%%%%%%%%%%%
%                     Fronti�re de la traduction r�vis�e                     %
%%%%%%%%%%%%%%%%%%%%%%%%%%%%%%%%%%%%%%%%%%%%%%%%%%%%%%%%%%%%%%%%%%%%%%%%%%%%%%

\begin{enumerate}
  \item Cr�ez/�ditez votre fichier source \LaTeX. Il s'agit d'un fichier texte
    pur. Sur les syst�mes Unix, tous les �diteurs cr�ent ce type de fichier.
    Sous Windows, assurez-vous que le fichier est sauvegard� en texte seul
    (ASCII ou encore ��plain text��). Choisissez pour votre fichier un nom
    avec le suffixe \texttt{.tex}. 
  \item Ex�cutez \LaTeX sur votre fichier. Si tout se passe bien, vous
    obtiendrez un nouveau fichier avec le suffixe \texttt{.dvi}. 
\begin{verbatim}
latex document.tex
\end{verbatim}
  \item � pr�sent, vous pouvez visualiser le r�sultat, le fichier DVI.
\begin{verbatim}
xdvi document.dvi
\end{verbatim}
    ou encore transformer le r�sultat en PostScript
\begin{verbatim}
dvips -Pcmz -o document.ps document.dvi
\end{verbatim}
    \texttt{\wi{xdvi}} et \texttt{\wi{dvips}} sont des logiciels %libres qui
    manipulent les fichiers DVI. Ils sont disponibles sur la plupart des
    syst�mes Unix. Sur les autres syst�mes, d'autres outils de manipulation
    des fichiers DVI sont disponibles. 
\end{enumerate}

\section{La mise en page du document}
 
\subsection {Classes de documents}\label{sec:documentclass}

La premi�re information dont \LaTeX a besoin en traitant un fichier source est
le type de document que son auteur est en train de cr�er. Ce type est sp�cifi�
par la commande \ci{documentclass}.
\begin{lscommand}
\ci{documentclass}\verb|[|\emph{options}\verb|]{|\emph{classe}\verb|}|
\end{lscommand}
Ici \emph{classe} indique le type de document � cr�er. Le
tableau~\ref{documentclasses} donne la liste des classes de documents
pr�sent�es dans cette introduction. \LaTeXe fournit d'autres classes pour
d'autres types de documents, notamment des lettres et des transparents. Le
param�tre \emph{\wi{option}s} permet de modifier le comportement de la classe
de document. Les options sont s�par�es par des virgules. Les principales
options disponibles sont pr�sent�es dans le tableau~\ref{options}.

\begin{table}[!thbp]
  \caption{Classes de documents} \label{documentclasses}
  \begin{lined}{12cm}
    \begin{description}
      \item [\normalfont\texttt{article}] pour des articles dans des revues
        scientifiques, des pr�sentations, des rapports courts, des
        documentations, des invitations, etc.  \index{article (classe)}
      \item [\normalfont\texttt{report}] pour des rapports plus longs
        contenant plusieurs chapitres, des petits livres, des th�ses, etc.
        \index{report (classe)} \index{rapport}
      \item [\normalfont\texttt{book}] pour des vrais livres.  \index{book
          (classe)} \index{livre}
      \item [\normalfont\texttt{slides}] pour des transparents. Cette classe
        utilise de grands caract�res sans serif. Voir �galement la classe
        Foil\TeX\footnote{%
          \texttt{CTAN:/tex-archive/macros/latex/contrib/supported/foiltex}}
        \index{slides@\textsf{slides}}\index{foiltex@\textsf{foiltex}}
        \index{transparents}
    \end{description}
  \end{lined}
\end{table}

\begin{table}[!hbp]
  \caption{Options de classes de document} \label{options}
  \begin{lined}{12cm}
    \begin{flushleft}
      \begin{description}
        \item[\normalfont\texttt{10pt}, \texttt{11pt}, \texttt{12pt}] \quad
          d�finit la taille de la police principale du document. Si aucune
          option n'est pr�sente, la taille par d�faut est de \texttt{10pt}.
          \index{taille!de la police par d�faut}
        \item[\normalfont\texttt{a4paper}, \texttt{letterpaper}, \dots] \quad
          d�finit la taille du papier. Le papier par d�faut est
          \texttt{letterpaper}, le format standard am�ricain. Les autres
          valeurs possibles sont�: \texttt{a5paper}, \texttt{b5paper},
          \texttt{executivepaper}, et \texttt{legalpaper}. 
          \index{legal (papier)}
          \index{taille!du papier} \index{A4 (papier)} \index{letter (papier)} 
          \index{A5 (papier)} \index{B5 (papier)} \index{executive (papier)}
          \index{papier!taille du} \index{papier!A4} \index{papier!A5}
          \index{papier!letter}
        \item[\normalfont\texttt{fleqn}] \quad aligne les formules
          math�matiques � gauche au lieu de les centrer.
          \index{fleqn@\texttt{fleqn}}
        \item[\normalfont\texttt{leqno}] \quad place la num�rotation des
          formules � gauche plut�t qu'� droite.  \index{leqno@\texttt{leqno}}
        \item[\normalfont\texttt{titlepage}, \texttt{notitlepage}] \quad
          indique si une nouvelle page doit �tre commenc�e apr�s le \wi{titre
            du document} ou non. La classe \texttt{article} continue par
          d�faut sur la m�me page contrairement aux classes \texttt{report} et
          \texttt{book}.  \index{titlepage@\texttt{titlepage}}
          \index{notitlepage@\texttt{notitlepage}}
        \item[\normalfont\texttt{twocolumn}] \quad demande � \LaTeX de
          formater le texte sur \wi{deux colonnes}.
          \index{twocolumn@\texttt{twocolumn}}
        \item[\normalfont\texttt{twoside, oneside}] \quad indique si la sortie
          se fera en \wi{recto-verso} ou en \wi{recto simple}. Par d�faut, les
          classes \texttt{article} et \texttt{report} sont en \wi{simple face}
          alors que la classe \texttt{book} est en \wi{double-face}.
          \index{twoside@\texttt{twoside}} \index{oneside@\texttt{oneside}}
        \item[\normalfont\texttt{openright, openany}] \quad fait commencer un
          chapitre sur la page de droite ou sur la prochaine page. Cette
          option n'a pas de sens avec la classe \texttt{article} qui ne
          conna�t pas la notion de chapitre. Par d�faut, la classe
          \texttt{report} commence les chapitres sur la prochaine page vierge
          alors que la classe \texttt{book} les commence toujours sur une page
          de droite.  \index{openright@\texttt{openright}}
          \index{openany@\texttt{openany}}
      \end{description}
    \end{flushleft}
  \end{lined}
\end{table}

Exemple�: un fichier source pour un document \LaTeX pourrait commencer par la
ligne
\begin{code}
  \ci{documentclass}\verb|[11pt,twoside,a4paper]{article}|
\end{code}
elle informe \LaTeX qu'il doit composer ce document comme un \emph{article}
avec une taille de caract�re de base de \emph{onze points} et qu'il devra
produire une mise en page pour une impression \emph{double face} sur du papier
au format \emph{A4}% \footnote{Sans l'option \texttt{a4paper}, le format de
papier sera am�ricain�: 8,5~$\times$~11 pouces, soit 216~$\times$~280 mm.}.

\subsection{Extensions}

\begin{table}[!tbp]
  \caption{Quelques extensions fournies avec \LaTeX} \label{extensions}
  \begin{lined}{11cm}
    \begin{description}
      \item[\normalfont\texttt{doc}] permet de documenter des programmes pour
        \LaTeX.\\ D�crite dans \texttt{doc.dtx}\footnote{Ce fichier devrait
          �tre intall� sur votre syst�me et vous devriez �tre capable de le
          formater avec la commande \texttt{latex doc.dtx}. Il en est de m�me
          pour les autres fichiers cit�s dans ce tableau.} et dans \companion.
      \item[\normalfont\pai{exscale}] fournit des versions de taille
        param�trable des polices math�matiques �tendues.\\ D�crite dans
        \texttt{ltexscale.dtx}.
      \item[\normalfont\pai{fontenc}] sp�cifie le \wi{codage} des polices de
        caract�re que \LaTeX va utiliser.\\ D�crite dans
        \texttt{ltoutenc.dtx}.
      \item[\normalfont\pai{ifthen}] fournit des commandes de la forme\\
        `if\dots then do\dots otherwise do\dots.'\\ D�crite dans
        \texttt{ifthen.dtx}, dans \companion et dans \desgraupes. 
      \item[\normalfont\pai{latexsym}] permet l'utilisation de la police des
        symboles \LaTeX.\\ D�crite dans \texttt{latexsym.dtx}, dans \companion
        et dans \desgraupes. 
      \item[\normalfont\pai{makeidx}] fournit des commandes pour r�aliser un
        index.\\ D�crite dans ce document, section~\ref{sec:indexing}, dans
        \companion et dans \desgraupes.
      \item[\normalfont\pai{syntonly}] analyse un document sans le formater.\\
        D�crite dans \texttt{syntonly.dtx} et dans \companion. Utile pour une
        v�rification rapide de la syntaxe.
      \item[\normalfont\pai{inputenc}] permet de sp�cifier le codage des
        caract�res utilis� dans le fichier source, parmi ASCII, ISO Latin-1,
        ISO Latin-2, 437/850 IBM code pages,  Apple Macintosh, Next,
        ANSI-Windows ou un codage d�fini par l'utilisateur.\\ D�crite dans
        \texttt{inputenc.dtx}. 
    \end{description}
  \end{lined}
\end{table}

\index{extension} 
En r�digeant votre document, vous remarquerez peut-�tre qu'il y a des domaines
o� les commandes de base de \LaTeX ne permettent pas d'exprimer ce que vous
voudriez. Si vous voulez inclure des graphiques, du texte en couleur ou du
code d'un programme dans votre document, il faut augmenter les possibilit�s de
\LaTeX gr�ce � des extensions.  Une extension est charg�e par la commande
\begin{lscommand}
\ci{usepackage}\verb|[|\emph{options}\verb|]{|\emph{extension}\verb|}|
\end{lscommand}
\emph{extension} est le nom de l'extension et \emph{options} une liste de
mots-cl�s qui d�clenchent certaines fonctions de l'extension. Certaines
extensions font partie de la distribution standard de \LaTeXe (reportez-vous
au tableau~\ref{extensions}). D'autres sont distribu�es � part. Le \guide peut
vous fournir plus d'informations sur les extensions install�es sur votre site.
\companion est la principale source d'information au sujet de \LaTeXe. Ce
livre contient la description de centaines d'extensions ainsi que les
informations n�cessaires pour �crire vos propres extensions � \LaTeXe.

\subsection{Styles de page}

\LaTeX propose trois combinaisons d'\wi{en-t�te}s et de \wi{pieds de page},
appel�es styles de page et d�finies par le param�tre \emph{style} de la
commande�:
\index{style de page!plain@\texttt{plain}}\index{plain@\texttt{plain}}
\index{style de page!headings@\texttt{headings}}
\index{headings@\texttt{headings}}
\index{style de page!empty@\texttt{empty}}\index{empty@\texttt{empty}}
\begin{lscommand}
  \ci{pagestyle}\verb|{|\emph{style}\verb|}|
\end{lscommand}
Le tableau~\ref{pagestyle} donne la liste des styles pr�d�finis.

\begin{table}[!hbp]
  \caption{Les styles de page de \LaTeX} \label{pagestyle}
  \begin{lined}{12cm}
    \begin{description}
      \item[\normalfont\texttt{plain}] imprime le num�ro de page au milieu du
        pied de page. C'est le style par d�faut.
      \item[\normalfont\texttt{headings}] imprime le titre du chapitre courant
        et le num�ro de page dans l'en-t�te de chaque page et laisse le pied
        de page vide. C'est � peu pr�s le style utilis� dans ce document.
      \item[\normalfont\texttt{empty}] laisse l'en-t�te et le pied de page
        vides. 
    \end{description}
  \end{lined}
\end{table}

On peut changer le style de la page en cours gr�ce � la commande
\begin{lscommand}
  \ci{thispagestyle}\verb|{|\emph{style}\verb|}|
\end{lscommand}

Au chapitre~\ref{chap:spec}, page~\pageref{sec:fancyhdr}, vous apprendrez
comment cr�er vos propres en-t�tes et pieds de pages.

\section{Les fichiers manipul�s}

L'utilisateur de \LaTeX est amen� � cotoyer un grand nombre de fichiers aux
suffixes divers, chaque suffixe renseigne sur le r�le du fichier, il est donc
utile d'en conna�tre la signification, voici les suffixes les plus courants�:
\begin{description}
  \item[\wi{.tex}] fichier source \TeX ou \LaTeX�;
  \item[\wi{.sty}] fichier contenant des commandes, que l'on charge dans le
    pr�ambule d'un document gr�ce � une commande \verb+\usepackage+�;
  \item[\wi{.dtx}] fichier contenant du code \LaTeX (commandes) document�, le
    lancement de \LaTeX sur un fichier \texttt{.dtx} en extrait la
    documentation.
  \item[\wi{.ins}] fichier permettant d'installer le contenu du
    fichier~\texttt{.dtx} de m�me nom. Une extension \LaTeX t�l�charg�e de
    l'Internet est compos�e d'un fichier \texttt{.dtx} et d'un \texttt{.ins}.
    Ex�cuter \LaTeX sur le fichier \texttt{.ins} pour extraire les fichiers �
    installer du \texttt{.dtx}.
  \item[\wi{.cls}] d�signe un fichier de \emph{classe} contenant la
    description d'un type de document, charg� par la commande
    \ci{documentclass};
  \item[\wi{.fd}] fichier contenant des d�finitions pour les polices de
    caract�res�; 
\end{description}

Les fichiers suivants sont produits par \LaTeX � partir du fichier source (de
suffixe~\texttt{.tex})�:
\begin{description}
  \item[\wi{.dvi}] signifie \emph{DeVice Independent}, c'est le fichier que
    l'on visualise � l'�cran et qui servira � l'impression (par \texttt{dvips}
    par exemple)�;
  \item[\wi{.log}] fichier contenant le compte-rendu d�taill� de la
    compilation du fichier source (avec les messages d'erreur �ventuels), 
  \item[\wi{.toc}] contient le mat�riel n�cessaire � la production de la table
    des mati�res, si celle-ci a �t� demand�e�;
  \item[\wi{.lof}] contient la liste num�rot�e des figures, si elle a �t�
    demand�e�;
  \item[\wi{.lot}] contient la liste num�rot�e des tableaux, si elle a �t�
    demand�e�;
  \item[\wi{.aux}] contient diverses informations utiles � \LaTeX, en
    particulier ce qui est n�cessaire au fonctionnement des r�f�rences
    crois�es. Le fichier \texttt{.aux} produit lors d'une ex�cution de \LaTeX
    est lu lors de l'ex�cution suivante�;
  \item[\wi{.idx}] fichier produit seulement si un index est demand�, il doit
    �tre trait� par \texttt{makeindex} (voir section~\ref{sec:indexing}
    page~\pageref{sec:indexing})�; \item[\wi{.ind}] fichier produit par
    \texttt{makeindex} � partir du~\texttt{.idx}, il contient l'index pr�t �
    �tre inclus dans le document�; \item[\wi{.ilg}] fichier contenant le
    compte-rendu du travail de \texttt{makeindex}.
\end{description}

\section{Gros documents}

Lorsque l'on travaille sur de gros documents, il peut �tre pratique de couper
le fichier source en plusieurs morceaux. \LaTeX a deux commandes qui vous
permettent de g�rer plusieurs fichiers sources.

\begin{lscommand}
  \ci{include}\verb|{|\emph{fichier}\verb|}|
\end{lscommand}
Vous pouvez utiliser cette commande dans le corps de votre document pour
ins�rer le contenu d'un autre fichier source. \LaTeX ajoute automatiquement le
suffixe \texttt{.tex} au nom sp�cifi�. Remarquez que \LaTeX va sauter une page
pour traiter le contenu de \emph{fichier}\texttt{.tex}. 

La seconde commande peut �tre utilis�e dans le pr�ambule. Elle permet de dire
� \LaTeX de n'inclure que certains des fichiers d�sign�s par les commandes
\verb|\include|.
\begin{lscommand}
  \ci{includeonly}\verb|{|\emph{fichier}\verb|,|\emph{fichier}%
    \verb|,|\ldots\verb|}|
\end{lscommand}
Apr�s avoir rencontr� cette commande dans le pr�ambule d'un document, seules
les commandes \ci{include} dont les fichiers sont cit�s en param�tre de la
commande \ci{includeonly} seront ex�cut�es. Attention, il ne doit pas y avoir
d'espace entre les noms de fichiers et les virgules. 

La commande \ci{include} saute une page avant de commencer le formatage du
texte inclus. Ceci est utile lorsqu'on utilise \ci{includeonly}, parce
qu'ainsi les sauts de pages ne bougeront pas, m�me si certains morceaux ne
sont pas inclus. Parfois ce comportement n'est pas souhaitable. Dans ce cas,
vous pouvez utiliser la commande�:
\begin{lscommand}
  \ci{input}\verb|{|\emph{fichier.tex}\verb|}|
\end{lscommand}
\noindent qui ins�re simplement le fichier indiqu� sans aucun traitement
sophistiqu�. 

\enlargethispage{\baselineskip}

Il est possible de demander � \LaTeX de simplement v�rifier la syntaxe d'un
document, sans produire de fichier~\texttt{.dvi} pour gagner du temps, en
utilisant l'extension~\texttt{syntonly}�:
\begin{verbatim}
\usepackage{syntonly}
\syntaxonly
\end{verbatim}
La v�rification termin�e, il suffit de mettre ces deux lignes (ou simplement
la seconde) en commentaire en pla�ant un~\texttt{\%} en t�te de ligne.

\endinput

%%% Local Variables: 
%%% mode: latex
%%% TeX-master: "flshort"
%%% End: 

%%%%%%%%%%%%%%%%%%%%%%%%%%%%%%%%%%%%%%%%%%%%%%%%%%%%%%%%%%%%%%%%%
% Contents: Typesetting Part of LaTeX2e Introduction
% $Id$
%%%%%%%%%%%%%%%%%%%%%%%%%%%%%%%%%%%%%%%%%%%%%%%%%%%%%%%%%%%%%%%%%

% Pour les informations de licence, voir contrib.tex.
% See contrib.tex for license information.

\chapter{Mise en page}
\thispagestyle{plain}

\begin{intro}
  Après la lecture du chapitre précédent vous connaissez maintenant
  les éléments de base qui constituent un document \LaTeX{}. Dans ce
  chapitre, nous allons compléter vos connaissances afin de vous
  rendre capables de créer des documents réalistes.
\end{intro}

\section{La structure du document et le langage}
\secby{Hanspeter Schmid}{hanspi@schmid-werren.ch}

La principale raison d'être d'un texte est de diffuser des idées, de
l'information ou de la connaissance au lecteur. Celui-ci comprendra
mieux le texte si ces idées sont bien structurées et il
ressentira d'autant mieux cette structure si la typographie utilisée
reflète la structure logique et sémantique du contenu.

Ce qui distingue \LaTeX{} des autres logiciels de traitement de texte
c'est qu'il suffit d'indiquer à \LaTeX{} la structure logique et
sémantique d'un texte. Il en déduit la forme typographique en fonction
des \enquote{règles} définies dans la classe de document et les différents
fichiers de style.

L'unité de texte la plus importante pour \LaTeX{} (et en typographie)
est le \wi{paragraphe}.  Le paragraphe est la forme typographique qui
contient une pensée cohérente ou qui développe une idée. Vous allez
apprendre dans les pages suivantes la différence entre un retour à la
ligne (obtenu avec la commande \texttt{\bs\bs}) et un changement de
paragraphe (obtenu en laissant une ligne vide dans le document
source). Une nouvelle réflexion doit débuter sur un nouveau
paragraphe, mais si vous poursuivez une réflexion déjà entamée, un
simple retour à la ligne suffit.

En général, on sous-estime complètement l'importance du découpage en
paragraphes. Certains ignorent même la signification d'un changement
de paragraphe ou bien, notamment avec \LaTeX{}, coupent des paragraphes
sans le savoir. Cette erreur est particulièrement fréquente lorsque
des équations sont présentes au milieu du texte. Étudiez les exemples
suivants et essayez de comprendre pourquoi des lignes vides
(changements de paragraphe) sont parfois utilisées avant et après
l'équation et parfois non. (Si vous ne comprenez pas suffisamment les
commandes utilisées, lisez d'abord la suite du chapitre puis revenez à
cette section.)

\begin{code}
\begin{verbatim}
% Exemple 1
\dots{} lorsqu'Einstein introduit sa formule
\begin{equation}
  e = m \cdot c^2 \; ,
\end{equation}
qui est en même temps la formule la plus connue et la
moins comprise de la physique.

% Exemple 2
\dots{} d'où vient la loi des courants de Kirchhoff :
\begin{equation}
  \sum_{k=1}{n} I_k = 0 \; .
\end{equation}

La loi des tensions de Kirchhof s'en déduit\dots

% Exemple 3
\dots{} qui a plusieurs avantages.

\begin{equation}
  I_D = I_F - I_R
\end{equation}
est le c\oe{}ur d'un modèle de transistor très
différent\dots
\end{verbatim}
\end{code}

L'unité de texte immédiatement inférieure est la phrase. Dans les
documents anglo-saxons, l'espace après le point terminant une phrase
est plus grande que celle qui suit un point après une
abréviation. (Ceci n'est pas vrai dans les règles de la typographie
française.) En général, \LaTeX{} se débrouille pour déterminer la
bonne largeur des espaces. S'il n'y arrive pas, vous verrez dans la
suite comment le forcer à faire quelque chose de correct.

La structure du texte s'étend même aux morceaux d'une phrase. Les
règles grammaticales de chaque langue gèrent la ponctuation de
manière très précise. Dans la plupart des langues, la virgule
représente une courte respiration dans le flux du langage. Si vous ne
savez pas trop où placer une virgule, lisez la phrase à voix haute en
respirant à chaque virgule. Si cela ne sonne pas naturellement à
certains endroits, supprimez la virgule ; au contraire, si vous
ressentez le besoin de respirer (ou de marquer une courte pause),
insérez une virgule à cet endroit.

Enfin, les paragraphes d'un texte sont également structurés au niveau
supérieur, en les regroupant en sections, chapitres, etc. L'effet
typographique d'une commande telle que
\begin{center}
\verb|\section{La structure du texte et du langage}|
\end{center}
\noindent est suffisamment évident pour
comprendre comment utiliser ces structures de haut niveau.

\section{Sauts de ligne et de page}

\subsection{Paragraphes justifiés}

\index{justification}
Les livres sont souvent composés de lignes qui ont toutes la même
longueur~; on dit qu'elles sont justifiées à droite. \LaTeX{} insère
des retours à la ligne et des espacements entre les mots de manière à
optimiser la présentation de l'ensemble d'un paragraphe. En cas de
besoin, il coupe les mots qui ne tiennent pas en entier sur une
ligne. La présentation exacte d'un paragraphe dépend de la classe de
document\footnote{Et des règles typographiques propres de chaque pays.
\NdT}. Normalement la première ligne d'un paragraphe est
en retrait par rapport à la marge gauche
et il n'y a pas d'espace vertical particulière entre deux
paragraphes (cf. section~\ref{parsp}).

Dans certains cas particuliers, il peut être nécessaire de demander à
\LaTeX{} de couper une ligne :
\begin{lscommand}
\ci{\bs} ou \ci{newline}
\end{lscommand}
\noindent commence une nouvelle ligne sans commencer un nouveau
paragraphe.

\begin{lscommand}
\ci{\bs*}
\end{lscommand}
\noindent empêche un saut de page après le saut de ligne demandé.

\begin{lscommand}
\ci{newpage}
\end{lscommand}
\noindent provoque un saut de page.

\begin{lscommand}
\ci{linebreak}\verb|[|\emph{n}\verb|]|,
\ci{nolinebreak}\verb|[|\emph{n}\verb|]|,
\ci{pagebreak}\verb|[|\emph{n}\verb|]|,
\ci{nopagebreak}\verb|[|\emph{n}\verb|]|
\end{lscommand}
\noindent indiquent les endroits où un saut de ligne ou de page devrait
apparaître ou non. L'action de ces commandes peut être paramétrée par l'auteur
grâce au paramètre optionnel \emph{n}
qui peut prendre une valeur entre zéro et quatre. En donnant à \emph{n}
une valeur inférieure à quatre, vous laissez à \LaTeX{} la possibilité
de ne pas tenir compte de votre commande si cela devait rendre le
résultat réellement laid.  Ne confondez pas ces commandes \enquote{break}
avec les commandes \enquote{new}.  Même lorsque vous utilisez
une commande \enquote{break}, \LaTeX{} essaye de justifier le bord
droit du texte et d'ajuster la longueur totale de la page, comme
expliqué plus loin ; cela peut mener à des trous disgracieux dans
votre texte.  Si vous voulez réellement commencer une \enquote{nouvelle}
ligne ou une \enquote{nouvelle} page, utilisez la commande \enquote{new}
correspondante.

\LaTeX{} essaye toujours de trouver les meilleurs endroits pour les
retours à la ligne.
S'il ne trouve pas de solution pour couper les lignes de
manière conforme à ses normes de qualité, il laisse dépasser un bout
de ligne sur la marge droite du paragraphe. \LaTeX{} émet alors le
message d'erreur
\enquote{\wiat{\texttt{overfull \bs hbox}}{overfull hbox}\footnote{Débordement horizontal.}}.
Cela se produit surtout quand \LaTeX{} ne trouve pas
de point de césure dans un mot.\footnote{Bien que \LaTeX{} signale un
  avertissement lorsque cela arrive et affiche la
  ligne qui pose problème, celle-ci n'est pas toujours facile à
  retrouver dans le texte. En utilisant l'option \texttt{draft} dans
  la commande \ci{documentclass}, ces lignes problématiques seront
  marquées d'une épaisse marque noire dans la marge de droite.}
En utilisant alors la commande \ci{sloppy},
vous pouvez demander à \LaTeX{} d'être moins exigeant. Il ne produira
 plus de lignes trop longues en ajoutant de l'espace entre les
mots du paragraphe, même si ceux-ci finissent trop espacés selon ses
critères. Dans ce cas le message
\enquote{\wiat{\texttt{underfull \bs hbox}}{underfull hbox}\footnote{Boîte horizontale pas assez pleine.}}
est produit. Souvent, malgré tout, le
résultat est acceptable. La commande \ci{fussy} agit dans l'autre
sens, au cas où vous voudriez voir \LaTeX{} revenir à ses exigences
normales.

\subsection{Césure} \label{hyph}
\index{cesure@césure}
\LaTeX{} coupe les mots en fin de ligne si nécessaire. Si l'algorithme de
césure\footnote{\eng{\wi{Hyphenation}} en anglais.} ne trouve pas
l'endroit correct pour couper un mot\footnote{Ce qui est normalement
plutôt rare. Si vous observez de nombreuses erreurs de césure, c'est
probablement un problème de spécification de la langue du document ou
du codage de sortie. Voir le paragraphe sur le support multilingue,
page~\pageref{international}.}, vous pouvez utiliser les
commandes suivantes pour informer \TeX{} de l'exception.

La commande :
\begin{lscommand}
\ci{hyphenation}\verb|{|\emph{liste de mots}\verb|}|
\end{lscommand}
\noindent permet de ne couper les mots cités en argument qu'aux
endroits indiqués par «~\verb|-|~». Cette commande doit être
placée dans le préambule et ne doit contenir que des mots composés de
lettres ou signes considérés comme normaux par \LaTeX{}. La casse des
caractères n'est pas prise en compte. Les informations de césure sont
associés au langage actif lors de l'invocation de la commande de
césure. Cela signifie que si vous placez une commande de césure dans
le préambule, cela influencera la césure de l'anglais \footnote{Par défaut
les documents sont supposés être en anglais. \NdT}. Si vous placez la
commande après \verb|\begin{document}| et que vous utilisez une
extension comme \pai{polyglossia} pour le support d'une autre langue,
alors les suggestions de césure seront actives pour le langage
activé via \pai{polyglossia}.

%MPG: let's do what we say... :-)
\hyphenation{Anti-cons-ti-tu-tion-nel-le-ment}
L'exemple ci-dessous permet à
\enquote{anticonstitutionnellement} et \enquote{Anticonstitutionnellement},
d'être coupés. Mais il empêche toute césure de
\enquote{FORTRAN}, \enquote{Fortran} ou \enquote{fortran}. Ni les
caractères spéciaux ni les symboles ne sont autorisés dans cette
commande.

\begin{code}
\verb|\hyphenation{FORTRAN}|\\
\verb|\hyphenation{Anti-cons-ti-tu-tion-nel-le-ment}|
\end{code}

La commande \verb|\hyphenation{|\emph{liste de mots}\verb|}| a un effet
\emph{global} sur toutes les occurrences des mots de la liste.
Si l'exception ne concerne qu'une occurrence d'un mot on utilise
la commande \ci{-} qui insère un point de  césure potentiel dans un
mot. Ces positions deviennent alors les \emph{seuls} points de césure
possibles pour cette occurrence du mot. Cette commande est
particulièrement utile pour les mots contenant des caractères
spéciaux, puisque \LaTeX{} ne réalise pas automatiquement la césure
pour ces derniers.
%\footnote{À moins d'utiliser les nouvelles \wi{polices DC}}

\begin{example}
I think this is: su\-per\-cal\-%
i\-frag\-i\-lis\-tic\-ex\-pi\-%
al\-i\-do\-cious
\end{example}

Normalement, en français, on ne coupe pas
la dernière syllabe d'un mot si elle est muette, mais il arrive qu'on
soit obligé de le faire, par exemple si on travaille sur des textes
étroits (cas de colonnes multiples).

Exemple : on pourra coder \verb+ils ex\-pri\-ment+
pour autoriser \emph{exceptionnellement} le rejet à la ligne suivante de
la syllabe muette \texttt{ment}.

Plusieurs mots peuvent être maintenus ensemble sur une ligne avec la
commande :
\begin{lscommand}
\ci{mbox}\verb|{|\emph{texte}\verb|}|
\end{lscommand}
\noindent
Elle a pour effet d'interdire toute coupure de ligne dans \emph{texte}.

\begin{example}
Mon num\'ero de t\'el\'ephone va
changer. \`A partir du 18 mai,
ce sera le  \mbox{0561 336 330}.

Le param\`etre
\mbox{\emph{nom du fichier}}
de la commande \texttt{input}
contient le nom du fichier
\`a lire.
\end{example}

\ci{fbox} est similaire à \ci{mbox}, à la différence qu'un cadre
visible sera en plus dessiné autour du contenu.

\section{Chaînes prêtes à l'emploi}

Dans les exemples précédents, vous avez découvert certaines commandes
permettant de produire le logo \LaTeX{} et quelques autres chaînes de
caractères spécifiques. Voici une liste de quelques-unes de ces
commandes :

\vspace{2ex}

\noindent
\begin{tabular}{@{}lll@{}}
Commande&Résultat&Description\\
\hline
\ci{today} & \today   & Date du jour\\
\ci{TeX} & \TeX       & Logo TeX\\
\ci{LaTeX} & \LaTeX   & Logo LaTeX\\
\ci{LaTeXe} & \LaTeXe & Sa version actuelle\\
\end{tabular}

\section{Caractères spéciaux et symboles}

\subsection{Guillemets}

Pour insérer des \wi{guillemets} n'utilisez pas le caractère \verb|"|
\index{""@\texttt{""}} comme sur une machine à écrire. En typographie,
il y a des guillemets ouvrants et fermants spécifiques. En anglais,
utilisez deux~\textasciigrave{} pour les guillemets ouvrants et
deux~\textquotesingle{}
pour les guillemets fermants.
En français, avec le paquet \pai{csquotes} et l'activation du français via \pai{polyglossia},
utilisez \ci{enquote} ou bien utilisez directement
\texttt{\guillemotleft} et
\texttt{\guillemotright} si vous disposez d'un moyen de saisir ces caractères.
\begin{example}
``Please press the `x' key.''

« Appuyez sur la touche `x'. »
\end{example}
%SC: Hmmm, I do not like much this translation
Je suis conscient que le rendu n'est pas idéal, mais il s'agit
effectivement d'un accent grave (\textasciigrave) pour l'ouverture et
d'une quote (\textquotesingle) (i.e. pas une apostrophe au sens
typographique du terme) pour la fermeture, et ce malgré ce que la
police choisie semble indiquer.

\subsection{Tirets}

\LaTeX{} connaît quatre types de \wi{tiret}s. Trois d'entre eux sont
obtenus en juxtaposant un nombre variable de tirets simples. Le
quatrième n'est  pas réellement un tiret~---~il s'agit du signe
mathématique moins. \index{-} \index{--} \index{---} \index{-@$-$}
\index{moins (signe)}

\begin{example}
belle-fille, pages 13-67\\
il parle ---~en vain~---
du passé.\\
oui~---~ou non ? \\
$0$, $1$ et $-1$
\end{example}

Notez que les exemples ci-dessus respectent les règles de la
typographie française concernant l'usage des tirets, qui diffèrent
des habitudes anglo-saxonnes, en particulier le double tiret n'est pas
utilisé en français.

\subsection{Tilde (\textasciitilde)}
\index{URL link}\index{tilde}
Un caractère souvent utilisé dans les adresses sur le web est le
tilde. Pour produire ce caractère avec \LaTeX{}, on peut utiliser
\verb|\~{}|, mais le résultat (\~{}) n'est pas tout à fait le symbole
attendu. Essayez ceci à la place :
\cih{sim}

\begin{example}
http://www.rich.edu/\~{}bush \\
http://www.clever.edu/$\sim$demo
\end{example}

Voir aussi l'extension \pai{hyperref} qui inclut une commande
\ci{url}.

\subsection{Barre oblique ou slash (/)}
\index{slash}\index{Barre oblique}
Pour obtenir une barre oblique entre deux mots, il suffit de l'écrire,
comme par exemple \texttt{lire/écrire}. Cependant \LaTeX{} considère
alors cela comme un seul mot et non deux. La césure est désactivée
pour ces mots, ce qui peut conduire à des erreurs de débordement de
ligne (\enquote{overfull hbox}). Pour surmonter ce problème, utilisez
\ci{slash} comme par exemple \verb|lire\slash écrire| qui autorise la
césure. Le caractère usuel de barre oblique (\texttt{/}) peut toujours
être utilisés pour des fractions ou des unités, par exemple \texttt{5 MB/s}.

\subsection{Symbole degré (\textdegree)}

L'exemple suivant montre comment obtenir un symbole \wi{degré} :

\begin{example}
Il fait $-30\,^{\circ}\mathrm{C}$.
Je vais bient\^ot devenir
supra-conducteur.
\end{example}

L'extension \pai{textcomp} fournit un symbole degré plus adapté, disponible
seul avec \ci{textdegree}, ou accompagné d'un C avec
\ci{textcelsius}.

\begin{example}
30 \textcelsius{} font
86 \textdegree{}F.
\end{example}

En français avec l'option \texttt{french} de \package{babel}, on dispose aussi
de la commande \ci{degres} qui donne un résultat similaire.

\subsection{Le symbole de l'euro \texorpdfstring{(\officialeuro)}{}}

Écrire sur tout sujet économique de nos jours requiert l'utilisation
du symbole de l'euro. De nombreuses polices de caractères contiennent
un symbole euro. Après avoir chargé l'extension \pai{textcomp} dans le
préambule
\begin{lscommand}
\ci{usepackage}\verb|{textcomp}|
\end{lscommand}
\noindent vous pouvez utiliser la commande
\begin{lscommand}
\ci{texteuro}
\end{lscommand}
\noindent pour y accéder.

Si votre police ne fournit pas son propre symbole de l'euro ou si vous
ne l'aimez pas, il vous reste d'autres possibilités.

Tout d'abord l'extension \pai{eurosym} qui fournit un symbol officiel
de l'euro~:
\begin{lscommand}
\ci{usepackage}\verb|[official]{eurosym}|
\end{lscommand}
Si vous préférez un symbole qui se marie bien à votre police, utilisez
plutôt \texttt{gen} à la place de \texttt{official}.

% Si les polices Adobe Eurofonts sont installées sur votre système (vous
% pouvez les obtenir gratuitement sur
% \url{ftp://ftp.adobe.com/pub/adobe/type/win/all}), vous pouvez
% utiliser l'extension \pai{europs} et la commande \ci{EUR} (pour
% un symbole de l'euro qui correspond à la police courante).
% ne fonctionne pas
% soit
% l'extension \pai{eurosans} et la commande \ci{euro} (pour l'\enquote{euro officiel}).

% L'extension \pai{marvosym} fournit également des symboles variés, y
% compris celui de l'euro, sous le nom \ci{EURtm}. Son défaut est de ne
% pas proposer des versions italiques et grasses de ce symbole.

\begin{table}[!htbp]
\caption{Un sac plein d'euros} \label{eurosymb}
\begin{lined}{10cm}
\begin{tabular}{llccc}
LM+textcomp  &\verb+\texteuro+ & \huge\texteuro &\huge\sffamily\texteuro
                                                &\huge\ttfamily\texteuro\\
eurosym      &\verb+\euro+ & \huge\officialeuro &\huge\sffamily\officialeuro
                                                &\huge\ttfamily\officialeuro\\
$[$gen$]$eurosym &\verb+\euro+ & \huge\geneuro  &\huge\sffamily\geneuro
                                                &\huge\ttfamily\geneuro\\
%europs       &\verb+\EUR + & \huge\EURtm        &\huge\EURhv
%                                                &\huge\EURcr\\
% eurosans     &\verb+\euro+ & \huge\EUROSANS  &\huge\sffamily\EUROSANS
%                                             & \huge\ttfamily\EUROSANS \\
% marvosym     &\verb+\EURtm+  & \huge\mvchr101  &\huge\mvchr101
%                                                &\huge\mvchr101
\end{tabular}
\medskip
\end{lined}
\end{table}

\subsection{Points de suspension (\dots)}

Sur une machine à écrire, une \wi{virgule} ou un \wi{point} occupent la
même largeur que les autres lettres. En typographie professionnelle,
le point occupe très peu de place et il est placé tout près du caractère
qui le précède. Il n'est donc pas possible d'utiliser trois points
consécutifs pour créer des \wi{points de suspension}. À la place on
utilise la commande spécifique :
%SC : upstream ajoute (low dots) après \ci{ldots}, mais ici on utilise dots
\begin{lscommand}
\ci{dots}
\end{lscommand}
\index{...@\dots}
\nonfrenchspacing
\begin{example}
Non pas comme \c{c}a...
mais ainsi :\\
New York, Tokyo, Budapest\dots
\end{example}
\frenchspacing

\subsection{Ligatures}

Certaines séquences de lettres ne sont pas composées simplement en
juxtaposant les différentes lettres les unes à la suite des autres,
mais en utilisant des symboles spéciaux.
\begin{code}
{\large ff fi fl ffi\dots}\quad
\`a la place de\quad {\large f{}f f{}i f{}l f{}f{}i\dots}
\end{code}
Ces \wi{ligature}s peuvent être désactivées en insérant un
\ci{mbox}\verb|{}| entre les lettres en question. Cela peut s'avérer
utile pour certains mots composés\footnote{Il n'existe pas d'exemple en
français. \NdT}.
\begin{example}
\Large Not shelfful\\
but shelf{}ful
\end{example}

\subsection{Accents et caractères spéciaux}

\LaTeX{} permet l'utilisation d'\wi{accent}s et de \wi{caractères
spéciaux} issus de nombreuses langues. Le tableau~\ref{accents} montre
tous les accents que l'on peut ajouter à la lettre o. Ils s'appliquent
naturellement aux autres lettres.

Pour placer un accent sur un i ou un j, il faut supprimer leur
point. Ceci s'obtient en tapant \verb|\i| et \verb|\j|.

\begin{example}
H\^otel, na\"\i ve, \'el\`eve,\\
sm\o rrebr\o d, !`Se\~norita!,\\
Sch\"onbrunner Schlo\ss{}
Stra\ss e
\end{example}

\begin{table}[!hbp]
\caption{Accents et caractères spéciaux} \label{accents}
\begin{lined}{10cm}
\begin{tabular}{*4{cl}}
\mstA{\`o} & \mstA{\'o} & \mstA{\^o} & \mstA{\~o} \\
\mstA{\=o} & \mstA{\.o} & \mstA{\"o} & \mstB{\c}{c}\\[6pt]
\mstB{\u}{o} & \mstB{\v}{o} & \mstB{\H}{o} & \mstB{\c}{o} \\
\mstB{\d}{o} & \mstB{\b}{o} & \mstB{\t}{oo} \\[6pt]
\mstA{\oe}  &  \mstA{\OE} & \mstA{\ae} & \mstA{\AE} \\
\mstA{\aa} &  \mstA{\AA} \\[6pt]
\mstA{\o}  & \mstA{\O} & \mstA{\l} & \mstA{\L} \\
\mstA{\i}  & \mstA{\j} & !` & \verb|!|\verb|`| & ?` & \verb|?|\verb|`|
\end{tabular}
\index{i et j@\i{} et \j{} sans points}\index{scandinaves (caractères)}
\index{ae@\ae}\index{umlaut}\index{accent!grave}\index{accent!aigu}
\index{accent!circonflexe}
\index{oe@\oe}\index{aa@\aa}
\index{cédille}

\bigskip
\end{lined}
\end{table}

\section{Support multilingue\label{international}}
\secby{Axel Kielhorn}{A.Kielhorn@web.de}%
\index{international}

Pour composer des documents dans des langues autres que l'anglais,
il y a plusieurs domaines pour lesquels
\LaTeX{} doit s'adapter aux spécificités de chaque langue :
\begin{enumerate}
\item Toutes les chaînes de caractères générées automatiquement
  \footnote{\enquote{Table des matières}, \enquote{Liste des figures}, \dots}
  doivent être traduites.
\item \LaTeX{} doit connaître les règles de césure de la nouvelle
      langue.
\item Certaines règles typographiques changent en fonction de la
      langue ou de la région géographique. Par exemple le français
      impose une espace avant le caractère deux-points (:).
\end{enumerate}

De plus, saisir du texte dans votre langage de prédilection peut
devenir fastidieux avec toutes les commandes de la
figure~\ref{accents}. Pour contourner ce problème, jusqu'à récemment
il fallait naviguer dans les eaux troubles des codages spécifiques des
langages tant pour la saisie que pour les polices. De nos jours, les
moteurs \TeX{} modernes sachant traiter l'UTF-8, ces problèmes ont été
considérablement allégés.

L'extension \pai{polyglossia}\cite{polyglossia} est un substitut au désormais obsolète paquet
\pai{babel}. Elle prend soin des motifs de césure et des chaînes de
textes générées automatiquement dans vos documents.

L'extension \pai{fontspec}\cite{fontspec} gère le chargement des
polices pour \hologo{XeLaTeX} et \hologo{LuaTeX}. La police par défaut
est Latin Modern Roman.

\subsection{Utilisation de Polyglossia}

Selon le moteur \TeX{} que vous utilisez, des commandes différentes
sont nécessaires dans le préambule de votre ddocument pour activer
correctement le support multilingue.
La figure~\ref{allinone} en page~\pageref{allinone} montre un exemple
de préambule qui se charge de tous les réglages nécessaires.

\begin{figure}[!bp]
\begin{lined}{10cm}
\begin{verbatim}
\usepackage{iftex}
\ifXeTeX
  \usepackage{fontspec}
\else
  \usepackage{luatextra}
\fi
\defaultfontfeatures{Ligatures=TeX}
\usepackage{polyglossia}
\end{verbatim}
\end{lined}
\caption[Préambule tout-en-un]{Préambule tout-en-un qui prend en compte \hologo{LuaLaTeX} et \hologo{XeLaTeX}} \label{allinone}
\end{figure}


Jusqu'à maintenant l'utilisation d'un système \hologo{TeX} Unicode
n'apportait aucun avantage. Cet état de fait change dès que l'on
quitte l'alphabet latin et que l'on utilise un langage plus
intéressant comme le grec ou le russe. Avec un système basé sur
Unicode, pour pouvez simplement\footnote{Pour des valeurs suffisamment
  petites de \enquote{simple}.} entrer les caractères natifs dans votre éditeur
et \hologo{TeX} les comprendra.

Écrire en plusieurs langues est facile, il suffit de spécifier les
langages dans le préambule\footnote{les noms des langages sont en
  anglais, il faut bien un socle commun... \NdT}. Cet exemple utilise
le paquet \pai{csquotes} qui génère le bon type d'apostrophes en
fonction du langage dans lequel vous écrivez. Notez qu'il doit être
chargé \emph{avant} de charger le support pour votre langage.

\begin{lscommand}
\verb|\usepackage[autostyle=true]{csquotes}|\\
\verb|\setdefaultlanguage{french}|\\
\verb|\setotherlanguage{german}|
\end{lscommand}
%
Pour écrire un paragraphe en allemand, vous utiliserez l'environnement
\texttt{german} :

\begin{example}
Texte en français.
\begin{german}
Deutscher \enquote{Text}.
\end{german}
Encore du \enquote{texte} en français.
\end{example}

Si vous avez seulement besoin d'un mot dans une langue autre, utilisez
alors la commande \verb|\text|\emph{langage} :

\begin{example}
Saviez-vous que
\textgerman{Gesundheit} est
en fait un mot allemand.
\end{example}

Cela peut sembler superflu puisque le seul avantage est alors une
césure correcte, mais pour un langage plus exotique le jeu en vaut la
chandelle.

Parfois la police principale du document ne contient pas les glyphes
requis pour le second langage. Par exemple Latin Modern ne contient pas de
caractères cyrilliques. La solution est alors de spécifier une
police à utiliser pour ce langage. Chaque fois qu'un nouveau langage
est activé, \pai{polyglossia} vérifiera d'abord si une police aura été
spécifiée pour ce langage.
Si la police \emph{computer modern} vous sied, vous pouvez essayer la
police \enquote{Computer Modern Unicode} en ajoutant les commandes
suivantes au préambule de votre document.

\medskip\noindent Pour \hologo{LuaLaTeX} c'est plutôt simple:
\begin{verbatim}
\setmainfont{CMU Serif}
\setsansfont{CMU Sans Serif}
\setmonofont{CMU Typewriter Text}
\end{verbatim}
\noindent Pour \hologo{XeLaTeX} il faut être un peu plus explicite:
\begin{verbatim}
\setmainfont{cmun}[
   Extension=.otf,UprightFont=*rm,ItalicFont=*ti,
   BoldFont=*bx,BoldItalicFont=*bi,
 ]
 \setsansfont{cmun}[
   Extension=.otf,UprightFont=*ss,ItalicFont=*si,
   BoldFont=*sx,BoldItalicFont=*so,
 ]
 \setmonofont{cmun}[
   Extension=.otf,UprightFont=*btl,ItalicFont=*bto,
   BoldFont=*tb,BoldItalicFont=*tx,
 ]
\end{verbatim}

Une fois les polices appropriées chargées, vous pouvez maintenant écrire :

\begin{example}
\textrussian{Правда} est
un journal russe.
\textgreek{ἀλήθεια} est la vérité
ou réalité en philosophie.
\end{example}

L'extension \pai{xgreek}\index{Grec}\cite{xgreek} offre le support
pour écrire autant en ancien grec qu'en grec moderne (monotonique ou
polytonique).

\subsubsection{Langages à l'écriture de droite à gauche.}

Certains langages s'écrivent de gauche à droite, d'autres de droite à
gauche (abbrévié en RTL, comme \enquote{right to left} en
anglais). \pai{polyglossia} doit faire appel à l'extension
\pai{bidi}\cite{bidi}\footnote{\texttt{bidi} ne fonctionne pas avec
  \hologo{LuaTeX}.} pour le support des langages RTL. L'extension
\pai{bidi} doit être la dernière à être chargée, même après
\pai{hyperref} à qui cette place est usuellement réservée (de plus,
comme \pai{polyglossia} charge \pai{bidi}, alors \pai{polyglossia}
devra être la dernière extension chargée).

L'extension \pai{xepersian}\index{Perse}\cite{xepersian} propose un
support de la langue perse. Elle fournit des commandes \LaTeX\
permettant d'écrire des commandes comme \verb|\section| en perse,
ce qui rend cette extension attrayante pour les natifs de cette
langue. \pai{xepersian} est la seule extension disposant du support du
kashida\index{kashida} avec \hologo{XeLaTeX}. Le développement d'une
extension avec support du syriaque selon un algorithme similaire est
en cours.

La police IranNastaliq fournie par le SCICT\footnote{Supreme Council of
+Information and Communication Technology -- Conseil suprême pour les
technologies de l'information et de la communication. \NdT} est
disponible sur son site web
\url{http://www.scict.ir/Portal/Home/Default.aspx}.

L'extension \pai{arabxetex}\cite{arabxetex} propose le support de
plusieurs langages utilisant un alphabet arabe :

\begin{itemize}
\item arab (arabe) \index{arabe}
\item persian (perse) \index{perse}
\item urdu \index{urdu}
\item sindhi (sindhî) \index{sindhî}
\item pashto (ou pachto ou pachtoune) \index{Pashto}\index{pachtoune}
\item ottoman (turc) \index{ottoman}\index{Turc}
\item kurdish (kurde) \index{kurde}
\item kashmiri (Cachemire) \index{cachemire}
\item malay (malais ou jawi) \index{malais}\index{Jawi}
\item uighur (ouïghoure) \index{ouighoure}
\end{itemize}

%SC : pas très sûr de la traduction, à quoi s'applique le "using"
Elle offre un mécanisme de mise en correspondance des polices qui
permet à \hologo{XeLaTeX} de traiter les entrées à l'aide d'une
transcription ASCII Arab\TeX.

Les polices avec support de plusieurs langues arabes sont offertes
par l'IRMUG\footnote{Iranian Mac User Group -- groupe iranien des
  utilisateurs de Mac \NdT.} sur
\url{http://wiki.irmug.org/index.php/X_Series_2}.

Il n'y a pas d'extension pour l'hébreu\index{hébreu} parce qu'aucune
n'est nécessaire, \pai{polyglossia} étant suffisante ici. Si cependant
vous souhaitez une police convenable pour de l'hébreu en Unicode, SBL
Hebrew est gratuite pour un usage non commercial et disponible sur
\url{http://www.sbl-site.org/educational/biblicalfonts.aspx}. Vous
pouvez également regarder à la police Ezra SIL distribuée sous la
licence SIL Open Font License sur
\url{http://www.sil.org/computing/catalog/show_software.asp?id=76}.

Rappelez-vous simplement d'utiliser les commandes suivantes pour les
utiliser :

\begin{lscommand}
\verb|\newfontfamily\hebrewfont[Script=Hebrew]{SBL Hebrew}| \\
\verb|\newfontfamily\hebrewfont[Script=Hebrew]{Ezra SIL}|
\end{lscommand}

\subsubsection{Chinois, japonais et coréen (CJK)}

L'extension \pai{xeCJK}\cite{xecjk} prend soin de la sélection des
polices et de la ponctuation pour ces langues.

\section{L'espace entre les mots}

Pour obtenir une marge droite alignée, \LaTeX{} insère des espaces
plus ou moins larges entre les mots. Après la ponctuation finale
d'une phrase, les règles de la typographie anglo-saxonne\footnote{Mais pas
  celles de la typographie française. C'est pourquoi l'exemple suivant reste
  en anglais. \NdT} veulent que
l'on insère une espace plus large.  Mais si un point suit une lettre
majuscule, \LaTeX{} considère qu'il s'agit d'une abréviation et insère
alors une espace normale.

Toute exception à ces règles doit être spécifiée par l'auteur du
document. Une contre-oblique qui précède une espace génère une espace qui ne
sera pas élargie par \LaTeX{}.  Un tilde «\verb|~|» produit
une espace interdisant tout saut de ligne (dit espace
\emph{insécable}).  \verb|~| est à utiliser pour éviter les coupures
indésirables : on code par exemple \verb|M.~Dupont|.  La
commande \verb|\@| avant un point indique que celui-ci termine une
phrase, même lorsqu'il suit une majuscule.
\cih{"@}
\index{~@$\sim$} \index{tilde@tilde ( \verb.~.)}
\index{., espace après}
\index{espace insécable}

\begin{otherlanguage}{english}
\begin{example}
  Mr.~Smith was happy to see her\\
  cf.~Fig.~5\\
  I like BASIC\@. What about you?
\end{example}
\end{otherlanguage}

L'ajout d'espace supplémentaire à la fin d'une phrase peut être
supprimé par la commande :
\begin{lscommand}
\ci{frenchspacing}
\end{lscommand}
\noindent qui est active par défaut avec l'option \pai{francais} de
l'extension \pai{babel}. Dans ce cas, la commande \verb|\@| n'est pas
nécessaire.



\section{Titres, chapitres et sections}

Pour aider le lecteur à suivre votre pensée, vous souhaitez séparer
vos documents en chapitres, sections ou sous-sections. \LaTeX{}
utilise pour cela des commandes qui prennent en argument le titre de
chaque élément. C'est à vous de les utiliser dans l'ordre.

Dans la classe de document \texttt{article}, les commandes de
sectionnement suivantes sont disponibles : \nopagebreak
\begin{lscommand}
\ci{section}\verb|{...}|\\
\ci{subsection}\verb|{...}|\\
\ci{subsubsection}\verb|{...}|\\
\ci{paragraph}\verb|{...}|\\
\ci{subparagraph}\verb|{...}|
\end{lscommand}

Si vous souhaitez découper votre document en plusieurs parties sans que cela influence la
numérotation des chapitres ou des sections vous pouvez utiliser la
commande :
\begin{lscommand}
\ci{part}\verb|{...}|
\end{lscommand}

Dans les classes \texttt{report} et \texttt{book}, une commande de
sectionnemnent supérieur est disponible (elle s'intercale
entre \verb|\part| et \verb|\section|) :
\begin{lscommand}
\ci{chapter}\verb|{...}|
\end{lscommand}

Puisque la classe \texttt{article} ne connaît pas les chapitres, il
est facile par exemple de regrouper des articles en tant que chapitres
d'un livre en remplacant le \texttt{\bs title} de chaque article par
\texttt{\bs chapter}.

L'espacement entre les sections, la numérotation et le
choix de la police et de la taille des titres sont gérés
automatiquement par \LaTeX{}.

Deux commandes de sectionnement ont un comportement spécial :
\begin{itemize}
\item la commande \ci{part} ne change pas la numérotation des
      chapitres ;
\item la commande \ci{appendix} ne prend pas d'argument. Elle bascule
      simplement la numérotation des chapitres\footnote{Pour la classe
      article, elle change la numérotation des sections} en lettres.
\end{itemize}

\LaTeX{} peut ensuite créer la table des matières en récupérant la
liste des titres et de leur numéro de page d'une exécution précédente
(fichier \texttt{.toc}). La commande :
\begin{lscommand}
\ci{tableofcontents}
\end{lscommand}
\noindent imprime la table des matières à l'endroit où la commande est
invoquée. Un document doit être traité (on dit aussi \enquote{compilé})
deux fois par \LaTeX{} pour avoir une table des matières
correcte. Dans certains cas, un troisième passage est même
nécessaire. \LaTeX{} vous indique quand c'est le cas.
%Remarque: j'ai (SC) créé un cas pathologique qui demande 5
%compilations, situé dans les exemple de mon GNUmakefile pour LaTeX.

Toutes les commandes citées ci-dessus existent dans une forme
\enquote{étoilée} obtenue en ajoutant une étoile \verb|*| au nom de la
commande. Ces commandes produisent des titres de sections qui
n'apparaissent pas dans la table des matières et qui ne sont pas
numérotés. On peut ainsi remplacer la commande
\verb|\section{Introduction}| par
\verb|\section*{Introduction}|.

Par défaut, les titres de section apparaissent dans la table des
matières exactement comme ils sont dans le texte. Parfois il n'est pas
possible de faire tenir un titre trop long dans la table des
matières. On peut donner un titre spécifique pour la table des
matières en argument optionnel avant le titre principal :
\begin{code}
\verb|\chapter[Le LAAS du CNRS]{Le laboratoire|\\
\verb|         d'analyse et d'architecture|\\
\verb|        des systèmes du Centre national|\\
\verb|        de la recherche scientifique}|
\end{code}

Le \wi{titre du document} est obtenu par la commande :
\begin{lscommand}
\ci{maketitle}
\end{lscommand}
Les éléments de ce titre sont définis par les commandes :
\begin{lscommand}
\ci{title}\verb|{...}|, \ci{author}\verb|{...}|
et éventuellement \ci{date}\verb|{...}|
\end{lscommand}
\noindent qui doivent être appelées avant \verb|\maketitle|. Dans
l'argument de la commande \ci{author}, vous pouvez citer plusieurs
auteurs en séparant leurs noms par des commandes \ci{and}.

Vous trouverez un exemple des commandes citées ci-dessus sur la
figure~\ref{document}, page~\pageref{document}.

En plus des commandes de sectionnement expliquées ci-dessus, \LaTeXe{}
a introduit trois nouvelles commandes destinées à être utilisées avec
la classe \texttt{book} :
\begin{description}
\item[\ci{frontmatter}] doit être la première commande après le
  \vadjust{\pagebreak[3]}%MPG: avoid underfull vbox on next page
  début du corps du document (\verb|\begin{document}|),
    elle introduit le prologue du document.
    Les numéros de pages sont alors en romain (i, ii, iii, etc.) et
    les sections non-numérotées, comme si vous utilisiez les variantes
    étoilées des commandes de sectionnement
    (p.e. \verb|\chapter*{Preface}|), mais les sections
    apparaissaient tout de même en table des matières ;

\item[\ci{mainmatter}] se place juste avant le début du premier
  (vrai) chapitre du livre,  la numérotation des pages se fait alors
  en chiffres arabes et le compteur de pages est remis à~1 ;

\item[\ci{appendix}] indique le début des appendices, les numéros
  des chapitres sont alors remplacés par des lettres majuscules (A, B,
  etc.) ;

\item[\ci{backmatter}] se place juste avant la bibliographie et les
  index. Avec les classes standard de document, cette commmande n'a
  aucun effet visible.
\end{description}

\section{Références croisées}

Dans les livres, rapports ou articles, on trouve souvent des
\wi{références croisées} vers des figures, des tableaux ou des passages
particuliers du texte. \LaTeX{} dispose des commandes suivantes pour
faire des références croisées :

\begin{lscommand}
\ci{label}\verb|{|\emph{marque}\verb|}|, \ci{ref}\verb|{|\emph{marque}\verb|}|
et \ci{pageref}\verb|{|\emph{marque}\verb|}|
\end{lscommand}
\noindent où \emph{marque} est un identificateur choisi par
l'utilisateur. \LaTeX{} remplace \verb|\ref| par le numéro de la
section, de la sous-section, de la figure, du tableau, ou du théorème
où la commande \verb|\label| correspondante a été
placée. \verb|\pageref| affichera la page de la commande \verb|\label|
correspondante.
L'utilisation de références croisées rend nécessaire de compiler deux fois le
document : à la première compilation les numéros correspondant aux étiquettes
\verb|\label{}| sont inscrits dans le fichier \texttt{.aux} et, à la
compilation suivante, \verb|\ref{}| et \verb|\pageref{}| peuvent imprimer
ces numéros%
\footnote{Ces commandes ne connaissent pas le type du numéro auquel
elles se réfèrent, elles utilisent le dernier numéro généré
automatiquement.}.

\begin{example}
Une référence à cette
section\label{ma-section}
ressemble à :
\enquote{voir section~\ref{ma-section},
page~\pageref{ma-section}.}
\end{example}

\section{Notes de bas de page}
La commande :
\begin{lscommand}
\ci{footnote}\verb|{|\emph{texte}\verb|}|
\end{lscommand}
\noindent imprime une note de bas de page en bas de la page en cours.
Les notes de bas de page doivent être placées après le mot où la
phrase auquel elles se réfèrent%
\footnote{La typographie française demande une espace fine avant la
marque de renvoi à la note. Celle-ci est insérée automatiquement par
\package{babel} si le français est la langue principale du document, depuis la
version 2.0 de \package{frenchb}. Auparavant, il fallait utiliser
\ci{AddThinSpaceBeforeFootnotes} dans le préambule. \NdT}
%SC: hmmm, is the following french good practice ?
%MPG: I don't think so, but I'm not sure. At least Lacroux says no.
% If we change that, we should take care to stop applying this rule ourselves!
Les notes qui se réfèrent à une (partie de) phrase devraient être
placées après une virgule ou un point.\footnote{Remarquez que les
  notes de bas de page détournent l'attention du lecteur du corps du
  document. Après tout, tout le monde lit les notes de bas de
  page~---~nous sommes une espèce curieuse, alors pourquoi ne pas plus
  simplement intégrer tout ce que vous souhaitez dire dans le corps du
  document ?\footnotemark}
\footnotetext{Un guide ne va pas forcément dans la direction qu'il
  indique :-).}
\nopagebreak[2]

\begin{example}
Les notes de bas de page
\footnote{Ceci est une note
	  de bas de page.}
sont très prisées par les
utilisateurs de \LaTeX{}.
\end{example}

\section{Souligner l'importance d'un mot}

Dans un manuscrit réalisé sur une machine à écrire, les mots
importants sont \texttt{valorisés en les \underline{soulignant}} ;
on peut obtenir ce résultat en \LaTeX{} avec la commande :
\begin{lscommand}
\ci{underline}\verb|{|\textit{texte}\verb|}|
\end{lscommand}

Dans un ouvrage
imprimé, on préfère les \emph{mettre en valeur}%
\footnote{\eng{Emphasize} en anglais.}.
La commande de mise en valeur est :
\begin{lscommand}
\ci{emph}\verb|{|\emph{texte}\verb|}|
\end{lscommand}

Son argument est le texte à mettre en valeur. En général, la police
\emph{italique} est utilisée pour la mise en valeur, sauf si le texte
est déja en italique, auquel cas on utilise une police romaine (droite).
En tant qu'auteur, ce n'est pas tant la police que le besoin de mettre
en valeur un texte particulier qui est important, d'où l'existence de
cette commande.

\begin{example}
\emph{Pour \emph{insister}
dans un passage déjà
mis en valeur, \LaTeX{}
utilise une police droite.}
\end{example}

Si vous souhaitez plus de contrôle sur la police et sa taille, la
section \ref{sec:fontsize} en page \pageref{sec:fontsize} donnera
quelques idées dans ce sens.

\section{Environnements} \label{env}

Pour composer du texte dans des contextes spécifiques, \LaTeX{}
définit des \wi{environnement}s différents pour appliquer divers types
de mise en page à des portions de texte potentiellement longues :
%MPG: rallongé pour remplir la page...

\begin{lscommand}
\ci{begin}\verb|{|\emph{nom}\verb|}|\quad
   \emph{contenu}\quad
\ci{end}\verb|{|\emph{nom}\verb|}|
\end{lscommand}
\noindent
\emph{nom} est le nom de l'environnement. Les environnements peuvent
être imbriqués, à condition que l'ordre de fermeture soit correct.
\begin{code}
\verb|\begin{aaa}...\begin{bbb}...\end{bbb}...\end{aaa}|
\end{code}
\noindent Les sections suivantes vous présentent (presque) tous les
environnements importants.

\subsection{Listes, énumérations et descriptions}

L'environnement \ei{itemize} est utilisé pour des listes simples,
\ei{enumerate} est utilisé pour des énumérations (listes
numérotées) et \ei{description} est utilisé pour des descriptions.
\cih{item}

% \begin{example}
% Les différents types de liste :
% \begin{itemize}
% \item \texttt{itemize}
% \item \texttt{enumerate}
% \item \texttt{description}
% \end{itemize}
% \end{example}

%MPG: déplacé ici et allongé pour remplir la page
Notez que l'option \pai{francais} de l'extension \pai{babel} utilise
une présentation des
listes simples qui respecte les règles typographiques françaises : utilisation
d'un tiret pour les listes simples au loin d'un point épais \enquote{\textbullet},
espaces verticaux réduits.
%MPG: and this is applied to all languages by default. So don't bother trying
%to make an example.

\begin{example}
\begin{enumerate}
\item Il est possible d'imbriquer
les environnements à sa guise :
\begin{itemize}
\item mais cela peut ne pas
  être  très beau,
\item ni facile à suivre.
\end{itemize}
\item Souvenez-vous :
\begin{description}
\item[Clarté :] les faits ne
vont pas devenir plus sensés
parce  qu'ils sont dans une liste,
\item[Synthèse :] une liste peut
cependant très bien
résumer des faits.
\end{description}
\end{enumerate}
\end{example}

\subsection{Alignements à gauche, à droite et centrage}

Les environnements \ei{flushleft} et \ei{flushright} produisent des
textes \wi{aligné}s à gauche ou à droite. L'environnement \ei{center}
produit un texte centré. Si vous n'utilisez pas la commande \ci{\bs}
pour indiquer les sauts de ligne, ceux-ci continuent d'être calculés
automatiquement par \LaTeX{}.

\begin{example}
\begin{flushleft}
Ce texte est\\
aligné à gauche.
\LaTeX{} n'essaye pas
d'aligner la marge droite.
\end{flushleft}
\end{example}

\begin{example}
\begin{flushright}
Ce texte est\\
aligné à droite.
\LaTeX{} n'essaye pas
d'aligner la marge gauche.
\end{flushright}
\end{example}

\begin{example}
\begin{center}
Au centre de la terre.
\end{center}
\end{example}

\subsection{Citations et vers}

L'environnement \ei{quote} est utile pour les citations, les phrases
importantes ou les exemples.

\begin{example}
Une règle typographique
simple pour la longueur
des lignes :
\begin{quote}
Une ligne ne devrait pas comporter
plus de 66~caractères.
\end{quote}
C'est pourquoi les pages
composées par \LaTeX{} ont des
marges importantes et
les journaux utilisent
souvent plusieurs colonnes.
\end{example}

Il existe deux autres environnements comparables : \ei{quotation} et
\ei{verse}. L'environnement \ei{quotation} est utile pour des
citations plus longues, couvrant plusieurs
paragraphes parce qu'il indente ceux-ci.
L'environnement \ei{verse} est utilisé pour la poésie, là
où les retours à la ligne sont importants. Les vers sont séparés par
des commandes \ci{\bs} et les strophes par une ligne vide\footnote{Les
puristes constateront que l'environnement \ei{verse} ne respecte pas
les règles de la typographie française : les rejets devraient être
préfixés par \enquote{[\iffalse]\fi} et alignés à droite sur le vers précédent.}.

\begin{example}
Voici le début d'une
fugue de Boris Vian :
\begin{flushleft}
\begin{verse}
Les gens qui n'ont plus
  rien à faire\\
Se suivent dans la rue comme\\
Des wagons de chemin de fer.

Fer fer fer\\
Fer fer fer\\
Fer quoi faire\\
Fer coiffeur.\\
\end{verse}
\end{flushleft}
\end{example}

\subsection{Résumé}

Lors d'une publication scientifique il est usuel de démarrer celle-ci
avec un résumé (\eng{abstract}), censé donner au lecteur une vue
d'ensemble de ce qu'il doit attendre du document. \LaTeX{} fournit un
environnement \ei{abstract} à cette fin. Normalement \ei{abstract} est
utilisé dans les documents de classe \texttt{article}.

\newenvironment{abstract}%
        {\begin{center}\begin{small}\begin{minipage}{0.8\textwidth}}%
        {\end{minipage}\end{small}\end{center}}
\begin{example}
\begin{abstract}
L'abstrait abstract résumé.
\end{abstract}
\end{example}

\subsection{Impression \emph{verbatim}}

Tout texte inclus entre \verb|\begin{|\ei{verbatim}\verb|}| et
\verb|\end{verbatim}| est imprimé tel quel, comme s'il avait été tapé
à la machine, avec tous les retours à la ligne et les espaces, sans
qu'aucune commande \LaTeX{} ne soit exécutée.

À l'intérieur d'un paragraphe, une fonctionnalité équivalente peut
être obtenue par
\begin{lscommand}
\ci{verb}\verb|+|\emph{texte}\verb|+|
\end{lscommand}
\noindent Le caractère \verb|+| est seulement un exemple de caractère
séparateur. Vous pouvez utiliser n'importe quel caractère, sauf les
lettres, \verb|*| ou l'espace. La plupart des exemples de commandes
\LaTeX{} dans ce document sont réalisés avec cette commande.

\begin{example}
La commande \verb|\dots| \dots

\begin{verbatim}
10 PRINT "HELLO WORLD ";
20 GOTO 10
\end{verbatim}
\end{example}

\begin{example}
\begin{verbatim*}
La version étoilée de
l'environnement  verbatim
met    les   espaces   en
évidence
\end{verbatim*}
\end{example}

La commande \ci{verb} peut également être utilisée avec une étoile :
\begin{example}
\verb*|comme ceci :-) |
\end{example}

L'environnement \texttt{verbatim} et la commande \verb|\verb| ne
peuvent être utilisés à l'intérieur d'autres commandes comme
\verb|\footnote{}|.


\subsection{Tableaux}

\newcommand{\mfr}[1]{\framebox{\rule{0pt}{0.7em}\texttt{#1}}}

L'environnement \ei{tabular} permet de réaliser des tableaux avec ou
sans lignes de séparation horizontales ou verticales. \LaTeX{}
ajuste automatiquement la largeur des colonnes.

L'argument \emph{description} de la commande :
\begin{lscommand}
\verb|\begin{tabular}[|\emph{position}\verb|]{|\emph{description}\verb|}|
\end{lscommand}
\noindent définit le format des colonnes du tableau. Utilisez un
\mfr{l} pour une colonne alignée à gauche, \mfr{r} pour
une colonne alignée à droite et \mfr{c} pour une colonne
centrée. \mfr{p{\{\emph{largeur}\}}} permet de réaliser une colonne
justifiée sur plusieurs lignes et enfin
\mfr{|} permet d'obtenir un filet vertical.
\index{"|@ \verb."|.}

Si le texte d'une colonne est trop large pour la page, \LaTeX{}
n'insèrera pas automatiquement de saut de ligne. Grâce à
\mfr{p\{\emph{largeur}\}} vous pouvez définir un type spécial de
colonne qui fera passer le texte à la ligne comme pour un paragraphe
usuel.

L'argument \emph{position} définit la position verticale du tableau
par rapport au texte environnant. Utilisez une des lettres \mfr{t},
\mfr{b} et \mfr{c} pour l'aligner en haut (\emph{top}), en bas
(\emph{bottom}) ou au centre (\emph{center}) respectivement.

À l'intérieur de l'environnement \texttt{tabular}, le caractère
\texttt{\&} est le séparateur de colonnes, \ci{\bs} commence une nouvelle
ligne et \ci{hline} insère un filet horizontal. Vous pouvez ajoutez
des filets partiels via la commande
\ci{cline}\texttt{\{}$i$\texttt{-}$j$\texttt{\}}, où i et j
sont les numéros de colonnes de début et de fin du filet.
\index{\&}

\begin{example}
\begin{tabular}{|r|l|}
\hline
7C0 & hexadécimal \\
3700 & octal \\
11111000000 & binaire \\
\hline \hline
1984 & décimal \\
\hline
\end{tabular}
\end{example}

\begin{example}
\begin{tabular}{|p{4.7cm}|}
\hline
Bienvenue dans ce
cadre.\\
Merci de votre visite.\\
\hline
\end{tabular}

\end{example}

La construction \mfr{@\{...\}} permet d'imposer le séparateur de
colonnes. Cette commande supprime l'espacement inter-colonnes et le
remplace par ce qui est indiqué entre les crochets. Une utilisation
courante de cette commande est présentée plus loin comme solution au
problème de l'alignement des nombres décimaux. Une autre utilisation
possible est de supprimer l'espacement dans un tableau avec
\mfr{@\{\}}.

\begin{example}
\begin{tabular}{@{} l @{}}
\hline
sans espace\\\hline
\end{tabular}
\end{example}
\begin{example}
\begin{tabular}{l}
\hline
avec espaces\\
\hline
\end{tabular}
\end{example}

%
% This part by Mike Ressler
%

\index{alignement décimal} S'il n'y a pas de commande prévue%
\footnote{Si les extensions de l'ensemble \enquote{tools} sont installées
          sur votre système, jetez un \oe il sur l'extension
          \pai{dcolumn} faite pour résoudre ce problème.}
pour aligner les nombres sur le point décimal (ou la virgule si on
respecte les règles françaises) nous pouvons \enquote{tricher} et
réaliser cet alignement en utilisant deux colonnes : la première
alignée à droite contient la partie entière et la seconde alignée à
gauche contient la partie décimale. La commande \verb|\@{,}| dans la
description du tableau remplace l'espace normale entre les colonnes par
une simple virgule, donnant l'impression d'une seule colonne alignée
sur le séparateur décimal. N'oubliez pas de remplacer dans votre
tableau le point ou la virgule
par un séparateur de colonnes (\verb|&|) ! Un titre peut être
placé au-dessus de cette \enquote{colonne virtuelle} (en fait, de ces deux
colonnes) en utilisant la commande \ci{multicolumn}.

\begin{example}
\begin{tabular}{c r @{,} l}
Expression       &
\multicolumn{2}{c}{Valeur} \\
\hline
$\pi$               & 3&1416  \\
$\pi^{\pi}$         & 36&46   \\
$(\pi^{\pi})^{\pi}$ & 80662&7 \\
$\pi^{-1}$          & 0&3183 \\
\end{tabular}
\end{example}
%MPG: added more lines, trying to fill the page...
Autre exemple d'utilisation de \verb+\multicolumn+ :
\begin{example}
\begin{tabular}{|l|l|}
\hline
\multicolumn{2}{|c|}{%
  \textbf{Nom}} \\
\hline
Dupont & Jules \\
Durand & Jacques \\
\hline
\end{tabular}
\end{example}

\LaTeX{} traite le contenu d'un environnement \texttt{tabular} comme
une boîte indivisible, en particulier il ne peut y avoir de coupure de
page. Pour réaliser de longs tableaux s'étendant sur plusieurs pages
il faut avoir recours aux extensions \pai{supertabular} ou
\pai{longtable}.

Parfois les tableaux par défaut de \LaTeX{} donnent une impression
d'étroitesse. Si vous voulez leur donner plus d'extension, vous pouvez
le faire en modifiant les valeurs de \ci{arraystretch} et \ci{tabcolsep} comme
dans l'exemple suivant.

\begin{example}
\begin{tabular}{|l|}
\hline
Ces lignes sont\\\hline
à l'étroit\\\hline
\end{tabular}

{\renewcommand{\arraystretch}{1.5}
\renewcommand{\tabcolsep}{0.2cm}
\begin{tabular}{|l|}
\hline
Un tableau\\\hline
moins étroit\\\hline
\end{tabular}}

\end{example}

Si vous voulez seulement augmenter la hauteur d'un ligne dans un
tableau, vous pouvez utiliser une réglure de largeur nulle
\footnote{En typographie professionnelle ceci est appelé un
\wi{montant}.}. Donnez à cette réglure\cih{rule} la hauteur voulue.

\begin{example}
\begin{tabular}{|c|}
\hline
\rule{1pt}{4ex}\'Etai\dots\\
\hline
\rule{0pt}{4ex} montant \\
\hline
\end{tabular}
\end{example}

Les \texttt{pt} et \texttt{ex} dans l'exemple ci-avant sont des unités
\TeX. Référez-vous au tableau \ref{units} en page \pageref{units} pour
en savoir plus sur les unités de \TeX.

Un certain nombre de commandes supplémentaires pour améliorer
l'environnement \texttt{tabular} sont disponibles dans le paquet
\pai{booktabs}. Celui-ci simplifie grandement la création de tables
d'aspect professionnel et à l'espacement correct.

\section{Inclusion de graphiques et d'images} \label{eps}

Comme expliqué dans la section précédente,
avec les environnements \ei{figure} et \texttt{table}, \LaTeX{}
fournit les mécanismes de base pour travailler avec des objets tels que
des images ou des graphiques.

Un ensemble de commandes bien adaptées à l'insertion de graphiques dans ces objets flottants est
fourni par l'extension \pai{graphicx}, développée par D.~P.~Carlisle. Elle
fait partie d'un ensemble d'extensions appelé \enquote{graphics}.
\footnote{\CTAN|pkg/graphics|.}.

Voici la marche à suivre pas à pas pour inclure une
figure dans un document :

\begin{enumerate}
\item exportez la figure de votre logiciel graphique au format EPS, PDF, PNG ou JPEG.
\item Si vous avez exporté votre figure sous forme de graphique
  vectoriel EPS, vous devez la convertir en PDF avant de
  l'utiliser. Il existe une commande \texttt{epstopdf} pour cette
  tâche précise. Notez qu'exporter vers le format EPS a du sens même
  si votre logiciel sait exporter vers le format PDF, étant donné que
  le PDF est souvent en pleine page et sera réduit lorsqu'importé dans
  un document, alors qu'EPS fournit un cadre qui définit la zone qui
  contient exactement la figure.
\item chargez l'extension \textsf{graphicx} dans le préambule de votre
      fichier source avec :
\begin{lscommand}
\verb|\usepackage{graphicx}|
\end{lscommand}
\item utilisez la commande :
\begin{lscommand}
\ci{includegraphics}\verb|[|\emph{clef}=\emph{valeur}, ... \verb|]{|\emph{fichier}\verb|}|
\end{lscommand}
\noindent pour insérer \emph{fichier} dans votre document. Le param<E8>tre
optionnel est une liste de paires de \emph{clefs} et de \emph{valeurs}
séparées par des virgules. Les \emph{clefs} permettent de modifier la
largeur, la hauteur, ou l'angle de rotation de la figure. Le
tableau~\ref{keyvals} présente les clefs les plus importantes.
\end{enumerate}

\begin{table}[tb]
\caption{Clefs pour l'extension \textsf{graphicx}}
\label{keyvals}
\begin{lined}{9cm}
\begin{tabular}{@{}ll}
\texttt{width}& définit la largeur de la figure\\
\texttt{height}& définit la hauteur de la figure\\
\texttt{angle}& (en degrés) tourne la figure dans le sens \\
&  des aiguilles d'une montre \\
\texttt{scale}& échelle de la figure
\end{tabular}

\bigskip
\end{lined}
\end{table}

L'exemple en figure~\ref{figureex} page~\pageref{figureex} devrait
aider à clarifier le fonctionnement de la commande.
\begin{figure}
\begin{lined}{9cm}
\begin{verbatim}
\includegraphics[angle=90,width=\textwidth]{test.png}
\end{verbatim}
\end{lined}
\caption{Code d'exemple pour inclure \texttt{test.png} dans un document.\label{figureex}}
\end{figure}

Cette commande inclut la figure stockée dans le fichier
\texttt{test.png}. La figure est \emph{d'abord} tournée de 90 degrés
puis ajustée pour que sa largeur finale soit de 10 cm. Les proportions
largeur/hauteur sont conservées, puisqu'aucune hauteur n'est spécifiée.

Pour plus d'informations, reportez vous à~\cite{graphics}.

\section{Objets flottants}

De nos jours, la plupart des publications contiennent un nombre
important de figures et de tableaux. Ces éléments nécessitent un
traitement particulier car ils ne peuvent être coupés par un
changement de page. On pourrait imaginer de commencer une nouvelle
page chaque fois qu'une figure ou un tableau ne rentrerait pas dans la
page en cours. Cette façon de faire laisserait de nombreuses pages à moitié
blanches, ce qui ne serait réellement pas beau.
\index{tableau}
\index{figure}

La solution est de laisser \enquote{flotter} les figures et les tableaux
qui ne rentrent pas sur la page en cours, vers une page suivante et de
compléter la page avec le texte qui suit l'objet \enquote{flottant}.
\LaTeX{} fournit deux
environnements pour les \wi{objets flottants} adaptés respectivement
aux figures (\ei{figure}) et aux tableaux (\ei{table}). Pour
faire le meilleur usage de ces deux environnements, il est important
de comprendre comment \LaTeX{} traite ces objets flottants de manière
interne. Dans le cas contraire ces objets deviendront une cause de
frustration intense
car \LaTeX{} ne les placera jamais à l'endroit où vous souhaitiez les
voir.

\bigskip
Commençons par regarder les commandes que \LaTeX{} propose pour les
objets flottants.Tout objet inclus dans un environnement \ei{figure}
ou \ei{table} est traité comme un objet flottant. Les deux
environnements flottants ont un paramètre optionnel :
\begin{lscommand}
\verb|\begin{figure}[|\emph{placement}\verb|]| ou
\verb|\begin{table}[|\emph{placement}\verb|]|
\end{lscommand}
\noindent
appelé \emph{placement}. Ce paramètre permet de dire à \LaTeX{} où
vous autorisez l'objet à flotter. Un \emph{placement} est composé
d'une chaîne de caractères représentant des \emph{placements
possibles}. Reportez-vous au tableau~\ref{tab:permiss}.


\begin{table}[!htbp]
\caption{Placements possibles}\label{tab:permiss}
\noindent \begin{minipage}{\textwidth}
\medskip
\begin{center}
\begin{tabular}{@{}cp{8cm}@{}}
Caractère & Emplacement pour l'objet flottant\dots\\
\hline
\rule{0pt}{1.05em}%
\texttt{h} & \emph{here}, ici, à l'emplacement dans
	     le texte où la commande se trouve. Utile pour les petits
	     objets.\\[0.3ex]
\texttt{t} & \emph{top}, en haut d'une page\\[0.3ex]
\texttt{b} & \emph{bottom}, en bas d'une page\\[0.3ex]
\texttt{p} & \emph{page}, sur une page à part ne contenant que des
             objets flottants.\\[0.3ex]
\texttt{!} & ici, sans prendre en compte les paramètres
             internes\footnote{tels que le nombre maximum d'objets
             flottants sur une page} qui autrement pourraient empêcher ce
             placement.
\end{tabular}
\end{center}
\end{minipage}
\end{table}

Un tableau flottant peut commencer par exemple par la ligne suivante :
\begin{code}
\verb|\begin{table}[!hbp]|
\end{code}
\noindent L'\wi{emplacement} \verb|[!hbp]| permet à \LaTeX{} de placer
le tableau soit sur place (\texttt{h}), soit en bas de page
(\texttt{b}) soit enfin sur une page à part (\texttt{p}), et
tout cela même si les règles internes de \LaTeX{} ne sont pas toutes
respectées (\texttt{!}). Si aucun placement n'est indiqué, les
classes standard utilisent \verb|[tbp]| par défaut.

\LaTeX{} place tous les objets flottants qu'il rencontre
en suivant les indications fournies par l'auteur. Si un objet ne peut
être placé sur la page en cours, il est placé soit dans la file des
figures soit dans la file des tableaux\footnote{Il s'agit de files
FIFO (\emph{First In, First Out}) : premier arrivé, premier servi.}.
Quand une nouvelle page est
entamée, \LaTeX{} essaye d'abord de voir si les objets en tête des
deux files pourraient être placés sur une page spéciale, à part.  Si
cela n'est pas possible, les objets en tête des deux files sont
traités comme s'ils venaient d'être trouvés dans le texte : \LaTeX{}
essaye de les placer selon leurs spécifications de placement (sauf
\texttt{h}, qui n'est plus possible). Tous les
nouveaux objets flottants rencontrés dans la suite du texte sont
ajoutés à la queue des files. \LaTeX{} respecte scrupuleusement
l'ordre d'apparition des objets flottants. C'est pourquoi un objet
flottant qui ne peut être placé dans le texte repousse tous les
autres à la fin du document.

D'où la règle :
\begin{quote}
Si \LaTeX{} ne place pas les objets flottants comme vous le souhaitez,
c'est souvent à cause d'un seul objet trop grand qui bouche l'une des
deux files d'objets flottants.
\end{quote}

Essayer d'imposer à \LaTeX{} un emplacement particulier pose souvent
problème : si l'objet flottant ne tient pas à l'emplacement demandé,
alors il est coincé et bloque le reste des objets flottants. En
particulier, l'utilisation de la seule option \verb+[h]+ pour un
flottant est une idée \emph{à proscrire}, les versions modernes de
\LaTeX{} changent d'ailleurs automatiquement l'option \verb+[h]+ en
\verb+[ht]+.

Voici quelques éléments supplémentaires qu'il est bon de connaître sur
les environnements \ei{table} et \ei{figure}.

Avec la commande :
\begin{lscommand}
\ci{caption}\verb|{|\emph{texte de la légende}\verb|}|
\end{lscommand}
\noindent
vous définissez une légende pour l'objet. Un numéro (incrémenté
 automatiquement) et le mot \enquote{Figure} ou
 \enquote{Table}\footnote{Avec l'extension \texttt{babel}, la
 présentation des légendes est modifiée pour obéir aux règles
 françaises.} sont ajoutés par \LaTeX.

Les deux commandes :
\begin{lscommand}
\ci{listoffigures} et \ci{listoftables}
\end{lscommand}
\noindent fonctionnent de la même manière que la commande
\verb|\tableofcontents| ; elles impriment respectivement la liste des
figures et des tableaux. Dans ces listes, la légende est reprise en
entier. Si vous désirez utiliser des légendes longues, vous pouvez
en donner une version courte entre crochets qui sera utilisée pour la
table :
\begin{code}
\verb|\caption[courte]{LLLLLoooooonnnnnggggguuuueee}|
\end{code}

Avec \ci{label} et \ci{ref} vous pouvez faire référence à votre objet
à l'intérieur de votre texte. La commande \ci{label} doit apparaître
\emph{après} la commande \ci{caption} d'une légende si vous voulez
référencer le numéro de cette légende.

L'exemple suivant dessine un carré et l'insère dans le document. Vous
pouvez utiliser cette commande pour réserver de la place pour une
illustration que vous allez coller sur le document terminé.

\begin{code}
\begin{verbatim}
La figure~\ref{blanche} est un exemple de Pop-Art.
\begin{figure}[!htbp]
\includegraphics[angle=90,width=\textwidth]{white-box.pdf}
\caption{White Box by Peter Markus Paulian.\label{white}}
\end{figure}
\end{verbatim}
\end{code}

Dans l'exemple ci-dessus\footnote{En supposant que la file
des figures soit vide.} \LaTeX{} va s'acharner~(\texttt{!}) à
placer la figure là où se trouve la commande~(\texttt{h}) dans le
texte. S'il n'y arrive pas, il essayera de la placer en
bas~(\texttt{b}) de la page. Enfin s'il ne peut la placer sur la
page courante, il essayera de créer une page à part avec d'autres
objets flottants. S'il n'y a pas suffisamment de tableaux en attente
pour remplir une page spécifique, \LaTeX{} continue et, au début de la
page suivante, réessayera de placer la figure comme si elle venait
d'apparaître dans le texte.

Dans certains cas il peut s'avérer nécessaire d'utiliser la commande :
\begin{lscommand}
\ci{clearpage} ou même \ci{cleardoublepage}
\end{lscommand}
Elle ordonne à \LaTeX{} de placer tous les objets en attente
immédiatement puis de commencer une nouvelle
page. \ci{cleardoublepage} commence une nouvelle page de droite.


\endinput

% Local Variables:
% TeX-master: "lshort2e"
% mode: latex
% mode: flyspell
% End:

%%%%%%%%%%%%%%%%%%%%%%%%%%%%%%%%%%%%%%%%%%%%%%%%%%%%%%%%%%%%%%%%%
% Contents: Math typesetting with LaTeX
% $Id: math.tex 169 2008-09-24 07:32:13Z oetiker $
%%%%%%%%%%%%%%%%%%%%%%%%%%%%%%%%%%%%%%%%%%%%%%%%%%%%%%%%%%%%%%%%%


% Pour les informations de licence, voir title.tex.
% See title.tex for license information.

\chapter{Formules Math�matiques}
\thispagestyle{plain}

\begin{intro}
  Vous �tes pr�ts ! Dans ce chapitre nous allons aborder l'atout
  majeur de \TeX{} : la composition de formules math�matiques.
  Mais attention, ce chapitre ne fait que d�crire les commandes de
  base. Bien que ce qui est expliqu� ici soit suffisant pour la
  majorit� des utilisateurs, ne d�sesp�rez pas si vous n'y trouvez pas
  la solution � votre probl�me de mise en forme d'une �quation
  math�matique. Il y a de fortes chances pour que la solution se
  trouve dans l'extension \pai{amsmath} de \AmS-\LaTeX{}.% 
\end{intro}
  
\section{L'ensemble \texorpdfstring{\AmS}{AMS}-\LaTeX{}}

Si vous souhaitez saisir des textes \wi{math�matiques} (avanc�s), vous
devriez utiliser \AmS-\LaTeX{}. Le paquet \AmS-\LaTeX{} est une
collection d'extensions et de classes pour la saisie
math�matique. Nous traiterons ici principalement de l'extension
\pai{amsmath} qui fait partie de ce paquet. \AmS-\LaTeX{} est produit
par l'\emph{American Mathematical Society} (NdT. la Soci�t�
Am�ricaine de Math�matiques) et est utilis�e extensivement pour la
mise en forme de math�matiques. \LaTeX{} seul fournit bien quelques
fonctionnalit�s et environnements basiques, mais ils sont
relativement limit�s (voire, la logique s'appliquerait plut�t dans
l'autre sens : \AmS-\LaTeX{} est \emph{illimit�} !) et parfois
incoh�rents.

\AmS-\LaTeX{} fait partie de la distribution de base et est fournie
avec toutes les distributions r�centes de \LaTeX{}.\footnote{Si la
  v�tre ne l'a pas, visitez
  \texttt{CTAN:macros/latex/required/amslatex}. Dans ce chapitre nous
  supposerons qu'\pai{amsmath} est charg� en pr�ambule, via
  {\ci{usepackage}\{amsmath\}}}.

\section{�quations simples}

Il y a deux fa�ons de mettre en forme des \wi{formules}
math�matiques~: au fil du texte � l'int�rieur d'un paragraph
(\emph{\wi{style en-ligne}}) ou le paragraphe peut �tre d�coup� pour
que la mise en forme soit s�par�e (\textit{\wi{style hors-texte}}). Les
\wi{�quation}s math�matiques \emph{dans} un paragraphe sont entr�es
entre deux signes \index{$@\texttt{\$}} \texttt{\$} :
\begin{example}
Ajoutez $a$ au carr� 
et $b$ au carr� pour obtenir
$c$ au carr�. Ou, en 
utilisant une approche plus
matheuse : $a^2 + b^2 = c^2$
\end{example}
\begin{example}
 \TeX{} se prononce 
$\tau\epsilon\chi$\\[5pt]
100~m$^{3}$ d'eau\\[5pt]
J'$\heartsuit$ \LaTeX{}
\end{example}

% just to avoid bad emacs colorizing because of too many $
%
Il vaut mieux composer les �quations ou les formules plus importantes
\og \emph{hors-texte} \fg{} plut�t que d�couper un paragraphe pour
cela, c'est-�-dire sur des lignes � part. � cette fin, on les
inclut entre \verb|\begin{|\ei{equation}\verb|}| et
\verb|\end{equation}|.\footnote{Il s'agit d'une commande d'
\textsf{amsmath}. Si vous n'y avez pas acc�s pour quelque obscure
raison, utilisez l'environnement \ei{displaymath} de \LaTeX{} � la
place.}
Vous pouvez ensuite utiliser  \ci{label} pour marquer un num�ro
d'�quation et vous y r�f�rer ailleurs dans le texte via la commande
\ci{eqref}. Si vous voulez plut�t nommer l'�quation, utiliser \ci{tag}
� la place. Vous ne pouvez pas utiliser \ci{eqref} avec \ci{tag}.

\begin{example}
Ajoutez $a$ au carr� 
et $b$ au carr� pour obtenir
$c$ au carr�. Ou, en 
utilisant une approche plus
matheuse :
 \begin{equation}
   a^2 + b^2 = c^2
 \end{equation}
Einstein a dit
 \begin{equation}
   E = mc^2 \label{intelligent}
 \end{equation}
Il n'a pas dit
 \begin{equation}
  1 + 1 = 3 \tag{idiot}
 \end{equation}
Voici une r�f�rence �
\eqref{intelligent}.
\end{example}

Si vous ne voulez pas que \LaTeX{} num�rote vos �quations, utilisez la
version �toil�e d'\texttt{equation}, \ei{equation*}, ou mieux encore,
entourez votre �quation par \ci{[} et \ci{]} :
\footnote{\index{equation!\textsf{amsmath}} \index{equation!\LaTeX{}}
Cela provient encore d'\textsf{amsmath}. Si vous ne l'avez pas charg�,
utilisez l'environnement \texttt{equation} de \LaTeX{} � la place. Les
noms des commandes \texttt{amsmath}/\LaTeX{} peuvent pr�ter �
confusion, mais ce n'est pas un vrai probl�me puisque tout le monde
utilise \texttt{amsmath} de toute fa�on. Il vaut mieux en r�gle
g�n�rale charger l'extension d�s le d�part parce que vous pourriez
l'utiliser plus tard. De plus les \texttt{equation} non num�rot�es de
\LaTeX{} entrent en conflit avec les \texttt{equation} num�rot�es
d'\AmS-\LaTeX{}.}.
\begin{example}
Ajoutez $a$ au carr� 
et $b$ au carr� pour obtenir
$c$ au carr�. Ou, en 
utilisant une approche plus
matheuse :
 \begin{equation*}
   a^2 + b^2 = c^2
 \end{equation*}
ou plus parcimonieusement :
 \[ a^2 + b^2 = c^2 \]
\end{example}


Remarquez que les expressions math�matiques sont format�es
diff�remment selon qu'elles sont compos�es \og \wi{en-ligne} \fg{} ou \og
\wi{hors-texte} \fg{} :
 \begin{example}
Style en-ligne :
 $\lim_{n \to \infty} 
 \sum_{k=1}^n \frac{1}{k^2} 
 = \frac{\pi^2}{6}$.
Style hors-text :
 \begin{equation}
  \lim_{n \to \infty} 
  \sum_{k=1}^n \frac{1}{k^2} 
  = \frac{\pi^2}{6}
 \end{equation}
\end{example}

En style en-ligne, utilisez la commande \ci{smash} sur des
(sous-)expressions math�matiques � plusieurs niveaux, que ce soit vers
le haut ou vers le bas. Cela incite \LaTeX{} � ne pas prendre en
compte la hauteur de ces expressions et permet d'avoir un interligne
r�gulier.

\begin{example}
Une expression math�matique
$d_{e_{e_p}}$ suivie par une
expression $h^{i^{g^h}}$. Par
opposition � une expression
avec smash \smash{$d_{e_{e_p}}$}
suivie par une expression
\smash{$h^{i^{g^h}}$}.
\end{example}


\subsection{Mode math�matique}


Il y a �galement des diff�rences notables entre le \wi{mode
  \emph{math�matique}} et le mode \emph{texte}. Par exemple, en mode
\emph{math�matique} :

\begin{enumerate}
\item \index{espacement!math�matique}
      la plupart des espaces et des retours � la ligne n'ont aucune
      signification. Les espaces sont d�duites de la logique de la
      formule ou indiqu�es � l'aide de commandes sp�cifiques telles
      que \ci{,}, \ci{quad} ou\ci{qquad} (nous reparlerons de cela en
      section~\ref{sec:math-spacing}) ;
\item les lignes vides ne sont pas autoris�es. Un seul paragraphe par
      formule ;
\item chaque lettre est consid�r�e comme �tant le nom d'une variable
      et sera imprim�e comme tel. Pour ins�rer du  texte
      normal (police et espacement standard) dans une formule, il 
      faut utiliser la commande \verb|\textrm{...}| (voir �galement la
      section \ref{sec:fontsz} en page \pageref{sec:fontsz}).
\end{enumerate}

\begin{example}
$\forall x \in \mathbf{R}:
 \qquad x^{2} \geq 0$
\end{example}
\begin{example}
$x^{2} \geq 0\qquad
 \text{pour tout}x\in\mathbf{R}$
 \end{example}
 
Une mode r�cente et contestable pousse � utiliser la police 
\og \wi{blackboard bold} \fg{} (Gras Tableau Noir, ainsi appel�e car c'est par
le doublement des verticales des lettres que l'on simule le gras
typographique lorsqu'on ne peut faire autrement) qui est obtenue par
la commande \ci{mathbb} de l'extension \pai{amssymb}
\footnote{\pai{amssymb} ne fait pas partie du paquet \AmS-\LaTeX{},
  mais fait peut-�tre partie de votre distribution \LaTeX{}. V�rifiez
  celle-ci ou visitez \texttt{CTAN:/fonts/amsfonts/latex/} pour
  l'obtenir.}
pour d�signer les ensembles de nombres entiers, r�els, etc.
\ifx\mathbb\undefined\else
L'exemple pr�c�dent devient :
\begin{example}
$x^{2} \geq 0\qquad
 \text{pour tout } x 
 \in \mathbb{R}$
\end{example}
\fi

R�f�rez-vous aux tableaux \ref{mathalpha} en page \pageref{mathalpha}
et \ref{mathfonts} en page \pageref{mathfonts} pour plus de polices
math�matiques.


\section{�l�ments d'une formule math�matique}


Cette section d�crit les commandes les plus importantes 
du mode math�matique. La plupart des commandes de cette section ne
n�cessitent pas \textsf{amsmath}, sauf mention explicite, mais
chargez-la tout de m�me.


\index{grec!alphabet}
Les lettres \textbf{grecques minuscules} sont saisies de la mani�re
suivante :
\verb|\alpha|, \verb|\beta|, \verb|\gamma|, etc. Les lettres
\textbf{grecques majuscules} sont quant � elles saisies ainsi :
\verb|\Gamma|, \verb|\Delta|, etc.
\footnote{Il n'y a pas de Alpha, Beta\ldots{} majuscule dans \LaTeXe{}
parce que c'est le m�me caract�re que le A,B\ldots{} romain. Lorsque le nouveau
codage math�matique sera termin�, cela changera.} sont saisies

R�f�rez-vous au tableau~\ref{greekletters} en
page~\pageref{greekletters} pour une liste de lettres grecques.
\begin{example}
$\lambda,\xi,\pi,\theta,
 \mu,\Phi,\Omega,\Delta$
\end{example}
 

Les \textbf{indices et exposants} sont
\index{indice}\index{exposant} positionn�s en utilisant les caract�res
\verb|_|\index{_@\verb"|_"|} et \verb|^|\index{^@\verb"|^"|}.

La plupart des commandes du mode math�matique ne s'applique qu'au
caract�re suivant. Pour qu'une commande s'applique � un ensemble de
caract�res, il faut les grouper en utilisant des accolades :
\verb|{...}|. 

Le tableau~\ref{binaryrel} en page \pageref{binaryrel}  liste de
nombreuses autres relations binaires comme $\subseteq$ et $\perp$.

\begin{example}
$p^3_{ij} \qquad
 m_\text{Knuth} \\[5pt]
 a^x+y \neq a^{x+y}\qquad 
 e^{x^2} \neq {e^x}^2$
\end{example}

%SC: nth root with the accent
La \textbf{\wi{racine carr�e}} est saisie via \ci{sqrt}. La
racine $n^\text{ieme}$ est produite par la commande
\verb|\sqrt[|$n$\verb|]|. La taille du symbole racine est calcul�e par
\LaTeX. Pour obtenir le symbole seul, utilisez \verb|\surd|.

Voyez d'autres sortes de fl�ches comme $\hookrightarrow$ et
$\rightleftharpoons$ dans le tableau~\ref{tab:arrows} en page
\pageref{tab:arrows}.
\begin{example}
$\sqrt{x} \Leftrightarrow x^{1/2}
 \quad \sqrt[3]{2}
 \quad \sqrt{x^{2} + \sqrt{y}}
 \quad \surd[x^2 + y^2]$
\end{example}


\index{points de suspension!verticaux}
\index{vertical!points de suspension}
\index{points de suspension!diagonaux}
En g�n�ral, les \textbf{\wi{point}s} indiquant une op�ration de
multiplication ne sont pas imprim�s. Cependant, il arrive qu'il soit
n�cessaire de les faire appara�tre pour aider la lecture. Utilisez
alors \ci{cdot} qui imprime un seul point centr�. \ci{cdots} imprime
des \textbf{\wi{points de suspension}} centr�s (� la mani�re de points
de suspension situ�s plus haut sur la ligne). \ci{ldots} imprime des
points de suspension normaux. En plus de ces commandes vous avez
�galement \ci{vdots} pour des points align�s verticalement et
\ci{ddots} qui imprime des \wi{points en diagonale}. Vous en trouverez
un exemple en section~\ref{sec:arraymat}.
\begin{example}
$\Psi = v_1 \cdot v_2
 \cdot \ldots \qquad 
 n! = 1 \cdot 2 
 \cdots (n-1) \cdot n$
\end{example}

Les commandes \ci{overline} et \ci{underline} cr�ent un \textbf{trait
horizontal} au-dessus ou au-dessous d'une expression :
\index{horizontal!trait}
\index{trait!horizontal}
\begin{example}
$0.\overline{3} = 
 \underline{\underline{1/3}}$
\end{example}

Les commandes \ci{overbrace} et \ci{underbrace} cr�ent une grande
\textbf{accolade horizontale} au-dessus ou au-dessous d'une
expression :
\index{horizontal!accolade}
\index{accolade!horizontale}
\begin{example}
$\underbrace{\overbrace{a+b+c}^6 
 \cdot \overbrace{d+e+f}^9}
 _\text{meaning of life} = 42$
\end{example}

\index{math�matiques!accents}
Pour ajouter des accents math�matiques tels que des \textbf{fl�ches}
ou des \textbf{\wi{tilde}s}, vous pouvez utiliser les commandes du
tableau~\ref{mathacc} p.~\pageref{mathacc}.  Les chapeaux et les
tildes larges, couvrant plusieurs caract�res, sont produits par les
commandes \ci{widetilde} et \ci{widehat}. Notez bien la diff�rence
entre \ci{hat} et \ci{widehat} ainsi que le placement de \ci{bar} pour
une variable indic�e.La commande \verb|'|\index{'@\verb"|'"|} produit
un \wi{prime} :
% a dash is --
\begin{example}
$f(x) = x^2 \qquad f'(x) 
 = 2x \qquad f''(x) = 2\\[5pt]
 \hat{XY} \quad \widehat{XY}
 \quad \bar{x_0} \quad \bar{x}_0$
\end{example}


Les \textbf{\wi{vecteurs}} sont en g�n�ral marqu�s en ajoutant une
fl�che au-dessus du nom de la variable. Ceci est obtenu par la
commande \ci{vec}. Pour coder le vecteur de $A$ � $B$, les commandes
\ci{overrightarrow} et \ci{overleftarrow} sont bien utiles :
\begin{example}
$\vec{a} \qquad
 \vec{AB} \qquad
 \overrightarrow{AB}$
\end{example}

Les noms des fonctions telles que sinus doivent �tre imprim�s � l'aide
d'une police droite et non en italique comme les variables. C'est
pourquoi \LaTeX{} fournit les commandes suivantes pour les fonctions
les plus utilis�es :
\index{math�matiques!fonctions}

\begin{tabular}{llllll}
\ci{arccos} &  \ci{cos}  &  \ci{csc} &  \ci{exp} &  \ci{ker}    & \ci{limsup} \\
\ci{arcsin} &  \ci{cosh} &  \ci{deg} &  \ci{gcd} &  \ci{lg}     & \ci{ln}     \\
\ci{arctan} &  \ci{cot}  &  \ci{det} &  \ci{hom} &  \ci{lim}    & \ci{log}    \\
\ci{arg}    &  \ci{coth} &  \ci{dim} &  \ci{inf} &  \ci{liminf} & \ci{max}    \\
\ci{sinh}   & \ci{sup}   &  \ci{tan}  & \ci{tanh}&  \ci{min}    & \ci{Pr}     \\
\ci{sec}    & \ci{sin} \\
\end{tabular}

\begin{example}
\[\lim_{x \rightarrow 0}
 \frac{\sin x}{x}=1\]
\end{example}

Les fonctions n'apparaissant pas dans la liste peuvent �tre d�clar�es
avec \ci{DeclareMathOperator}. Il y a m�me une version �toil�e pour
les fonctions avec des limites. Cette commande ne peut �tre utilis�e
qu'en pr�ambule, aussi les lignes comment�es de l'exemple doivent �tre
ajout�es en pr�ambule pour fonctionner.

\begin{example}
%\DeclareMathOperator{\argh}{argh}
%\DeclareMathOperator*{\nut}{Nut}
\[3\argh = 2\nut_{x=1}\]
\end{example}

Pour la fonction \wi{modulo}, il y a deux commandes possibles : 
\ci{bmod} pour l'op�rateur binaire et 
\ci{pmod} pour l'op�rateur unaire :
\begin{example}
$a\bmod b \\
 x\equiv a \pmod{b}$
\end{example}

Un trait de \textbf{\wi{fraction}} est produit par la commande
\ci{frac}\verb|{|\emph{num�rateur}\verb|}{|\emph{d�nominateur}\verb|}|. Pour
les �quations en-ligne, la fraction est r�duite pour tenir sur la
ligne. Ce style peut aussi s'obtenir hors-text avec
\ci{tfrac}. L'inverse, c'est-�-dire obtenir des fractions en-ligne
selon le style hors-texte, est obtenu avec \ci{dfrac}.  La forme
utilisant un \emph{slash} ($1/2$) est souvent pr�f�rable pour des
petits �l�ments.
\begin{example}
En style hors-texte :
\[3/8 \qquad \frac{3}{8} 
 \qquad \tfrac{3}{8} \]
\end{example}

\begin{example}
En style en-ligne :
$1\frac{1}{2}$~heures \qquad
$1\dfrac{1}{2}$~heures
\end{example}

Ici nous utilisons la commande \ci{partial} utilis�e habituellement
pour une \wi{d�riv�e partielle} :
\begin{example}
\[\sqrt{\frac{x^2}{k+1}}\qquad
  x^\frac{2}{k+1}\qquad
  \frac{\partial^2f}
  {\partial x^2} \]
\end{example}

Pour imprimer des coefficients binomiaux (� l'am�ricaine) ou d'autres
structures semblables, utilisez la commande \ci{binom} d'\pai{amsmath} :
%SC: maybe not the right translation for Pascal's rule
\begin{example}
La r�gle de Pascal est
\begin{equation*}
 \binom{n}{k} =\binom{n-1}{k}
 + \binom{n-1}{k-1}
\end{equation*}
\end{example}
 
Il est parfois utile, notamment pour des \wi{relations binaires}, de
pouvoir superposer des symboles.  La commande
\ci{stackrel}\verb|{#1}{#2}| place l'argument \verb|#1| en taille
r�duite au-dessus de l'argument \verb|#2|, lui-m�me mis en position
normale :
\begin{example}
\begin{equation*}
 f_n(x) \stackrel{*}{\approx} 1
\end{equation*}
\end{example}

Les \textbf{\wi{int�grale}s} sont produites par la commande \ci{int},
les \textbf{\wi{somme}s} par la commande \ci{sum} et les produits par la
commande \ci{prod}. Les limites
inf�rieures et sup�rieures sont indiqu�es avec~\verb|_| et~\verb|^|
comme pour les indices et les exposants :
\begin{example}
\begin{equation*}
\sum_{i=1}^n \qquad
\int_0^{\frac{\pi}{2}} \qquad
\prod_\epsilon
\end{equation*}
\end{example}

Pour superposer des indices, l'extension \pai{amsmath} propose la
commande \ci{substack} :
\begin{example}
\begin{equation*}
\sum^n_{\substack{0<i<n \\ 
        j\subseteq i}}
   P(i,j) = Q(i,j)
\end{equation*}
\end{example}

\LaTeX{} fournit toutes sortes de symboles pour les
\textbf{\wi{crochets} et autres \wi{d�limiteurs}} (par exemple
$[\;\langle\;\|\;\updownarrow$). Les \wi{parenth�ses} et les crochets
sont obtenus avec les caract�res correspondants, les \wi{accolades}
avec \verb|\{|, mais les autres d�limiteurs ne sont obtenus que par des
commandes sp�ciales (par exemple \verb|\updownarrow|) :
\begin{example}
\begin{equation*}
{a,b,c} \neq \{a,b,c\}
\end{equation*}
\end{example}

Si vous ajoutez \ci{left} avant un d�limiteur ouvrant ou
\ci{right} avant un d�limiteur fermant, \LaTeX{} d�termine
automatiquement la taille appropri�e pour ce caract�re. Remarquez
qu'il est n�cessaire de fermer chaque d�limiteur ouvrant
(\ci{left}) avec un d�limiteur fermant (\ci{right}). Si vous
ne voulez pas de d�limiteur fermant, utilisez le d�limiteur invisible
\og \ci{right.} \fg{} :
\begin{example}
\begin{equation*}
1 + \left(\frac{1}{1-x^{2}}
    \right)^3 \qquad 
\left. \ddagger \frac{~}{~}\right)
\end{equation*}
\end{example}

Dans certains cas, il est n�cessaire d'indiquer la taille exacte des
d�limiteurs math�matiques � la main. Vous pouvez alors utiliser les
commandes \ci{big}, \ci{Big}, \ci{bigg} et \ci{Bigg} comme pr�fixes
des commandes qui impriment les d�limiteurs :
\begin{example}
$\Big((x+1)(x-1)\Big)^{2}$\\
$\big( \Big( \bigg( \Bigg( \quad
\big\} \Big\} \bigg\} \Bigg\} \quad
\big\| \Big\| \bigg\| \Bigg\| \quad
\big\Downarrow \Big\Downarrow 
\bigg\Downarrow \Bigg\Downarrow$
\end{example}

Pour une liste de tous les d�limiteurs disponibles, reportez-vous au
tableau~\ref{tab:delimiters}, page~\pageref{tab:delimiters}.



\section{Alignements verticaux}
\label{sec:vert}

\subsection{�quations longues}
\index{equation!longue}


Pour les formules qui prennent plusieurs lignes ou pour des
\wi{syst�mes d'�quations} \index{equations@�quations!syst�me d'},
utilisez les environnements \ei{align} et \verb|align*| plut�t que
\texttt{equation} ou \texttt{equation*}.%
\footnote{L'environnement \ei{align} provient
  d'\textsf{amsmath}. L'environnement similaire de \LaTeX{} sans
  \textsf{amsmath} est \ei{eqnarray}, mais il est d�conseill� de
  l'utiliser pour des raisons d'incoh�rences d'espaces et de
  r�f�rences.}
Avec \ei{align} chaque ligne est num�rot�e, alors que la variante
\verb|align*| ne produit aucun num�ro.

Les environnements \ei{align} centrent l'�quation par rapport au signe
\verb|&|. La commande \verb|\\| s�pare les lignes.  Pour �num�rer
seulement certaines �quations particuli�res, utilisez \ci{nonumber}
pour supprimer la num�rotation en la pla�ant avant le saut de ligne
\verb|\\| :
\begin{example}
\begin{align}
f(x) &= (a+b)(a-b) \label{1}\\
     &= a^2-ab+ba-b^2  \\ 
     &= a^2+b^2 \tag{faux}
\end{align}
Ceci se r�f�re � \eqref{1}.
\end{example}

\index{equations@�quations!longues}
Les \textbf{�quations longues} ne sont pas d�coup�es automatiquement
en morceaux harmonieux. L'auteur doit indiquer o� les couper et
comment indenter la suite :
\begin{example}
\begin{align}
f(x) &= 3x^5 + x^4 + 2x^3 
                \nonumber \\
     &\qquad + 9x^2 + 12x + 23 \\
     &= g(x) - h(x)
\end{align}
\end{example}
L'extension \pai{amsmath} propose d'autres environnements utiles :
\verb|flalign|, \verb|gather|, \verb|multline| et
\verb|split|. Consultez la documentation de l'extension pour en savoir
plus.

\subsection{Tableaux et matrices}
\label{sec:arraymat}

Pour imprimer des \textbf{matrices}, utilisez l'environnement
\ei{array}. Il fonctionne de mani�re similaire � l'environnement
\texttt{tabular}. La commande \verb|\\| est utilis�e pour s�parer les
lignes :
\begin{example}
\begin{equation*}
 \mathbf{X} = \left( 
  \begin{array}{ccc}
   x_1 & x_2 & \ldots \\
   x_3 & x_4 & \ldots \\
   \vdots & \vdots & \ddots
  \end{array} \right)
\end{equation*}
\end{example}

L'environnement \ei{array} peut �galement �tre utilis� pour imprimer
des \wi{fonctions!par parties} en utilisant \og \verb|.| \fg{} comme
d�limiteur (invisible) de droite :%
\footnote{Si vous avez de nombreuses constructions de ce genre,
  essayer l'environnement \ei{cases} d'\textsf{amsmath} peut en valoir
  la peine, puisqu'il simplifie ce type de syntaxe.}
\begin{example}
\begin{equation*}
|x| = \left\{
 \begin{array}{rl}
  -x & \text{si } x < 0\\
   0 & \text{si } x = 0\\
   x & \text{si } x > 0
 \end{array} \right.
\end{equation*}
\end{example}


\ei{array} peut aussi servir � mettre en page des
matrices\index{matrice}, mais \pai{amsmath} fournit une meilleure
solution via les environnements \ei{matrix}. Il y en a six versions
(avec des d�limiteurs diff�rents) : \ei{matrix} (aucun), \ei{pmatrix}
$($, \ei{bmatrix} $[$, \ei{Bmatrix} $\{$, \ei{vmatrix} $\vert$ et
\ei{Vmatrix} $\Vert$. Vous n'avez pas � sp�cifier le nombre de
colonnes comme avec \ei{array}. Leur nombre maximal est de 10 mais il
est modifiable (bien que ce ne soit pas fr�quent d'avoir besoin de 10
colonnes ou plus !) :
\begin{example}
\begin{equation*}
 \begin{matrix} 
   1 & 2 \\
   3 & 4 
 \end{matrix} \qquad
 \begin{bmatrix} 
   1 & 2 & 3 \\
   4 & 5 & 6 \\ 
   7 & 8 & 9
 \end{bmatrix}
\end{equation*}
\end{example}

\section{Espacement en mode math�matique}
\label{sec:math-spacing}

\index{espacement!math�matique}
\index{math�matiques!espaces}

Si l'espacement choisi par \LaTeX{} dans une formule n'est pas
satisfaisant, il peut �tre ajust� en ins�rant des commandes
d'espacement. Les plus importantes sont : \ci{,} pour une espace fine
($\frac{3}{18}\:\textrm{quad}$, \demowidth{0.166em}), \ci{:} pour une
espace moyenne ($\frac{4}{18}\:\textrm{quad}$, \demowidth{0.222em}) et
\ci{;} pour une espace grande ($\frac{5}{18}\: \textrm{quad}$,
\demowidth{0.277em}).  L'espace �chapp�e \verb*|\ | cr�e une espace
moyenne similaire � l'espace entre mots. and \ci{quad}
(\demowidth{1em}) et \ci{qquad} (\demowidth{2em}) produisent des
espaces plus larges.  La largeur d'un \ci{quad}
%SC: \footnote{cadratin} ?? A v�rifier -mh
correspond � la largeur du caract�re \og M \fg{} dans la police courante. 
La commande \verb|\!|\cih{"!} produit une espace fine n�gative
de $-\frac{3}{18}\:\textrm{quad}$ ($-$\demowidth{0.166em}). Remarquez
que dans l'exemple suivant, \og d \fg{} est imprim�e en police romaine :

\begin{example}
\begin{equation*}
 \int_1^2 \ln x \mathrm{d}x \qquad
 \int_1^2 \ln x \,\mathrm{d}x
\end{equation*}
\end{example}


Dans l'exemple suivant, nous d�finissons une commande \ci{ud} qui
produit \og $\,\mathrm{d}$ \fg{} (remarquez l'espace
\demowidth{0.166em} avant le $\text{d}$), de mani�re � ne pas avoir �
le saisir � chaque fois. La commande  \ci{newcommand} est plac�e en
pr�ambule.
%  More on
% \ci{newcommand} in section~\ref{} on page \pageref{}. To Do: Add label and
% reference to "Customising LaTeX" -> "New Commands, Environments and Packages"
% -> "New Commands".
\begin{example}
\newcommand{\ud}{\,\mathrm{d}}

\begin{equation*}
 \int_a^b f(x)\ud x 
\end{equation*}
\end{example}

Lorsque vous utilisez des int�grales multiples, vous constatez que
l'espace entre celles-ci est trop grand. Vous pouvez certes utiliser
\ci{!}, mais \AmS-\LaTeX{} propose un ensemble de commandes pour
r�aliser cet ajustement : \ci{iint}, \ci{iiint},
\ci{iiiint} et \ci{idotsint}.

\begin{example}
\newcommand{\ud}{\,\mathrm{d}}

\[ \int\int f(x)g(y) 
                  \ud x \ud y \]
\[ \int\!\!\!\int 
         f(x)g(y) \ud x \ud y \]
\[ \iint f(x)g(y) \ud x \ud y \]
\end{example}

Reportez-vous au document \texttt{testmath.tex} distribu� avec
\AmS-\LaTeX{}, au chapitre 8 de \companion{} ou au chapitre 9 de
\desgraupes{} pour plus de d�tails.


\subsection{Fant�mes\texorpdfstring{\dots}{...}}


Il arrive que \LaTeX{} en fasse un peu trop dans des alignements
verticaux d'indices ou d'exposants. La commande \ci{phantom} permet
de r�server de l'espace pour des caract�res qui ne seront pas
imprim�s, comme le montrent les exemples suivants :
\begin{example}
\begin{equation*}
{}^{14}_{6}\text{C}
\qquad \text{� comparer �} \qquad
{}^{14}_{\phantom{1}6}\text{C}
\end{equation*}
\end{example}

Si vous souhaitez mettre en forme des isotopes comme dans l'exemple
ci-avant, l'extension \pai{mhchem} d�di�e aux formules chimiques peut
vous y aider.


\section{Manipuler les polices math�matiques}\label{sec:fontsz}
Plusieurs polices math�matiques sont list�e au tableau~\ref{mathalpha}
page \pageref{mathalpha}.
\begin{example}
 $\Re \qquad
  \mathcal{R} \qquad
  \mathfrak{R} \qquad
  \mathbb{R} \qquad $  
\end{example}
Ces deux derni�res n�cessitent \pai{amssymb} ou \pai{amsfonts}.

Malgr� tout, il peut �tre n�cessaire d'indiquer � \LaTeX{} la taille
exacte. En mode math�matique, la taille de la police est d�termin�e
par les quatre commandes :
\begin{flushleft}
\ci{displaystyle}~($\displaystyle 123$),
 \ci{textstyle}~($\textstyle 123$), 
\ci{scriptstyle}~($\scriptstyle 123$) et
\ci{scriptscriptstyle}~($\scriptscriptstyle 123$).
\end{flushleft}

Si $\sum$ est plac� dans une fraction, il sera imprim� en-ligne �
moins d'indiquer � \LaTeX{} le contraire :
\begin{example}
\begin{equation*}
 R = \frac{\displaystyle{ 
   \sum_{i=1}^n (x_i-\bar{x})
   (y_i- \bar{y})}} 
   {\displaystyle{\left[
   \sum_{i=1}^n(x_i-\bar{x})^2
   \sum_{i=1}^n(y_i-\bar{y})^2
   \right]^{1/2}}}
\end{equation*}    
\end{example}

Changer de style modifie �galement la fa�on dont les limites et les
grands op�rateurs sont affich�s.
 
% This is not a math accent, and no maths book would be set this way.
% mathop gets the spacing right.


\section{Symboles gras}
\index{symboles!gras}

Il est relativement ardu d'obtenir des symboles gras avec \LaTeX{} ;
cela est sans doute fait expr�s car les typographes amateurs ont
tendance � en abuser. La commande de changement de graisse
\verb|\mathbf| permet d'obtenir des caract�res gras, mais romains
(donc droits) alors que les symboles math�matiques sont normalement en
italique. De plus elle ne fonctionne pas sur les minuscules
grecques. Il y a bien une commande \ci{boldmath}, mais \emph{elle ne
  peut �tre utilis�e qu'en dehors du mode math�matique}. Cependant
elle fonctionne aussi pour les symboles :
\begin{example}
$\mu, M \qquad 
\mathbf{\mu}, \mathbf{M}$
\qquad \boldmath{$\mu, M$}
\end{example}

L'extension \pai{amsbsy} (appel�e par \pai{amsmath}) ainsi que
\pai{bm} dans le paquet \texttt{tools} simplifient beaucoup ce
probl�me puisqu'ils fournissent une commande \ci{boldsymbol} :

\begin{example}
$\mu, M \qquad
\boldsymbol{\mu}, \boldsymbol{M}$
\end{example}


\section{Th�or�mes, lemmes, etc.}

En r�digeant des documents math�matiques, on a besoin d'un moyen de
pr�senter des lemmes, des d�finitions, des axiomes et d'autres
structures similaires :
\begin{lscommand}
\ci{newtheorem}\verb|{|\emph{nom}\verb|}[|\emph{compteur}\verb|]{|%
         \emph{texte}\verb|}[|\emph{section}\verb|]|
\end{lscommand}
L'argument \emph{nom} est un mot-clef utilis� pour identifier le
th�or�me. L'argument \emph{texte} d�finit le nom r�el du th�or�me tel
qu'il sera imprim� dans le document final.

Les arguments entre crochets sont optionnels. Ils servent �
indiquer la num�rotation � utiliser sur le th�or�me. Utilisez
le \emph{compteur} pour indiquer le \emph{nom} d'un th�or�me d�j�
d�clar�. Le nouveau th�or�me sera alors num�rot� dans la m�me
s�quence. Avec \emph{section} vous indiquez dans quel niveau de
sectionnement vous voulez num�roter votre th�or�me.

Apr�s avoir ex�cut� \ci{newtheorem} dans le pr�ambule de votre
document, vous pouvez utiliser la commande suivante :

\begin{code}
\verb|\begin{|\emph{nom}\verb|}[|\emph{texte}\verb|]|\\
Ceci est mon premier th�or�me\\
\verb|\end{|\emph{nom}\verb|}|     
\end{code}

L'extension \pai{amsthm} (qui fait partie d'\AmS-LaTeX) met �
disposition la commande
\ci{newtheoremstyle}\verb|{|\emph{style}\verb|}| qui vous montrera que
le choix de style pour les th�or�mes se limite � trois styles
pr�f�finis : \texttt{definition} (titre gras, corps romain),
\texttt{plain} (titre gras, corps italique) or \texttt{remark} (titre
italique, corps romain).

%SC: ah, the initial translation missed the subtle sarcasm here
Voil� pour la th�orie. Les exemples qui suivent devraient lever tout
doute et montrer clairement que l'environnement \verb|\newtheorem| est
trop complexe � comprendre.

% actually define things
\theoremstyle{definition} \newtheorem{loi}{loi}
\theoremstyle{plain}      \newtheorem{decret}[loi]{D�cret}
\theoremstyle{remark}     \newtheorem*{lechef}{Le chef}

D'abord d�finissez les th�or�mes :

\begin{verbatim}
\theoremstyle{definition} \newtheorem{loi}{loi}
\theoremstyle{plain}      \newtheorem{decret}[loi]{D�cret}
\theoremstyle{remark}     \newtheorem*{lechef}{Le chef}
\end{verbatim}

\begin{example}
\begin{loi} \label{chef}
Le chef a raison.
\end{loi}
\begin{decret}[Important]
Le chef a toujours raison,
voir la loi~\ref{chef}.
\end{decret}
\begin{lechef}
Et si le chef a tort, se r�f�rer
� la loi~\ref{chef}.
\end{lechef}
\end{example}


Le th�or�me \og decret \fg{} utilise le m�me compteur que le
th�or�me \og loi \fg{}, donc il obtient un num�ro dans la m�me
s�quence que les autres \og lois \fg{}. L'argument entre crochets permet
de sp�cifier un titre ou quelque chose de ce genre pour le th�or�me.
\begin{example}
\newtheorem{mur}{Murphy}[section]
\begin{mur} Tout ce qui peut 
aller mal ira mal.\end{mur}
\end{example}

Le th�or�me \og Murphy \fg{} est num�rot� � l'int�rieur de la section en
cours. On aurait pu utiliser un autre niveau tel que \texttt{chapter}
ou \texttt{subsection}. 

L'extension \pai{amsthm} fournit aussi l'environnement de preuve
\ei{proof}.

\begin{example}
\begin{proof}
 Trivial, utilisez
\[E=mc^2\]
\end{proof}
\end{example}

La commande \ci{qedhere} permet de d�placer le symbole de fin de
preuve (CQFD) pour les cas o� il finirait seul sur une ligne.

\begin{example}
\begin{proof}
 Trivial, utilisez
\[E=mc^2 \qedhere\]
\end{proof}
\end{example}

Si vous voulez personnaliser vos th�or�mes au point pr�s, l'extension
\pai{ntheorem} vous offrira la pl�thore d'options dont vous avez
besoin.


\endinput

%

% Local Variables:
% TeX-master: "lshort"
% mode: latex
% mode: flyspell
% End:
 
%%%%%%%%%%%%%%%%%%%%%%%%%%%%%%%%%%%%%%%%%%%%%%%%%%%%%%%%%%%%%%%%%
% Contents: TeX and LaTeX and AMS symbols for Maths
% $Id$
%%%%%%%%%%%%%%%%%%%%%%%%%%%%%%%%%%%%%%%%%%%%%%%%%%%%%%%%%%%%%%%%%

% Pour les informations de licence, voir contrib.tex.
% See contrib.tex for license information.



\section{Liste des symboles mathématiques}  \label{symbols}

\index{symboles!mathématiques}

Les tableaux suivants montrent tous les symboles accessibles en mode
\emph{mathématique}.

%
% Conditional Text in case the AMS Fonts are installed
%
Remarquez que certains tableaux montrent des symboles qui ne sont
accessibles qu'après avoir chargé l'extension \pai{amssymb} dans le
préambule\footnote{Ces tables sont dérivées du
fichier \texttt{symbols.tex} de David~Carlisle et modifiées selon les
suggestions de Josef~Tkadlec.}. Si les extensions et
les polices de l'\AmS{} ne sont pas installées sur votre système, vous
pouvez les récupérer sur\\
\CTANref|pkg/amslatex|. Il existe une
liste beaucoup plus complète de symboles sur
 \CTANref|info/symbols/comprehensive|.

\begin{table}[!h]
\caption{Accents en mode mathématique}  \label{mathacc}
\begin{symbols}{*3{cl}}
\mstW{\hat}{a}   & \mstW{\check}{a} & \mstW{\tilde}{a}       \\
\mstW{\grave}{a} & \mstW{\dot}{a}   & \mstW{\ddot}{a}        \\
\mstW{\bar}{a}   & \mstW{\vec}{a}   & \mstW{\widehat}{AAA}   \\
\mstW{\acute}{a} & \mstW{\breve}{a} & \mstW{\widetilde}{AAA} \\
\mstW{\mathring}{a}
\end{symbols}
\end{table}


\begin{table}[!h]
\caption{Alphabet grec}\label{greekletters}
\bigskip
Certaines lettres n'ont pas leur équivalent en majuscule comme
\ci{Alpha}, \ci{Beta}\ldots{} parce qu'elles ressemblent aux lettres
romaines normales : A, B\ldots
\begin{symbols}{*4{cl}}
 \mstX{\alpha}     & \mstX{\theta}     & \mstX{o}          & \mstX{\upsilon}  \\
 \mstX{\beta}      & \mstX{\vartheta}  & \mstX{\pi}        & \mstX{\phi}      \\
 \mstX{\gamma}     & \mstX{\iota}      & \mstX{\varpi}     & \mstX{\varphi}   \\
 \mstX{\delta}     & \mstX{\kappa}     & \mstX{\rho}       & \mstX{\chi}      \\
 \mstX{\epsilon}   & \mstX{\lambda}    & \mstX{\varrho}    & \mstX{\psi}      \\
 \mstX{\varepsilon}& \mstX{\mu}        & \mstX{\sigma}     & \mstX{\omega}    \\
 \mstX{\zeta}      & \mstX{\nu}        & \mstX{\varsigma}  &               \\
 \mstX{\eta}       & \mstX{\xi}        & \mstX{\tau} & \\
 \mstX{\Gamma}     & \mstX{\Lambda}    & \mstX{\Sigma}     & \mstX{\Psi}      \\
 \mstX{\Delta}     & \mstX{\Xi}        & \mstX{\Upsilon}   & \mstX{\Omega}    \\
 \mstX{\Theta}     & \mstX{\Pi}        & \mstX{\Phi}
\end{symbols}
\end{table}

\clearpage

\begin{table}[!tbp]
\caption{Relations binaires} \label{binaryrel}
\bigskip
Vous pouvez produire la négation de ces symboles en les préfixant par
la commande \ci{not}.
\begin{symbols}{*3{cl}}
 \mstX{<}           & \mstX{>}           & \mstX{=}          \\
 \mstX{\leq}ou \verb|\le|   & \mstX{\geq}ou \verb|\ge|   & \mstX{\equiv}     \\
 \mstX{\ll}         & \mstX{\gg}         & \mstX{\doteq}     \\
 \mstX{\prec}       & \mstX{\succ}       & \mstX{\sim}       \\
 \mstX{\preceq}     & \mstX{\succeq}     & \mstX{\simeq}     \\
 \mstX{\subset}     & \mstX{\supset}     & \mstX{\approx}    \\
 \mstX{\subseteq}   & \mstX{\supseteq}   & \mstX{\cong}      \\
 \mstX{\sqsubset}$^a$ & \mstX{\sqsupset}$^a$ & \mstX{\Join}$^a$    \\
 \mstX{\sqsubseteq} & \mstX{\sqsupseteq} & \mstX{\bowtie}    \\
 \mstX{\in}         & \mstX{\ni}, \verb|\owns|  & \mstX{\propto}    \\
 \mstX{\vdash}      & \mstX{\dashv}      & \mstX{\models}    \\
 \mstX{\mid}        & \mstX{\parallel}   & \mstX{\perp}      \\
 \mstX{\smile}      & \mstX{\frown}      & \mstX{\asymp}     \\
 \mstX{:}           & \mstX{\notin}      & \mstX{\neq}ou \verb|\ne|
\end{symbols}
\end{table}

\begin{table}[!tbp]
\caption{Opérateurs binaires}
\begin{symbols}{*3{cl}}
 \mstX{+}              & \mstX{-}              & &                 \\
 \mstX{\pm}            & \mstX{\mp}            & \mstX{\triangleleft} \\
 \mstX{\cdot}          & \mstX{\div}           & \mstX{\triangleright}\\
 \mstX{\times}         & \mstX{\setminus}      & \mstX{\star}         \\
 \mstX{\cup}           & \mstX{\cap}           & \mstX{\ast}          \\
 \mstX{\sqcup}         & \mstX{\sqcap}         & \mstX{\circ}         \\
 \mstX{\vee}, \verb|\lor|     & \mstX{\wedge}, \verb|\land|  & \mstX{\bullet}       \\
 \mstX{\oplus}         & \mstX{\ominus}        & \mstX{\diamond}      \\
 \mstX{\odot}          & \mstX{\oslash}        & \mstX{\uplus}        \\
 \mstX{\otimes}        & \mstX{\bigcirc}       & \mstX{\amalg}        \\
 \mstX{\bigtriangleup} &\mstX{\bigtriangledown}& \mstX{\dagger}       \\
 \mstX{\lhd}$^a$         & \mstX{\rhd}$^a$         & \mstX{\ddagger}      \\
 \mstX{\unlhd}$^a$       & \mstX{\unrhd}$^a$       & \mstX{\wr}
\end{symbols}
\centerline{\footnotesize $^a$Utilisez l'extension \textsf{latexsym}
pour avoir accès à ces symboles}
\end{table}

\clearpage

\begin{table}[!tbp]
\caption{Opérateurs n-aires}
\begin{symbols}{*4{cl}}
 \mstX{\sum}      & \mstX{\bigcup}   & \mstX{\bigvee}  \\
 \mstX{\prod}     & \mstX{\bigcap}   & \mstX{\bigwedge} \\
 \mstX{\coprod}   & \mstX{\bigsqcup} & \mstX{\biguplus} \\
 \mstX{\int}      & \mstX{\oint}     & \mstX{\bigodot} \\
 \mstX{\bigoplus} & \mstX{\bigotimes} & \\
\end{symbols}

\end{table}


\begin{table}[!tbp]
\caption{Flèches} \label{tab:arrows}
\begin{symbols}{*2{cl}}
 \mstX{\leftarrow}or \verb|\gets|& \mstX{\longleftarrow} \\
 \mstX{\rightarrow}or \verb|\to|& \mstX{\longrightarrow} \\
 \mstX{\leftrightarrow}    & \mstX{\longleftrightarrow} \\
 \mstX{\Leftarrow}         & \mstX{\Longleftarrow}     \\
 \mstX{\Rightarrow}        & \mstX{\Longrightarrow}    \\
 \mstX{\Leftrightarrow}    & \mstX{\Longleftrightarrow}\\
 \mstX{\mapsto}            & \mstX{\longmapsto}        \\
 \mstX{\hookleftarrow}     & \mstX{\hookrightarrow}    \\
 \mstX{\leftharpoonup}     & \mstX{\rightharpoonup}    \\
 \mstX{\leftharpoondown}   & \mstX{\rightharpoondown}  \\
 \mstX{\rightleftharpoons} & \mstX{\iff}(bigger spaces) \\
 \mstX{\uparrow}   & \mstX{\downarrow} \\
 \mstX{\updownarrow} & \mstX{\Uparrow} \\
 \mstX{\Downarrow} &  \mstX{\Updownarrow} \\
 \mstX{\nearrow} &  \mstX{\searrow} \\
  \mstX{\swarrow} & \mstX{\nwarrow} \\
 \mstX{\leadsto}$^a$
\end{symbols}
\centerline{\footnotesize $^a$Utilisez l'extension \textsf{latexsym}
pour obtenir ces symboles}
\end{table}

\begin{table}[!tbp]
\caption{Flèches en tant qu'accents}  \label{arrowacc}
\begin{symbols}{*2{cl}}
\mstW{\overrightarrow}{AB}     & \mstW{\underrightarrow}{AB}     \\
\mstW{\overleftarrow}{AB}      & \mstW{\underleftarrow}{AB}      \\
\mstW{\overleftrightarrow}{AB} & \mstW{\underleftrightarrow}{AB} \\
\end{symbols}
\end{table}


\clearpage

\begin{table}[!tbp]
\caption{Délimiteurs}\label{tab:delimiters}
\begin{symbols}{*3{cl}}
 \mstX{(}            & \mstX{)}            & \mstX{\uparrow} \\
 \mstX{[}ou \verb|\lbrack|   & \mstX{]}ou \verb|\rbrack|  & \mstX{\downarrow}   \\
 \mstX{\{}ou \verb|\lbrace|  & \mstX{\}}ou \verb|\rbrace|  & \mstX{\updownarrow} \\
 \mstX{\langle}      & \mstX{\rangle}      &  \mstX{\Uparrow} \\
 \mstX{|}ou \verb|\vert| & \mstX{\|}ou \verb|\Vert| & \mstX{\Downarrow} \\
  \mstX{/}            & \mstX{\backslash}   &   \mstX{\Updownarrow}  \\
 \mstX{\lfloor}      & \mstX{\rfloor}      &  \\
 \mstX{\rceil}       &  \mstX{\lceil}  &&\\
\end{symbols}
\end{table}

\begin{table}[!tbp]
\caption{Grands délimiteurs}
\begin{symbols}{*3{cl}}
 \mstY{\lgroup}      & \mstY{\rgroup}      & \mstY{\lmoustache}  \\
 \mstY{\arrowvert}   & \mstY{\Arrowvert}   & \mstY{\bracevert} \\
 \mstY{\rmoustache} \\
\end{symbols}
\end{table}


\begin{table}[!tbp]
\caption{Symboles divers}
\begin{symbols}{*4{cl}}
 \mstX{\dots}       & \mstX{\cdots}      & \mstX{\vdots}      & \mstX{\ddots}     \\
 \mstX{\hbar}       & \mstX{\imath}      & \mstX{\jmath}      & \mstX{\ell}       \\
 \mstX{\Re}         & \mstX{\Im}         & \mstX{\aleph}      & \mstX{\wp}        \\
 \mstX{\forall}     & \mstX{\exists}     & \mstX{\mho}$^a$      & \mstX{\partial}   \\
 \mstX{'}           & \mstX{\prime}      & \mstX{\emptyset}   & \mstX{\infty}     \\
 \mstX{\nabla}      & \mstX{\triangle}   & \mstX{\Box}$^a$     & \mstX{\Diamond}$^a$ \\
 \mstX{\bot}        & \mstX{\top}        & \mstX{\angle}      & \mstX{\surd}      \\
\mstX{\diamondsuit} & \mstX{\heartsuit}  & \mstX{\clubsuit}   & \mstX{\spadesuit} \\
 \mstX{\neg}ou \verb|\lnot| & \mstX{\flat}       & \mstX{\natural}    & \mstX{\sharp}
\end{symbols}
\centerline{\footnotesize $^a$Utilisez l'extension \textsf{latexsym}
pour obtenir ces symboles}
\end{table}


\clearpage

\begin{table}[!tbp]
\caption{Symboles non-mathématiques}
\bigskip
Ces symboles peuvent également être utilisés en mode \emph{texte}.
\begin{symbols}{*4{cl}}
 \mstSC{\dag}  &  \mstSC{\S}  &  \mstSC{\copyright} &  \mstSC{\textregistered}  \\
 \mstSC{\ddag} &  \mstSC{\P}  &  \mstSC{\pounds}    &  \mstSC{\%}               \\
\end{symbols}
\end{table}

%
%
% If the AMS Stuff is not available, we drop out right here :-)
%

\begin{table}[!tbp]
\caption{Délimiteurs de l'\AmS}\label{AMSD}
\bigskip
\begin{symbols}{*4{cl}}
\mstX{\ulcorner}&\mstX{\urcorner}&\mstX{\llcorner}&\mstX{\lrcorner}
\end{symbols}
\end{table}

\begin{table}[!tbp]
\caption{Caractères grecs et hébreux de l'\AmS}
\begin{symbols}{*5{cl}}
\mstX{\digamma}     &\mstX{\varkappa} & \mstX{\beth} &\mstX{\gimel} & \mstX{\daleth}
\end{symbols}
\end{table}

\clearpage

\begin{table}[tbp]
  \caption{Alphabets mathématiques} \label{mathalpha}
\bigskip Voir le tableau~\ref{mathfonts} page~\pageref{mathfonts} pour
d'autres polices mathématiques.
\begin{symbols}{@{}*3l@{}}
Exemple& Commande & Extension à utiliser\\
\hline
\rule{0pt}{1.05em}$\mathrm{ABCDE abcde 1234}$
        & \verb|\mathrm{ABCDE abcde 1234}|
        &       \\
$\mathit{ABCDE abcde 1234}$
        & \verb|\mathit{ABCDE abcde 1234}|
        &       \\
$\mathnormal{ABCDE abcde 1234}$
        & \verb|\mathnormal{ABCDE abcde 1234}|
        &  \\
$\mathcal{ABCDE abcde 1234}$
        & \verb|\mathcal{ABCDE abcde 1234}|
        &  \\
$\mathscr{ABCDE abcde 1234}$
        &\verb|\mathscr{ABCDE abcde 1234}|
        &\pai{mathrsfs}\\
$\mathfrak{ABCDE abcde 1234}$
        & \verb|\mathfrak{ABCDE abcde 1234}|
        &\pai{amsfonts}  ou \textsf{amssymb}  \\
$\mathbb{ABCDE abcde 1234}$
        & \verb|\mathbb{ABCDE abcde 1234}|
        &\pai{amsfonts}  ou \textsf{amssymb} \\
\end{symbols}
\end{table}

\begin{table}[!tbp]
\caption{Opérateurs binaires de l'\AmS}
\begin{symbols}{*3{cl}}
 \mstX{\dotplus}        & \mstX{\centerdot}      &       \\
 \mstX{\ltimes}         & \mstX{\rtimes}         & \mstX{\divideontimes} \\
 \mstX{\doublecup}      & \mstX{\doublecap}	   & \mstX{\smallsetminus} \\
 \mstX{\veebar}         & \mstX{\barwedge}       & \mstX{\doublebarwedge}\\
 \mstX{\boxplus}        & \mstX{\boxminus}       & \mstX{\circleddash}   \\
 \mstX{\boxtimes}       & \mstX{\boxdot}         & \mstX{\circledcirc}   \\
 \mstX{\intercal}       & \mstX{\circledast}     & \mstX{\rightthreetimes} \\
 \mstX{\curlyvee}       & \mstX{\curlywedge}     & \mstX{\leftthreetimes}
\end{symbols}
\end{table}

\clearpage

\begin{table}[!tbp]
\caption{Relations binaires de l'\AmS}
\begin{symbols}{*3{cl}}
 \mstX{\lessdot}           & \mstX{\gtrdot}            & \mstX{\doteqdot} \\
 \mstX{\leqslant}          & \mstX{\geqslant}          & \mstX{\risingdotseq}     \\
 \mstX{\eqslantless}       & \mstX{\eqslantgtr}        & \mstX{\fallingdotseq}    \\
 \mstX{\leqq}              & \mstX{\geqq}              & \mstX{\eqcirc}           \\
 \mstX{\lll}ou \verb|\llless| & \mstX{\ggg}            & \mstX{\circeq}  \\
 \mstX{\lesssim}           & \mstX{\gtrsim}            & \mstX{\triangleq}        \\
 \mstX{\lessapprox}        & \mstX{\gtrapprox}         & \mstX{\bumpeq}           \\
 \mstX{\lessgtr}           & \mstX{\gtrless}           & \mstX{\Bumpeq}           \\
 \mstX{\lesseqgtr}         & \mstX{\gtreqless}         & \mstX{\thicksim}         \\
 \mstX{\lesseqqgtr}        & \mstX{\gtreqqless}        & \mstX{\thickapprox}      \\
 \mstX{\preccurlyeq}       & \mstX{\succcurlyeq}       & \mstX{\approxeq}         \\
 \mstX{\curlyeqprec}       & \mstX{\curlyeqsucc}       & \mstX{\backsim}          \\
 \mstX{\precsim}           & \mstX{\succsim}           & \mstX{\backsimeq}        \\
 \mstX{\precapprox}        & \mstX{\succapprox}        & \mstX{\vDash}            \\
 \mstX{\subseteqq}         & \mstX{\supseteqq}         & \mstX{\Vdash}            \\
 \mstX{\shortparallel}     & \mstX{\Supset}            & \mstX{\Vvdash}           \\
 \mstX{\blacktriangleleft} & \mstX{\sqsupset}          & \mstX{\backepsilon}      \\
 \mstX{\vartriangleright}  & \mstX{\because}           & \mstX{\varpropto}        \\
 \mstX{\blacktriangleright}& \mstX{\Subset}            & \mstX{\between}          \\
 \mstX{\trianglerighteq}   & \mstX{\smallfrown}        & \mstX{\pitchfork}        \\
 \mstX{\vartriangleleft}   & \mstX{\shortmid} 	 & \mstX{\smallsmile} 	\\
 \mstX{\trianglelefteq}    & \mstX{\therefore} 	 & \mstX{\sqsubset}
\end{symbols}
\end{table}

\begin{table}[!tbp]
\caption{Flèches de l'\AmS}
\begin{symbols}{*2{cl}}
 \mstX{\dashleftarrow}      & \mstX{\dashrightarrow}     \\
 \mstX{\leftleftarrows}     & \mstX{\rightrightarrows}   \\
 \mstX{\leftrightarrows}    & \mstX{\rightleftarrows}    \\
 \mstX{\Lleftarrow}         & \mstX{\Rrightarrow}        \\
 \mstX{\twoheadleftarrow}   & \mstX{\twoheadrightarrow}  \\
 \mstX{\leftarrowtail}      & \mstX{\rightarrowtail}     \\
 \mstX{\leftrightharpoons}  & \mstX{\rightleftharpoons}  \\
 \mstX{\Lsh}                & \mstX{\Rsh}                \\
 \mstX{\looparrowleft}      & \mstX{\looparrowright}     \\
 \mstX{\curvearrowleft}     & \mstX{\curvearrowright}    \\
 \mstX{\circlearrowleft}    & \mstX{\circlearrowright}   \\
 \mstX{\multimap}  &  \mstX{\upuparrows}  \\
 \mstX{\downdownarrows} & \mstX{\upharpoonleft} \\
 \mstX{\upharpoonright} & \mstX{\downharpoonright} \\
 \mstX{\rightsquigarrow} & \mstX{\leftrightsquigarrow} \\
\end{symbols}
\end{table}

\clearpage

\begin{table}[!tbp]
\caption{Négations des relations binaires et des flèches de l'\AmS}\label{AMSNBR}
\begin{symbols}{*3{cl}}
 \mstX{\nless}           & \mstX{\ngtr}            & \mstX{\varsubsetneqq}  \\
 \mstX{\lneq}            & \mstX{\gneq}            & \mstX{\varsupsetneqq}  \\
 \mstX{\nleq}            & \mstX{\ngeq}            & \mstX{\nsubseteqq}     \\
 \mstX{\nleqslant}       & \mstX{\ngeqslant}       & \mstX{\nsupseteqq}     \\
 \mstX{\lneqq}           & \mstX{\gneqq}           & \mstX{\nmid}           \\
 \mstX{\lvertneqq}       & \mstX{\gvertneqq}       & \mstX{\nparallel}      \\
 \mstX{\nleqq}           & \mstX{\ngeqq}           & \mstX{\nshortmid}      \\
 \mstX{\lnsim}           & \mstX{\gnsim}           & \mstX{\nshortparallel} \\
 \mstX{\lnapprox}        & \mstX{\gnapprox}        & \mstX{\nsim}           \\
 \mstX{\nprec}           & \mstX{\nsucc}           & \mstX{\ncong}          \\
 \mstX{\npreceq}         & \mstX{\nsucceq}         & \mstX{\nvdash}         \\
 \mstX{\precneqq}        & \mstX{\succneqq}        & \mstX{\nvDash}         \\
 \mstX{\precnsim}        & \mstX{\succnsim}        & \mstX{\nVdash}         \\
 \mstX{\precnapprox}     & \mstX{\succnapprox}     & \mstX{\nVDash}         \\
 \mstX{\subsetneq}       & \mstX{\supsetneq}       & \mstX{\ntriangleleft}  \\
 \mstX{\varsubsetneq}    & \mstX{\varsupsetneq}    & \mstX{\ntriangleright} \\
 \mstX{\nsubseteq}       & \mstX{\nsupseteq}       & \mstX{\ntrianglelefteq}\\
 \mstX{\subsetneqq}      & \mstX{\supsetneqq}      &\mstX{\ntrianglerighteq}\\[0.5ex]
 \mstX{\nleftarrow}      & \mstX{\nrightarrow}     & \mstX{\nleftrightarrow}\\
 \mstX{\nLeftarrow}      & \mstX{\nRightarrow}     & \mstX{\nLeftrightarrow}

\end{symbols}
\end{table}

\begin{table}[!tbp]
\caption{Symboles divers de l'\AmS} \label{AMSmisc}
\begin{symbols}{*3{cl}}
 \mstX{\hbar}             & \mstX{\hslash}           & \mstX{\Bbbk}            \\
 \mstX{\square}           & \mstX{\blacksquare}      & \mstX{\circledS}        \\
 \mstX{\vartriangle}      & \mstX{\blacktriangle}    & \mstX{\complement}      \\
 \mstX{\triangledown}     &\mstX{\blacktriangledown} & \mstX{\Game}            \\
 \mstX{\lozenge}          & \mstX{\blacklozenge}     & \mstX{\bigstar}         \\
 \mstX{\angle}            & \mstX{\measuredangle}    & \\
 \mstX{\diagup}           & \mstX{\diagdown}         & \mstX{\backprime}       \\
 \mstX{\nexists}          & \mstX{\Finv}             & \mstX{\varnothing}      \\
 \mstX{\eth}              & \mstX{\sphericalangle}   & \mstX{\mho}
\end{symbols}
\end{table}





\endinput

%

% Local Variables:
% TeX-master: "lshort2e"
% mode: latex
% mode: flyspell
% End:

%%%%%%%%%%%%%%%%%%%%%%%%%%%%%%%%%%%%%%%%%%%%%%%%%%%%%%%%%%%%%%%%%
% Contents: Specialties of the LaTeX system
% $Id$
%%%%%%%%%%%%%%%%%%%%%%%%%%%%%%%%%%%%%%%%%%%%%%%%%%%%%%%%%%%%%%%%%

% Pour les informations de licence, voir contrib.tex.
% See contrib.tex for license information.


\chapter{Compléments}
\label{chap:spec}
\thispagestyle{plain}

\begin{intro}
%  Ne lisez pas ce chapitre ! Si vous vous sentez à l'aise, vous pouvez
%  écrire vos premiers documents avec \LaTeX{} dès maintenant. Le but
%  de ce chapitre est d'ajouter un peu de piment à votre connaissance
%  de \LaTeX{}. Une description bien plus complète des possibilités et
%  des améliorations possibles se trouve dans le {\normalfont\manual{}} et
%  dans {\normalfont\companion}.

Pour rédiger un document important, \LaTeX{} vous fournit des outils
pour réaliser un index, une liste de références bibliographiques et
d'autres choses. Des descriptions bien plus complètes de ces possibilités
et des améliorations possibles avec \LaTeX{} se trouvent
dans le {\normalfont\manual{}},dans {\normalfont\companion{}}
et dans {\normalfont\desgraupes{}}.
\end{intro}

\section{Références bibliographiques}

L'environnement \ei{thebibliography} permet de produire une liste de
références bibliographiques.  Chaque référence
commence par
\begin{lscommand}
\ci{bibitem}\verb|[|\emph{label}\verb|]{|\emph{marque}\verb|}|
\end{lscommand}
La \emph{marque} est utilisée pour citer la référence dans le
document.
\begin{lscommand}
\ci{cite}\verb|{|\emph{marque}\verb|}|
\end{lscommand}
Si vous n'utilisez pas l'option \emph{label}, les références sont
automatiquement numérotées. Le paramètre qui suit
\verb|\begin{thebibliography}| définit la largeur du décrochement
utilisé pour placer ces numéros. Dans l'exemple ci-après,
\verb|{99}| indique à \LaTeX{} que le décrochement ne devrait jamais
être plus large que le nombre 99.

\begin{example}
Partl~\cite{pa}
propose que\dots

{\small
\begin{thebibliography}{99}
\bibitem{pa} H.~Partl:
\emph{German \TeX},
TUGboat Vol.~9, No.~1 ('88)
\end{thebibliography}
}
\end{example}
\chaptermark{Compléments} % w need to fix the damage done by the
                           %bibliography example.
\thispagestyle{fancyplain}


Pour des projets plus importants, il est recommandé d'utiliser l'outil
BiB\TeX{}. Celui-ci est fourni avec la plupart des installations de
\TeX{}. Il permet de maintenir une base de données de références
bibliographiques et d'en extraire la liste des références citées dans
votre document. La génération des listes de références  par
BiB\TeX{} utilise un mécanisme de feuilles de style qui permettent
de réaliser tous les types de présentations habituellement demandés.

\section{Index} \label{sec:indexing}
L'\wi{index} est un élément fort utile pour de nombreux
ouvrages. \LaTeX{} et le programme associé
\texttt{makeindex}\footnote{Sur les systèmes qui ne supportent pas les
noms de fichiers de plus de huit caractères, ce programme s'appelle
\texttt{makeidx}.} permettent de créer des index assez
facilement. Cette introduction présente seulement les commandes
élémentaires de gestion d'un index. Pour une description plus
détaillée, reportez-vous à~\companion ou à \desgraupes{}.

Pour utiliser cette fonctionnalité, l'extension \pai{makeidx} doit être
chargée dans le préambule avec :
\begin{lscommand}
\verb|\usepackage{makeidx}|
\end{lscommand}
\pagebreak[3]
La  création de l'index doit être activée par la commande :
\begin{lscommand}
  \ci{makeindex}
\end{lscommand}
\noindent placée dans le préambule.

Le contenu de l'index est défini par une série de commandes :
\begin{lscommand}
  \ci{index}\verb|{|\emph{clef@entrée formatée}\verb|}|
\end{lscommand}
\noindent
où \emph{entrée formatée} est ce qui doit apparaître dans l'index, et
\emph{clef} est utilisée pour le tri. La partie \emph{entrée formattée} est
optionnelle. Vous insérez des commandes
 \ci{index} aux endroits du texte que vous voulez voir référencés par
 l'index. Le tableau~\ref{index} explique cela
 avec plusieurs exemples.

\begin{table}[!htp]
\caption{Exemples de clefs d'index}
\label{index}
\begin{center}
\begin{tabular}{@{}lll@{}}
  \textbf{Exemple} &\textbf{Résultat} &\textbf{Commentaires}\\\hline
  \rule{0pt}{1.05em}\verb|\index{hello}| &hello, 1 &Entrée normale\\
\verb|\index{hello!Peter}|   &\hspace*{2ex}Peter, 3 &Sous-entrée de 'hello'\\
\verb|\index{Sam@\textsl{Sam}}|     &\textsl{Sam}, 2& Entrée formatée\\
\verb|\index{Lin@\textbf{Lin}}|     &\textbf{Lin}, 7& Entrée formatée\\
\verb|\index{Kaese@K\"ase}|     &\textbf{K\"ase}, 33& Entrée formatée\\
\verb.\index{ecole@\'ecole}.     &\'ecole, 4& Entrée formatée\\
\verb.\index{Jenny|textbf}.     &Jenny, \textbf{3}& Numéro de page formaté\\
\verb.\index{Joe|textit}.     &Joe, \textit{5}& Numéro de page formaté\\
\end{tabular}
\end{center}
\end{table}

Quand le fichier source est traité par \LaTeX{}, chaque commande
\verb|\index| crée une entrée adaptée contenant le numéro de la page
en cours dans le fichier qui porte le même nom de base que le fichier
source, mais avec le suffixe \eei{.idx}. Ce fichier est ensuite traité
par le programme \texttt{makeindex} :

\begin{lscommand}
  \texttt{makeindex} \emph{nom de fichier}
\end{lscommand}
\index{makeindex@\texttt{makeindex}}
Le programme \texttt{makeindex} crée un index trié dans le fichier
\eei{.ind}. Ensuite, la prochaine fois que  le fichier source sera
traité, le contenu du fichier \texttt{.ind} sera
inclus à l'endroit où \LaTeX{} rencontrera la commande :
\begin{lscommand}
  \ci{printindex}
\end{lscommand}

L'extension \pai{showidx} fournie avec \LaTeXe{} permet de visualiser
les entrées de l'index dans la marge gauche du texte. Cela permet la
relecture et la mise au point de l'index.

Remarquez également que la command \ci{index} peut affecter votre mise
en page si vous n'y prenez pas garde.

\begin{example}
Mon mot \index{mot}. Différent
de mot\index{mot}. Notez la
position du point final.
\end{example}

Le programme \texttt{makeindex} standard ne traite malheureusement pas
correctement les caractères accentués dans les clefs : il les place
systématiquement en tête de l'ordre alphabétique.
Pour obtenir un classement correct des clés contenant des caractères
accentués (le «é» doit être classé comme un «e»), on peut utiliser
le caractère \texttt{@} : la dernière ligne du tableau~\ref{index}
produira une entrée \og école \fg{} dans l'index, classée comme s'il s'agissait
de \og ecole \fg{}.


\section{En-têtes améliorés}
\label{sec:fancyhdr}

L'extension \pai{fancyhdr}%
\footnote{disponible sur
          \CTAN|macros/latex/contrib/supported/fancyhdr|.},
développée par Piet van Oostrum, offre quelques commandes simples
permettant de personnaliser les entêtes et les pieds de page de votre
document. Si vous regardez en haut de cette page, vous verrez un
résultat possible de l'utilisation de cette extension.

La difficulté principale pour personnaliser les en-têtes et les pieds
de page consiste à mettre à jour le nom de la section ou du chapitre
en cours utilisés par ces éléments. \LaTeX{} réalise cela en deux
étapes. Dans la définition des en-têtes et pieds de page
les commandes \ci{leftmark} et \ci{rightmark} sont utilisées pour désigner
respectivement les noms du chapitre et la section courants.
La valeur de ces commandes est redéfinie chaque fois
qu'un nouveau chapitre ou une nouvelle section commence.

Pour plus de souplesse, la commande \verb|\chapter| et ses collègues
ne redéfinissent pas \ci{leftmark} et \ci{rightmark}
directement. Elles appellent les commandes appelées \ci{chaptermark},
\ci{sectionmark} et \ci{subsectionmark} qui sont chargées de redéfinir
\ci{leftmark} et/ou \ci{rightmark}, selon la présentation
désirée.

Ainsi, si vous voulez modifier la présentation du nom du chapitre
courant dans l'en-tête, vous n'aurez qu'à redéfinir la commande
\ci{chaptermark}. \cih{sectionmark}\cih{subsectionmark}

La figure~\ref{fancyhdr} montre un exemple de configuration de l'extension
\pai{fancyhdr} qui se rapproche de la présentation utilisée pour ce
document. La documentation complète de cette extension se trouve à
l'adresse mentionnée dans la note de bas de page.

\begin{figure}[!htbp]
\begin{lined}{\textwidth}
\begin{verbatim}
\documentclass{book}
\usepackage{fancyhdr}
\pagestyle{fancy}
% Ceci permet d'avoir les noms de chapitre et de section
% en minuscules
\renewcommand{\chaptermark}[1]{
        \markboth{#1}{}}
\renewcommand{\sectionmark}[1]{
        \markright{\thesection\ #1}}
\fancyhf{}     % supprime les en-têtes et pieds
\fancyhead[LE,RO]{\bfseries\thepage}% Left Even, Right Odd
\fancyhead[LO]{\bfseries\rightmark} % Left Odd
\fancyhead[RE]{\bfseries\leftmark}  % Right Even
\renewcommand{\headrulewidth}{0.5pt}% filet en haut de page
\addtolength{\headheight}{0.5pt}    % espace pour le filet
\renewcommand{\footrulewidth}{0pt}  % pas de filet en bas
\fancypagestyle{plain}{ % pages de tetes de chapitre
   \fancyhead{}         % supprime l'entete
   \renewcommand{\headrulewidth}{0pt} % et le filet
}
\end{verbatim}
\end{lined}
\caption{Exemple de configuration de l'extension \pai{fancyhdr}}%
\label{fancyhdr}
\end{figure}


\section{L'extension verbatim}

Plus haut dans ce document, vous avez appris à utiliser
l'\emph{environnement} \ei{verbatim}. Dans cette section vous allez découvrir
l'\emph{extension} \pai{verbatim}. L'extension \pai{verbatim} est une
nouvelle implémentation de l'environnement du même nom qui corrige
certaines de ses limitations. En soi cela n'est pas spectaculaire,
mais ce package s'adjoint de nouvelles fonctionnalités
qui justifient que cette extension soit citée ici. L'extension
\pai{verbatim} propose la commande~:

\begin{lscommand}
\ci{verbatiminput}\verb|{|\emph{nom de fichier}\verb|}|
\end{lscommand}

\noindent qui permet d'inclure un fichier ASCII brut dans votre
document, comme s'il se trouvait à l'intérieur d'un environnement
\ei{verbatim}.

Puisque l'extension \pai{verbatim} fait partie de l'ensemble \og
\texttt{tools} \fg{}, elle devrait être déjà disponible sur la plupart
des systèmes. Pour en savoir plus au sujet de cette extension,
reportez-vous à~\cite{verbatim}.

\section{Installation d'extensions}
\label{sec:Packages}

La plupart des  installations \LaTeX{} fournissent en standard un grand
nombre d'extensions, mais il arrive que justement celle dont on aurait
besoin manque, ou qu'une extension nécessite une mise à jour.
L'endroit le plus adéquat pour rechercher les versions officielles
des  extensions est le CTAN (\url{http://www.ctan.org/}).

Les extensions, telles \pai{geometry}, \pai{hyphenat} et beaucoup
d'autres, sont en général fournies sous la forme de deux fichiers, l'un
de suffixe \texttt{.dtx}, l'autre de suffixe \texttt{.ins}.  Souvent
un fichier \texttt{readme.txt} leur est joint et donne une brève
description de l'extension. Le mieux est alors de commencer par la
lecture de ce fichier.

Quoi qu'il en soit, une fois que vous avez copié les fichiers de
l'extension sur votre machine, vous dever les manipuler de manière à
(a) informer votre distribution \TeX\ de cette nouvelle extension et
(b) obtenir sa documentation. Voici la manière de procéder :

\begin{enumerate}
\item exécuter \LaTeX{} sur le fichier \texttt{.ins}. Ceci produira les
  fichiers \eei{.sty}, \eei{.def}, etc., dont \LaTeX{} a besoin.
\item déplacer ces fichiers dans un répertoire adéquat, en général
  c'est dans \texttt{\ldots/texmf/tex/latex} ou dans
  \texttt{\ldots/\emph{localtexmf}/tex/latex}.
\item mettre à jour la base de données des noms de fichiers, la commande
  dépend de votre distribution \LaTeX{} :
  \TeX{}live -- \texttt{texhash}; web2c -- \texttt{mktexlsr};
  MiK\TeX{} -- \texttt{initexmf -{}-update-fndb} ou via l'interface
  graphique; la commande peut également être \texttt{texconfig rehash}.
\end{enumerate}

\noindent Il faut ensuite  extraire la documentation du fichier
\texttt{.dtx} :
\begin{enumerate}
\item exécuter \hologo{XeLaTeX} sur le fichier \texttt{.dtx}. Cela produira un
      fichier \texttt{.pdf}. Noter que plusieurs exécutions de
      \hologo{XeLaTeX} peuvent être nécessaires pour produire les références
      croisées complètes.
\item vérifier si \LaTeX{} a produit un fichier \texttt{.idx}. Si ce
      n'est pas le cas, alors la documentation n'a pas d'index. Passer à l'étape 5.
\item pour produire l'index, exécuter la commande suivante :
\begin{lscommand}
\texttt{makeindex -s gind.ist \textit{nom}}
\end{lscommand}
(où \textit{nom} représente le nom du fichier principal, sans
suffixe).
\item exécuter \LaTeX{} sur le fichier \texttt{.dtx} une fois de plus.
\item enfin, produire un fichier PostScript ou PDF à imprimer pour une
      lecture plus confortable.
\end{enumerate}

Parfois vous constaterez qu'un fichier \texttt{.glo} (glossaire) a
également été produit. Exécutez la commande suivante entre les étapes
4 et 5 :

\texttt{makeindex -s gglo.ist -o \textit{nom}\texttt{.gls}
\textit{nom}\texttt{.glo}}

\noindent Et n'oubliez pas de re-exécuter \LaTeX{} sur le fichier
\texttt{.dtx} avant de passer à l'étape 5.

%%%%%%%%%%%%%%%%%%%%%%%%%%%%%%%%%%%%%%%%%%%%%%%%%%%%%%%%%%%%%%%%%
% Contents: Chapter on pdfLaTeX
% French original by Daniel Flipo 14/07/2004
%%%%%%%%%%%%%%%%%%%%%%%%%%%%%%%%%%%%%%%%%%%%%%%%%%%%%%%%%%%%%%%%%


\section{\LaTeX{} et PDF} \label{sec:pdftex}\index{PDF}
\secby{Daniel Flipo}{Daniel.Flipo@univ-lille1.fr}%

PDF est un format de document \wi{hypertexte} et portable. De la même manière que
dans une page Web, certains mots sont marqués comme des
hyperliens. Ils renvoient vers d'autres endroits du document voir vers
d'autres documents. En cliquant sur un hyperlien vous
serez transportés sur la destination de ce lien. Dans le contexte de
\LaTeX{}, cela signifie que toute occurrence de \ci{ref} et de
\ci{pageref} peut devenir un hyperlien. De plus, la table des matières,
l'index et d'autres structures similaires deviendront aussi des
collections d'hyperliens.

La plupart des pages Web de nos jours sont écrites en HTML
\emph{(HyperText Markup Language)}. Ce format a deux défauts majeurs
pour écrire des documents scientifiques :
\begin{enumerate}
\item L'inclusion de formules mathématiques n'est généralement pas
  possible. Bien qu'il y ait un standard pour cela, la plupart des
  navigateurs ne le prennent pas en compte ou n'ont pas les polices
  requises;
\item L'impression de documents HTML est possible mais les résultats
  varient énormément selon la plateforme et le navigateur, et loin des
  standards de qualité du monde \LaTeX{}.
\end{enumerate}

Il y a de nombreuses tentatives de création de traducteurs de \LaTeX{}
vers HTML. Certaines rencontrèrent un certain succès dans le sens où
elles peuvent produire des pages Web lisibles à partir d'un fichier
d'entrée \LaTeX{} standard. Mais toutes évitent certaines parties
délicates pour obtenir ce résultat. Dès que l'on utilise des
fonctionnalités plus complexes de \LaTeX{} ou des extensions externes,
les choses tendent à partir à vau-l'eau. Les auteurs qui souhaitent
préserver la qualité typographique unique de leurs documents, même sur
le Web, se tournent vers le format PDF \emph{(Portable Document
  Format)} qui préserve la mise en page du document et autorise la
navigation hypertextuelle. La plupart des navigateurs disposent de
plus d'extensions pour l'affichage direct de documents PDF.

Tous les moteurs \TeX{} modernes sont capables de générer du PDF
nativement. Si vous avez lu cette introduction jusqu'ici vous êtes
déjà familier du processus.

\subsection{Liens hypertextuels}
\label{ssec:pdfhyperref}

L'extension \pai{hyperref} ajoute deux sympathiques fonctionnalités à
vos fichiers \LaTeX{} en PDF:

\begin{enumerate}
\item Le format de papier est réglé selon ce que vous avez spécifié
  dans l'invocation de classe de document
\item Toutes les références de votre document sont transformées en
  liens hypertexte.
\end{enumerate}

Ajoutez simplement \verb+\usepackage[pdftex]{hyperref}+ en tant que
\emph{dernière} commande de votre préambule.

De nombreuses options modifient le comportement de l'extension
\pai{hyperref} :
\begin{itemize}
\item soit en tant qu'une liste séparée par des virgules après
  l'option pdftex\\
  \verb+\usepackage{hyperref}+ ;
\item soit sur des lignes à part avec la commande
  \verb+\hypersetup{+\emph{options}\verb+}+.
\end{itemize}

Dans la liste suivante, les valeurs par défaut des options sont
affichées avec une police droite.

\begin{flushleft}
\begin{description}
  \item [\texttt{bookmarks (=true,\textit{false})}] montrer ou cacher
    les marque-pages lors de l'affichage du document ;
  \item [\texttt{unicode (=false,\textit{true})}] permettre d'utiliser
    des caractères non-latins dans les marque-pages du lecteur de PDF ;
  \item [\texttt{pdftoolbar (=true,\textit{false})}] montrer ou cacher
    la barre d'outils du lecteur de PDF ;
  \item [\texttt{pdfmenubar (=true,\textit{false})}] montrer ou cacher
    le menu du lecteur de PDF~;
  \item [\texttt{pdffitwindow (=false,\textit{true})}] ajuster le
    grossissement initial lors de l'affichage du fichier PDF ;
  \item [\texttt{pdftitle (=\{texte\})}] définir le titre affiché dans
    la fenêtre \texttt{Document Info} du lecteur de PDF ;
  \item [\texttt{pdfauthor (=\{texte\})}] le nom de l'auteur ;
  \item [\texttt{pdfnewwindow (=false,\textit{true})}] définir si une
    nouvelle fenêtre doit être ouverte lorsqu'un lien conduit hors du
    document courant ;
  \item [\texttt{colorlinks (=false,\textit{true})}] entourer les
    liens par des liserés colorés (\texttt{false}) ou colorer le texte
    des liens (\texttt{true}). La couleur de ces liens peut
    être configurée via les options suivantes (les couleurs par défaut
    sont indiquées) :
    \begin{description}
    \item [\texttt{linkcolor (=red)}] couleur des liens
      internes (sections, pages, etc) ;
    \item [\texttt{citecolor (=green)}] couleur des
      liens de citations bibliographiques ;
    \item [\texttt{filecolor (=magenta)}] couleur des
      liens vers des fichiers ;
    \item [\texttt{urlcolor (=cyan)}] couleur des liens
      URL (adresse électronique, adresse web).
    \end{description}
\end{description}
\end{flushleft}

Si les valeurs par défaut vous plaisent, utilisez
\begin{code}
\begin{verbatim}
\usepackage{hyperref}
\end{verbatim}
\end{code}

Pour avoir la liste des marque-pages ouverte et des liens en couleur
(les valeurs \texttt{=true} sont optionnelles) :
\begin{code}
\begin{verbatim}
\usepackage[bookmarks,colorlinks]{hyperref}
\end{verbatim}
\end{code}

Lors de la création de fichiers PDFs en vue d'impression, les liens
colorés peuvent finir grisés dans le résultat final, ce qui les rend
difficiles à lire. Vous pouvez utiliser des liserés colorés qui ne
seront pas imprimés :
\begin{code}
\begin{verbatim}
\usepackage{hyperref}
\hypersetup{colorlinks=false}
\end{verbatim}
\end{code}
\noindent ou noircir les liens :
\begin{code}
\begin{verbatim}
\usepackage{hyperref}
\hypersetup{colorlinks,
            citecolor=black,
            filecolor=black,
            linkcolor=black,
            urlcolor=black,
            }
\end{verbatim}
\end{code}

Lorsque vous souhaitez simplement fournir des informations en section
\texttt{Document Info} pour le fichier PDF :
\begin{code}
\begin{verbatim}
\usepackage[pdfauthor={Pierre Desproges},
            pdftitle={Des femmes qui tombent},
            ]{hyperref}
\end{verbatim}
\end{code}

\vspace{\baselineskip}

En plus des hyperliens automatiques pour les références croisées, il
est possible d'insérer des liens explicites via
\begin{lscommand}
\ci{href}\verb|{|\emph{url}\verb|}{|\emph{text}\verb|}|
\end{lscommand}

Le code
\begin{code}
\begin{verbatim}
Le site web du \href{http://www.ctan.org}{CTAN}.
\end{verbatim}
\end{code}
produit la sortie \og \href{http://www.ctan.org}{CTAN} \fg{}; cliquer
sur le mot \og \textcolor{magenta}{CTAN} \fg{} vous amènera au site
web du CTAN.

Si la destination du lien n'est pas une URL mais un fichier local,
vous pouvez utiliser la commande \ci{href} sans la partie 'http://' :
\begin{verbatim}
  Le document complet est \href{manuel.pdf}{ici}
\end{verbatim}
Elle produit le texte \og Le document complet est
\textcolor{cyan}{ici} \fg{}. Cliquer sur le mot \og
\textcolor{cyan}{ici} \fg{} ouvrira le fichier \texttt{manuel.pdf} (le
nom de fichier est relatif au document courant).

L'auteur d'un article peut souhaiter que ses lecteurs puissent lui
envoyer facilement des emails en utilisant la commande \ci{href} à
l'intérieur de la commande \ci{author} sur la page de titre du
document :
\begin{code}
\begin{verbatim}
\author{Mary Oetiker <\href{mailto:mary@oetiker.ch}%
       {mary@oetiker.ch}>
\end{verbatim}
\end{code}
Remarquez que j'ai écrit le lien de manière à ce que mon adresse email
apparaisse non seulement dans le lien mais aussi sur la page
elle-même. J'ai fait cela parce que le lien\\
\verb+\href{mailto:mary@oetiker.ch}{Mary Oetiker}+,\\
s'il fonctionnerait bien en listant le PDF, ne serait plus visible une fois
la page imprimée.

\subsection{Problème de liens}

Des messages tels que celui-ci :
\begin{verbatim}
! pdfTeX warning (ext4): destination with the same
  identifier (name{page.1}) has been already used,
  duplicate ignored
\end{verbatim}
apparaissent lors de la réinitialisation d'un compteur, par exemple
lors de l'utilisation de la commande \ci{mainmatter} fournie par la
classe de document \texttt{book}. Elle remet à 1 le numéro de page
page avant le premier chapitre du livre. Cependant, la préface du
livre a aussi une page numérotée~1 : les liens vers la \og page~1
\fg{} ne seraient alors plus uniques, d'où l'avertissement \og
\verb+duplicate+ has been \verb+ignored+ \fg{}.

La contre-mesure consiste en l'ajout de \texttt{plainpages=false} aux
options d'hyperref. Ça ne résout le problème que pour le compteur de
pages, malheureusement.
Une solution encore plus radicale est d'utiliser l'option\\
\texttt{hypertexnames=false} mais cela empêchera les liens de pages en
index de fonctionner.

\subsection{Problèmes de marque-pages}

Le texte affiché par les marque-pages ne correspond pas toujours à vos
attentes : comme ceux-ci sont \og juste du texte \fg{}, encore moins de
caractères que pour \LaTeX{} sont disponibles. Hyperref remarquera ce
type de problème et notifiera un avertissement :
\begin{code}
\begin{verbatim}
Package hyperref Warning:
Token not allowed in a PDFDocEncoded string:
\end{verbatim}
\end{code}
Vous pouvez contourner ce problème en fournissant une chaîne de
caractères pour les marque-pages pour remplacer le texte à problèmes :
\begin{lscommand}
\ci{texorpdfstring}\verb|{|\emph{Texte \TeX{}}\verb|}{|\emph{Texte de marque-pages}\verb|}|
\end{lscommand}


Les expressions mathématiques sont les coupables idéales pour ce genre
de problèmes :
\begin{code}
\begin{verbatim}
\section{\texorpdfstring{$E=mc^2$}%
        {E=mc**2}}
\end{verbatim}
\end{code}
transforme \verb+$E=mc^2$+ en \og E=mc**2 \fg{} dans la zone de
marque-pages.

Si votre document est écrit en Unicode et que vous utilisez l'option
\verb+unicode+ de l'extension \pai{hyperref}, vous pouvez utiliser des
caractères Unicode dans les marque-pages. Vous aurez ainsi une bien
plus grande sélection parmi les caractères utilisables avec
\ci{texorpdfstring}.

\section{Utiliser \hologo{XeLaTeX} et PDF}
\label{sec:xetex}\index{PDF}\index{XeTeX@\hologo{XeTeX}}\index{XeLaTeX@\hologo{XeLaTeX}}

\secby{Axel Kielhorn}{A.Kielhorn@web.de}%

La plupart des propos de la précédente section sont également valides
pour \hologo{XeLaTeX}.

Le Wiki situé sur le site \url{http://wiki.xelatex.org/doku.php}
rassemble les informations concernant \hologo{XeTeX} et \hologo{XeLaTeX}.

\subsection{Les polices}

En plus des polices usuelles basées sur \texttt{tfm}, \hologo{XeLaTeX}
sait utiliser toute police connue du système d'exploitation. Supposons
que les polices \texttt{Linux Libertine} soient installées, vous
pouvez les utiliser via la commande
\begin{code}
\begin{verbatim}
\usepackage{fontspec}
\setmainfont[Ligatures=TeX]{Linux Libertine}
\end{verbatim}
\end{code}
%
ajoutée dans le préambule. La détection des versions italique et
grasse de la police devrait également être effective, ainsi
\verb|\textit| et \verb|\textbf| fonctionneront comme de
coutume. Lorsque la police utilise la technologie OpenType, vous avez
accès à de nombreuses fonctionnalités qui nécessitaient naguère le
passage à une autre police ou l'utilisation de polices virtuelles. La
fonctionnalité principale est l'obtention d'un ensemble de caractères
étendu : une police peut contenir des caractères latins, grecs et
cyrilliques ainsi que les ligatures correspondantes.

%SC: Pas sûr de la traduction de "lower case" pour des chiffres, ça
%fait penser au vieux style de chiffres à la française
De nombreuses polices contiennent au moins deux types de chiffres,
ceux à l'alignement normal et ceux en style ancien (ou en minuscules)
qui s'étendent partiellement en dessous de la ligne. Elles peuvent
contenir des chiffres proportionnés (le \og 1 \fg prend moins de place
que le \og 0 \fg) ou à espacement fixe mieux adaptés pour des tableaux.

\begin{code}
\begin{verbatim}
\newfontfamily\LLln[Numbers=Lining]{(font)}
\newfontfamily\LLos[Numbers=OldStyle]{(font)}
\newfontfamily\LLlnm[Numbers=Lining,Numbers=Monospaced]{(font)}
\newfontfamily\LLosm[Numbers=OldStyle,Numbers=Monospaced]{(font)}
\end{verbatim}
\end{code}

Presque toutes les polices OpenType contiennent les ligatures standard
(fl fi ffi) mais il existe aussi des ligatures rares ou historiques
comme st, ct et tz. Si leur utilité dans un rapport technique est
discutable, elles ont leur place dans un roman. Pour activer ces
ligatures utilisez l'une des commandes suivantes :

\begin{code}
\begin{verbatim}
\setmainfont[Ligatures=Rare]{(font)}
\setmainfont[Ligatures=Historic]{(font)}
\setmainfont[Ligatures=Historic,Ligatures=Rare]{(font)}
\end{verbatim}
\end{code}

Toutes les polices ne contiennent pas les deux ensembles de ligatures,
aussi consultez la documentation de la police ou essayez-la
directement. Parfois ces ligatures dépendent du langage utilisé, par
exemple la ligature (fk), inconnue en français, est utilisée en
polonais. Il vous faudra ajouter
\begin{code}
\begin{verbatim}
\setmainfont[Language=Polish]{(font)}
\end{verbatim}
\end{code}
pour activer les ligatures polonaises.

Certaines polices (comme la police commerciale Adobe Garamond Premier
Pro) contiennent aussi des glyphes alternatifs activés par défaut dans
le \hologo{XeLaTeX} fourni avec \TeX Live~2010\footnote{Ce
  comportement n'était pas présent car désactivé par défaut dans des
  versions précédentes.}. Le résultat de son utilisation est par
exemple un \og Q \fg majuscule stylisé dont la queue rejoint par
en-dessous le \og u \fg qui suit. Pour désactiver cette fonctionnalité,
vous devrez définir la police sans les traitements contextuels :

\begin{code}
\begin{verbatim}
\setmainfont[Contextuals=NoAlternate]{(font)}
\end{verbatim}
\end{code}

Pour en savoir plus sur les polices et \hologo{XeLaTeX}, référez-vous
au manuel de \pai{fontspec}.

\subsubsection{Où se procurer des polices OpenType ?}

Si \texttt{TeXLive} est installé, vous en avez quelques-unes à
disposition dans \url{.../texmf-dist/fonts/opentype} : il ne vous
reste plus qu'à les installer dans votre système d'exploitation. Cette
collection ne comprend pas \texttt{DejaVu}, qui elle est disponible
sur le site \url{http://dejavu-fonts.org/}.

Assurez-vous que chacune des polices n'est bien installée \emph{qu'une
seule fois}, sinon vous obtiendrez des résultats au mieux \og
intéressants \fg.

Vous pouvez utiliser toute police installée sur votre ordinateur,
rappelez-vous cependant que d'autres utilisateurs ne les ont peut-être
pas. Par exemple, la police Zapfino utilisée dans le manuel de
\pai{fontspec} existe sous Mac OSX mais pas sous Windows.\footnote{À
  noter qu'une version commerciale appelée Zapfino Extra existe.}

\subsubsection{Saisie de caractères Unicode}

Le nombre de caractères dans une police a assurément augmenté, mais ce
n'est pas le cas du nombre de touches des claviers. Comment, alors,
saisir des caractères non-ASCII ?

Si vous écrivez beaucoup de texte dans une autre langue, vous pouvez
installer un clavier pour ce langage et imprimer les positions des
caractères (la plupart des systèmes d'exploitation ont une forme ou
une autre de clavier virtuel, il suffit de faire une copie d'écran).

Si vous n'avez pas souvent besoin de caractères exotiques, vous pouvez
simplement le choisir dans la palette des caractères.

Certains environnements (comme p.e. le système X Window) offrent des
méthodes spécifiques pour entrer des caractères non-ASCII. Il en va de
même de certains éditeurs (comme Vim et Emacs). Consultez les
documentations des outils que vous utilisez.

\subsection{Compatibilité entre \hologo{XeLaTeX} et \hologo{pdfLaTeX}}
\label{sec:comp-entre-holog}

Certaines différences existent entre \hologo{XeLaTeX} et
\hologo{pdfLaTeX}.

\begin{itemize}
\item Un document \hologo{XeLaTeX} doit être écrit en Unicode (UTF-8)
  alors qu'un document \hologo{pdfLaTeX} peut utiliser d'autres types
  de codages.
\item L'extension \pai{microtype} ne fonctionne pas encore avec
  \hologo{XeLaTeX} et le support pour la saillie de caractères
  (technique connue sous le nom de \og character protrusion \fg en
  anglais) est en cours de développement.
\item Tout ce qui a trait aux polices doit être vérifié (sauf si vous
  vous cantonnez à la police Latin Modern).
\end{itemize}


\section{Créer des présentations}
\label{sec:beamer}
\secby{Daniel Flipo}{Daniel.Flipo@univ-lille1.fr}
Vous pouvez présenter les résultats de vos travaux sur un tableau
noir, avec des transparents ou directement depuis votre ordinateur
portable grâce à un logiciel de présentation.

\wi{pdf\LaTeX} combiné à la classe \pai{beamer} permettent la création
de présentations en PDF très semblables à ce que vous feriez avec
LibreOffice ou PowerPoint dans un bon jour, mais bien plus portable
puisque des lecteurs PDF sont disponibles sur bien plus de systèmes.

La classe \pai{beamer} utilise \pai{graphicx}, \pai{color} et
\pai{hyperref} avec des options adaptées aux présentations sur
écrans.
%SC: this comment does not serve any purpose anymore and should be removed from fr and eng versions
%La figure~\ref{fig:pdfscr} contient un exemple de fichier minimal à
%compiler avec \wi{pdf\LaTeX} et le
%résultat produit.

% Écran capturé par ImageMagick (man ImageMagick) fonction « import »
% et convertie en jpg toujours par ImageMagick.


\begin{figure}[htbp]
\begin{verbatim}
\documentclass[10pt]{beamer}
\usepackage[utf8x]{inputenc}
\mode<beamer>{%
  \usetheme[hideothersubsections,
            right,width=22mm]{Goettingen}
}

\title{Une simple présentation}
\author[D. Flipo]{Daniel Flipo}
\institute{U.S.T.L. \& GUTenberg}
\titlegraphic{\includegraphics[width=20mm]{USTL}}
\date{2005}

\begin{document}

\begin{frame}<handout:0>
  \titlepage
\end{frame}

\section{Un exemple}

\begin{frame}
  \frametitle{Choses à faire un dimanche après-midi}
  \begin{block}{On peut \ldots}
    \begin{itemize}
      \item sortir le chien\dots \pause
      \item lire un livre\pause
      \item rendre fou un chat\pause
    \end{itemize}
  \end{block}
  et bien d'autres choses
\end{frame}
\end{document}
\end{verbatim}
  \caption{Exemple de code pour la classe \pai{beamer}}
  \label{fig:code-beamer}
\end{figure}

À la compilation du code de la figure~\ref{fig:code-beamer}
\footnote{Attention, l'exemple suppose un codage d'entrée en
  utf8, voyez la ligne d'appel de l'extension \texttt{inputenc}. \NdT}
avec \wi{pdf\LaTeX}, vous obtiendrez un fichier avec une page de titre
et une deuxième page dont les points seront révélés un à un lorsque
vous avancez dans la présentation.

Un avantage notable de la classe beamer est sa capacité à produire un
fichier PDF utilisable directement sans avoir à passer par une étape
\PSi{} comme avec \pai{prosper} ou sans demander un post-traitement
comme les présentations créées avec l'extension \pai{ppower4}.

Avec la classe \pai{beamer} vous pouvez produire plus versions (modes)
de votre document à partir de votre fichier d'entrée. Celui-ci peut
contenir des instructions spéciales pour les différents modes entre
chevrons (crochets obliques \verb|<| et \verb|>|). Les modes suivants
sont disponibles :
\begin{description}
\item[beamer] pour une présentation PDF comme au-dessus ;
\item[trans] pour les transparents ;
\item[handout] pour la version imprimée.
\end{description}
Le mode par défaut est \texttt{beamer}. Vous pouvez le changer via les
options globales de la classe, comme
\verb|\documentclass[10pt,handout]{beamer}| pour obtenir les notes
associées.

L'aspect de la présentation dépend du thème choisi. Vous pouvez soit
utiliser les thèmes livrés avec la classe beamer, soit en créer
un. Pour plus d'information, voyez la documentation dans
\texttt{beameruserguide.pdf}.

Observons maintenant le code de la figure~\ref{fig:code-beamer} de
plus près.

Pour la version à l'écran de la présentation (\verb|\mode<beamer>|) nous
avons choisi le thème \emph{Goettingen}. Celui-ci propose un panneau
de navigation intégré à la table des matières. Les options nous permettent
de choisir la taille de ce panneau (22~mm dans notre cas) et sa
position (à droite du corps du texte). L'option
\emph{hideothersubsections} montre les titres de sections, mais
uniquement les sous-sections de la section en cours. Il n'y a pas
d'autre réglage spécifique pour \verb|\mode<trans>| et
\verb|\mode<handout>|, ils apparaîtront selon leur mise en page
usuelle.

Les commandes \verb|\title{}|, \verb|\author{}|, \verb|\institute{}|,
et\\ \verb|\titlegraphic{}| permettent de remplir la page de titre. Les
arguments optionnels de \verb|\title[]{}| et \verb|\author[]{}|
servent à spécifier une version spéciale du titre et de l'auteur
(respectivement) qui apparaîtront sur le panneau fourni par le thème
\emph{Goettingen}.

Les titres et sous-titres du panneau proviennent de l'utilisation des
commandes usuelles \verb|\section{}| et
\verb|\subsection{}|. Celles-ci doivent être placées \emph{en dehors}
de l'environnement \ei{frame}.

Les petites icônes de navigation au bas de l'écran permettent elles
aussi de naviguer dans le document. Leur présence n'est pas dépendante
du thème choisi.

Le contenu de chaque transparent ou chaque écran doit être placé dans
un environnement \ei{frame}. L'option entre chevrons ((\verb|<| et
\verb|>|) nous permet de supprimer un cadre en particulier de l'une des
versions de la présentation. Dans notre exemple, la première page
n'apparaîtra pas dans la version imprimée à cause de l'option
\verb|<handout:0>|.

Il est particulièrement recommandé de donner un titre à chaque
transparent, en plus du transparent de titre. Ce titre est fourni via
la commande \verb|\frametitle{}|. Pour obtenir un sous-titre,
utilisez \verb|\framesubtitle{}| ou une commande \ei{block} comme dans
l'exemple. Remarquez également que le contenu des commandes \verb|\section{}|
et \verb|\subsection{}| n'apparaît pas sur le transparent.

La commande \verb|\pause| dans l'environnement itemize fait que les points
vont se révéler un à un. D'autres effets de présentations sont
disponibles via les commandes \verb|\only|, \verb|\uncover|,
\verb|\alt| et \verb|\temporal|. Vous pouvez aussi utiliser les
chevrons à de nombreux endroits pour personnaliser encore plus votre
présentation.

Dans tous les cas, nous vous recommandons fortement de lire la
documentation de la classe beamer \texttt{beameruserguide.pdf} pour
tout savoir sur les possibilités non évoquées ici. Cette classe est
développée activement, consultez son site
(\url{http://latex-beamer.sourceforge.net/})
pour plus d'informations récentes.



% Local Variables:
% TeX-master: "lshort2e"
% mode: latex
% mode: flyspell
% End:

%%%%%%%%%%%%%%%%%%%%%%%%%%%%%%%%%%%%%%%%%%%%%%%%%%%%%%%%%%%%%%%%%
%%%%%%%%%%%%%%%%%%%%%%%%%%%%%%%%%%%%%%%%%%%%%%%%%%%%%%%%%%%%%%%%%
\setcounter{chapter}{4}
\newcommand{\graphicscompanion}{\emph{The \LaTeX{} Graphics Companion}~\cite{graphicscompanion}} 
\newcommand{\hobby}{\emph{A User's Manual for MetaPost}~\cite{metapost}}
\newcommand{\hoenig}{\emph{\TeX{} Unbound}~\cite{unbound}}
\newcommand{\graphicsinlatex}{\emph{Graphics in \LaTeXe{}}~\cite{ursoswald}}

\chapter{Produire des graphiques math�matiques}

\begin{intro}
De nombreuses personnes utilisent \LaTeX{} pour mettre en page leur
texte. Mais si l'approche \og orient�e structure \fg{} est si
commode, \LaTeX{} fournit �galement des possibilit�s de production
graphique sur la base de descriptions textuelles quelque peu
limit�es. De nombreuses extensions ont �t� cr��es pour surmonter ces
limitations. Vous d�couvrirez quelques-une d'entre elles dans cette
section.
\end{intro}

\section{Vue d'ensemble}

L'environnement \ei{picture} permet la programmation directe d'images
en \LaTeX. Vous pourrez en trouver une description d�taill�e dans le
\manual. D'un c�t�, cet environnement a des contraintes s�rieuses,
comme les pentes des segments ou les rayons des cercles restreints �
un choix limit� de valeurs. D'un autre c�t�, l'environnement
\ei{picture} de \LaTeXe{} am�ne la commande \ci{qbezier}, avec \og
\texttt{q} \fg{} pour \og quadratique \fg{}. Des courbes fr�quemment
utilis�es comme des cercles, des ellipses ou des cat�naires peuvent
�tre approch�s de mani�re satisfaisantes par des courbes de B\'ezier,
bien que cela requiert quelques efforts math�matiques. Si en plus un
langage de programmation comme Java est utilis� pour g�n�rer des blocs
\ci{qbezier} en fichiers d'entr�e, l'environnement \ei{picture}
devient soudain plus puissant.

Bien que programmer des images directement en \LaTeX{} soit limit� et
souvent fatiguant, il reste des raisons pour continuer � le faire : en
effet les documents produits ainsi sont de petite taille (en octets)
et il n'y a pas de fichier graphique � adjoindre.

Des environnements comme \pai{epic} et \pai{eepic} (d�crits dans
\companion, par exemple), ou \pai{pstricks} aident � �liminer les
restrictions qui entravent l'environnement \ei{picture} originel. Ils
renforcent ainsi les capacit�s graphiques de \LaTeX.

Alors que les deux premi�res extensions pr�cit�es se contentent
d'am�liorer \ei{picture}, l'environnement \pai{pstricks} fournit son
propre environnement de dessin, \ei{pspicture}. La puissance de
\pai{pstricks} provient de son usage intensif des capacit�s de
\PSi{}. De plus, de nombreux environnements ont �t�
�crits avec des objectifs pr�cis en t�te. L'un d'entre eux est
\texorpdfstring{\Xy}{Xy}-pic que nous pr�sentons � la fin de ce
chapitre. \graphicscompanion{} d�crit en d�tail une grande vari�t� de
ces environnements (� ne pas confondre avec \companion).

Le plus puissant outil graphique li� � \LaTeX{} est \texttt{MetaPost},
le fr�re jumeau de \texttt{METAFONT} par Donald
E. Knuth. \texttt{MetaPost} poss�de le langage math�matique puissant
et sophistiqu� de \texttt{METAFONT}. � la diff�rence de
\texttt{METAFONT} qui g�n�re des bitmaps, \texttt{MetaPost} g�n�re du
\PSi{} encapsul�, qui peut �tre import� dans \LaTeX. Pour
une pr�sentation, voyez \hobby ou le tutoriel de \cite{ursoswald}.

Vous pourrez trouver une discussion tr�s d�taill� des strat�gies
\LaTeX{} et \TeX{} pour les images (et les polices) dans \hoenig.


\section{L'extension \texttt{picture}}
\secby{Urs Oswald}{osurs@bluewin.ch}

\subsection{Commandes de base}

L'une des deux commandes
\footnote{Croyez-le ou non, l'environnement picture fonctionne
  directement, sans avoir � charger quelque extensions \LaTeXe{}
  suppl�mentaire que ce soit.}
suivantes cr�e un environnement \ei{picture}~:
\begin{lscommand}
\ci{begin}\verb|{picture}(|$x,y$\verb|)|\ldots\ci{end}\verb|{picture}|
\end{lscommand}
\noindent ou
\begin{lscommand}
\ci{begin}\verb|{picture}(|$x,y$\verb|)(|$x_0,y_0$\verb|)|\ldots\ci{end}\verb|{picture}|
\end{lscommand}
Les nombres $x,\,y,\,x_0,\,y_0$ se rapportent � une unit� de longueur
\ci{unitlength} qui peut �tre r�initialis�e � tout moment (sauf �
l'int�rieur d'un environnement \ei{picture}) avec une commande telle
que
\begin{lscommand}
\ci{setlength}\verb|{|\ci{unitlength}\verb|}{1.2cm}|
\end{lscommand}
La valeur par d�faut d'\ci{unitlength} est \texttt{1pt}. La premi�re
paire $(x,y)$ r�serve un espace rectangulaire pour l'image dans le
document. La deuxi�me paire optionnelle associe des coordonn�es
arbitraires au coin inf�rieur gauche du rectangle r�serv�.

La plupart des commandes de dessin suivent l'une des deux formes
suivantes :
\begin{lscommand}
\ci{put}\verb|(|$x,y$\verb|){|\emph{objet}\verb|}|
\end{lscommand}
\noindent ou
\begin{lscommand}
\ci{multiput}\verb|(|$x,y$\verb|)(|$\Delta x,\Delta y$\verb|){|$n$\verb|}{|\emph{objet}\verb|}|
\end{lscommand}
Les courbes de B\'ezier font exception : elles sont dessin�es avec la
commande
\begin{lscommand}
\ci{qbezier}\verb|(|$x_1,y_1$\verb|)(|$x_2,y_2$\verb|)(|$x_3,y_3$\verb|)|
\end{lscommand}

\subsection{Segments}

\begin{example}
\setlength{\unitlength}{5cm}
\begin{picture}(1,1)
  \put(0,0){\line(0,1){1}}
  \put(0,0){\line(1,0){1}}  
  \put(0,0){\line(1,1){1}}  
  \put(0,0){\line(1,2){.5}}
  \put(0,0){\line(1,3){.3333}}
  \put(0,0){\line(1,4){.25}}  
  \put(0,0){\line(1,5){.2}}
  \put(0,0){\line(1,6){.1667}}
  \put(0,0){\line(2,1){1}}
  \put(0,0){\line(2,3){.6667}}
  \put(0,0){\line(2,5){.4}}
  \put(0,0){\line(3,1){1}}  
  \put(0,0){\line(3,2){1}}
  \put(0,0){\line(3,4){.75}}
  \put(0,0){\line(3,5){.6}}
  \put(0,0){\line(4,1){1}}
  \put(0,0){\line(4,3){1}}  
  \put(0,0){\line(4,5){.8}}
  \put(0,0){\line(5,1){1}}
  \put(0,0){\line(5,2){1}}
  \put(0,0){\line(5,3){1}}
  \put(0,0){\line(5,4){1}}
  \put(0,0){\line(5,6){.8333}}
  \put(0,0){\line(6,1){1}}
  \put(0,0){\line(6,5){1}}
\end{picture}
\end{example}
Les segments sont dessin�s via
\begin{lscommand}
\ci{put}\verb|(|$x,y$\verb|){|\ci{line}\verb|(|$x_1,y_1$\verb|){|$longueur$\verb|}}|
\end{lscommand}
La commande \ci{line} a deux arguments :
\begin{enumerate}
  \item Un vecteur de direction ;
  \item Une longueur.
\end{enumerate}
Les composants du vecteur de direction sont restreints aux entiers
\[
  -6,\,-5,\,\ldots,\,5,\,6,
\]
et doivent �tre premiers entre eux (pas de diviseur en commun sauf
1). La figure ci-dessus illustre les 25 valeurs possibles de pentes
pour le premier quadrant. La longueur est relative �
\ci{unitlength}. L'argument de longueur est la coordonn�e verticale
dans le cas d'un segment vertical, horizontale dans tous les autres
cas.

\subsection{Fl�ches}

\begin{example}
\setlength{\unitlength}{1mm}
\begin{picture}(60,40)
  \put(30,20){\vector(1,0){30}}
  \put(30,20){\vector(4,1){20}}
  \put(30,20){\vector(3,1){25}}
  \put(30,20){\vector(2,1){30}}
  \put(30,20){\vector(1,2){10}}
  \thicklines
  \put(30,20){\vector(-4,1){30}}
  \put(30,20){\vector(-1,4){5}}
  \thinlines
  \put(30,20){\vector(-1,-1){5}}
  \put(30,20){\vector(-1,-4){5}}
\end{picture}
\end{example}
Les fl�ches sont dessin�es via
\begin{lscommand}
\ci{put}\verb|(|$x,y$\verb|){|\ci{vector}\verb|(|$x_1,y_1$\verb|){|$longueur$\verb|}}|
\end{lscommand}
Pour les fl�ches, les composants du vecteur de direction sont encore
plus restreints que pour les segments, aux entiers
\[
  -4,\,-3,\,\ldots,\,3,\,4.
\]
qui doivent aussi �tre premiers entre eux. Remarquez l'effet de la
commande \ci{thicklines} sur les fl�ches qui pointent vers le coin
sup�rieur gauche.

\subsection{Cercles}

\begin{example}
\setlength{\unitlength}{1mm}
\begin{picture}(60, 40)
  \put(20,30){\circle{1}}
  \put(20,30){\circle{2}}
  \put(20,30){\circle{4}}
  \put(20,30){\circle{8}}
  \put(20,30){\circle{16}}
  \put(20,30){\circle{32}}
  
  \put(40,30){\circle{1}}
  \put(40,30){\circle{2}}
  \put(40,30){\circle{3}}
  \put(40,30){\circle{4}}
  \put(40,30){\circle{5}}
  \put(40,30){\circle{6}}
  \put(40,30){\circle{7}}
  \put(40,30){\circle{8}}
  \put(40,30){\circle{9}}
  \put(40,30){\circle{10}}
  \put(40,30){\circle{11}}
  \put(40,30){\circle{12}}
  \put(40,30){\circle{13}}
  \put(40,30){\circle{14}}
  
  \put(15,10){\circle*{1}}
  \put(20,10){\circle*{2}}
  \put(25,10){\circle*{3}}
  \put(30,10){\circle*{4}}
  \put(35,10){\circle*{5}}
\end{picture}
\end{example}
La commande
\begin{lscommand}
  \ci{put}\verb|(|$x,y$\verb|){|\ci{circle}\verb|{|\emph{diam�tre}\verb|}}|
\end{lscommand}
\noindent dessine un cercle de centre $(x,y)$ et de diam�tre
\emph{diam�tre} (pas le rayon). L'environnement \ei{picture} admet
seulement des cercles de moins de 14\,mm, et m�me en-dessous de cette
limite tous les diam�tres ne sont pas autoris�s. La commande
\ci{circle*} produit quant � elle des disques (i.e. des cercles
remplis).

Comme pour les segments, il peut devenir n�cessaire de recourir � des
extensions suppl�mentaires comme \pai{eepic} ou \pai{pstricks}. Pour
une description d�taill�e de ces extensions, voyez \graphicscompanion.

Il existe aussi une possibilit� dans l'environnement \ei{picture}. �
condition de ne pas �tre effray� par les calculs requis (ou en les
laissant � un programme d�di�), des cercles et des ellipses de tailles
arbitraires peuvent �tre construits � base de courbes de B\'ezier,
quadratiques. Voyez \graphicsinlatex{} pour des exemples et des
fichiers source Java.


\subsection{Texte et formules}

\begin{example}
\setlength{\unitlength}{1cm}
\begin{picture}(6,5)
  \thicklines
  \put(1,0.5){\line(2,1){3}}
  \put(4,2){\line(-2,1){2}}
  \put(2,3){\line(-2,-5){1}}
  \put(0.7,0.3){$A$}
  \put(4.05,1.9){$B$}
  \put(1.7,2.95){$C$}
  \put(3.1,2.5){$a$}
  \put(1.3,1.7){$b$}
  \put(2.5,1.05){$c$}
  \put(0.3,4){$F=
    \sqrt{s(s-a)(s-b)(s-c)}$}  
  \put(3.5,0.4){$\displaystyle
    s:=\frac{a+b+c}{2}$}
\end{picture}
\end{example}
Comme le montre l'exemple ci-dessus, la commande \ci{put} permet
d'ins�rer du texte et des formules dans un environnement \ei{picture}
comme � l'accoutum�e.

\subsection{Les commandes \ci{multiput} et \ci{linethickness}}

\begin{example}
\setlength{\unitlength}{2mm}
\begin{picture}(30,20)
  \linethickness{0.075mm}
  \multiput(0,0)(1,0){31}%
    {\line(0,1){20}}
  \multiput(0,0)(0,1){21}%
    {\line(1,0){30}}
  \linethickness{0.15mm}    
  \multiput(0,0)(5,0){7}%
    {\line(0,1){20}}
  \multiput(0,0)(0,5){5}%
    {\line(1,0){30}}
  \linethickness{0.3mm}    
  \multiput(5,0)(10,0){3}%
    {\line(0,1){20}}
  \multiput(0,5)(0,10){2}%
    {\line(1,0){30}}
\end{picture}
\end{example}
La commande
\begin{lscommand}
  \ci{multiput}\verb|(|$x,y$\verb|)(|$\Delta x,\Delta y$\verb|){|$n$\verb|}{|\emph{objet}\verb|}|
\end{lscommand}
\noindent poss�de 4 param�tres : le point de d�part, le vecteur de
translation d'un objet � l'autre, le nombre d'objets et l'objet �
dessiner. La commande \ci{linethickness} s'applique aux segments
horizontaux et verticaux mais pas aux segments obliques ni aux
cercles. Elle s'applique cependant aux courbes de B\'ezier.

\subsection{Ovales. Les commandes \ci{thinlines} et \ci{thicklines}}

\begin{example}
\setlength{\unitlength}{1cm}
\begin{picture}(6,4)
  \linethickness{0.075mm}
  \multiput(0,0)(1,0){7}%
    {\line(0,1){4}}
  \multiput(0,0)(0,1){5}%
    {\line(1,0){6}}
  \thicklines
  \put(2,3){\oval(3,1.8)} 
  \thinlines
  \put(3,2){\oval(3,1.8)} 
  \thicklines
  \put(2,1){\oval(3,1.8)[tl]} 
  \put(4,1){\oval(3,1.8)[b]} 
  \put(4,3){\oval(3,1.8)[r]} 
  \put(3,1.5){\oval(1.8,0.4)}     
\end{picture}
\end{example}
La commande
\begin{lscommand}
  \ci{put}\verb|(|$x,y$\verb|){|\ci{oval}\verb|(|$w,h$\verb|)}|
\end{lscommand}
\noindent or
\begin{lscommand}
  \ci{put}\verb|(|$x,y$\verb|){|\ci{oval}\verb|(|$w,h$\verb|)[|\emph{position}\verb|]}|
\end{lscommand}
\noindent produit un ovale centr� en $(x,y)$, de largeur $w$ et
d'hauteur $h$. Les param�tres optionnels de \emph{position}
\texttt{b}, \texttt{t}, \texttt{l} et \texttt{r} se rapportent � \og
haut \fg{}, \og bas \fg{},\og gauche \fg{} et \og droite \fg{}
respectivement. Ils peuvent �tre combin�s comme l'illustre l'exemple
plus haut.

L'�paisseur de trait peut �tre contr�l�es par deux sortes de
commandes : \\
\ci{linethickness}\verb|{|\emph{longueur}\verb|}|
d'un c�t�, \ci{thinlines} et \ci{thicklines} de l'autre. Alors que
\ci{linethickness}\verb|{|\emph{longueur}\verb|}| ne s'applique qu'aux
lignes horizontales, verticales et aux courbes de B\'ezier,
\ci{thinlines} et \ci{thicklines} s'appliquent aux segments obliques
ainsi qu'aux cercles et aux ovales.


\subsection{Usage multiple d'images pr�d�finies}

\begin{example}
\setlength{\unitlength}{0.5mm}
\begin{picture}(120,168)
\newsavebox{\foldera}% declaration
\savebox{\foldera}
  (40,32)[bl]{% definition 
  \multiput(0,0)(0,28){2}
    {\line(1,0){40}}
  \multiput(0,0)(40,0){2}
    {\line(0,1){28}}
  \put(1,28){\oval(2,2)[tl]}
  \put(1,29){\line(1,0){5}}
  \put(9,29){\oval(6,6)[tl]}
  \put(9,32){\line(1,0){8}}
  \put(17,29){\oval(6,6)[tr]}
  \put(20,29){\line(1,0){19}}
  \put(39,28){\oval(2,2)[tr]}  
}
\newsavebox{\folderb}% declaration
\savebox{\folderb}
  (40,32)[l]{%         definition 
  \put(0,14){\line(1,0){8}}
  \put(8,0){\usebox{\foldera}}
}
\put(34,26){\line(0,1){102}} 
\put(14,128){\usebox{\foldera}}
\multiput(34,86)(0,-37){3}
  {\usebox{\folderb}} 
\end{picture}
\end{example}
Une bo�te d'image peut �tre \emph{d�clar�e} par la commande
\begin{lscommand}
  \ci{newsavebox}\verb|{|\emph{nom}\verb|}|
\end{lscommand}
\noindent puis \emph{d�finie} par
\begin{lscommand}
  \ci{savebox}\verb|{|\emph{nom}\verb|}(|\emph{largeur,hauteur}\verb|)[|\emph{position}\verb|]{|\emph{contenu}\verb|}|
\end{lscommand}
\noindent et enfin dessin�e arbitrairement souvent via
\begin{lscommand}
  \ci{put}\verb|(|$x,y$\verb|)|\ci{usebox}\verb|{|\emph{nom}\verb|}|
\end{lscommand}

Le param�tre optionnel de \emph{position} a pour effet de d�finir
le \og point d'ancrage \fg{} de l'image embo�t�e. Dans l'exemple il
est d�fini comme \texttt{bl} qui met le point d'ancrage dans le coin
inf�rieur gauche de l'image embo�t�e. Les autres positions possibles
sont \texttt{t}op (en haut) et \texttt{r}ight (� droite).

Le param�tre \emph{nom} se rapporte � un registre de stockage
\LaTeX{}: il s'agit donc d'une commande (ce qui explique l'usage
d'antislashs dans l'exemple). Les images embo�t�es peuvent �tre
imbriqu�es : \ci{foldera} est utilis�e � l'int�rieur de la d�finition
de \ci{folderb} dans l'exemple.

La commande \ci{oval} a d� �tre utilis�e car la commande \ci{line} ne
fonctionne pas avec des segments dont la longueur est inf�rieure �
3\,mm.

\subsection{Courbes de B\'ezier}

\begin{example}
\setlength{\unitlength}{1cm}
\begin{picture}(6,4)
  \linethickness{0.075mm}
  \multiput(0,0)(1,0){7}
    {\line(0,1){4}}
  \multiput(0,0)(0,1){5}
    {\line(1,0){6}}
  \thicklines
  \put(0.5,0.5){\line(1,5){0.5}}    
  \put(1,3){\line(4,1){2}} 
  \qbezier(0.5,0.5)(1,3)(3,3.5)
  \thinlines   
  \put(2.5,2){\line(2,-1){3}}
  \put(5.5,0.5){\line(-1,5){0.5}}
  \linethickness{1mm}
  \qbezier(2.5,2)(5.5,0.5)(5,3)
  \thinlines
  \qbezier(4,2)(4,3)(3,3)
  \qbezier(3,3)(2,3)(2,2)
  \qbezier(2,2)(2,1)(3,1)
  \qbezier(3,1)(4,1)(4,2)
\end{picture}
\end{example}
Cet exemple illustre l'inad�quation du d�coupage d'un cercle en 4
courbes de B\'ezier. Au moins 8 d'entre elles sont requises. La figure
montre � nouveau l'effet de la commande \ci{linethickness} sur les
lignes horizontales et verticales, ainsi que des commandes
\ci{thinlines} et \ci{thicklines} sur les segments obliques. Elle
montre enfin l'effet des deux sortes de commandes sur les courbes de
B\'ezier, chaque commande supplantant toutes les pr�c�dentes.

Soient $P_1=(x_1,\,y_1),\,P_2=(x_2,\,y_2)$ les points de d�but et fin
et $m_1,\,m_2$ leurs pentes respectives d'une courbe de B\'ezier. Le
point de contr�le interm�diaire est donn� par les �quations
suivantes~:
\begin{equation} \label{zwischenpunkt}
  \left\{
    \begin{array}{rcl}
      x & = & \displaystyle \frac{m_2 x_2-m_1x_1-(y_2-y_1)}{m_2-m_1}, \\
      y & = & y_i+m_i(x-x_i)\qquad (i=1,\,2).
    \end{array}
  \right.
\end{equation}
\noindent R�f�rez-vous � \graphicsinlatex{} pour un programme Java qui
g�n�re les lignes de commandes \ci{qbezier} n�cessaires.

\subsection{Cat�naire}

\begin{example}
\setlength{\unitlength}{1.3cm}
\begin{picture}(4.3,3.6)(-2.5,-0.25)
  \put(-2,0){\vector(1,0){4.4}}
  \put(2.45,-.05){$x$}
  \put(0,0){\vector(0,1){3.2}}
  \put(0,3.35){\makebox(0,0){$y$}}
  \qbezier(0.0,0.0)(1.2384,0.0)
    (2.0,2.7622) 
  \qbezier(0.0,0.0)(-1.2384,0.0)
    (-2.0,2.7622)
  \linethickness{.075mm}
  \multiput(-2,0)(1,0){5}
    {\line(0,1){3}}
  \multiput(-2,0)(0,1){4}
    {\line(1,0){4}}
  \linethickness{.2mm}
  \put( .3,.12763){\line(1,0){.4}}
  \put(.5,-.07237){\line(0,1){.4}}
  \put(-.7,.12763){\line(1,0){.4}}
  \put(-.5,-.07237){\line(0,1){.4}}
  \put(.8,.54308){\line(1,0){.4}}
  \put(1,.34308){\line(0,1){.4}}
  \put(-1.2,.54308){\line(1,0){.4}}
  \put(-1,.34308){\line(0,1){.4}}
  \put(1.3,1.35241){\line(1,0){.4}}
  \put(1.5,1.15241){\line(0,1){.4}}
  \put(-1.7,1.35241){\line(1,0){.4}}
  \put(-1.5,1.15241){\line(0,1){.4}}
  \put(-2.5,-0.25){\circle*{0.2}}
\end{picture}
\end{example}

Dans cette figure, chaque moiti� symm�trique du cat�naire $y=\cosh x
-1$ est approch�e par une courbe de B\'ezier. La moiti� droite
s'ach�ve au point \((2,\,2.7622)\) avec une pente \(m=3.6269\). �
l'aide de l'�quation (\ref{zwischenpunkt}) nous pouvons calculer les
points de contr�le interm�diaires. Ils s'av�rent �tre $(1.2384,\,0)$
et $(-1.2384,\,0)$. Les croix indiquent les points du cat�naire
\emph{r�el}. L'erreur est n�gligeable, de moins d'un pourcent.

Cet exemple signale l'usage du param�tre optionel de la commande
\verb|\begin{picture}|. L'image est d�finie selon des coordonn�es \og
  math�matiques \fg{} commodes, tandis que par la commande
\begin{lscommand} 
  \ci{begin}\verb|{picture}(4.3,3.6)(-2.5,-0.25)|
\end{lscommand}
\noindent son coin inf�rieur gauche (marqu� par une disque noir) est
associ� aux coordonn�es $(-2.5,-0.25)$.

%SC: not a physicist here, hope I got this one right :-P
\subsection{La rapidit� dans la th�orie de la relativit� restreinte}

\begin{example}
\setlength{\unitlength}{1cm}
\begin{picture}(6,4)(-3,-2)
  \put(-2.5,0){\vector(1,0){5}}
  \put(2.7,-0.1){$\chi$}
  \put(0,-1.5){\vector(0,1){3}}
  \multiput(-2.5,1)(0.4,0){13}
    {\line(1,0){0.2}}
  \multiput(-2.5,-1)(0.4,0){13}
    {\line(1,0){0.2}}
  \put(0.2,1.4)
    {$\beta=v/c=\tanh\chi$}
  \qbezier(0,0)(0.8853,0.8853)
    (2,0.9640)
  \qbezier(0,0)(-0.8853,-0.8853)
    (-2,-0.9640)
  \put(-3,-2){\circle*{0.2}}
\end{picture}
\end{example}
Les points de contr�le des deux courbes de B\'ezier ont �t� calcul�es
gr�ce aux formules (\ref{zwischenpunkt}). La branche positive est
d�termin�e par $P_1=(0,\,0),\,m_1=1$ and $P_2=(2,\,\tanh
2),\,m_2=1/\cosh^2 2$. L'image est � nouveau d�finie selon des
coordonn�es math�matiques commodes et le coin inf�rieur gauche est
associ� aux coordonn�es math�matiques $(-3,-2)$ (le disque noir).

\section{\texorpdfstring{\Xy}{Xy}-pic}
\secby{Alberto Manuel Brand\~ao Sim\~oes}{albie@alfarrabio.di.uminho.pt}
\pai{xy} est une extension d�di�e au dessin de diagrammes. Elle est
invoqu�e par l'ajout dans votre pr�ambule de la ligne :
\begin{lscommand}
\verb|\usepackage[|\emph{options}\verb|]{xy}|
\end{lscommand}
\emph{options} est une liste des fonctions que vous voulez voir
\Xy-pic charger. Ces options sont surtout utiles pour d�boguer
l'extension. Je vous sugg�re d'utiliser l'option \verb!all! afin de
faire charger toutes les commandes \Xy{} par \LaTeX{}.

Les diagrammes \Xy-pic sont dessin�s sur une trame bas�e sur un
matrice, o� chaque �l�ment de dessin correspond � un �l�ment de la
matrice :
\begin{example}
\begin{displaymath}
\xymatrix{A & B \\
          C & D }
\end{displaymath}
\end{example}
La commande \ci{xymatrix} requiert le mode math�matique. Ici nous
avons sp�cifi� deux lignes et deux colonnes. Pour faire de cette
matrice un diagramme, nous ajoutons des fl�ches via la commande
\ci{ar}.
\begin{example}
\begin{displaymath}
\xymatrix{ A \ar[r] & B \ar[d] \\
           D \ar[u] & C \ar[l] }
\end{displaymath}
\end{example}
La commande de fl�che est plac�e sur la cellule d'origine de la
fl�che. Les arguments correspondent � sa direction (\texttt{u}p vers
le haut, \texttt{d}own vers le bas, \texttt{r}ight vers la droite et
\texttt{l}eft vers la gauche).


\begin{example}
\begin{displaymath}
\xymatrix{
  A \ar[d] \ar[dr] \ar[r] & B \\
  D                       & C }
\end{displaymath}
\end{example}
Les fl�ches en diagonales sont obtenus en utilisant plus d'une
direction. Vous pouvez aussi r�p�ter les directions pour avoir de plus
grandes fl�ches.
\begin{example}
\begin{displaymath}
\xymatrix{
  A \ar[d] \ar[dr] \ar[drr] & & \\
  B                      & C & D }
\end{displaymath}
\end{example}

Nous pouvons dessiner des diagrammes plus int�ressants en ajoutant des
�tiquettes aux fl�ches gr�ce aux op�rateurs d'indice et d'exposant.
\begin{example}
\begin{displaymath}
\xymatrix{
  A \ar[r]^f \ar[d]_g &
             B \ar[d]^{g'} \\
  D \ar[r]_{f'}       & C }
\end{displaymath}
\end{example}

Comme l'indique l'exemple, vous utilisez ces op�rateurs en mode
math�matique. La seule diff�rence est que l'exposant signifie \og
au-dessus de la fl�che \fg{} et l'indice \og sous la fl�che \fg{}. il
existe un troisi�me op�rateur, la barre verticale: \verb+|+. Elle fait
se placer le texte \emph{dans} la fl�che.
\begin{example}
\begin{displaymath}
\xymatrix{
  A \ar[r]|f \ar[d]|g &
             B \ar[d]|{g'} \\
  D \ar[r]|{f'}       & C }
\end{displaymath}
\end{example}

Pour dessiner une fl�che trou�e, utilisez \verb!\ar[...]|\hole!.

Dans certaines circonstances, il peut �tre important de distinguer les
diff�rents types de fl�ches. Ceci peut �tre obtenu en leur assignant
des �tiquettes ou en changeant leur apparence :

\begin{example}
\shorthandoff{"}
\begin{displaymath}
\xymatrix{
 \bullet\ar@{->}[rr] && \bullet\\
 \bullet\ar@{.<}[rr] && \bullet\\
 \bullet\ar@{~)}[rr] && \bullet\\
 \bullet\ar@{=(}[rr] && \bullet\\
 \bullet\ar@{~/}[rr]  && \bullet\\
 \bullet\ar@{^{(}->}[rr]  && \bullet\\
 \bullet\ar@2{->}[rr]  && \bullet\\
 \bullet\ar@3{->}[rr]  && \bullet\\
 \bullet\ar@{=+}[rr]   && \bullet
}
\end{displaymath}
\shorthandon{"}
\end{example}

Notez la diff�rence entre les deux diagrammes suivants:

\begin{example}
\begin{displaymath}
\xymatrix{
 \bullet \ar[r] 
         \ar@{.>}[r] & 
 \bullet
}
\end{displaymath}
\end{example}

\begin{example}
\begin{displaymath}
\xymatrix{
 \bullet \ar@/^/[r] 
         \ar@/_/@{.>}[r] &
 \bullet
}
\end{displaymath}
\end{example}

Les modifieurs entre les slashs d�finissent la mani�re dont les
courbes sont dessin�es. \Xy-pic offre de nombreuses fa�ons
d'influencer le dessin des courbes. Pour plus d'informations, veuillez
consultez la documentation d'\Xy-pic.

% \begin{example}
% \begin{lscommand}
% \ci{dum}
% \end{lscommand}
% \end{example}


%%%%%%%%%%%%%%%%%%%%%%%%%%%%%%%%%%%%%%%%%%%%%%%%%%%%%%%%%%%%%%%%%
% Contents: Customising LaTeX output
% $Id: custom.tex 485 2011-11-24 13:03:07Z oetiker $
%%%%%%%%%%%%%%%%%%%%%%%%%%%%%%%%%%%%%%%%%%%%%%%%%%%%%%%%%%%%%%%%%

% Pour les informations de licence, voir contrib.tex.
% See contrib.tex for license information.

\chapter{Personnalisation de \LaTeX}

\begin{intro}
Les documents produits avec les commandes que vous avez apprises
jusqu'ici sont de tr�s bonne qualit� aux yeux d'un vaste public. M�me s'ils
manquent de fantaisie, ils ob�issent � toutes les r�gles de l'art de
la typographie, ce qui les rend agr�ables � lire.

Mais il y a des situations o� \LaTeX{} ne propose pas de commande ou
d'environnement adapt� � vos besoins, ou bien o� le r�sultat obtenu par
une commande existante peut ne pas r�pondre  � votre attente.

Dans ce chapitre, vous allez avoir un aper�u sur la mani�re
d'enrichir les commandes de \LaTeX{} et de modifier la pr�sentation
par d�faut.
\end{intro}

\section{Vos propres commandes, environnements et extensions}

Vous avez s�rement constat� que toutes les commandes d�crites dans cet
ouvrage sont pr�sent�es dans un cadre et sont r�f�renc�es dans l'index
qui se trouve � la fin. Au lieu d'utiliser � chaque fois l'ensemble
des commandes \LaTeX{} n�cessaires, nous avons cr�� une extension dans
laquelle nous avons d�fini de nouvelles commandes et de nouveaux
environnement adapt�s � cet usage. Ainsi nous pouvons simplement
�crire :

\begin{example}
\begin{lscommand}
\ci{dum}
\end{lscommand}
\end{example}

Dans cet exemple, nous utilisons � la fois un nouvel environnement
appel� \ei{lscommand} qui est responsable du trac� du cadre et une
nouvelle commande appel�e \ci{ci} qui compose le nom de la commande et
qui produit l'entr�e correspondante dans l'index. Vous pouvez le
v�rifier en cherchant la commande \ci{dum} dans l'index � la fin de ce 
document ; vous
y trouverez une entr�e pointant vers chaque page o� la commande dum
est mentionn�e.

Si nous d�cidons un jour que nous ne souhaitons plus voir de cadre
autour du nom des commandes, il nous suffira de modifier la d�finition
de l'environnement \ei{lscommand} pour d�finir un nouveau style.  C'est
bien plus simple (et efficace) que de parcourir tout le document pour
remplacer une � une toutes les commandes qui tracent les cadres.

% Dans le premier chapitre, nous avons vu que \LaTeX{} avait besoin de
% conna�tre la structure logique d'un document pour pouvoir le formater
% correctement. Ceci est une id�e int�ressante, mais en pratique on
% atteint assez vite ses limites quand on ne trouve pas dans \LaTeX{} la
% commande sp�cialis�e pour exprimer ce que l'on voudrait.
% 
% Une solution serait de combiner plusieurs commandes \LaTeX{} pour
% obtenir le r�sultat d�sir�. Ponctuellement, cela ne pose pas de
% probl�me ; mais si cela se reproduit, r�p�ter chaque fois la m�me
% combinaison peut devenir long et fastidieux. De plus, si vous souhaitez
% plus tard modifier la pr�sentation, il faudra aller modifier chaque
% combinaison de commandes tout au long du texte.
% 
% Pour r�soudre ce probl�me, \LaTeX{} vous permet de d�finir vos propres
% commandes et vos propres environnements.

\subsection{Nouvelles commandes}

Pour ajouter de nouvelles commandes, utilisez la commande :
\begin{lscommand}
\ci{newcommand}\verb|{|%
       \emph{nom}\verb|}[|\emph{num}\verb|]{|\emph{d�finition}\verb|}|
\end{lscommand}
\noindent Cette commande prend principalement deux
arguments : le \emph{nom} de la commande � cr�er et sa
\emph{d�finition}. L'argument \emph{num} entre crochets est
optionnel. Il indique le nombre de param�tres  qu'utilisera la
nouvelle commande (jusqu'� 9).

Les deux exemples ci-dessous vous aiderons � saisir le principe.
Le premier exemple d�finit une nouvelle commande appel�e \ci{ucil}
qui est une abr�viation de \og une courte introduction �
\LaTeXe \fg{}. Une telle commande pourrait �tre utile si vous aviez �
citer de nombreuses fois le titre de ce livre.

\begin{example}
\newcommand{\ucil}
    {Une courte (?) 
     introduction � \LaTeXe}
% dans le document :
Voici \og \ucil \fg\dots
\end{example}

L'exemple suivant montre comment utiliser l'argument \emph{num}. La
balise \verb|#1| est remplac�e par le param�tre r�el.
Pour utiliser plus d'un param�tre, continuez avec \verb|#2|, etc.

\begin{example}
\newcommand{\uxil}[2]
    {Une \emph{#1}  
     introduction � #2}
% dans le document :
\begin{itemize}
  \item \uxil{courte}{\LaTeXe}
  \item \uxil{longue}{Word}
\end{itemize}
\end{example}

\LaTeX{} ne vous permet pas de cr�er une nouvelle commande si celle-ci
existe d�j�. Si vous voulez explicitement remplacer une commande
existante, utilisez \ci{renewcommand}. Elle utilise la m�me syntaxe
que \verb|\newcommand|. 

Dans certains cas, vous pouvez avoir besoin de
\ci{providecommand}. Cette commande fonctionne comme
\verb|\newcommand|, mais si la nouvelle commande est d�j� d�finie,
\LaTeXe{} ignore  la nouvelle d�finition.

\LaTeX{} supprime en g�n�ral les espaces qui suivent une commande
(voir page~\pageref{whitespace}).

\subsection{Nouveaux environnements}

De mani�re analogue � la commande \verb|\newcommand|, il est possible
de d�finir de nouveaux environnements.  La commande
\ci{newenvironment} se pr�sente de la mani�re suivante :

\begin{lscommand}
\ci{newenvironment}\verb|{|%
       \emph{nom}\verb|}[|\emph{num}\verb|]{|%
       \emph{avant}\verb|}{|\emph{apr�s}\verb|}|
\end{lscommand}

Il est ici aussi possible d'utiliser
\ci{newenvironment} avec un param�tre optionnel. Le contenu de
l'argument \emph{avant} est ex�cut� avant que le contenu de
l'environnement ne soit trait�. Le contenu de l'argument \emph{apr�s}
est trait� lorsque l'on rencontre la commande
\verb|\end{|\emph{nom}\verb|}|.

L'exemple ci-dessous illustre l'utilisation de \verb|\newenvironment|.

\begin{example}
\newenvironment{king}
 {\rule{1ex}{1ex}%
      \hspace{\stretch{1}}}
 {\hspace{\stretch{1}}%
      \rule{1ex}{1ex}}

\begin{king} 
Mes chers sujets, \dots
\end{king}
\end{example}

L'argument \emph{num} est utilis� de la m�me fa�on que pour la
commande \verb|\newcommand|. \LaTeX{} vous emp�che de red�finir un
environnement qui existe d�j�. Si jamais vous vouliez red�finir un
environnement existant, utilisez \ci{renewenvironment} qui utilise la
m�me syntaxe que \verb|\newenvironment|.

Les commandes utilis�es dans l'exemple ci-dessus seront pr�sent�es
plus loin. Pour la commande \ci{rule}, voir page~\pageref{sec:rule},
pour \ci{stretch}, voir page~\pageref{cmd:stretch} enfin, pour plus
d'informations sur \ci{hspace}, voir page~\pageref{sec:hspace}.

\subsection{Espaces surnum�raires}

La cr�ation d'un nouvel environnement est souvent accompagn�e du
probl�me r�current d'espaces surnum�raires qui peuvent avoir des
effets ind�sirables. Par exemple lorsque vous voulez cr�er un
environnement de titre qui supprime sa propre indentation ainsi que
celle du paragraphe qui le suit imm�diatement. La commande
\ci{ignorespaces} dans le bloc de d�but de l'environnement lui fera
ignorer tout espace situ� apr�s l'ex�cution du bloc de d�but. Le bloc
de fin est plus d�licat car un processus sp�cifique se d�roule � la
fin d'un environnement. Avec la commande \ci{ignorespacesafterend}
\LaTeX{} ajoutera un \ci{ignorespaces} qui ignorera les espaces apr�s
que le processus sp�cifique se soit d�roul�.

\begin{example}
\newenvironment{simple}%
 {\noindent}%
 {\par\noindent}

\begin{simple}
Voyez l'espace\\� gauche.
\end{simple}
Ici\\aussi.
\end{example}

\begin{example}
\newenvironment{correct}%
 {\noindent\ignorespaces}%
 {\par\noindent%
   \ignorespacesafterend}

\begin{correct}
Plus d'espace\\� gauche.
\end{correct}
Ici\\non plus.
\end{example}

\subsection{\LaTeX{} en ligne de commande}

Si vous travaillez sur un syst�me de type Unix, il est probable que
vous utilisiez des Makefiles pour compiler vos projets \LaTeX{}. Dans
ce contexte il pourrait �tre int�ressant de pouvoir produire des
versions sensiblement diff�rentes du m�me document par le simple
changement de param�tres de ligne de commande lors de l'invocation de
\LaTeX{}. Si vous ajoutez la structure suivante � votre document :

\begin{verbatim}
\usepackage{ifthen}
\ifthenelse{\equal{\noiretblanc}{true}}{
  % mode noir et blanc; faire ceci..
}{
  % mode couleur; faire cela..
}
\end{verbatim}

Alors vous pouvez invoquer \LaTeX{} de la mani�re suivante:
\begin{verbatim}
latex '\newcommand{\noiretblanc}{true}\input{document.tex}'
\end{verbatim}

D'abord la commande \verb|\noiretblanc| est d�finie, puis le document
est effectivement lu. En modifiant \verb|\noiretblanc| �
\texttt{false} ce serait la version couleur du document qui serait
produite.

\subsection{Votre propre extension}

Si vous d�finissez beaucoup de nouveaux environnements et de nouvelles
commandes, le pr�ambule de votre document va se rallonger
dangereusement. Il peut alors devenir int�ressant de cr�er une
extension contenant toutes ces nouvelles d�finitions. Avec 
la commande \ci{usepackage} vous pourrez rendre disponible votre
extension dans votre document.

\begin{figure}[!htbp]
\begin{lined}{\textwidth}
\begin{verbatim}
% Exemple d'extension par Tobias Oetiker
\ProvidesPackage{demopack}
\newcommand{\ucil}{Une courte (?) introduction
                   � \LaTeXe}
\newcommand{\uxil}[1]{Une \emph{#1}  
                       introduction � \LaTeXe}
\newenvironment{king}{\begin{quote}}{\end{quote}}
\end{verbatim}
\end{lined}
\caption{Exemple d'extension} \label{package}
\end{figure}

�crire une extension consiste principalement � copier le pr�ambule de
votre document dans un fichier � part, dont le nom se termine par
\texttt{.sty}. Il y a une commande sp�cifique � utiliser sur la
premi�re ligne de votre extension~:

\begin{lscommand}
\ci{ProvidesPackage}\verb|{|\emph{nom de l'extension}\verb|}|
\end{lscommand}

\noindent  \ci{ProvidesPackage} indique � \LaTeXe{} le nom de
l'extension afin notamment de lui permettre de produire des messages
d'erreur significatifs. La figure~\ref{package} montre un exemple
d'extension simple qui reprend les commandes d�finies dans les
exemples pr�c�dents. 

\section{Polices et tailles des caract�res}
\label{sec:fontsize}
\index{police}
\index{taille!des polices}

\subsection{Commandes de changement de police}
\LaTeX{} choisit la police de caract�res et sa taille en fonction de la
structure logique du document (sections, notes de bas de
page, etc. ). Dans certains cas, on voudrait pouvoir changer la taille
de la police � la main. Pour cela, utilisez les commandes list�es dans
les tableaux~\ref{fonts} et~\ref{sizes}. La taille exacte de chaque
police est un choix qui d�pend de la classe de document et de ses
options. La table~\ref{tab:pointsizes} donne les tailles absolues en
points pour les commandes pr�sentes dans les classes de document
standard. 

\begin{example}
{\small Les romains 
petits et \textbf{gras}  
r�gn�rent sur}
{\Large toute la belle 
et grande \textit{Italie}.}
\end{example}

%MPG: ajout� la note juste pour remplir la page...
Une caract�ristique importante de \LaTeXe{} est que les diff�rents
attributs d'une police sont ind�pendants\footnote{Ou presque. Par exemple,
  comme l'italique et les petites capitales rel�vent d'un m�me attribut pour
  \LaTeX, il lui est impossible de concevoir des petites capitales italiques.
  \NdT}. Cela signifie que vous
pouvez ex�cuter des commandes de changement de taille ou m�me de
changement de police tout en conservant l'attribut gras ou
italique. 

% Cela peut para�tre �vident � quelqu'un qui  d�bute avec
% \LaTeXe{}, mais �a ne l'est certainement pas pour quelqu'un qui �tait
% habitu� � \LaTeX{}~2.09.

En mode \emph{math�matique}, vous pouvez utiliser les commandes de
changement de police pour quitter provisoirement le mode math�matique
et saisir du texte normal. Pour changer les attributs de la police en
mode math�matique, il existe un jeu de commandes
sp�ciales. Reportez-vous au tableau~\ref{mathfonts}.


\begin{table}[!bp]
\caption{Polices} \label{fonts}
\begin{lined}{\textwidth}
%
% Alan suggested not to tell about the other form of the command
% eg \verb|\sffamily| or \verb|\bfseries|. This seems a good thing to me.
%
\begin{tabular}{@{}r@{ }l@{\qquad}r@{ }l@{}}
\fni{textrm}\verb|{...}|        &      \textrm{\wi{romain}}&
\fni{texttt}\verb|{...}|        &      \texttt{machine � �crire}\\
\fni{textsf}\verb|{...}|        &      \textsf{\wi{sans serif}}\\[6pt]
\fni{textmd}\verb|{...}|        &      \textmd{moyen}&
\fni{textbf}\verb|{...}|        &      \textbf{\wi{gras}}\\[6pt]
\fni{textup}\verb|{...}|        &      \textup{\wi{droit}}&
\fni{textit}\verb|{...}|        &      \textit{\wi{italique}}\\
\fni{textsl}\verb|{...}|        &      \textsl{\wi{pench�}}&
\fni{textsc}\verb|{...}|        &      \textsc{\wi{Petites Capitales}}\\[6pt]
\ci{emph}\verb|{...}|           &      \emph{en �vidence} &
\fni{textnormal}\verb|{...}|    &      \textnormal{par d�faut}
\end{tabular}

\bigskip
\end{lined}
\end{table}


\begin{table}[!bp]
\index{taille!pr�d�finies}
\caption{Tailles des polices} \label{sizes}
\begin{lined}{12cm}
\begin{tabular}{@{}ll}
\fni{tiny}      & \tiny        minuscule \\
\fni{scriptsize}   & \scriptsize  tr�s petit\\
\fni{footnotesize} & \footnotesize  assez petit \\
\fni{small}        &  \small            petit \\
\fni{normalsize}   &  \normalsize  normal  \\
\fni{large}        &  \large       grand
\end{tabular}%
\qquad\begin{tabular}{ll@{}}
\fni{Large}        &  \Large       plus grand \\[5pt]
\fni{LARGE}        &  \LARGE       tr�s grand \\[5pt]
\fni{huge}         &  \huge        �norme \\[5pt]
\fni{Huge}         &  \Huge        g�ant
\end{tabular}

\bigskip
\end{lined}
\end{table}

\begin{table}[!tbp]
\caption{Tailles en points dans les classes standard}\label{tab:pointsizes}
\label{tab:sizes}
\begin{lined}{12cm}
\begin{tabular}{lrrr}
\multicolumn{1}{c}{taille} &
\multicolumn{1}{c}{10pt (d�faut) } &
           \multicolumn{1}{c}{option 11pt}  &
           \multicolumn{1}{c}{option 12pt}\\
\verb|\tiny|       & 5pt  & 6pt & 6pt\\
\verb|\scriptsize| & 7pt  & 8pt & 8pt\\
\verb|\footnotesize| & 8pt & 9pt & 10pt \\
\verb|\small|        & 9pt & 10pt & 11pt \\
\verb|\normalsize| & 10pt & 11pt & 12pt \\
\verb|\large|      & 12pt & 12pt & 14pt \\
\verb|\Large|      & 14pt & 14pt & 17pt \\
\verb|\LARGE|      & 17pt & 17pt & 20pt\\
\verb|\huge|       & 20pt & 20pt & 25pt\\
\verb|\Huge|       & 25pt & 25pt & 25pt\\
\end{tabular}

\bigskip
\end{lined}
\end{table}

\begin{table}[!htbp]
\caption{Polices math�matiques} \label{mathfonts}
\begin{lined}{0.8\textwidth}
\begin{tabular}{@{}ll@{}}
\fni{mathrm}\verb|{...}|&     $\mathrm{Police\ romaine}$\\
\fni{mathbf}\verb|{...}|&     $\mathbf{Police\ grasse}$\\
\fni{mathsf}\verb|{...}|&     $\mathsf{Police\ sans\ serif}$\\
\fni{mathtt}\verb|{...}|&     $\mathtt{Police\ typewriter}$\\
\fni{mathit}\verb|{...}|&     $\mathit{Police\ italique}$\\
\fni{mathcal}\verb|{...}|&    $\mathcal{Police\ cursive}$\\
\fni{mathnormal}\verb|{...}|& $\mathnormal{Police\ normale}$\\
\end{tabular}

% \begin{tabular}{@{}lll@{}}
% \textit{Commande}&\textit{Exemple}&    \textit{R�sultat}\\[6pt]
% \fni{mathcal}\verb|{...}|&    \verb|$\mathcal{B}=c$|&     $\mathcal{B}=c$\\
% \fni{mathrm}\verb|{...}|&     \verb|$\mathrm{K}_2$|&      $\mathrm{K}_2$\\
% \fni{mathbf}\verb|{...}|&     \verb|$\sum x=\mathbf{v}$|& $\sum x=\mathbf{v}$\\
% \fni{mathsf}\verb|{...}|&     \verb|$\mathsf{G\times R}$|&        $\mathsf{G\times R}$\\
% \fni{mathtt}\verb|{...}|&     \verb|$\mathtt{L}(b,c)$|&   $\mathtt{L}(b,c)$\\
% \fni{mathnormal}\verb|{...}|& \verb|$\mathnormal{R_{19}}| &\\
% & \verb|    \neq R_{19}$|& $\mathnormal{R_{19}}\neq R_{19}$\\
% \fni{mathit}\verb|{...}|&     \verb|$\mathit{ffi}\neq ffi$|& $\mathit{ffi}\neq ffi$
% \end{tabular}

\bigskip
\end{lined}
\end{table}

En compl�ment des commandes de changement de taille, les
\wi{accolade}s jouent un r�le essentiel. Elles sont utilis�es pour
former des \emph{\wi{groupe}s} qui limitent la
port�e de la plupart des commandes de \LaTeX{}.

\begin{example}
Il aime les  {\LARGE grands et  
{\small les petits} 
caract�res}. 
\end{example}

Les commandes de changement de taille modifient �galement
l'interligne, mais seulement si le paragraphe se termine dans la
port�e de la commande de changement de taille. L'accolade fermante
\verb|}| ne
doit donc pas �tre plac�e trop t�t. Remarquez la position de la
commande \verb|\par| dans les deux exemples suivants 
\footnote{\ci{par} est �quivalent � une ligne vide} :

\begin{example}
{\Large Ne lisez pas ceci !
 Ce n'est pas vrai !
 Croyez-moi !\par}
\end{example}
\begin{example}
{\Large Ce n'est pas vrai. Mais
n'oubliez pas que je suis un
menteur.}\par
\end{example}

Si vous voulez utiliser une commande de modification de la taille pour
tout un paragraphe ou m�me plus, vous pouvez utiliser la syntaxe des
environnements � la place des commandes. 

%MPG: ajout de \raggedright:
\begin{example}
\begin{Large} \raggedright
Ceci n'est pas vrai
Mais, qu'est-ce qui l'est
de nos jours\dots\par
\end{Large}
\end{example}

\noindent Cela vous �vitera d'avoir � compter les accolades
fermantes. 

\subsection{Attention danger}

Il est dangereux d'utiliser de telles commandes de changement
explicite de police tout au long de vos documents, en effet ces
commandes vont � l'encontre de la philosophie de \LaTeX{} qui est de
s�parer les aspects logiques et visuels d'un document. Cela signifie
que si vous voulez utiliser en plusieurs endroits la m�me commande de
changement de style afin de mettre en valeur un type particulier
\vadjust{\pagebreak[3]}%MPG: fighting underfull \vbox'es
d'information, vous devriez utiliser \ci{newcommand} pour d�finir une
nouvelle commande en ins�rant ainsi la commande de changement de style
dans une enveloppe logique.

\begin{example}
% dans le pr�ambule ou dans 
% une extension :
\newcommand{\danger}[1]{%
 \textbf{#1}}
% dans le document :
D�fense d'\danger{entrer}.
Cette pi�ce contient des
\danger{machines} d'origine 
inconnue.
\end{example}

Cette approche sera pr�cieuse si vous d�cidez plus tard
d'utiliser une autre repr�sentation typographique du danger que
\verb|\textbf|. Elle �vitera d'avoir � rechercher et remplacer une �
une toutes les occurrences de \verb|\textbf| correspondant � la notion
de danger. 

-Remarquez la diff�rence entre demander � \LaTeX{} de \emph{mettre en
-valeur} un mot et lui demander de changer de \emph{police de
-caract�res}. La commande \ci{emph} prend en compte le contexte alors
que les commandes de police sont absolues.

\begin{example}
\textit{Vous pouvez aussi
  \emph{mettre en valeur}\\
  du texte en italique,}
\textsf{ou dans une police
  \emph{sans-serif},}
\texttt{ou dans une police
  \emph{machine � �crire}.}
\end{example}

\subsection{Un conseil}

Pour conclure cette promenade au pays des commandes de changement de
police, voici un (mauvais) conseil :
\begin{quote}
  \underline{\textbf{N'oubliez pas\Huge!}} \textit{Plus}
  \textsf{V\textbf{\LARGE O}\texttt{u}\textsl{s}} \Huge utilisez
  \tiny de polices \footnotesize \textbf{dans} un \small \texttt{document}
  \large \textit{Plus} \normalsize il \textsc{devient} 
  \textsl{\textsf{lisible} et bien pr�s\large e\Large n\LARGE t\huge �}.
\end{quote}


\section{Espacement}
 
\subsection{Entre les lignes}

\index{interligne} Pour utiliser un interligne plus grand pour un
document, vous pouvez utiliser la commande
\begin{lscommand}
\ci{linespread}\verb|{|\emph{facteur}\verb|}|
\end{lscommand}
\noindent dans le pr�ambule de votre document. Utilisez
\verb|\linespread{1.3}| pour un interligne \og un et demi \fg{} et
\verb|\linespread{1.6}| pour un \og double \fg{}
interligne. L'interligne par d�faut est~1.
\index{double interligne}

%Ce changement a souvent des effets ind�sirables dans les
%notes de bas de page, les tableaux, etc.  C'est assez compliqu� �
%g�rer, mais l'extension \pai{setspace} est pr�vue � cet effet.

%SC: hmmm, good translation for drastic ?
Notez que l'effet de la commande \ci{linespread} est drastique et
inappropri� pour la publication. Si vous avez une bonne raison de
changer l'interligne, vous pouvez utiliser la commande :
\pagebreak[3]%MPG: avoid underfull \vbox
\begin{lscommand}
\verb|\setlength{\baselineskip}{1.5\baselineskip}|
\end{lscommand}

\begin{example}
{\setlength{\baselineskip}%
           {1.5\baselineskip}
Ce paragraphe est formatt� avec
un interligne fix� � 1,5 de ce
qu'il �tait avant. Remarquez la
commande par � la fin du
paragraphe.\par}

Ce paragraphe a un objet pr�cis
et montre qu'apr�s l'accolade
fermante tout redevient normal.
\end{example}

\subsection{Mise en page d'un paragraphe}\label{parsp}

Il y a deux param�tres qui jouent sur l'apparence d'un paragraphe. En
ins�rant une d�finition telle que :
\begin{code}
\ci{setlength}\verb|{|\ci{parindent}\verb|}{0pt}| \\
\verb|\setlength{|\ci{parskip}\verb|}{1ex plus 0.5ex minus 0.2ex}|
\end{code}
dans le pr�ambule, vous supprimez le retrait des d�buts de paragraphe
(1\iere{} d�finition) et vous augmentez l'espace entre deux paragraphes
(2\ieme{} d�finition).

Les arguments \emph{plus} et \emph{minus} de la deuxi�me d�finition
indiquent � \TeX{} de quelle taille il est autoris� � �tendre et
r�tr�cir l'espace entre paragraphes, si cela lui est n�cessaire pour
faire tenir un paragraphe dans une m�me page.

Attention, la deuxi�me d�finition a �galement une influence sur la
table des mati�res : ses lignes deviennent �galement plus
espac�es. Pour �viter cela, vous pouvez d�placer ces commandes du
pr�ambule vers le corps du document, apr�s la commande
\ci{tableofcontents} (ou bien ne pas les utiliser du tout, car la
typographie professionnelle pr�f�re utiliser l'indentation plut�t que
l'espacement pour s�parer les paragraphes).

Pour indenter un paragraphe qui ne l'est pas, utilisez la commande :
\begin{lscommand}
\ci{indent}
\end{lscommand}
\noindent au d�but du paragraphe\footnote{Pour indenter
 syst�matiquement le premier
 paragraphe apr�s le titre d'une section, utilisez\footnotemark l'extension
 \pai{indentfirst} de l'ensemble \package{tools}.}. Bien s�r cela ne
 marche que si \verb|\parindent| n'est pas nul.
 \footnotetext{Ou bien, si vous �crivez en fran�ais, ne faites rien :
   \pai{babel} s'en est charg� pour vous ! \NdT}

Pour cr�er un paragraphe sans indentation, utilisez :
\begin{lscommand}
\ci{noindent}
\end{lscommand}
\noindent 
en t�te du paragraphe. 

L'option \pai{francais} de l'extension \pai{babel} modifie ici aussi
les r�gles par d�faut de 
\LaTeX{} pour s'adapter aux r�gles fran�aises. 

Il est possible de commencer un paragraphe par une lettrine en utilisant
l'extension \pai{lettrine}%
\footnote{\texttt{CTAN:/macros/latex/contrib/supported/lettrine/}} :
\begin{lscommand}
\ci{lettrine}\verb|[|\emph{options}\verb|]{|\emph{lettrine}\verb|}|\verb|{|\emph{texte}\verb|}|
\end{lscommand} 
La lettrine de la page~\pageref{lettrine} s'obtient par la commande :
\begin{code}
\verb|\lettrine{C}{e document}|
\end{code}

\subsection{Espacement horizontal}
\label{sec:hspace}

\LaTeX{} d�termine l'espacement entre les mots et les phrases
automatiquement. Pour ajouter de l'espacement horizontal, utilisez :
\index{espacement!horizontal}
\begin{lscommand}
\ci{hspace}\verb|{|\emph{longueur}\verb|}|
\end{lscommand}
Si une telle espace doit �tre conserv�e, m�me lorsqu'elle tombe en d�but ou en
fin de ligne, utilisez \verb|\hspace*|. Dans le cas le plus simple,
\emph{longueur} est simplement un nombre suivi d'une unit�. Les unit�s
les plus importantes sont list�es dans le tableau~\ref{units}.
\index{unit�s}\index{dimensions}

\begin{example}
Ceci\hspace{1.5cm}est une espace 
de 1.5 cm.
\end{example}
\suppressfloats
\begin{table}[tbp]
\caption{Unit�s \TeX} \label{units}\index{unit�s}
\begin{lined}{9.5cm} 
\begin{tabular}{@{}ll@{}}
\texttt{mm} & millim�tre \quad \demowidth{1mm} \\
\texttt{cm} & centim�tre = 10~mm  \quad \demowidth{1cm}                     \\
\texttt{in} & pouce\footnote{\emph{Inch} en anglais. \NdT.}
	      $=$ 25,4~mm \quad \demowidth{1in}                    \\
\texttt{pt} & point $\approx 1/72$~pouce $\approx 0,35$~mm  \quad\demowidth{1pt}\\
\texttt{em} & largeur d'un  ``M'' dans la police courante \quad \demowidth{1em}\\
\texttt{ex} & hauteur d'un ``x'' dans la police courante \quad \demowidth{1ex}
\end{tabular}

\bigskip
\end{lined}
\end{table}

La commande : 
\begin{lscommand}
\ci{stretch}\verb|{|\emph{n}\verb|}|
\end{lscommand} 
\label{cmd:stretch}
\noindent produit une espace �lastique. Elle s'�tend jusqu'� ce que tout
l'espace libre sur la ligne soit occup�. Si plusieurs commandes
\verb|\hspace{\stretch{|\emph{n}\verb|}}| sont ex�cut�es sur la m�me
ligne, les espaces s'�tendent proportionnellement � leurs facteurs
d'�lasticit� \emph{n} respectifs.


\begin{example}
x\hspace{\stretch{1}}%
x\hspace{\stretch{3}}x
\end{example}

Lors que l'espacement horizontal est utilis� en conjonction avec du
texte, il est pr�f�rable de faire que l'espace soit ajust� en fonction
de la taille de la police courante. Ceci peut �tre r�alis� en
utilisant les unit�s relatives \texttt{em} et \texttt{ex}:

\begin{example}
{\Large{}plus gr\hspace{1em}and}\\
{\tiny{}minuscu\hspace{1em}le}
\end{example}

\subsection{Espacement vertical}

L'espacement vertical entre les paragraphes, sections,
sous-sections\dots{} est d�termin� automatiquement par \LaTeX{}. 
En cas de besoin, de l'espace suppl�mentaire 
\emph{entre deux paragraphes} peut �tre ins�r� avec la commande :
\begin{lscommand}
\ci{vspace}\verb|{|\emph{longueur}\verb|}|
\end{lscommand}

Cette commande doit normalement �tre utilis�e entre deux lignes
vides. Si l'espacement doit �tre conserv� en haut ou en bas d'une
page, utilisez la version �toil�e, i.e. \verb|\vspace*|.
\index{espacement!vertical}
\index{vertical!espacement}

La commande \ci{stretch} en association avec \verb|\pagebreak| permet
d'imprimer du texte sur la derni�re ligne d'une page ou de centrer
verticalement du texte sur une page.
\begin{code}
\begin{verbatim}
Du texte \dots

\vspace{\stretch{1}}
Ceci sera imprim� sur la derni�re ligne.\pagebreak
\end{verbatim}
\end{code}

De l'espace suppl�mentaire entre deux lignes du \emph{m�me} paragraphe
ou � l'int�rieur d'une table peut �tre obtenu par la commande :
\begin{lscommand}
\ci{\bs}\verb|[|\emph{longueur}\verb|]|
\end{lscommand}

Les commandes \ci{bigskip} et \ci{smallskip} permettent de cr�er des
espacements verticaux pr�d�finis sans se pr�occuper des dimensions
exactes. 

\section{Disposition d'une page}

\begin{figure}[!hp]
\begin{center}
\makeatletter\@mylayout\makeatother
\end{center}
\vspace*{1.8cm}
\caption[Param�tres de disposition de page pour ce document]{Param�tres de disposition de page pour ce document. Utilisez l'extension \pai{layout} pour afficher la disposition de page de votre document}
\label{fig:layout}
\cih{footskip}
\cih{headheight}
\cih{headsep}
\cih{marginparpush}
\cih{marginparsep}
\cih{marginparwidth}
\cih{oddsidemargin}
\cih{paperheight}
\cih{paperwidth}
\cih{textheight}
\cih{textwidth}
\cih{topmargin}
\end{figure}
\index{disposition d'une page}
\LaTeXe{} permet d'indiquer la taille du papier en param�tre de  la commande
\verb|\documentclass|. Il d�finit ensuite automatiquement les
\wi{marges} les mieux adapt�es. Parfois, on peut ne pas �tre satisfait par les
valeurs pr�d�finies et vouloir les modifier.
%no idea why this is needed here ...
\thispagestyle{fancyplain}
La figure~\ref{fig:layout} montre tous les param�tres qui peuvent �tre
modifi�s. Cette figure a �t� r�alis�e avec l'extension \pai{layout} de
l'ensemble tools
\footnote{\CTANref|pkg/tools|}.

\textbf{Attendez !}\dots{} avant de vous lancer dans \og �largissons un peu
ce texte \fg{}, prenez deux secondes pour r�fl�chir. Comme souvent avec
\LaTeX{}, il y a de bonnes raisons pour disposer les pages de cette
fa�on. 

Sans doute, compar� avec une page standard produite avec MS Word, une page de
\LaTeX{} � l'air horriblement �troite. Mais regardez votre livre
pr�f�r�\footnote{Un vrai livre, imprim� par un grand �diteur\dots} et
comptez le nombre de caract�res sur une ligne normale. Vous verrez
qu'il n'y a gu�re plus de soixante-six caract�res par
ligne. L'exp�rience montre qu'un texte devient moins lisible si le
nombre de caract�res par ligne d�passe cette valeur, cela parce qu'il
devient plus difficile pour les yeux de passer de la fin d'une ligne
au d�but de la ligne suivante. Ceci explique aussi que les journaux
utilisent plusieurs colonnes.

Ainsi, si vous �largissez le corps du texte, ayez conscience que vous
le rendez aussi moins lisible. Ceci dit, si vous tenez � modifier les
param�tres qui contr�lent la 
disposition d'une page, voici comment proc�der :

\LaTeX{} dispose de deux commandes pour modifier ces param�tres. Elles
sont g�n�ralement utilis�es dans le pr�ambule.

La premi�re commande affecte une valeur fixe  au param�tre sp�cifi� :
\begin{lscommand}
\ci{setlength}\verb|{|\emph{param�tre}\verb|}{|\emph{longueur}\verb|}|
\end{lscommand}

La deuxi�me commande ajoute une longueur � ce param�tre :
\begin{lscommand}
\ci{addtolength}\verb|{|\emph{param�tre}\verb|}{|\emph{longueur}\verb|}|
\end{lscommand} 

La deuxi�me commande est en pratique plus utile que \ci{setlength},
parce qu'elle fonctionne relativement � la taille
par d�faut. Pour ajouter un centim�tre � la largeur du texte, nous
utiliserions les commandes suivantes dans le pr�ambule :
\begin{code}
\verb|\addtolength{\hoffset}{-0.5cm}|\\
\verb|\addtolength{\textwidth}{1cm}|
\end{code}

Dans ce contexte, il peut �tre int�ressant d'utiliser l'extension
\pai{calc}. Elle permet d'utiliser des expressions alg�briques
traditionnelles en argument de \verb|\setlength| ainsi que partout o�
l'on utilise des valeurs num�riques comme arguments de macros.

\section{Jouons un peu avec les dimensions}

Autant que possible nous �vitons d'utiliser des dimensions absolues
dans des documents \LaTeX{}. Nous essayons plut�t de les d�finir
relativement aux dimensions d'autres �l�ments de la page. La largeur
d'une figure sera ainsi \verb|\textwidth| afin de lui faire occuper
toute la largeur de la page.

Les trois commandes suivantes permettent de d�terminer la largeur, la
hauteur et la profondeur d'une cha�ne de caract�res.

\begin{lscommand}
\ci{settoheight}\verb|{|\emph{variable}\verb|}{|\emph{texte}\verb|}|\\
\ci{settodepth}\verb|{|\emph{variable}\verb|}{|\emph{texte}\verb|}|\\
\ci{settowidth}\verb|{|\emph{variable}\verb|}{|\emph{texte}\verb|}|
\end{lscommand}

\bigbreak
L'exemple ci-dessous montre une utilisation possible de ces 
commandes : 

\begin{example}
\flushleft
\newenvironment{vardesc}[1]{%
  \settowidth{\parindent}{#1\ }
  \makebox[0pt][r]{#1\ }}{}
\begin{displaymath}
a^2+b^2=c^2
\end{displaymath}

\begin{vardesc}{O� :}$a$, 
$b$ sont les cot�s adjacents �
l'angle droit d'un triangle 
rectangle,\par
$c$ est l'hypoth�nuse du 
triangle,\par
$d$ n'est pas utilis� ici. 
\'Etonnant non ?
\end{vardesc}
\end{example}


\section{Bo�tes}
\index{bo�te}

\LaTeX{} construit ses pages en empilant des bo�tes. Au commencement
chaque caract�re est une petite bo�te qui est ensuite coll�e �
d'autres bo�tes-caract�res pour former un mot. Ceux-ci sont alors
assembl�s � d'autres mots, avec une colle sp�ciale qui est �lastique
pour permettre de comprimer ou d'�tirer des s�ries de mots afin de
remplir exactement une ligne sur la page.

Reconnaissons qu'il s'agit d'une description simpliste de ce
qui se passe r�ellement, mais le fait est l� : \TeX{} travaille avec
des bo�tes et de la colle. Les caract�res ne sont pas les seuls �
pouvoir former des bo�tes. Virtuellement tout peut �tre mis dans des
bo�tes, y compris d'autres bo�tes. Chaque bo�te est ensuite trait�e
par \LaTeX{} comme s'il s'agissait d'un simple caract�re. 

Dans de pr�c�dents chapitres vous avez d�j� rencontr� quelques
bo�tes, m�me si nous ne l'avons pas signal�. L'environnement
\verb|tabular| et la commande \verb|\includegraphics|, par exemple,
produisent tous les deux des bo�tes. Cela signifie que vous pouvez
facilement aligner deux illustrations ou deux tables c�te � c�te. Il
suffit de s'assurer que la somme de leurs largeurs ne d�passe pas la
largeur du texte.

Il est aussi possible d'emballer un paragraphe dans une bo�te  :
\begin{lscommand}
\ci{parbox}\verb|[|\emph{pos}\verb|]{|\emph{largeur}\verb|}{|\emph{texte}\verb|}|
\end{lscommand}

\noindent on peut �galement utiliser un environnement :
\begin{lscommand}
\verb|\begin{|\ei{minipage}\verb|}[|\emph{pos}\verb|]{|\emph{largeur}\verb|}| texte
\verb|\end{|\ei{minipage}\verb|}|
\end{lscommand}

\noindent le param�tre \texttt{pos} peut �tre l'une des lettres
\texttt{c}, \texttt{t} ou \texttt{b} pour contr�ler l'alignement
vertical de la bo�te par rapport � la base du texte
pr�c�dent. \texttt{largeur} est une dimension indiquant la largeur de la
bo�te. La diff�rence majeure entre \ei{minipage} et \ci{parbox} est
qu'il est possible d'utiliser quasiment n'importe quelle commande ou
environnement dans \ei{minipage}, alors que ce n'est pas le cas\footnote{En
  fait, ce sont essentiellement les environnements et commandes de type
  \emph{verbatim} qui posent probl�me. \NdT} avec
\ci{parbox}.

Alors que \ci{parbox} englobe tout un paragraphe en r�alisant coupure
des lignes et tout le reste, il existe �galement une cat�gorie de
commandes de gestion des bo�tes qui ne travaillent que sur des
�l�ments align�s horizontalement. L'une d'elles nous est d�j�
connue : il s'agit de \ci{mbox}. Celle-ci combine simplement une s�rie de
bo�tes pour en former une nouvelle; elle peut �tre utilis�e pour
emp�cher \LaTeX{} de couper une ligne entre deux mots. Puisqu'il est
possible de placer des bo�tes dans d'autres bo�tes, ces constructeurs de
bo�tes horizontales sont extr�mement flexibles.

\begin{lscommand}
\ci{makebox}\verb|[|\emph{largeur}\verb|][|\emph{pos}\verb|]{|\emph{texte}\verb|}|
\end{lscommand}

\noindent Le param�tre \texttt{largeur} d�finit la largeur de la bo�te vue de
l'ext�rieur\footnote{Cela signifie qu'elle peut �tre plus petite que
la largeur du contenu de la bo�te. Dans un cas extr�me on peut m�me
r�gler la largeur � \texttt{0pt}; ainsi le texte dans la bo�te
sera plac� sans influencer les bo�tes adjacentes.}.  En plus des
expressions exprimant une longueur vous pouvez �galement utiliser
\ci{width}, \ci{height}, \ci{depth} et \ci{totalheight} � l'int�rieur
du param�tre \texttt{largeur}. Leurs valeurs sont obtenues � partir
des dimensions r�elles du \texttt{texte}. Le param�tre \texttt{pos}
est une lettre parmi \texttt{c} (\textbf{c}enter) pour centrer le
texte, \texttt{l} (flush\textbf{l}eft) pour l'aligner � gauche,
\texttt{r} (flush\textbf{r}ight) pour l'aligner � droite, ou
\texttt{s} (\textbf{s}pread) pour le r�partir horizontalement dans la
bo�te.

La commande \ci{framebox} fonctionne de la m�me fa�on que
\verb|\makebox|, mais en plus elle dessine un cadre autour du texte. 

L'exemple suivant vous montre quelques choses que l'on peut faire avec
les commandes \ci{makebox} et \ci{framebox} :

\typeout{Next message about under-full at line 4 is OK!}%MPG
\begin{example}
\makebox[\textwidth]{%
    c e n t r \'e}\par
\makebox[\textwidth][s]{%
� t i r �}\par
\framebox[1.1\width]{Whoua 
    le cadre !} \par
\framebox[0.8\width][r]{Rat\'e, 
    je suis trop large} \par
\framebox[1cm][l]{c'est aussi 
    mon cas.} 
Pouvez-vous lire ceci ?
\end{example}
\typeout{Previous message about under-full at line 4 was OK!}%MPG

Maintenant que nous savons contr�ler l'alignement horizontal, la suite
logique est de voir comment g�rer l'alignement vertical\footnote{Le
contr�le total est obtenu en contr�lant en m�me temps l'alignement
horizontal et l'alignement vertical.}. Pas de probl�me avec
\LaTeX{}. La commande :

\begin{lscommand}
\ci{raisebox}\verb|{|\emph{�l�vation}\verb|}[|\emph{profondeur}\verb|][|\emph{hauteur}\verb|]{|\emph{texte}\verb|}|
\end{lscommand}

\noindent permet de d�finir les propri�t�s verticales d'une bo�te. La
\emph{profondeur} correspond � une extension \emph{sous} la ligne de
base du texte, la \emph{hauteur} � une extension \emph{au-dessus} de
cette ligne. Vous pouvez utiliser \ci{width}, \ci{height}, \ci{depth}
et \ci{totalheight} dans les trois premiers param�tres afin d'agir en
fonction de la taille du texte contenu dans la bo�te.

\begin{example}
\raisebox{0pt}[0pt][0pt]{\Large%
\textbf{Aaaa\raisebox{-0.3ex}{a}%
\raisebox{-0.7ex}{aa}%
\raisebox{-1.2ex}{r}%
\raisebox{-2.2ex}{g}%
\raisebox{-4.5ex}{h}}}
cria-t-il, mais la ligne suivante
ne remarqua pas qu'une chose
horrible lui �tait arriv�e.
\end{example}

\section{Filets}
\label{sec:rule}

\index{filet}

%%% XXX Je ne sais pas quel est le terme technique exact pour 'strut'.

Quelques pages plus haut vous avez peut-�tre remarqu� la commande :

\begin{lscommand}
\ci{rule}\verb|[|\emph{�l�vation}\verb|]{|\emph{largeur}\verb|}{|\emph{hauteur}\verb|}|
\end{lscommand}

\noindent En utilisation normale, elle produit une simple bo�te
noire. 

\begin{example}
\rule{3mm}{.1pt}%
\rule[-1mm]{5mm}{1cm}%
\rule{3mm}{.1pt}%
\rule[1mm]{1cm}{5mm}%
\rule{3mm}{.1pt}
\end{example}

\index{horizontal!filet}
\noindent C'est utile pour produire des lignes horizontales et
verticales. La ligne horizontale sur la page de titre par exemple a
�t� trac�e � l'aide d'une commande \ci{rule}, par exemple.

\bigskip
{\flushright Fin.\par}

\endinput

%

% Local Variables:
% TeX-master: "lshort2e"
% mode: latex
% mode: flyspell
% End:

\backmatter
% bibliographie pour flshort (encod� en latin1)
%
% Pour les informations de licence, voir flshort.tex.
% See flshort.tex for licence information.
%
\svnid{$Id$}

\begin{thebibliography}{99}
\addcontentsline{toc}{chapter}{\numberline{}\bibname} 
\thispagestyle{plain}

\bibitem{manual} \textsc{Lamport}, Leslie.  \newblock \emph{{\LaTeX:}
    A Document Preparation System}.  \newblock Addison-Wesley, 1994.
    2\ieme{} �dition.\\ ISBN~0-201-52983-1.

\bibitem{texbook} \textsc{Knuth}, Donald~E.  \newblock 
    \textit{The \TeX{}book,} Volume~A de \textit{Computers and
    Typesetting}. Addison-Wesley, 1984.
    2\ieme{} �dition.\\
    ISBN~0-201-13448-9.

\bibitem{companion} \textsc{Goossens}, Michel ; \textsc{Mittelbach},
  Frank  et \textsc{Samarin}, Alexander.
  \newblock \emph{The {\LaTeX} Companion}.  \newblock
  Addison-Wesley, 1994. \\
  ISBN~0-201-54199-8.
 
\bibitem{desgraupes} \textsc{Desgraupes}, Bernard.
  \newblock \emph{{\LaTeX} Apprentissage, guide et r�f�rence}. \newblock
  Vuibert, 2000.\\
  ISBN~2-7117-8658-7.

\bibitem{local} Chaque installation de \LaTeX{} devrait fournir un 
  document appel�
  \emph{\LaTeX{} Local Guide} qui explique les particularit�s de
  cette installation. Malheureusement certains administrateurs syst�me
  paresseux ne fournissent pas ce document. Dans ce cas, demandez de
  l'aide aux
  autres utilisateurs autour de vous ou au gourou local de \LaTeX{}.

\bibitem{usrguide} \LaTeX3 Project Team.  \newblock \emph{\LaTeXe~for
    authors}.  \newblock Distribu� avec \LaTeXe{} dans
  \texttt{usrguide.tex}.

\bibitem{clsguide} \LaTeX3 Project Team.  \newblock \emph{\LaTeXe~for
    Class and Package writers}.  \newblock Distribu� avec \LaTeXe{} dans
   \texttt{clsguide.tex}.

\bibitem{fntguide} \LaTeX3 Project Team.  \newblock \emph{\LaTeXe~Font
    selection}.  \newblock Distribu� avec \LaTeXe{} dans
  \texttt{fntguide.tex}.

\bibitem{graphics} \textsc{Carlisle}, David~P. \newblock \emph{Packages in the
    `graphics' bundle}.  \newblock Distribu� avec les extensions
    � graphics � dans \texttt{grfguide.tex}.

\bibitem{verbatim} \textsc{Sch�pf}, Rainer ; \textsc{Raichle}, Bernd
    et \textsc{Rowley} Chris.  
    \newblock \emph{A New Implementation of \LaTeX's verbatim
    Environments}.
    \newblock Distribu� avec l'ensemble � tools � dans
    \texttt{verbatim.dtx}.

\bibitem{amsguide} American Mathematical Society \newblock
    \emph{\AmS-\LaTeX{} Version 1.2 User's guide}. \newblock Distribu�
    avec les extensions \AmS-\LaTeX{} dans \texttt{amsldoc.tex}. 

\bibitem{french} \textsc{Gaulle}, Bernard. \newblock \emph{Notice 
  d'utilisation du style french multilingue}. \newblock Disponible
  avec l'extension \texttt{french} sur \texttt{http://frenchpro.free.fr/}.

\bibitem{ftypo} \textsc{Perrousseaux}, Yves. \newblock \emph{Manuel de
  typographie fran�aise �l�mentaire}. \newblock Ateliers Perrousseaux
  �diteur, 1995.\\
  ISBN~2-911220-00-5.

\bibitem{eps} \textsc{Reckdahl}, Keith.  \newblock \emph{Using EPS Graphics in
    \LaTeXe{} Documents} qui explique tout ce que vous avez toujours
    voulu savoir et m�me plus sur les fichiers PostScript et leur
    utilisation avec \LaTeX{}. \newblock Disponible en ligne sur
    \texttt{CTAN:/info/epslatex.ps}
\end{thebibliography}


%%% Local Variables: 
%%% mode: latex
%%% TeX-master: "lshort2e"
%%% End: 

\refstepcounter{chapter}
\addcontentsline{toc}{chapter}{Index} 
\printindex
\refstepcounter{chapter}
\label{verylast}
\mbox{}
\end{document}



%

% Local Variables:
% TeX-master: "lshort2e"
% mode: latex
% mode: flyspell
% End:
