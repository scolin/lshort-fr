%%%%%%%%%%%%%%%%%%%%%%%%%%%%%%%%%%%%%%%%%%%%%%%%%%%%%%%%%%%%%%%%%
% Contents: Typesetting Part of LaTeX2e Introduction
% $Id$
%%%%%%%%%%%%%%%%%%%%%%%%%%%%%%%%%%%%%%%%%%%%%%%%%%%%%%%%%%%%%%%%%

% Pour les informations de licence, voir contrib.tex.
% See contrib.tex for license information.

\chapter{Mise en page}
\thispagestyle{plain}

\begin{intro}
  Après la lecture du chapitre précédent vous connaissez maintenant
  les éléments de base qui constituent un document \LaTeX{}. Dans ce
  chapitre, nous allons compléter vos connaissances afin de vous
  rendre capables de créer des documents réalistes.
\end{intro}

\section{La structure du document et le langage}
\secby{Hanspeter Schmid}{hanspi@schmid-werren.ch}

La principale raison d'être d'un texte est de diffuser des idées, de
l'information ou de la connaissance au lecteur. Celui-ci comprendra
mieux le texte si ces idées sont bien structurées et il
ressentira d'autant mieux cette structure si la typographie utilisée
reflète la structure logique et sémantique du contenu.

Ce qui distingue \LaTeX{} des autres logiciels de traitement de texte
c'est qu'il suffit d'indiquer à \LaTeX{} la structure logique et
sémantique d'un texte. Il en déduit la forme typographique en fonction
des \enquote{règles} définies dans la classe de document et les différents
fichiers de style.

L'unité de texte la plus importante pour \LaTeX{} (et en typographie)
est le \wi{paragraphe}.  Le paragraphe est la forme typographique qui
contient une pensée cohérente ou qui développe une idée. Vous allez
apprendre dans les pages suivantes la différence entre un retour à la
ligne (obtenu avec la commande \texttt{\bs\bs}) et un changement de
paragraphe (obtenu en laissant une ligne vide dans le document
source). Une nouvelle réflexion doit débuter sur un nouveau
paragraphe, mais si vous poursuivez une réflexion déjà entamée, un
simple retour à la ligne suffit.

En général, on sous-estime complètement l'importance du découpage en
paragraphes. Certains ignorent même la signification d'un changement
de paragraphe ou bien, notamment avec \LaTeX{}, coupent des paragraphes
sans le savoir. Cette erreur est particulièrement fréquente lorsque
des équations sont présentes au milieu du texte. Étudiez les exemples
suivants et essayez de comprendre pourquoi des lignes vides
(changements de paragraphe) sont parfois utilisées avant et après
l'équation et parfois non. (Si vous ne comprenez pas suffisamment les
commandes utilisées, lisez d'abord la suite du chapitre puis revenez à
cette section.)

\begin{code}
\begin{verbatim}
% Exemple 1
\dots{} lorsqu'Einstein introduit sa formule
\begin{equation}
  e = m \cdot c^2 \; ,
\end{equation}
qui est en même temps la formule la plus connue et la
moins comprise de la physique.

% Exemple 2
\dots{} d'où vient la loi des courants de Kirchhoff :
\begin{equation}
  \sum_{k=1}{n} I_k = 0 \; .
\end{equation}

La loi des tensions de Kirchhof s'en déduit\dots

% Exemple 3
\dots{} qui a plusieurs avantages.

\begin{equation}
  I_D = I_F - I_R
\end{equation}
est le c\oe{}ur d'un modèle de transistor très
différent\dots
\end{verbatim}
\end{code}

L'unité de texte immédiatement inférieure est la phrase. Dans les
documents anglo-saxons, l'espace après le point terminant une phrase
est plus grande que celle qui suit un point après une
abréviation. (Ceci n'est pas vrai dans les règles de la typographie
française.) En général, \LaTeX{} se débrouille pour déterminer la
bonne largeur des espaces. S'il n'y arrive pas, vous verrez dans la
suite comment le forcer à faire quelque chose de correct.

La structure du texte s'étend même aux morceaux d'une phrase. Les
règles grammaticales de chaque langue gèrent la ponctuation de
manière très précise. Dans la plupart des langues, la virgule
représente une courte respiration dans le flux du langage. Si vous ne
savez pas trop où placer une virgule, lisez la phrase à voix haute en
respirant à chaque virgule. Si cela ne sonne pas naturellement à
certains endroits, supprimez la virgule ; au contraire, si vous
ressentez le besoin de respirer (ou de marquer une courte pause),
insérez une virgule à cet endroit.

Enfin, les paragraphes d'un texte sont également structurés au niveau
supérieur, en les regroupant en sections, chapitres, etc. L'effet
typographique d'une commande telle que
\begin{center}
\verb|\section{La structure du texte et du langage}|
\end{center}
\noindent est suffisamment évident pour
comprendre comment utiliser ces structures de haut niveau.

\section{Sauts de ligne et de page}

\subsection{Paragraphes justifiés}

\index{justification}
Les livres sont souvent composés de lignes qui ont toutes la même
longueur~; on dit qu'elles sont justifiées à droite. \LaTeX{} insère
des retours à la ligne et des espacements entre les mots de manière à
optimiser la présentation de l'ensemble d'un paragraphe. En cas de
besoin, il coupe les mots qui ne tiennent pas en entier sur une
ligne. La présentation exacte d'un paragraphe dépend de la classe de
document\footnote{Et des règles typographiques propres de chaque pays.
\NdT}. Normalement la première ligne d'un paragraphe est
en retrait par rapport à la marge gauche
et il n'y a pas d'espace vertical particulière entre deux
paragraphes (cf. section~\ref{parsp}).

Dans certains cas particuliers, il peut être nécessaire de demander à
\LaTeX{} de couper une ligne :
\begin{lscommand}
\ci{\bs} ou \ci{newline}
\end{lscommand}
\noindent commence une nouvelle ligne sans commencer un nouveau
paragraphe.

\begin{lscommand}
\ci{\bs*}
\end{lscommand}
\noindent empêche un saut de page après le saut de ligne demandé.

\begin{lscommand}
\ci{newpage}
\end{lscommand}
\noindent provoque un saut de page.

\begin{lscommand}
\ci{linebreak}\verb|[|\emph{n}\verb|]|,
\ci{nolinebreak}\verb|[|\emph{n}\verb|]|,
\ci{pagebreak}\verb|[|\emph{n}\verb|]|,
\ci{nopagebreak}\verb|[|\emph{n}\verb|]|
\end{lscommand}
\noindent indiquent les endroits où un saut de ligne ou de page devrait
apparaître ou non. L'action de ces commandes peut être paramétrée par l'auteur
grâce au paramètre optionnel \emph{n}
qui peut prendre une valeur entre zéro et quatre. En donnant à \emph{n}
une valeur inférieure à quatre, vous laissez à \LaTeX{} la possibilité
de ne pas tenir compte de votre commande si cela devait rendre le
résultat réellement laid.  Ne confondez pas ces commandes \enquote{break}
avec les commandes \enquote{new}.  Même lorsque vous utilisez
une commande \enquote{break}, \LaTeX{} essaye de justifier le bord
droit du texte et d'ajuster la longueur totale de la page, comme
expliqué plus loin ; cela peut mener à des trous disgracieux dans
votre texte.  Si vous voulez réellement commencer une \enquote{nouvelle}
ligne ou une \enquote{nouvelle} page, utilisez la commande \enquote{new}
correspondante.

\LaTeX{} essaye toujours de trouver les meilleurs endroits pour les
retours à la ligne.
S'il ne trouve pas de solution pour couper les lignes de
manière conforme à ses normes de qualité, il laisse dépasser un bout
de ligne sur la marge droite du paragraphe. \LaTeX{} émet alors le
message d'erreur
\enquote{\wiat{\texttt{overfull \bs hbox}}{overfull hbox}\footnote{Débordement horizontal.}}.
Cela se produit surtout quand \LaTeX{} ne trouve pas
de point de césure dans un mot.\footnote{Bien que \LaTeX{} signale un
  avertissement lorsque cela arrive et affiche la
  ligne qui pose problème, celle-ci n'est pas toujours facile à
  retrouver dans le texte. En utilisant l'option \texttt{draft} dans
  la commande \ci{documentclass}, ces lignes problématiques seront
  marquées d'une épaisse marque noire dans la marge de droite.}
En utilisant alors la commande \ci{sloppy},
vous pouvez demander à \LaTeX{} d'être moins exigeant. Il ne produira
 plus de lignes trop longues en ajoutant de l'espace entre les
mots du paragraphe, même si ceux-ci finissent trop espacés selon ses
critères. Dans ce cas le message
\enquote{\wiat{\texttt{underfull \bs hbox}}{underfull hbox}\footnote{Boîte horizontale pas assez pleine.}}
est produit. Souvent, malgré tout, le
résultat est acceptable. La commande \ci{fussy} agit dans l'autre
sens, au cas où vous voudriez voir \LaTeX{} revenir à ses exigences
normales.

\subsection{Césure} \label{hyph}
\index{cesure@césure}
\LaTeX{} coupe les mots en fin de ligne si nécessaire. Si l'algorithme de
césure\footnote{\eng{\wi{Hyphenation}} en anglais.} ne trouve pas
l'endroit correct pour couper un mot\footnote{Ce qui est normalement
plutôt rare. Si vous observez de nombreuses erreurs de césure, c'est
probablement un problème de spécification de la langue du document ou
du codage de sortie. Voir le paragraphe sur le support multilingue,
page~\pageref{international}.}, vous pouvez utiliser les
commandes suivantes pour informer \TeX{} de l'exception.

La commande :
\begin{lscommand}
\ci{hyphenation}\verb|{|\emph{liste de mots}\verb|}|
\end{lscommand}
\noindent permet de ne couper les mots cités en argument qu'aux
endroits indiqués par «~\verb|-|~». Cette commande doit être
placée dans le préambule et ne doit contenir que des mots composés de
lettres ou signes considérés comme normaux par \LaTeX{}. La casse des
caractères n'est pas prise en compte. Les informations de césure sont
associés au langage actif lors de l'invocation de la commande de
césure. Cela signifie que si vous placez une commande de césure dans
le préambule, cela influencera la césure de l'anglais \footnote{Par défaut
les documents sont supposés être en anglais. \NdT}. Si vous placez la
commande après \verb|\begin{document}| et que vous utilisez une
extension comme \pai{polyglossia} pour le support d'une autre langue,
alors les suggestions de césure seront actives pour le langage
activé via \pai{polyglossia}.

%MPG: let's do what we say... :-)
\hyphenation{Anti-cons-ti-tu-tion-nel-le-ment}
L'exemple ci-dessous permet à
\enquote{anticonstitutionnellement} et \enquote{Anticonstitutionnellement},
d'être coupés. Mais il empêche toute césure de
\enquote{FORTRAN}, \enquote{Fortran} ou \enquote{fortran}. Ni les
caractères spéciaux ni les symboles ne sont autorisés dans cette
commande.

\begin{code}
\verb|\hyphenation{FORTRAN}|\\
\verb|\hyphenation{Anti-cons-ti-tu-tion-nel-le-ment}|
\end{code}

La commande \verb|\hyphenation{|\emph{liste de mots}\verb|}| a un effet
\emph{global} sur toutes les occurrences des mots de la liste.
Si l'exception ne concerne qu'une occurrence d'un mot on utilise
la commande \ci{-} qui insère un point de  césure potentiel dans un
mot. Ces positions deviennent alors les \emph{seuls} points de césure
possibles pour cette occurrence du mot. Cette commande est
particulièrement utile pour les mots contenant des caractères
spéciaux, puisque \LaTeX{} ne réalise pas automatiquement la césure
pour ces derniers.
%\footnote{À moins d'utiliser les nouvelles \wi{polices DC}}

\begin{example}
I think this is: su\-per\-cal\-%
i\-frag\-i\-lis\-tic\-ex\-pi\-%
al\-i\-do\-cious
\end{example}

Normalement, en français, on ne coupe pas
la dernière syllabe d'un mot si elle est muette, mais il arrive qu'on
soit obligé de le faire, par exemple si on travaille sur des textes
étroits (cas de colonnes multiples).

Exemple : on pourra coder \verb+ils ex\-pri\-ment+
pour autoriser \emph{exceptionnellement} le rejet à la ligne suivante de
la syllabe muette \texttt{ment}.

Plusieurs mots peuvent être maintenus ensemble sur une ligne avec la
commande :
\begin{lscommand}
\ci{mbox}\verb|{|\emph{texte}\verb|}|
\end{lscommand}
\noindent
Elle a pour effet d'interdire toute coupure de ligne dans \emph{texte}.

\begin{example}
Mon num\'ero de t\'el\'ephone va
changer. \`A partir du 18 mai,
ce sera le  \mbox{0561 336 330}.

Le param\`etre
\mbox{\emph{nom du fichier}}
de la commande \texttt{input}
contient le nom du fichier
\`a lire.
\end{example}

\ci{fbox} est similaire à \ci{mbox}, à la différence qu'un cadre
visible sera en plus dessiné autour du contenu.

\section{Chaînes prêtes à l'emploi}

Dans les exemples précédents, vous avez découvert certaines commandes
permettant de produire le logo \LaTeX{} et quelques autres chaînes de
caractères spécifiques. Voici une liste de quelques-unes de ces
commandes :

\vspace{2ex}

\noindent
\begin{tabular}{@{}lll@{}}
Commande&Résultat&Description\\
\hline
\ci{today} & \today   & Date du jour\\
\ci{TeX} & \TeX       & Logo TeX\\
\ci{LaTeX} & \LaTeX   & Logo LaTeX\\
\ci{LaTeXe} & \LaTeXe & Sa version actuelle\\
\end{tabular}

\section{Caractères spéciaux et symboles}

\subsection{Guillemets}

Pour insérer des \wi{guillemets} n'utilisez pas le caractère \verb|"|
\index{""@\texttt{""}} comme sur une machine à écrire. En typographie,
il y a des guillemets ouvrants et fermants spécifiques. En anglais,
utilisez deux~\textasciigrave{} pour les guillemets ouvrants et
deux~\textquotesingle{}
pour les guillemets fermants.
En français, avec le paquet \pai{csquotes} et l'activation du français via \pai{polyglossia},
utilisez \ci{enquote} ou bien utilisez directement
\texttt{\guillemotleft} et
\texttt{\guillemotright} si vous disposez d'un moyen de saisir ces caractères.
\begin{example}
``Please press the `x' key.''

« Appuyez sur la touche `x'. »
\end{example}
%SC: Hmmm, I do not like much this translation
Je suis conscient que le rendu n'est pas idéal, mais il s'agit
effectivement d'un accent grave (\textasciigrave) pour l'ouverture et
d'une quote (\textquotesingle) (i.e. pas une apostrophe au sens
typographique du terme) pour la fermeture, et ce malgré ce que la
police choisie semble indiquer.

\subsection{Tirets}

\LaTeX{} connaît quatre types de \wi{tiret}s. Trois d'entre eux sont
obtenus en juxtaposant un nombre variable de tirets simples. Le
quatrième n'est  pas réellement un tiret~---~il s'agit du signe
mathématique moins. \index{-} \index{--} \index{---} \index{-@$-$}
\index{moins (signe)}

\begin{example}
belle-fille, pages 13-67\\
il parle ---~en vain~---
du passé.\\
oui~---~ou non ? \\
$0$, $1$ et $-1$
\end{example}

Notez que les exemples ci-dessus respectent les règles de la
typographie française concernant l'usage des tirets, qui diffèrent
des habitudes anglo-saxonnes, en particulier le double tiret n'est pas
utilisé en français.

\subsection{Tilde (\textasciitilde)}
\index{URL link}\index{tilde}
Un caractère souvent utilisé dans les adresses sur le web est le
tilde. Pour produire ce caractère avec \LaTeX{}, on peut utiliser
\verb|\~{}|, mais le résultat (\~{}) n'est pas tout à fait le symbole
attendu. Essayez ceci à la place :
\cih{sim}

\begin{example}
http://www.rich.edu/\~{}bush \\
http://www.clever.edu/$\sim$demo
\end{example}

Voir aussi l'extension \pai{hyperref} qui inclut une commande
\ci{url}.

\subsection{Barre oblique ou slash (/)}
\index{slash}\index{Barre oblique}
Pour obtenir une barre oblique entre deux mots, il suffit de l'écrire,
comme par exemple \texttt{lire/écrire}. Cependant \LaTeX{} considère
alors cela comme un seul mot et non deux. La césure est désactivée
pour ces mots, ce qui peut conduire à des erreurs de débordement de
ligne (\enquote{overfull hbox}). Pour surmonter ce problème, utilisez
\ci{slash} comme par exemple \verb|lire\slash écrire| qui autorise la
césure. Le caractère usuel de barre oblique (\texttt{/}) peut toujours
être utilisés pour des fractions ou des unités, par exemple \texttt{5 MB/s}.

\subsection{Symbole degré (\textdegree)}

L'exemple suivant montre comment obtenir un symbole \wi{degré} :

\begin{example}
Il fait $-30\,^{\circ}\mathrm{C}$.
Je vais bient\^ot devenir
supra-conducteur.
\end{example}

L'extension \pai{textcomp} fournit un symbole degré plus adapté, disponible
seul avec \ci{textdegree}, ou accompagné d'un C avec
\ci{textcelsius}.

\begin{example}
30 \textcelsius{} font
86 \textdegree{}F.
\end{example}

En français avec l'option \texttt{french} de \package{babel}, on dispose aussi
de la commande \ci{degres} qui donne un résultat similaire.

\subsection{Le symbole de l'euro \texorpdfstring{(\officialeuro)}{}}

Écrire sur tout sujet économique de nos jours requiert l'utilisation
du symbole de l'euro. De nombreuses polices de caractères contiennent
un symbole euro. Après avoir chargé l'extension \pai{textcomp} dans le
préambule
\begin{lscommand}
\ci{usepackage}\verb|{textcomp}|
\end{lscommand}
\noindent vous pouvez utiliser la commande
\begin{lscommand}
\ci{texteuro}
\end{lscommand}
\noindent pour y accéder.

Si votre police ne fournit pas son propre symbole de l'euro ou si vous
ne l'aimez pas, il vous reste d'autres possibilités.

Tout d'abord l'extension \pai{eurosym} qui fournit un symbol officiel
de l'euro~:
\begin{lscommand}
\ci{usepackage}\verb|[official]{eurosym}|
\end{lscommand}
Si vous préférez un symbole qui se marie bien à votre police, utilisez
plutôt \texttt{gen} à la place de \texttt{official}.

% Si les polices Adobe Eurofonts sont installées sur votre système (vous
% pouvez les obtenir gratuitement sur
% \url{ftp://ftp.adobe.com/pub/adobe/type/win/all}), vous pouvez
% utiliser l'extension \pai{europs} et la commande \ci{EUR} (pour
% un symbole de l'euro qui correspond à la police courante).
% ne fonctionne pas
% soit
% l'extension \pai{eurosans} et la commande \ci{euro} (pour l'\enquote{euro officiel}).

% L'extension \pai{marvosym} fournit également des symboles variés, y
% compris celui de l'euro, sous le nom \ci{EURtm}. Son défaut est de ne
% pas proposer des versions italiques et grasses de ce symbole.

\begin{table}[!htbp]
\caption{Un sac plein d'euros} \label{eurosymb}
\begin{lined}{10cm}
\begin{tabular}{llccc}
LM+textcomp  &\verb+\texteuro+ & \huge\texteuro &\huge\sffamily\texteuro
                                                &\huge\ttfamily\texteuro\\
eurosym      &\verb+\euro+ & \huge\officialeuro &\huge\sffamily\officialeuro
                                                &\huge\ttfamily\officialeuro\\
$[$gen$]$eurosym &\verb+\euro+ & \huge\geneuro  &\huge\sffamily\geneuro
                                                &\huge\ttfamily\geneuro\\
%europs       &\verb+\EUR + & \huge\EURtm        &\huge\EURhv
%                                                &\huge\EURcr\\
% eurosans     &\verb+\euro+ & \huge\EUROSANS  &\huge\sffamily\EUROSANS
%                                             & \huge\ttfamily\EUROSANS \\
% marvosym     &\verb+\EURtm+  & \huge\mvchr101  &\huge\mvchr101
%                                                &\huge\mvchr101
\end{tabular}
\medskip
\end{lined}
\end{table}

\subsection{Points de suspension (\dots)}

Sur une machine à écrire, une \wi{virgule} ou un \wi{point} occupent la
même largeur que les autres lettres. En typographie professionnelle,
le point occupe très peu de place et il est placé tout près du caractère
qui le précède. Il n'est donc pas possible d'utiliser trois points
consécutifs pour créer des \wi{points de suspension}. À la place on
utilise la commande spécifique :
%SC : upstream ajoute (low dots) après \ci{ldots}, mais ici on utilise dots
\begin{lscommand}
\ci{dots}
\end{lscommand}
\index{...@\dots}
\nonfrenchspacing
\begin{example}
Non pas comme \c{c}a...
mais ainsi :\\
New York, Tokyo, Budapest\dots
\end{example}
\frenchspacing

\subsection{Ligatures}

Certaines séquences de lettres ne sont pas composées simplement en
juxtaposant les différentes lettres les unes à la suite des autres,
mais en utilisant des symboles spéciaux.
\begin{code}
{\large ff fi fl ffi\dots}\quad
\`a la place de\quad {\large f{}f f{}i f{}l f{}f{}i\dots}
\end{code}
Ces \wi{ligature}s peuvent être désactivées en insérant un
\ci{mbox}\verb|{}| entre les lettres en question. Cela peut s'avérer
utile pour certains mots composés\footnote{Il n'existe pas d'exemple en
français. \NdT}.
\begin{example}
\Large Not shelfful\\
but shelf{}ful
\end{example}

\subsection{Accents et caractères spéciaux}

\LaTeX{} permet l'utilisation d'\wi{accent}s et de \wi{caractères
spéciaux} issus de nombreuses langues. Le tableau~\ref{accents} montre
tous les accents que l'on peut ajouter à la lettre o. Ils s'appliquent
naturellement aux autres lettres.

Pour placer un accent sur un i ou un j, il faut supprimer leur
point. Ceci s'obtient en tapant \verb|\i| et \verb|\j|.

\begin{example}
H\^otel, na\"\i ve, \'el\`eve,\\
sm\o rrebr\o d, !`Se\~norita!,\\
Sch\"onbrunner Schlo\ss{}
Stra\ss e
\end{example}

\begin{table}[!hbp]
\caption{Accents et caractères spéciaux} \label{accents}
\begin{lined}{10cm}
\begin{tabular}{*4{cl}}
\mstA{\`o} & \mstA{\'o} & \mstA{\^o} & \mstA{\~o} \\
\mstA{\=o} & \mstA{\.o} & \mstA{\"o} & \mstB{\c}{c}\\[6pt]
\mstB{\u}{o} & \mstB{\v}{o} & \mstB{\H}{o} & \mstB{\c}{o} \\
\mstB{\d}{o} & \mstB{\b}{o} & \mstB{\t}{oo} \\[6pt]
\mstA{\oe}  &  \mstA{\OE} & \mstA{\ae} & \mstA{\AE} \\
\mstA{\aa} &  \mstA{\AA} \\[6pt]
\mstA{\o}  & \mstA{\O} & \mstA{\l} & \mstA{\L} \\
\mstA{\i}  & \mstA{\j} & !` & \verb|!|\verb|`| & ?` & \verb|?|\verb|`|
\end{tabular}
\index{i et j@\i{} et \j{} sans points}\index{scandinaves (caractères)}
\index{ae@\ae}\index{umlaut}\index{accent!grave}\index{accent!aigu}
\index{accent!circonflexe}
\index{oe@\oe}\index{aa@\aa}
\index{cédille}

\bigskip
\end{lined}
\end{table}

\section{Support multilingue\label{international}}
\secby{Axel Kielhorn}{A.Kielhorn@web.de}%
\index{international}

Pour composer des documents dans des langues autres que l'anglais,
il y a plusieurs domaines pour lesquels
\LaTeX{} doit s'adapter aux spécificités de chaque langue :
\begin{enumerate}
\item Toutes les chaînes de caractères générées automatiquement
  \footnote{\enquote{Table des matières}, \enquote{Liste des figures}, \dots}
  doivent être traduites.
\item \LaTeX{} doit connaître les règles de césure de la nouvelle
      langue.
\item Certaines règles typographiques changent en fonction de la
      langue ou de la région géographique. Par exemple le français
      impose une espace avant le caractère deux-points (:).
\end{enumerate}

De plus, saisir du texte dans votre langage de prédilection peut
devenir fastidieux avec toutes les commandes de la
figure~\ref{accents}. Pour contourner ce problème, jusqu'à récemment
il fallait naviguer dans les eaux troubles des codages spécifiques des
langages tant pour la saisie que pour les polices. De nos jours, les
moteurs \TeX{} modernes sachant traiter l'UTF-8, ces problèmes ont été
considérablement allégés.

L'extension \pai{polyglossia}\cite{polyglossia} est un substitut au désormais obsolète paquet
\pai{babel}. Elle prend soin des motifs de césure et des chaînes de
textes générées automatiquement dans vos documents.

L'extension \pai{fontspec}\cite{fontspec} gère le chargement des
polices pour \hologo{XeLaTeX} et \hologo{LuaTeX}. La police par défaut
est Latin Modern Roman.

\subsection{Utilisation de Polyglossia}

Selon le moteur \TeX{} que vous utilisez, des commandes différentes
sont nécessaires dans le préambule de votre ddocument pour activer
correctement le support multilingue.
La figure~\ref{allinone} en page~\pageref{allinone} montre un exemple
de préambule qui se charge de tous les réglages nécessaires.

\begin{figure}[!bp]
\begin{lined}{10cm}
\begin{verbatim}
\usepackage{iftex}
\ifXeTeX
  \usepackage{fontspec}
\else
  \usepackage{luatextra}
\fi
\defaultfontfeatures{Ligatures=TeX}
\usepackage{polyglossia}
\end{verbatim}
\end{lined}
\caption[Préambule tout-en-un]{Préambule tout-en-un qui prend en compte \hologo{LuaLaTeX} et \hologo{XeLaTeX}} \label{allinone}
\end{figure}


Jusqu'à maintenant l'utilisation d'un système \hologo{TeX} Unicode
n'apportait aucun avantage. Cet état de fait change dès que l'on
quitte l'alphabet latin et que l'on utilise un langage plus
intéressant comme le grec ou le russe. Avec un système basé sur
Unicode, pour pouvez simplement\footnote{Pour des valeurs suffisamment
  petites de \enquote{simple}.} entrer les caractères natifs dans votre éditeur
et \hologo{TeX} les comprendra.

Écrire en plusieurs langues est facile, il suffit de spécifier les
langages dans le préambule\footnote{les noms des langages sont en
  anglais, il faut bien un socle commun... \NdT}. Cet exemple utilise
le paquet \pai{csquotes} qui génère le bon type d'apostrophes en
fonction du langage dans lequel vous écrivez. Notez qu'il doit être
chargé \emph{avant} de charger le support pour votre langage.

\begin{lscommand}
\verb|\usepackage[autostyle=true]{csquotes}|\\
\verb|\setdefaultlanguage{french}|\\
\verb|\setotherlanguage{german}|
\end{lscommand}
%
Pour écrire un paragraphe en allemand, vous utiliserez l'environnement
\texttt{german} :

\begin{example}
Texte en français.
\begin{german}
Deutscher \enquote{Text}.
\end{german}
Encore du \enquote{texte} en français.
\end{example}

Si vous avez seulement besoin d'un mot dans une langue autre, utilisez
alors la commande \verb|\text|\emph{langage} :

\begin{example}
Saviez-vous que
\textgerman{Gesundheit} est
en fait un mot allemand.
\end{example}

Cela peut sembler superflu puisque le seul avantage est alors une
césure correcte, mais pour un langage plus exotique le jeu en vaut la
chandelle.

Parfois la police principale du document ne contient pas les glyphes
requis pour le second langage. Par exemple Latin Modern ne contient pas de
caractères cyrilliques. La solution est alors de spécifier une
police à utiliser pour ce langage. Chaque fois qu'un nouveau langage
est activé, \pai{polyglossia} vérifiera d'abord si une police aura été
spécifiée pour ce langage.
Si la police \emph{computer modern} vous sied, vous pouvez essayer la
police \enquote{Computer Modern Unicode} en ajoutant les commandes
suivantes au préambule de votre document.

\medskip\noindent Pour \hologo{LuaLaTeX} c'est plutôt simple:
\begin{verbatim}
\setmainfont{CMU Serif}
\setsansfont{CMU Sans Serif}
\setmonofont{CMU Typewriter Text}
\end{verbatim}
\noindent Pour \hologo{XeLaTeX} il faut être un peu plus explicite:
\begin{verbatim}
\setmainfont{cmun}[
   Extension=.otf,UprightFont=*rm,ItalicFont=*ti,
   BoldFont=*bx,BoldItalicFont=*bi,
 ]
 \setsansfont{cmun}[
   Extension=.otf,UprightFont=*ss,ItalicFont=*si,
   BoldFont=*sx,BoldItalicFont=*so,
 ]
 \setmonofont{cmun}[
   Extension=.otf,UprightFont=*btl,ItalicFont=*bto,
   BoldFont=*tb,BoldItalicFont=*tx,
 ]
\end{verbatim}

Une fois les polices appropriées chargées, vous pouvez maintenant écrire :

\begin{example}
\textrussian{Правда} est
un journal russe.
\textgreek{ἀλήθεια} est la vérité
ou réalité en philosophie.
\end{example}

L'extension \pai{xgreek}\index{Grec}\cite{xgreek} offre le support
pour écrire autant en ancien grec qu'en grec moderne (monotonique ou
polytonique).

\subsubsection{Langages à l'écriture de droite à gauche.}

Certains langages s'écrivent de gauche à droite, d'autres de droite à
gauche (abbrévié en RTL, comme \enquote{right to left} en
anglais). \pai{polyglossia} doit faire appel à l'extension
\pai{bidi}\cite{bidi}\footnote{\texttt{bidi} ne fonctionne pas avec
  \hologo{LuaTeX}.} pour le support des langages RTL. L'extension
\pai{bidi} doit être la dernière à être chargée, même après
\pai{hyperref} à qui cette place est usuellement réservée (de plus,
comme \pai{polyglossia} charge \pai{bidi}, alors \pai{polyglossia}
devra être la dernière extension chargée).

L'extension \pai{xepersian}\index{Perse}\cite{xepersian} propose un
support de la langue perse. Elle fournit des commandes \LaTeX\
permettant d'écrire des commandes comme \verb|\section| en perse,
ce qui rend cette extension attrayante pour les natifs de cette
langue. \pai{xepersian} est la seule extension disposant du support du
kashida\index{kashida} avec \hologo{XeLaTeX}. Le développement d'une
extension avec support du syriaque selon un algorithme similaire est
en cours.

La police IranNastaliq fournie par le SCICT\footnote{Supreme Council of
+Information and Communication Technology -- Conseil suprême pour les
technologies de l'information et de la communication. \NdT} est
disponible sur son site web
\url{http://www.scict.ir/Portal/Home/Default.aspx}.

L'extension \pai{arabxetex}\cite{arabxetex} propose le support de
plusieurs langages utilisant un alphabet arabe :

\begin{itemize}
\item arab (arabe) \index{arabe}
\item persian (perse) \index{perse}
\item urdu \index{urdu}
\item sindhi (sindhî) \index{sindhî}
\item pashto (ou pachto ou pachtoune) \index{Pashto}\index{pachtoune}
\item ottoman (turc) \index{ottoman}\index{Turc}
\item kurdish (kurde) \index{kurde}
\item kashmiri (Cachemire) \index{cachemire}
\item malay (malais ou jawi) \index{malais}\index{Jawi}
\item uighur (ouïghoure) \index{ouighoure}
\end{itemize}

%SC : pas très sûr de la traduction, à quoi s'applique le "using"
Elle offre un mécanisme de mise en correspondance des polices qui
permet à \hologo{XeLaTeX} de traiter les entrées à l'aide d'une
transcription ASCII Arab\TeX.

Les polices avec support de plusieurs langues arabes sont offertes
par l'IRMUG\footnote{Iranian Mac User Group -- groupe iranien des
  utilisateurs de Mac \NdT.} sur
\url{http://wiki.irmug.org/index.php/X_Series_2}.

Il n'y a pas d'extension pour l'hébreu\index{hébreu} parce qu'aucune
n'est nécessaire, \pai{polyglossia} étant suffisante ici. Si cependant
vous souhaitez une police convenable pour de l'hébreu en Unicode, SBL
Hebrew est gratuite pour un usage non commercial et disponible sur
\url{http://www.sbl-site.org/educational/biblicalfonts.aspx}. Vous
pouvez également regarder à la police Ezra SIL distribuée sous la
licence SIL Open Font License sur
\url{http://www.sil.org/computing/catalog/show_software.asp?id=76}.

Rappelez-vous simplement d'utiliser les commandes suivantes pour les
utiliser :

\begin{lscommand}
\verb|\newfontfamily\hebrewfont[Script=Hebrew]{SBL Hebrew}| \\
\verb|\newfontfamily\hebrewfont[Script=Hebrew]{Ezra SIL}|
\end{lscommand}

\subsubsection{Chinois, japonais et coréen (CJK)}

L'extension \pai{xeCJK}\cite{xecjk} prend soin de la sélection des
polices et de la ponctuation pour ces langues.

\section{L'espace entre les mots}

Pour obtenir une marge droite alignée, \LaTeX{} insère des espaces
plus ou moins larges entre les mots. Après la ponctuation finale
d'une phrase, les règles de la typographie anglo-saxonne\footnote{Mais pas
  celles de la typographie française. C'est pourquoi l'exemple suivant reste
  en anglais. \NdT} veulent que
l'on insère une espace plus large.  Mais si un point suit une lettre
majuscule, \LaTeX{} considère qu'il s'agit d'une abréviation et insère
alors une espace normale.

Toute exception à ces règles doit être spécifiée par l'auteur du
document. Une contre-oblique qui précède une espace génère une espace qui ne
sera pas élargie par \LaTeX{}.  Un tilde «\verb|~|» produit
une espace interdisant tout saut de ligne (dit espace
\emph{insécable}).  \verb|~| est à utiliser pour éviter les coupures
indésirables : on code par exemple \verb|M.~Dupont|.  La
commande \verb|\@| avant un point indique que celui-ci termine une
phrase, même lorsqu'il suit une majuscule.
\cih{"@}
\index{~@$\sim$} \index{tilde@tilde ( \verb.~.)}
\index{., espace après}
\index{espace insécable}

\begin{otherlanguage}{english}
\begin{example}
  Mr.~Smith was happy to see her\\
  cf.~Fig.~5\\
  I like BASIC\@. What about you?
\end{example}
\end{otherlanguage}

L'ajout d'espace supplémentaire à la fin d'une phrase peut être
supprimé par la commande :
\begin{lscommand}
\ci{frenchspacing}
\end{lscommand}
\noindent qui est active par défaut avec l'option \pai{francais} de
l'extension \pai{babel}. Dans ce cas, la commande \verb|\@| n'est pas
nécessaire.



\section{Titres, chapitres et sections}

Pour aider le lecteur à suivre votre pensée, vous souhaitez séparer
vos documents en chapitres, sections ou sous-sections. \LaTeX{}
utilise pour cela des commandes qui prennent en argument le titre de
chaque élément. C'est à vous de les utiliser dans l'ordre.

Dans la classe de document \texttt{article}, les commandes de
sectionnement suivantes sont disponibles : \nopagebreak
\begin{lscommand}
\ci{section}\verb|{...}|\\
\ci{subsection}\verb|{...}|\\
\ci{subsubsection}\verb|{...}|\\
\ci{paragraph}\verb|{...}|\\
\ci{subparagraph}\verb|{...}|
\end{lscommand}

Si vous souhaitez découper votre document en plusieurs parties sans que cela influence la
numérotation des chapitres ou des sections vous pouvez utiliser la
commande :
\begin{lscommand}
\ci{part}\verb|{...}|
\end{lscommand}

Dans les classes \texttt{report} et \texttt{book}, une commande de
sectionnemnent supérieur est disponible (elle s'intercale
entre \verb|\part| et \verb|\section|) :
\begin{lscommand}
\ci{chapter}\verb|{...}|
\end{lscommand}

Puisque la classe \texttt{article} ne connaît pas les chapitres, il
est facile par exemple de regrouper des articles en tant que chapitres
d'un livre en remplacant le \texttt{\bs title} de chaque article par
\texttt{\bs chapter}.

L'espacement entre les sections, la numérotation et le
choix de la police et de la taille des titres sont gérés
automatiquement par \LaTeX{}.

Deux commandes de sectionnement ont un comportement spécial :
\begin{itemize}
\item la commande \ci{part} ne change pas la numérotation des
      chapitres ;
\item la commande \ci{appendix} ne prend pas d'argument. Elle bascule
      simplement la numérotation des chapitres\footnote{Pour la classe
      article, elle change la numérotation des sections} en lettres.
\end{itemize}

\LaTeX{} peut ensuite créer la table des matières en récupérant la
liste des titres et de leur numéro de page d'une exécution précédente
(fichier \texttt{.toc}). La commande :
\begin{lscommand}
\ci{tableofcontents}
\end{lscommand}
\noindent imprime la table des matières à l'endroit où la commande est
invoquée. Un document doit être traité (on dit aussi \enquote{compilé})
deux fois par \LaTeX{} pour avoir une table des matières
correcte. Dans certains cas, un troisième passage est même
nécessaire. \LaTeX{} vous indique quand c'est le cas.
%Remarque: j'ai (SC) créé un cas pathologique qui demande 5
%compilations, situé dans les exemple de mon GNUmakefile pour LaTeX.

Toutes les commandes citées ci-dessus existent dans une forme
\enquote{étoilée} obtenue en ajoutant une étoile \verb|*| au nom de la
commande. Ces commandes produisent des titres de sections qui
n'apparaissent pas dans la table des matières et qui ne sont pas
numérotés. On peut ainsi remplacer la commande
\verb|\section{Introduction}| par
\verb|\section*{Introduction}|.

Par défaut, les titres de section apparaissent dans la table des
matières exactement comme ils sont dans le texte. Parfois il n'est pas
possible de faire tenir un titre trop long dans la table des
matières. On peut donner un titre spécifique pour la table des
matières en argument optionnel avant le titre principal :
\begin{code}
\verb|\chapter[Le LAAS du CNRS]{Le laboratoire|\\
\verb|         d'analyse et d'architecture|\\
\verb|        des systèmes du Centre national|\\
\verb|        de la recherche scientifique}|
\end{code}

Le \wi{titre du document} est obtenu par la commande :
\begin{lscommand}
\ci{maketitle}
\end{lscommand}
Les éléments de ce titre sont définis par les commandes :
\begin{lscommand}
\ci{title}\verb|{...}|, \ci{author}\verb|{...}|
et éventuellement \ci{date}\verb|{...}|
\end{lscommand}
\noindent qui doivent être appelées avant \verb|\maketitle|. Dans
l'argument de la commande \ci{author}, vous pouvez citer plusieurs
auteurs en séparant leurs noms par des commandes \ci{and}.

Vous trouverez un exemple des commandes citées ci-dessus sur la
figure~\ref{document}, page~\pageref{document}.

En plus des commandes de sectionnement expliquées ci-dessus, \LaTeXe{}
a introduit trois nouvelles commandes destinées à être utilisées avec
la classe \texttt{book} :
\begin{description}
\item[\ci{frontmatter}] doit être la première commande après le
  \vadjust{\pagebreak[3]}%MPG: avoid underfull vbox on next page
  début du corps du document (\verb|\begin{document}|),
    elle introduit le prologue du document.
    Les numéros de pages sont alors en romain (i, ii, iii, etc.) et
    les sections non-numérotées, comme si vous utilisiez les variantes
    étoilées des commandes de sectionnement
    (p.e. \verb|\chapter*{Preface}|), mais les sections
    apparaissaient tout de même en table des matières ;

\item[\ci{mainmatter}] se place juste avant le début du premier
  (vrai) chapitre du livre,  la numérotation des pages se fait alors
  en chiffres arabes et le compteur de pages est remis à~1 ;

\item[\ci{appendix}] indique le début des appendices, les numéros
  des chapitres sont alors remplacés par des lettres majuscules (A, B,
  etc.) ;

\item[\ci{backmatter}] se place juste avant la bibliographie et les
  index. Avec les classes standard de document, cette commmande n'a
  aucun effet visible.
\end{description}

\section{Références croisées}

Dans les livres, rapports ou articles, on trouve souvent des
\wi{références croisées} vers des figures, des tableaux ou des passages
particuliers du texte. \LaTeX{} dispose des commandes suivantes pour
faire des références croisées :

\begin{lscommand}
\ci{label}\verb|{|\emph{marque}\verb|}|, \ci{ref}\verb|{|\emph{marque}\verb|}|
et \ci{pageref}\verb|{|\emph{marque}\verb|}|
\end{lscommand}
\noindent où \emph{marque} est un identificateur choisi par
l'utilisateur. \LaTeX{} remplace \verb|\ref| par le numéro de la
section, de la sous-section, de la figure, du tableau, ou du théorème
où la commande \verb|\label| correspondante a été
placée. \verb|\pageref| affichera la page de la commande \verb|\label|
correspondante.
L'utilisation de références croisées rend nécessaire de compiler deux fois le
document : à la première compilation les numéros correspondant aux étiquettes
\verb|\label{}| sont inscrits dans le fichier \texttt{.aux} et, à la
compilation suivante, \verb|\ref{}| et \verb|\pageref{}| peuvent imprimer
ces numéros%
\footnote{Ces commandes ne connaissent pas le type du numéro auquel
elles se réfèrent, elles utilisent le dernier numéro généré
automatiquement.}.

\begin{example}
Une référence à cette
section\label{ma-section}
ressemble à :
\enquote{voir section~\ref{ma-section},
page~\pageref{ma-section}.}
\end{example}

\section{Notes de bas de page}
La commande :
\begin{lscommand}
\ci{footnote}\verb|{|\emph{texte}\verb|}|
\end{lscommand}
\noindent imprime une note de bas de page en bas de la page en cours.
Les notes de bas de page doivent être placées après le mot où la
phrase auquel elles se réfèrent%
\footnote{La typographie française demande une espace fine avant la
marque de renvoi à la note. Celle-ci est insérée automatiquement par
\package{babel} si le français est la langue principale du document, depuis la
version 2.0 de \package{frenchb}. Auparavant, il fallait utiliser
\ci{AddThinSpaceBeforeFootnotes} dans le préambule. \NdT}
%SC: hmmm, is the following french good practice ?
%MPG: I don't think so, but I'm not sure. At least Lacroux says no.
% If we change that, we should take care to stop applying this rule ourselves!
Les notes qui se réfèrent à une (partie de) phrase devraient être
placées après une virgule ou un point.\footnote{Remarquez que les
  notes de bas de page détournent l'attention du lecteur du corps du
  document. Après tout, tout le monde lit les notes de bas de
  page~---~nous sommes une espèce curieuse, alors pourquoi ne pas plus
  simplement intégrer tout ce que vous souhaitez dire dans le corps du
  document ?\footnotemark}
\footnotetext{Un guide ne va pas forcément dans la direction qu'il
  indique :-).}
\nopagebreak[2]

\begin{example}
Les notes de bas de page
\footnote{Ceci est une note
	  de bas de page.}
sont très prisées par les
utilisateurs de \LaTeX{}.
\end{example}

\section{Souligner l'importance d'un mot}

Dans un manuscrit réalisé sur une machine à écrire, les mots
importants sont \texttt{valorisés en les \underline{soulignant}} ;
on peut obtenir ce résultat en \LaTeX{} avec la commande :
\begin{lscommand}
\ci{underline}\verb|{|\textit{texte}\verb|}|
\end{lscommand}

Dans un ouvrage
imprimé, on préfère les \emph{mettre en valeur}%
\footnote{\eng{Emphasize} en anglais.}.
La commande de mise en valeur est :
\begin{lscommand}
\ci{emph}\verb|{|\emph{texte}\verb|}|
\end{lscommand}

Son argument est le texte à mettre en valeur. En général, la police
\emph{italique} est utilisée pour la mise en valeur, sauf si le texte
est déja en italique, auquel cas on utilise une police romaine (droite).
En tant qu'auteur, ce n'est pas tant la police que le besoin de mettre
en valeur un texte particulier qui est important, d'où l'existence de
cette commande.

\begin{example}
\emph{Pour \emph{insister}
dans un passage déjà
mis en valeur, \LaTeX{}
utilise une police droite.}
\end{example}

Si vous souhaitez plus de contrôle sur la police et sa taille, la
section \ref{sec:fontsize} en page \pageref{sec:fontsize} donnera
quelques idées dans ce sens.

\section{Environnements} \label{env}

Pour composer du texte dans des contextes spécifiques, \LaTeX{}
définit des \wi{environnement}s différents pour appliquer divers types
de mise en page à des portions de texte potentiellement longues :
%MPG: rallongé pour remplir la page...

\begin{lscommand}
\ci{begin}\verb|{|\emph{nom}\verb|}|\quad
   \emph{contenu}\quad
\ci{end}\verb|{|\emph{nom}\verb|}|
\end{lscommand}
\noindent
\emph{nom} est le nom de l'environnement. Les environnements peuvent
être imbriqués, à condition que l'ordre de fermeture soit correct.
\begin{code}
\verb|\begin{aaa}...\begin{bbb}...\end{bbb}...\end{aaa}|
\end{code}
\noindent Les sections suivantes vous présentent (presque) tous les
environnements importants.

\subsection{Listes, énumérations et descriptions}

L'environnement \ei{itemize} est utilisé pour des listes simples,
\ei{enumerate} est utilisé pour des énumérations (listes
numérotées) et \ei{description} est utilisé pour des descriptions.
\cih{item}

% \begin{example}
% Les différents types de liste :
% \begin{itemize}
% \item \texttt{itemize}
% \item \texttt{enumerate}
% \item \texttt{description}
% \end{itemize}
% \end{example}

%MPG: déplacé ici et allongé pour remplir la page
Notez que l'option \pai{francais} de l'extension \pai{babel} utilise
une présentation des
listes simples qui respecte les règles typographiques françaises : utilisation
d'un tiret pour les listes simples au loin d'un point épais \enquote{\textbullet},
espaces verticaux réduits.
%MPG: and this is applied to all languages by default. So don't bother trying
%to make an example.

\begin{example}
\begin{enumerate}
\item Il est possible d'imbriquer
les environnements à sa guise :
\begin{itemize}
\item mais cela peut ne pas
  être  très beau,
\item ni facile à suivre.
\end{itemize}
\item Souvenez-vous :
\begin{description}
\item[Clarté :] les faits ne
vont pas devenir plus sensés
parce  qu'ils sont dans une liste,
\item[Synthèse :] une liste peut
cependant très bien
résumer des faits.
\end{description}
\end{enumerate}
\end{example}

\subsection{Alignements à gauche, à droite et centrage}

Les environnements \ei{flushleft} et \ei{flushright} produisent des
textes \wi{aligné}s à gauche ou à droite. L'environnement \ei{center}
produit un texte centré. Si vous n'utilisez pas la commande \ci{\bs}
pour indiquer les sauts de ligne, ceux-ci continuent d'être calculés
automatiquement par \LaTeX{}.

\begin{example}
\begin{flushleft}
Ce texte est\\
aligné à gauche.
\LaTeX{} n'essaye pas
d'aligner la marge droite.
\end{flushleft}
\end{example}

\begin{example}
\begin{flushright}
Ce texte est\\
aligné à droite.
\LaTeX{} n'essaye pas
d'aligner la marge gauche.
\end{flushright}
\end{example}

\begin{example}
\begin{center}
Au centre de la terre.
\end{center}
\end{example}

\subsection{Citations et vers}

L'environnement \ei{quote} est utile pour les citations, les phrases
importantes ou les exemples.

\begin{example}
Une règle typographique
simple pour la longueur
des lignes :
\begin{quote}
Une ligne ne devrait pas comporter
plus de 66~caractères.
\end{quote}
C'est pourquoi les pages
composées par \LaTeX{} ont des
marges importantes et
les journaux utilisent
souvent plusieurs colonnes.
\end{example}

Il existe deux autres environnements comparables : \ei{quotation} et
\ei{verse}. L'environnement \ei{quotation} est utile pour des
citations plus longues, couvrant plusieurs
paragraphes parce qu'il indente ceux-ci.
L'environnement \ei{verse} est utilisé pour la poésie, là
où les retours à la ligne sont importants. Les vers sont séparés par
des commandes \ci{\bs} et les strophes par une ligne vide\footnote{Les
puristes constateront que l'environnement \ei{verse} ne respecte pas
les règles de la typographie française : les rejets devraient être
préfixés par \enquote{[\iffalse]\fi} et alignés à droite sur le vers précédent.}.

\begin{example}
Voici le début d'une
fugue de Boris Vian :
\begin{flushleft}
\begin{verse}
Les gens qui n'ont plus
  rien à faire\\
Se suivent dans la rue comme\\
Des wagons de chemin de fer.

Fer fer fer\\
Fer fer fer\\
Fer quoi faire\\
Fer coiffeur.\\
\end{verse}
\end{flushleft}
\end{example}

\subsection{Résumé}

Lors d'une publication scientifique il est usuel de démarrer celle-ci
avec un résumé (\eng{abstract}), censé donner au lecteur une vue
d'ensemble de ce qu'il doit attendre du document. \LaTeX{} fournit un
environnement \ei{abstract} à cette fin. Normalement \ei{abstract} est
utilisé dans les documents de classe \texttt{article}.

\newenvironment{abstract}%
        {\begin{center}\begin{small}\begin{minipage}{0.8\textwidth}}%
        {\end{minipage}\end{small}\end{center}}
\begin{example}
\begin{abstract}
L'abstrait abstract résumé.
\end{abstract}
\end{example}

\subsection{Impression \emph{verbatim}}

Tout texte inclus entre \verb|\begin{|\ei{verbatim}\verb|}| et
\verb|\end{verbatim}| est imprimé tel quel, comme s'il avait été tapé
à la machine, avec tous les retours à la ligne et les espaces, sans
qu'aucune commande \LaTeX{} ne soit exécutée.

À l'intérieur d'un paragraphe, une fonctionnalité équivalente peut
être obtenue par
\begin{lscommand}
\ci{verb}\verb|+|\emph{texte}\verb|+|
\end{lscommand}
\noindent Le caractère \verb|+| est seulement un exemple de caractère
séparateur. Vous pouvez utiliser n'importe quel caractère, sauf les
lettres, \verb|*| ou l'espace. La plupart des exemples de commandes
\LaTeX{} dans ce document sont réalisés avec cette commande.

\begin{example}
La commande \verb|\dots| \dots

\begin{verbatim}
10 PRINT "HELLO WORLD ";
20 GOTO 10
\end{verbatim}
\end{example}

\begin{example}
\begin{verbatim*}
La version étoilée de
l'environnement  verbatim
met    les   espaces   en
évidence
\end{verbatim*}
\end{example}

La commande \ci{verb} peut également être utilisée avec une étoile :
\begin{example}
\verb*|comme ceci :-) |
\end{example}

L'environnement \texttt{verbatim} et la commande \verb|\verb| ne
peuvent être utilisés à l'intérieur d'autres commandes comme
\verb|\footnote{}|.


\subsection{Tableaux}

\newcommand{\mfr}[1]{\framebox{\rule{0pt}{0.7em}\texttt{#1}}}

L'environnement \ei{tabular} permet de réaliser des tableaux avec ou
sans lignes de séparation horizontales ou verticales. \LaTeX{}
ajuste automatiquement la largeur des colonnes.

L'argument \emph{description} de la commande :
\begin{lscommand}
\verb|\begin{tabular}[|\emph{position}\verb|]{|\emph{description}\verb|}|
\end{lscommand}
\noindent définit le format des colonnes du tableau. Utilisez un
\mfr{l} pour une colonne alignée à gauche, \mfr{r} pour
une colonne alignée à droite et \mfr{c} pour une colonne
centrée. \mfr{p{\{\emph{largeur}\}}} permet de réaliser une colonne
justifiée sur plusieurs lignes et enfin
\mfr{|} permet d'obtenir un filet vertical.
\index{"|@ \verb."|.}

Si le texte d'une colonne est trop large pour la page, \LaTeX{}
n'insèrera pas automatiquement de saut de ligne. Grâce à
\mfr{p\{\emph{largeur}\}} vous pouvez définir un type spécial de
colonne qui fera passer le texte à la ligne comme pour un paragraphe
usuel.

L'argument \emph{position} définit la position verticale du tableau
par rapport au texte environnant. Utilisez une des lettres \mfr{t},
\mfr{b} et \mfr{c} pour l'aligner en haut (\emph{top}), en bas
(\emph{bottom}) ou au centre (\emph{center}) respectivement.

À l'intérieur de l'environnement \texttt{tabular}, le caractère
\texttt{\&} est le séparateur de colonnes, \ci{\bs} commence une nouvelle
ligne et \ci{hline} insère un filet horizontal. Vous pouvez ajoutez
des filets partiels via la commande
\ci{cline}\texttt{\{}$i$\texttt{-}$j$\texttt{\}}, où i et j
sont les numéros de colonnes de début et de fin du filet.
\index{\&}

\begin{example}
\begin{tabular}{|r|l|}
\hline
7C0 & hexadécimal \\
3700 & octal \\
11111000000 & binaire \\
\hline \hline
1984 & décimal \\
\hline
\end{tabular}
\end{example}

\begin{example}
\begin{tabular}{|p{4.7cm}|}
\hline
Bienvenue dans ce
cadre.\\
Merci de votre visite.\\
\hline
\end{tabular}

\end{example}

La construction \mfr{@\{...\}} permet d'imposer le séparateur de
colonnes. Cette commande supprime l'espacement inter-colonnes et le
remplace par ce qui est indiqué entre les crochets. Une utilisation
courante de cette commande est présentée plus loin comme solution au
problème de l'alignement des nombres décimaux. Une autre utilisation
possible est de supprimer l'espacement dans un tableau avec
\mfr{@\{\}}.

\begin{example}
\begin{tabular}{@{} l @{}}
\hline
sans espace\\\hline
\end{tabular}
\end{example}
\begin{example}
\begin{tabular}{l}
\hline
avec espaces\\
\hline
\end{tabular}
\end{example}

%
% This part by Mike Ressler
%

\index{alignement décimal} S'il n'y a pas de commande prévue%
\footnote{Si les extensions de l'ensemble \enquote{tools} sont installées
          sur votre système, jetez un \oe il sur l'extension
          \pai{dcolumn} faite pour résoudre ce problème.}
pour aligner les nombres sur le point décimal (ou la virgule si on
respecte les règles françaises) nous pouvons \enquote{tricher} et
réaliser cet alignement en utilisant deux colonnes : la première
alignée à droite contient la partie entière et la seconde alignée à
gauche contient la partie décimale. La commande \verb|\@{,}| dans la
description du tableau remplace l'espace normale entre les colonnes par
une simple virgule, donnant l'impression d'une seule colonne alignée
sur le séparateur décimal. N'oubliez pas de remplacer dans votre
tableau le point ou la virgule
par un séparateur de colonnes (\verb|&|) ! Un titre peut être
placé au-dessus de cette \enquote{colonne virtuelle} (en fait, de ces deux
colonnes) en utilisant la commande \ci{multicolumn}.

\begin{example}
\begin{tabular}{c r @{,} l}
Expression       &
\multicolumn{2}{c}{Valeur} \\
\hline
$\pi$               & 3&1416  \\
$\pi^{\pi}$         & 36&46   \\
$(\pi^{\pi})^{\pi}$ & 80662&7 \\
$\pi^{-1}$          & 0&3183 \\
\end{tabular}
\end{example}
%MPG: added more lines, trying to fill the page...
Autre exemple d'utilisation de \verb+\multicolumn+ :
\begin{example}
\begin{tabular}{|l|l|}
\hline
\multicolumn{2}{|c|}{%
  \textbf{Nom}} \\
\hline
Dupont & Jules \\
Durand & Jacques \\
\hline
\end{tabular}
\end{example}

\LaTeX{} traite le contenu d'un environnement \texttt{tabular} comme
une boîte indivisible, en particulier il ne peut y avoir de coupure de
page. Pour réaliser de longs tableaux s'étendant sur plusieurs pages
il faut avoir recours aux extensions \pai{supertabular} ou
\pai{longtable}.

Parfois les tableaux par défaut de \LaTeX{} donnent une impression
d'étroitesse. Si vous voulez leur donner plus d'extension, vous pouvez
le faire en modifiant les valeurs de \ci{arraystretch} et \ci{tabcolsep} comme
dans l'exemple suivant.

\begin{example}
\begin{tabular}{|l|}
\hline
Ces lignes sont\\\hline
à l'étroit\\\hline
\end{tabular}

{\renewcommand{\arraystretch}{1.5}
\renewcommand{\tabcolsep}{0.2cm}
\begin{tabular}{|l|}
\hline
Un tableau\\\hline
moins étroit\\\hline
\end{tabular}}

\end{example}

Si vous voulez seulement augmenter la hauteur d'un ligne dans un
tableau, vous pouvez utiliser une réglure de largeur nulle
\footnote{En typographie professionnelle ceci est appelé un
\wi{montant}.}. Donnez à cette réglure\cih{rule} la hauteur voulue.

\begin{example}
\begin{tabular}{|c|}
\hline
\rule{1pt}{4ex}\'Etai\dots\\
\hline
\rule{0pt}{4ex} montant \\
\hline
\end{tabular}
\end{example}

Les \texttt{pt} et \texttt{ex} dans l'exemple ci-avant sont des unités
\TeX. Référez-vous au tableau \ref{units} en page \pageref{units} pour
en savoir plus sur les unités de \TeX.

Un certain nombre de commandes supplémentaires pour améliorer
l'environnement \texttt{tabular} sont disponibles dans le paquet
\pai{booktabs}. Celui-ci simplifie grandement la création de tables
d'aspect professionnel et à l'espacement correct.

\section{Inclusion de graphiques et d'images} \label{eps}

Comme expliqué dans la section précédente,
avec les environnements \ei{figure} et \texttt{table}, \LaTeX{}
fournit les mécanismes de base pour travailler avec des objets tels que
des images ou des graphiques.

Un ensemble de commandes bien adaptées à l'insertion de graphiques dans ces objets flottants est
fourni par l'extension \pai{graphicx}, développée par D.~P.~Carlisle. Elle
fait partie d'un ensemble d'extensions appelé \enquote{graphics}.
\footnote{\CTAN|pkg/graphics|.}.

Voici la marche à suivre pas à pas pour inclure une
figure dans un document :

\begin{enumerate}
\item exportez la figure de votre logiciel graphique au format EPS, PDF, PNG ou JPEG.
\item Si vous avez exporté votre figure sous forme de graphique
  vectoriel EPS, vous devez la convertir en PDF avant de
  l'utiliser. Il existe une commande \texttt{epstopdf} pour cette
  tâche précise. Notez qu'exporter vers le format EPS a du sens même
  si votre logiciel sait exporter vers le format PDF, étant donné que
  le PDF est souvent en pleine page et sera réduit lorsqu'importé dans
  un document, alors qu'EPS fournit un cadre qui définit la zone qui
  contient exactement la figure.
\item chargez l'extension \textsf{graphicx} dans le préambule de votre
      fichier source avec :
\begin{lscommand}
\verb|\usepackage{graphicx}|
\end{lscommand}
\item utilisez la commande :
\begin{lscommand}
\ci{includegraphics}\verb|[|\emph{clef}=\emph{valeur}, ... \verb|]{|\emph{fichier}\verb|}|
\end{lscommand}
\noindent pour insérer \emph{fichier} dans votre document. Le param<E8>tre
optionnel est une liste de paires de \emph{clefs} et de \emph{valeurs}
séparées par des virgules. Les \emph{clefs} permettent de modifier la
largeur, la hauteur, ou l'angle de rotation de la figure. Le
tableau~\ref{keyvals} présente les clefs les plus importantes.
\end{enumerate}

\begin{table}[tb]
\caption{Clefs pour l'extension \textsf{graphicx}}
\label{keyvals}
\begin{lined}{9cm}
\begin{tabular}{@{}ll}
\texttt{width}& définit la largeur de la figure\\
\texttt{height}& définit la hauteur de la figure\\
\texttt{angle}& (en degrés) tourne la figure dans le sens \\
&  des aiguilles d'une montre \\
\texttt{scale}& échelle de la figure
\end{tabular}

\bigskip
\end{lined}
\end{table}

L'exemple en figure~\ref{figureex} page~\pageref{figureex} devrait
aider à clarifier le fonctionnement de la commande.
\begin{figure}
\begin{lined}{9cm}
\begin{verbatim}
\includegraphics[angle=90,width=\textwidth]{test.png}
\end{verbatim}
\end{lined}
\caption{Code d'exemple pour inclure \texttt{test.png} dans un document.\label{figureex}}
\end{figure}

Cette commande inclut la figure stockée dans le fichier
\texttt{test.png}. La figure est \emph{d'abord} tournée de 90 degrés
puis ajustée pour que sa largeur finale soit de 10 cm. Les proportions
largeur/hauteur sont conservées, puisqu'aucune hauteur n'est spécifiée.

Pour plus d'informations, reportez vous à~\cite{graphics}.

\section{Objets flottants}

De nos jours, la plupart des publications contiennent un nombre
important de figures et de tableaux. Ces éléments nécessitent un
traitement particulier car ils ne peuvent être coupés par un
changement de page. On pourrait imaginer de commencer une nouvelle
page chaque fois qu'une figure ou un tableau ne rentrerait pas dans la
page en cours. Cette façon de faire laisserait de nombreuses pages à moitié
blanches, ce qui ne serait réellement pas beau.
\index{tableau}
\index{figure}

La solution est de laisser \enquote{flotter} les figures et les tableaux
qui ne rentrent pas sur la page en cours, vers une page suivante et de
compléter la page avec le texte qui suit l'objet \enquote{flottant}.
\LaTeX{} fournit deux
environnements pour les \wi{objets flottants} adaptés respectivement
aux figures (\ei{figure}) et aux tableaux (\ei{table}). Pour
faire le meilleur usage de ces deux environnements, il est important
de comprendre comment \LaTeX{} traite ces objets flottants de manière
interne. Dans le cas contraire ces objets deviendront une cause de
frustration intense
car \LaTeX{} ne les placera jamais à l'endroit où vous souhaitiez les
voir.

\bigskip
Commençons par regarder les commandes que \LaTeX{} propose pour les
objets flottants.Tout objet inclus dans un environnement \ei{figure}
ou \ei{table} est traité comme un objet flottant. Les deux
environnements flottants ont un paramètre optionnel :
\begin{lscommand}
\verb|\begin{figure}[|\emph{placement}\verb|]| ou
\verb|\begin{table}[|\emph{placement}\verb|]|
\end{lscommand}
\noindent
appelé \emph{placement}. Ce paramètre permet de dire à \LaTeX{} où
vous autorisez l'objet à flotter. Un \emph{placement} est composé
d'une chaîne de caractères représentant des \emph{placements
possibles}. Reportez-vous au tableau~\ref{tab:permiss}.


\begin{table}[!htbp]
\caption{Placements possibles}\label{tab:permiss}
\noindent \begin{minipage}{\textwidth}
\medskip
\begin{center}
\begin{tabular}{@{}cp{8cm}@{}}
Caractère & Emplacement pour l'objet flottant\dots\\
\hline
\rule{0pt}{1.05em}%
\texttt{h} & \emph{here}, ici, à l'emplacement dans
	     le texte où la commande se trouve. Utile pour les petits
	     objets.\\[0.3ex]
\texttt{t} & \emph{top}, en haut d'une page\\[0.3ex]
\texttt{b} & \emph{bottom}, en bas d'une page\\[0.3ex]
\texttt{p} & \emph{page}, sur une page à part ne contenant que des
             objets flottants.\\[0.3ex]
\texttt{!} & ici, sans prendre en compte les paramètres
             internes\footnote{tels que le nombre maximum d'objets
             flottants sur une page} qui autrement pourraient empêcher ce
             placement.
\end{tabular}
\end{center}
\end{minipage}
\end{table}

Un tableau flottant peut commencer par exemple par la ligne suivante :
\begin{code}
\verb|\begin{table}[!hbp]|
\end{code}
\noindent L'\wi{emplacement} \verb|[!hbp]| permet à \LaTeX{} de placer
le tableau soit sur place (\texttt{h}), soit en bas de page
(\texttt{b}) soit enfin sur une page à part (\texttt{p}), et
tout cela même si les règles internes de \LaTeX{} ne sont pas toutes
respectées (\texttt{!}). Si aucun placement n'est indiqué, les
classes standard utilisent \verb|[tbp]| par défaut.

\LaTeX{} place tous les objets flottants qu'il rencontre
en suivant les indications fournies par l'auteur. Si un objet ne peut
être placé sur la page en cours, il est placé soit dans la file des
figures soit dans la file des tableaux\footnote{Il s'agit de files
FIFO (\emph{First In, First Out}) : premier arrivé, premier servi.}.
Quand une nouvelle page est
entamée, \LaTeX{} essaye d'abord de voir si les objets en tête des
deux files pourraient être placés sur une page spéciale, à part.  Si
cela n'est pas possible, les objets en tête des deux files sont
traités comme s'ils venaient d'être trouvés dans le texte : \LaTeX{}
essaye de les placer selon leurs spécifications de placement (sauf
\texttt{h}, qui n'est plus possible). Tous les
nouveaux objets flottants rencontrés dans la suite du texte sont
ajoutés à la queue des files. \LaTeX{} respecte scrupuleusement
l'ordre d'apparition des objets flottants. C'est pourquoi un objet
flottant qui ne peut être placé dans le texte repousse tous les
autres à la fin du document.

D'où la règle :
\begin{quote}
Si \LaTeX{} ne place pas les objets flottants comme vous le souhaitez,
c'est souvent à cause d'un seul objet trop grand qui bouche l'une des
deux files d'objets flottants.
\end{quote}

Essayer d'imposer à \LaTeX{} un emplacement particulier pose souvent
problème : si l'objet flottant ne tient pas à l'emplacement demandé,
alors il est coincé et bloque le reste des objets flottants. En
particulier, l'utilisation de la seule option \verb+[h]+ pour un
flottant est une idée \emph{à proscrire}, les versions modernes de
\LaTeX{} changent d'ailleurs automatiquement l'option \verb+[h]+ en
\verb+[ht]+.

Voici quelques éléments supplémentaires qu'il est bon de connaître sur
les environnements \ei{table} et \ei{figure}.

Avec la commande :
\begin{lscommand}
\ci{caption}\verb|{|\emph{texte de la légende}\verb|}|
\end{lscommand}
\noindent
vous définissez une légende pour l'objet. Un numéro (incrémenté
 automatiquement) et le mot \enquote{Figure} ou
 \enquote{Table}\footnote{Avec l'extension \texttt{babel}, la
 présentation des légendes est modifiée pour obéir aux règles
 françaises.} sont ajoutés par \LaTeX.

Les deux commandes :
\begin{lscommand}
\ci{listoffigures} et \ci{listoftables}
\end{lscommand}
\noindent fonctionnent de la même manière que la commande
\verb|\tableofcontents| ; elles impriment respectivement la liste des
figures et des tableaux. Dans ces listes, la légende est reprise en
entier. Si vous désirez utiliser des légendes longues, vous pouvez
en donner une version courte entre crochets qui sera utilisée pour la
table :
\begin{code}
\verb|\caption[courte]{LLLLLoooooonnnnnggggguuuueee}|
\end{code}

Avec \ci{label} et \ci{ref} vous pouvez faire référence à votre objet
à l'intérieur de votre texte. La commande \ci{label} doit apparaître
\emph{après} la commande \ci{caption} d'une légende si vous voulez
référencer le numéro de cette légende.

L'exemple suivant dessine un carré et l'insère dans le document. Vous
pouvez utiliser cette commande pour réserver de la place pour une
illustration que vous allez coller sur le document terminé.

\begin{code}
\begin{verbatim}
La figure~\ref{blanche} est un exemple de Pop-Art.
\begin{figure}[!htbp]
\includegraphics[angle=90,width=\textwidth]{white-box.pdf}
\caption{White Box by Peter Markus Paulian.\label{white}}
\end{figure}
\end{verbatim}
\end{code}

Dans l'exemple ci-dessus\footnote{En supposant que la file
des figures soit vide.} \LaTeX{} va s'acharner~(\texttt{!}) à
placer la figure là où se trouve la commande~(\texttt{h}) dans le
texte. S'il n'y arrive pas, il essayera de la placer en
bas~(\texttt{b}) de la page. Enfin s'il ne peut la placer sur la
page courante, il essayera de créer une page à part avec d'autres
objets flottants. S'il n'y a pas suffisamment de tableaux en attente
pour remplir une page spécifique, \LaTeX{} continue et, au début de la
page suivante, réessayera de placer la figure comme si elle venait
d'apparaître dans le texte.

Dans certains cas il peut s'avérer nécessaire d'utiliser la commande :
\begin{lscommand}
\ci{clearpage} ou même \ci{cleardoublepage}
\end{lscommand}
Elle ordonne à \LaTeX{} de placer tous les objets en attente
immédiatement puis de commencer une nouvelle
page. \ci{cleardoublepage} commence une nouvelle page de droite.


\endinput

% Local Variables:
% TeX-master: "lshort2e"
% mode: latex
% mode: flyspell
% End:
