%%%%%%%%%%%%%%%%%%%%%%%%%%%%%%%%%%%%%%%%%%%%%%%%%%%%%%%%%%%%%%%%%
% Contents: Things you need to know
% $Id$
%%%%%%%%%%%%%%%%%%%%%%%%%%%%%%%%%%%%%%%%%%%%%%%%%%%%%%%%%%%%%%%%%

% Pour les informations de licence, voir contrib.tex.
% See contrib.tex for license information.


\chapter{Ce qu'il faut savoir}
\thispagestyle{plain}

\begin{intro}
La première partie de ce chapitre couvre rapidement la philosophie et
l'histoire de \LaTeXe{}. La
deuxième partie met l'accent sur les structures fondamentales d'un
document \LaTeX{}. Après avoir lu ce chapitre, vous devriez avoir une
idée d'ensemble du fonctionnement de \LaTeX{} qui vous aidera à
mieux comprendre les chapitres suivants.
\end{intro}

\section{Un peu d'histoire}

\subsection{\TeX}

\TeX{} est un programme écrit par \index{Knuth, Donald E.}Donald
E.~Knuth~\cite{texbook}.  Il est conçu pour la composition de textes
et de formules mathématiques.

Knuth a commencé le développement de \TeX{} en 1977 pour tenter
d'exploiter les possibilités du matériel d'impression numérique qui commençait
à s'introduire dans le milieu de l'édition de l'époque. En particulier, il
souhaitait contrecarrer la baisse de qualité typographique qui touchait ses
propres livres et articles. Le \TeX{} que nous connaissons aujourd'hui a été
publié en 1982, avec de légères améliorations en 1989 visant à mieux gérer les
caractères 8 bits ainsi que plusieurs langages. \TeX{} est connu pour sa très
grande stabilité, sa capacité à fonctionner sur toutes sortes d'ordinateurs,
et son absence quasi-totale de bogues. Son numéro de version converge vers
$\pi$ et vaut actuellement\footnote{Au moment de la traduction\dots \NdT}
$3.141592653$.


\TeX{} se prononce \enquote{Tech}, avec un \enquote{ch} comme dans le mot
écossais \enquote{Loch}.
\footnote{Il est à noter que l'orthographe particulière de \TeX{}
  tend à laisser la prononciation suivre la façon dont les lettres qui
  le composent sont prononcées dans le pays où il est utilisé. Les
  allemands, plutôt que d'utiliser le \enquote{ch} de \enquote{Ach},
  préfèrent celui de \enquote{Pech}, ce qui donnerait comme
  prononciation en français \enquote{tèche}. À propos de ce point,
  Knuth a écrit dans le Wikipedia allemand : \emph{Je ne m'offusque
    pas que les gens prononcent \TeX{} de la manière qu'ils préfèrent
    \ldots{} et en allemand, nombreux sont ceux qui utilisent le
    \enquote{ch} doux car le X suit la voyelle \enquote{e}, pas comme le
    \enquote{ch} qui suit le \enquote{a}. En Russie, \enquote{tex} est
    un mot commun qui se prononce \enquote{tyekh}. Je crois cependant
    que la meilleure prononciation est la grecque, où l'on a le \enquote{ch}
    dur de \enquote{ach} et \enquote{Loch}.}}
Le « ch » provient de l'alphabet grec où X est la lettre « chi ». \TeX{} est aussi la
première syllabe du mot d'origine grecque \emph{technique}.
En alphabet phonétique cela donne
\textbf{[tex]}\dots{} Dans un environnement ASCII, \TeX{}
devient TeX.

\subsection{\LaTeX}

\LaTeX{} permet à un auteur de mettre en page et imprimer son travail
avec la meilleure qualité typographique en utilisant un format
professionnel pré-défini. \LaTeX{} a été écrit par \index{Lamport,
  Leslie}Leslie Lamport~\cite{manual}. Il utilise \TeX{} comme outil
de mise en page. Actuellement \LaTeX{} est maintenu par
l'équipe du \index{Projet \LaTeX{}}Projet \LaTeX{}.

% En 1994, \LaTeX{} a été mis à jour par
% l'équipe \index{LaTeX3@\LaTeX 3}\LaTeX 3, menée par \index{Mittelbach,
% Frank}Frank Mittelbach, afin de réaliser certaines améliorations
% demandées depuis longtemps et de fusionner toutes les variantes qui
% s'étaient développées depuis la sortie de \index{LaTeX 2.09@\LaTeX{}
% 2.09}\LaTeX{} 2.09 quelques années auparavent. Pour distinguer cette
% nouvelle version des précédentes, elle est appelée \index{LaTeX
% 2e@\LaTeXe}\LaTeXe. Ce document est relatif à \LaTeXe.

\LaTeX{} se prononce \textbf{[latex]}. Si vous
voulez faire référence à \LaTeX{} dans un environnement
ASCII, utilisez LaTeX. \LaTeXe{} se prononce
\textbf{[latex d\o{}z\o{}]} et s'écrit
\texttt{LaTeX2e}.

En anglais, cela donne \textbf{[la\i{}tex]} et \textbf{[la\i{}tex tu:
i:]}.

% La figure~\ref{components}, page~\pageref{components} montre
% l'interaction entre les différents éléments d'un système \TeX{}. Cette
% figure est extraite de \texttt{wots.tex} de Kees van der Laan.

% \begin{figure}[btp]
% \begin{lined}{0.8\textwidth}
% \begin{center}
% \input{kees.fig}
% \end{center}
% \end{lined}
% \caption{Éléments d'un système \TeX{}} \label{components}
% \end{figure}

\section{Les bases}

\subsection{Auteur, éditeur et typographe}

Pour publier un texte, un auteur confie son manuscrit  à une maison
d'édition. L'éditeur décide alors de la mise en page du document
(largeur des colonnes, polices de caractères, présentation des
en-têtes, \dots). L'éditeur note ses instructions sur le manuscrit et
le passe à un technicien typographe qui réalise la mise en page
en suivant ces instructions.

Un éditeur humain essaye de comprendre ce que l'auteur avait en tête en
écrivant le manuscrit. Il décide de la présentation des en-têtes de chapitres,
citation, exemples, formules, etc. en fonction de son expérience
professionnelle et du contenu du manuscrit.

Dans un environnement \LaTeX{}, celui-ci joue le rôle de l'éditeur et
utilise \TeX{} comme typographe pour la composition. Mais \LaTeX{}
n'est qu'un programme et a donc besoin de plus de directives. L'auteur
doit en particulier lui fournir de l'information sur la structure logique de son
document. Cette information est insérée dans le texte sous la forme de
\enquote{commandes \LaTeX{}}.

Cette approche est totalement différente de l'approche
\wi{WYSIWYG}\footnote{What you see is what you get -- Ce que vous
voyez est ce qui sera imprimé.}  utilisée par les traitements de texte
modernes tels que \emph{\mbox{Microsoft} \mbox{Word}} ou
\emph{\mbox{LibreOffice}}. Avec ces programmes, l'auteur définit la
mise en page du document de manière interactive pendant la saisie du
texte. Il voit à l'écran à quoi ressemblera le document final une fois
imprimé.

Avec \LaTeX{}, il n'est normalement pas possible de voir le résultat
final durant la saisie du texte, mais celui-ci peut être
pré-visualisé après traitement du fichier par \LaTeX{}. Des corrections
peuvent alors être apportées avant d'envoyer la version définitive
à l'impression.

\subsection{Choix de la mise en page}

La typographie est un métier (un art ?). Les auteurs inexpérimentés
font souvent de graves erreurs en considérant que la mise en page est
avant tout une question d'esthétique : \enquote{si un document est beau, il
est bien conçu}. Mais un document doit être lu et non accroché dans
une galerie d'art. La lisibilité et la compréhensibilité sont bien
plus importantes que l'apparence. Par exemple :
\begin{itemize}
\item la taille de la police et la numérotation des en-têtes doivent
      être choisies afin de mettre en évidence la structure des
      chapitres et des sections ;
\item les lignes ne doivent pas être trop longues pour ne pas fatiguer
      la vue du lecteur, mais cependant assez pour remplir la page de manière
      harmonieuse.
\end{itemize}

Avec un logiciel \wi{WYSIWYG}, l'auteur produit généralement des
documents esthétiquement plaisants (quoi que\dots) mais très peu ou
mal structurés. \LaTeX{} empêche de telles erreurs de formatage en
forçant l'auteur à décrire la structure logique de son document et en
choisissant lui-même la mise en page la plus
\footnote{L'un des traducteurs n'est pas aussi optimiste et pense que \LaTeX{}
  choisit seulement \emph{une} mise en pages appropriée, ce qui n'est déjà pas
  mal. \NdT} appropriée.


\subsection{Avantages et inconvénients}

Un sujet de discussion qui  revient souvent quand des gens du monde
\wi{WYSIWYG} rencontrent des utilisateurs de \LaTeX{} est le
suivant : \enquote{les \wi{avantages de \LaTeX{}} par rapport à un
traitement de texte classique} ou bien le contraire.
La meilleure chose à faire quand une telle discussion démarre, est de
garder son calme, car souvent cela dégénère. Mais parfois on ne peut y
échapper\dots

\medskip Voici donc quelques arguments. Les principaux avantages de
\LaTeX{} par rapport à un traitement de texte traditionnel sont :

\begin{itemize}

\item mise en page professionnelle qui donne aux documents l'air de
      sortir de l'atelier d'un imprimeur ;
\item la composition des formules mathématiques se fait de manière
      pratique~;
\item il suffit de connaître quelques commandes de
      base pour décrire la structure logique du document.
      Il n'est pas nécessaire de se préoccuper de la mise en page ;
\item des structures complexes telles que des notes de bas de page,
      des renvois, la table des matières ou les références
      bibliographiques sont faciles à produire ;
\item pour la plupart des tâches de typographie qui ne sont pas
      directement gérées par \LaTeX{}, il existe des extensions
      gratuites : par exemple pour inclure des figures
      \PSi{} ou pour formater une bibliographie selon un
      standard précis. La majorité de ces extensions sont décrites dans
      \companion{} et dans \desgraupes{} (en français) ;
\item \LaTeX{} encourage les auteurs à écrire des documents bien
      structurés, parce que c'est ainsi qu'il fonctionne (en
      décrivant la structure) ;
\item \TeX{}, l'outil de formatage de \LaTeXe{}, est réellement
      portable et gratuit. Le système peut ainsi fonctionner sur
      quasiment tout type de machine existante.
%
% Add examples ...
%
\end{itemize}

\medskip

\LaTeX{} a également quelques inconvénients ; il est difficile
pour moi d'en trouver, mais d'autres vous en citeront des centaines  :

\begin{itemize}
\item \LaTeX{} ne fonctionne pas bien pour ceux qui ont vendu leur
      âme ;
\item bien que quelques paramètres des mises en page pré-définies
      puissent être personnalisés, la mise au point d'une présentation
      entièrement nouvelle est difficile et demande beaucoup de
      temps\footnote{La rumeur dit que c'est un des points qui
      devrait être améliorés dans la future version \LaTeX
      3.}\index{LaTeX3@\LaTeX 3} ;
%SC: the bit about the hamster was not translated here, too bad
\item écrire des documents mal organisés et mal structurés est très
      difficile.
\item il est possible que votre hamster, malgré des débuts encourageants, ne
      parvienne jamais à bien comprendre la notion de balisage logique.
 \end{itemize}

\section{Fichiers source \LaTeX{}}

L'entrée de \LaTeX{} est un fichier texte brut. Sous Unix/Linux les
fichiers textes sont communs. Sous Windows, d'aucuns utiliseraient
Notepad pour créer un fichier texte. Ce fichier contient le texte
de votre document ainsi que les commandes qui vont dire à \LaTeX{}
comment mettre en page le texte. On appelle ce fichier \emph{\wi{fichier
source}}\footnote{Couramment élidé en « le source » dans la
conversation. \NdT}. Si vous utilisez un environnement de
développement \LaTeX{} intégré, il contiendra un programme pour créer des
fichiers sources \LaTeX{} au format texte.

\subsection{Espaces}

Les caractères d'espacement, tels que les blancs ou les tabulations,
sont traités de manière unique comme \enquote{\wi{espace}} par
\LaTeX{}. Plusieurs \wi{blancs} \emph{consécutifs} sont considérés
comme \emph{une seule} espace\footnote{En langage typographique,
    \emph{espace} est un mot féminin. \NdT}.  Les espaces en début
de ligne sont en général ignorées et un retour à la ligne unique est
traité comme une espace.  \index{espace!en début de ligne}

Une ligne vide entre deux lignes de texte marque la fin d'un
paragraphe. \emph{Plusieurs} lignes vides sont considérées comme
\emph{une seule} ligne vide. Le texte ci-dessous est un exemple. À
gauche se trouve le contenu du fichier source et à droite le
résultat formaté.

\begin{example}
Saisir un ou      plusieurs
espaces entre  les     mots
n'a pas d'importance.

Une ligne vide commence
un nouveau paragraphe.
\end{example}

\subsection{Caractères spéciaux}

Les symboles suivants sont des \wi{caractères réservés} qui, soit ont
une signification spéciale dans \LaTeX{}, soit ne sont pas disponibles
dans toutes les polices. Si vous les saisissez directement dans votre
texte, ils ne seront pas imprimés mais forceront \LaTeX{} à faire des
choses que vous n'avez pas voulues.

\begin{code}
\verb.$ & % # _ { }  ~  ^  \ . %$
 \end{code}

\index{Backslash|see{Contre-oblique}}
\index{Antislash|see{Contre-oblique}}
\index{Barre oblique inverse|see{Contre-oblique}}
Comme vous le voyez, certains de ces caractères peuvent être utilisés
dans vos documents en les préfixant par une \wi{contre-oblique}\footnote{Aussi nommée
  \enquote{barre oblique inverse}, ou parfois \nolang{antislash} d'après l'anglais
  \eng{backslash}. \NdT } :

\begin{example}
\# \$ \% \^{} \& \_ \{ \} \~{}
\textbackslash
\end{example}

Les autres symboles et bien d'autres encore peuvent être obtenus avec des
commandes spéciales à l'intérieur de formules mathématiques ou comme
accents. La contre-oblique \textbackslash{} ne peut pas être saisie en ajoutant
une contre-oblique devant (\verb|\\|) : cette séquence est utilisée pour
indiquer les coupures de ligne ; utilisez la commande
\ci{textbackslash}\footnote{Fournie par l'extension \pai{textcomp}. \NdT} pour
cela.

\subsection{Commandes \LaTeX{}}

Les \wi{commandes} \LaTeX{} sont sensibles à la casse des caractères
(majuscules ou minuscules) et respectent l'un des deux formats
suivants :

\begin{itemize}
\item soit elles commencent par une \wi{contre-oblique} \verb|\| et ont un
      nom composé uniquement de lettres. Un nom de commande est
      terminé par une espace, un chiffre ou tout autre caractère qui
      n'est pas une lettre ;
\index{contre-oblique}
\item soit elles sont composées d'une contre-oblique et d'exactement
  un caractère autre qu'une lettre.
\item plusieurs commandes ont aussi une variante \enquote{étoilée} où
  une étoile (*) est ajoutée au nom de la commande.
\end{itemize}

%SC: toujours vrai au vu de la modif ci-dessus ?
%
% \\* doesn't comply !
%
%MPG: l'étoile ne fait pas partie de la commande, c'est la commande \\ suivie
%d'une étoile.

%SC: ibidem
%
% Can \3 be a valid command ? (jacoboni)
%
%MPG: of course!
\label{whitespace}
\LaTeX{} ignore les espaces après les commandes. Si vous souhaitez
obtenir un blanc après une commande\index{espace!après une commande},
il faut ou bien insérer un paramètre vide \verb|{}| suivi d'un blanc
ou bien utiliser une commande d'espacement spécifique de \LaTeX{}. Le
paramètre vide \verb|{}| empêche \LaTeX{} d'ignorer les blancs après une
commande.

\begin{example}
Les nouveaux utilisateurs \TeX peuvent
oublier des espaces après une commande. %rendu incorrect
Les utilisateurs \TeX{} expérimentés sont
\TeX perts et savent comment utiliser les
espaces. %rendu correct
\end{example}

Certaines commandes prennent un \wi{paramètre} fourni entre
\wi{accolades}~\verb|{ }|. Certaines commandes acceptent des
\wi{paramètres optionnels} qui suivent le nom de la commande entre
\wi{crochets}~\verb|[ ]|.
\begin{code}
\verb|\|\textit{commande}\verb|[|\textit{paramètre optionnel}\verb|]{|\textit{paramètre}\verb|}|
\end{code}
L'exemple suivant montre quelques commandes \LaTeX{}. Ne vous
tracassez pas pour les comprendre, elles seront expliquées plus loin.

\begin{example}
\textsl{Penchez}-vous !
\end{example}
\begin{example}
S'il vous plait, passez \`a la
ligne ici.\newline
Merci !
\end{example}

\subsection{Commentaires}
\index{commentaires}

Quand \LaTeX{} rencontre un caractère \verb|%| dans le fichier
source, il ignore le reste de la ligne en cours, le changement de ligne
et tous les espaces au début de la ligne\footnote{Ceci est habituel et n'est pas propre à \texttt{\%}.
  \NdT}  suivante.

C'est utile pour ajouter des notes qui n'apparaîtront pas dans la
version imprimée.

\begin{example}
% Démonstration :
Ceci est un % mauvais
exemple: anticonstitu%
       tionnellement
\end{example}

Le caractère \verb|%| peut également être utilisé pour couper des
lignes trop longues dans le fichier d'entrée, lorsqu'aucun espace ou
coupure n'est autorisé.

Pour créer des commentaires plus longs, on peut utilier
l'environnement \ei{comment} fourni par l'extension \pai{verbatim} à
ajouter en préambule.  Vous apprendrez plus loin à utiliser une
extension.

\begin{example}
Voici un autre exemple
\begin{comment}
Limité mais démonstratif
\end{comment}
de commentaires.
\end{example}

Notez cependant que cet environnement n'est pas utilisable à
l'intérieur d'autres environnements complexes, tels que certains
environnements mathématiques par exemple.

\section{Structure du fichier source}
\label{sec:structure}

Quand \LaTeX{} analyse un fichier source, il s'attend à y trouver une
certaine structure. C'est pourquoi chaque fichier source doit
commencer par la commande :
\begin{code}
\verb|\documentclass{...}|
\end{code}
Elle indique quel type de document vous voulez écrire. Après cela vous
pouvez insérer des commandes qui vont influencer le style du document
ou vous pouvez charger des \wi{extension}s qui ajoutent de nouvelles
fonctionnalités au système \LaTeX{}. Pour charger une extension,
utilisez la commande :
\begin{code}
\verb|\usepackage{...}|
\end{code}

Quand tout le travail de préparation est fait\footnote{La partie entre
\texttt{\bs{}documentclass} et
\texttt{\bs{}begin$\mathtt{\{}$document$\mathtt{\}}$} est appelée le
\emph{\wi{préambule}}.}, vous pouvez commencer le corps du texte avec
la commande :
\begin{code}
\verb|\begin{document}|
\end{code}

Maintenant vous pouvez saisir votre texte et y insérer des commandes
\LaTeX{}. À la fin de votre document, utilisez la commande
\begin{code}
\verb|\end{document}|
\end{code}
pour dire à \LaTeX{} qu'il en a fini. Tout ce qui suivra dans le
fichier source sera ignoré.

La figure~\ref{mini} montre le contenu d'un document \LaTeXe{}
minimal. Un fichier source plus complet est présenté sur la
figure~\ref{document}.

\begin{figure}[hbp]
\begin{lined}{6cm}
\begin{verbatim}
\documentclass{article}
\begin{document}
Small is beautiful.
\end{document}
\end{verbatim}
\end{lined}
\caption{Un fichier \LaTeX{} minimal} \label{mini}
\end{figure}

\begin{figure}[htbp]
\begin{lined}{10cm}
\begin{verbatim}
\documentclass[a4paper,11pt]{article}
\usepackage[T1]{fontenc}
\usepackage[english,francais]{babel}
\author{P.~Tar}
\title{Le Minimalisme}
\begin{document}
\maketitle
% insérer la table des matières
\tableofcontents
\section{Quelques mots descriptifs}
Et bien, ici commence mon \oe uvre.
\section{Au revoir, monde}
\ldots{} Et ainsi s'ach\`eve mon ouvrage.
\end{document}
\end{verbatim}
\end{lined}
\caption[Exemple d'un article de revue plus réaliste]{Exemple d'un
  article de revue plus réaliste. Les commandes que vous voyez dans
  cet exemple vous seront expliquées plus tard dans cette introduction.} \label{document}
\end{figure}

\section{Utilisation typique en ligne de commande}

Vous brûlez probablement d'envie d'essayer l'exemple présenté
page~\pageref{mini}. Voici quelques informations : \LaTeX{} lui-même
ne propose pas d'interface graphique ni de jolis boutons à cliquer. Il
s'agit simplement d'un programme qui \enquote{digère} votre fichier
source. Certaines installations de \LaTeX{} ajoutent une interface
graphique permettant de cliquer pour lancer la \enquote{compilation} de votre
document. Sur d'autres systèmes il faudra probablement taper quelques
lignes de commande, aussi voici comment convaincre \LaTeX{} de
compiler votre fichier d'entrée sur un système à interface
textuelle. Notez cependant que ces explications supposent que \LaTeX{}
soit déjà installé et fonctionnel sur votre ordinateur.\footnote{
C'est le cas de toute bonne déclinaison d'Unix, et \ldots{} les
Puristes utilisent Unix, donc \ldots{} \texttt{;-)}}
%NdT: je préfère éviter tout sexisme, ici ;-)

\begin{enumerate}
\item Créez/éditez votre fichier source \LaTeX{}. Il s'agit d'un
      fichier texte pur. Sur les systèmes Unix, tous les éditeurs
      créent ce type de fichier. Sous Windows, assurez-vous que le
      fichier est sauvegardé en texte seul (\eng{plain text}). Choisissez pour
      votre fichier un nom avec le suffixe \eei{.tex}.

\item

%SC: je suis conscient de l'abus de langage sur «terminal» plutôt que
% «émulateur de terminal», mais je préfère laisser comme ça.
%MPG: j'approuve.

Ouvrez un terminal ou une ligne de commande, déplacez-vous via la
commande \texttt{cd} dans le répertoire où se trouve votre fichier et
exécutez \LaTeX{} sur celui-ci. Si tout se passe bien, vous
obtiendrez un nouveau fichier avec le suffixe \texttt{.pdf}. Il peut
être nécessaire d'exécuter \LaTeX{} plusieurs fois afin que la table
des matières et les références croisées soient à jour. S'il y a une
erreur dans votre fichier, \LaTeX{} vous le signalera et s'arrêtera de le traiter.
Appuyez sur la combinaison \texttt{Ctrl-D} pour revenir à la ligne de
commande.
\begin{lscommand}
\verb+xelatex document.tex+
\end{lscommand}

\end{enumerate}

% \clearpage
\section{La mise en page du document}

\subsection {Classes de documents}\label{sec:documentclass}

La première information dont \LaTeX{} a besoin en traitant un fichier
source est le type de document que son auteur est en train de
créer. Ce type est spécifié par la commande \ci{documentclass}.
\begin{lscommand}
\ci{documentclass}\verb|[|\emph{options}\verb|]{|\emph{classe}\verb|}|
\end{lscommand}
Ici \emph{classe} indique le type de document à créer. Le
tableau~\ref{documentclasses} donne la liste des classes de documents
présentées dans cette introduction. \LaTeXe{} fournit d'autres classes
pour d'autres types de documents, notamment des lettres et des
transparents. Le paramètre \emph{\wi{option}s} permet de modifier le
comportement de la classe de document. Les options sont séparées par
des virgules. Les principales options disponibles sont présentées dans
le tableau~\ref{options}.


\begin{table}[!bp]
\caption{Classes de documents} \label{documentclasses}
\begin{lined}{\textwidth}
\begin{description}

\item [\normalfont\texttt{article}] pour des articles dans des revues
      scientifiques, des présentations, des rapports courts, des
      documentations, des invitations, etc.
  \index{article (classe)}
\item [\normalfont\texttt{proc}] pour des comptes-rendus de conférence.
  Cette classe est basée sur la classe \texttt{article}.
  \index{classe proc}
  \index{conférence}
\item [\normalfont\texttt{minimal}] est aussi réduite que
  possible. Elle définit uniquement une taille de papier et une police
  de base. Elle est utilisée principalement à des fins de déboguage.
  \index{classe minimal}
\item [\normalfont\texttt{report}] pour des rapports plus longs
      contenant plusieurs chapitres, des petits livres, des thèses, etc.
  \index{report (classe)}
  \index{rapport}
\item [\normalfont\texttt{book}] pour des vrais livres.
  \index{book (classe)}
  \index{livre}
\item [\normalfont\texttt{slides}] pour des transparents. Cette classe
      utilise de grands caractères sans serif. Voir également la
      classe Beamer.
   \index{slides@\textsf{slides}}
   \index{transparents}
\end{description}
\end{lined}
\end{table}

\begin{table}[!bp]
\caption{Options de classes de document} \label{options}
\begin{lined}{\textwidth}
\begin{flushleft}
\begin{description}
%SC: mpg suggère \changelabels{\ttfamily}\raggedright (me semble une bonne idée)
\item[\normalfont\texttt{10pt}, \texttt{11pt}, \texttt{12pt}] \quad
définit la taille de la police principale du document. Si aucune
option n'est présente, la taille par défaut est de \texttt{10pt}.
 \index{taille!de la police par défaut}
\item[\normalfont\texttt{a4paper}, \texttt{letterpaper}, \dots] \tth{a4paper} \tti{letterpaper} \quad
définit la taille du papier. La taille par défaut est
\texttt{letterpaper}, le format standard américain. Les autres valeurs
possibles sont : \texttt{a5paper}, \texttt{b5paper}, \texttt{executivepaper}
  et \texttt{legalpaper}. \index{legal (papier)}
  \index{taille!du papier}\index{A4 (papier)}\index{letter (papier)} \index{A5
    (papier)}\index{B5 (papier)}\index{executive (papier)}
\index{papier!taille du}
\index{papier!A4}
\index{papier!A5}
\index{papier!letter}

\item[\normalfont\texttt{fleqn}] \quad aligne les formules
  mathématiques à gauche au lieu de les centrer.
\index{fleqn@\texttt{fleqn}}

\item[\normalfont\texttt{leqno}] \quad place la numérotation des
formules à gauche plutôt qu'à droite.
\index{leqno@\texttt{leqno}}

\item[\normalfont\texttt{titlepage}, \texttt{notitlepage}] \quad
indique si une nouvelle page doit être commencée après le \wi{titre du
document} ou non. La classe \texttt{article} continue par défaut sur
la même page contrairement aux classes \texttt{report} et
\texttt{book}.  \index{titlepage@\texttt{titlepage}}
\index{notitlepage@\texttt{notitlepage}}

\item[\normalfont\texttt{onecolumn}, \texttt{twocolumn}] \quad
demandent à \LaTeX{} de formater le texte sur une seule colonne
(\wi{deux colonnes}, respectivement).
\index{onecolumn@\texttt{onecolumn}}
\index{twocolumn@\texttt{twocolumn}}

\item[\normalfont\texttt{twoside, oneside}] \quad indique si la sortie
se fera en \wi{recto-verso} ou en \wi{recto simple}. Par défaut, les classes
\texttt{article} et \texttt{report} sont en \wi{simple face}
alors que la classe \texttt{book} est en \wi{double-face}.
\index{twoside@\texttt{twoside}}
\index{oneside@\texttt{oneside}}

\item[\normalfont\texttt{landscape}] \quad change la disposition du
  mode portrait au mode paysage\footnote{Aussi connus sous les noms
    \enquote{à la française} et \enquote{à l'italienne}, respectivement. \NdT}.
\index{landscape@\texttt{landscape}}

\item[\normalfont\texttt{openright, openany}] \quad fait commencer un
chapitre sur la page de droite ou sur la prochaine page. Cette option
n'a pas de sens avec la classe \texttt{article} qui ne connaît pas la
notion de chapitre. Par défaut, la classe \texttt{report} commence les
chapitres sur la prochaine page vierge alors que la classe
\texttt{book} les commence toujours sur une page de droite.
\index{openright@\texttt{openright}}
\index{openany@\texttt{openany}}

\end{description}
\end{flushleft}
\end{lined}
\end{table}

Exemple : un fichier source pour un document \LaTeX{} pourrait
commencer par la ligne
\begin{code}
\ci{documentclass}\verb|[11pt,twoside,a4paper]{article}|
\end{code}
elle informe \LaTeX{} qu'il doit composer ce document comme un
\emph{article} avec une taille de caractère de base de \emph{onze
points} et qu'il devra produire une mise en page pour une impression
\emph{double face} sur du papier au format \emph{A4}%
\footnote{Sans l'option \texttt{a4paper}, le format de papier sera
américain : 8,5~$\times$~11 pouces, soit 216~$\times$~280 mm.}.
\pagebreak[2]

\subsection{Extensions}
\index{extension}
En rédigeant votre document, vous remarquerez peut-être qu'il y a des
domaines où les commandes de base de \LaTeX{} ne permettent pas
d'exprimer ce que vous voudriez. Si vous voulez inclure des
graphiques, du texte en couleur ou du code d'un programme dans votre
document, il faut augmenter les possibilités de \LaTeX{} grâce à des
extensions.
Une extension est chargée par la commande
\begin{lscommand}
\ci{usepackage}\verb|[|\emph{options}\verb|]{|\emph{extension}\verb|}|
\end{lscommand}
où \emph{extension} est le nom de l'extension et \emph{options} une liste
de mots-clés qui déclenchent certaines fonctions de
l'extension. La commande \ci{usepackage} s'utilise dans le préambule
du document. Voir la section \ref{sec:structure} pour plus de détails.

Certaines extensions font partie de la distribution
standard de \LaTeXe{} (reportez-vous au
tableau~\ref{extensions}). D'autres sont distribuées à part. Le
\guide{} peut vous fournir plus d'informations sur les extensions
installées sur votre site. \companion{} est la principale source
d'information au sujet de \LaTeXe{}. Ce livre contient la description
de centaines d'extensions ainsi que les informations nécessaires pour
écrire vos propres extensions à \LaTeXe.

Les distributions \TeX{} modernes sont fournies avec un très grand
nombre d'extensions préinstallées. Vous pouvez utiliser la commande
\texttt{texdoc} (sous \texlive et Mac\TeX) ou \texttt{mthelp} (sous Mik\TeX{})
pour accéder à la documentation d'une extension.


\begin{table}[!tbp]
\caption{Quelques extensions fournies avec \LaTeX} \label{extensions}
\begin{lined}{\textwidth}
%SC: mpg suggère : \begin{description} \changelabels{\normalfont}
\begin{description}
\item[\normalfont\texttt{doc}] permet de documenter des programmes
 pour  \LaTeX{}.\\
 Décrite dans \texttt{doc.pdf}\footnote{Ce fichier devrait être installé
 sur votre système et vous devriez être capable de le trouver via
 la commande \texttt{texdoc} ou \texttt{mthelp} selon votre distribution. Il en est de même pour les autres
 fichiers cités dans ce tableau.} et dans \companion.

\item[\normalfont\pai{exscale}] fournit des versions de taille
  paramétrable des polices mathématiques étendues.\\
  Décrite dans \texttt{ltexscale.pdf}.

\item[\normalfont\pai{fontenc}] spécifie le \wi{codage} des polices
  de caractère que \LaTeX{} va utiliser.\\
  Décrite dans \texttt{ltoutenc.pdf}.

\item[\normalfont\pai{ifthen}] fournit des commandes de la forme\\
  `if\dots then do\dots otherwise do\dots.'\\
  Décrite dans \texttt{ifthen.pdf}, dans \companion{} et dans
  \desgraupes{}.

\item[\normalfont\pai{latexsym}] permet l'utilisation de la police des
  symboles \LaTeX{}.\\
  Décrite dans \texttt{latexsym.pdf}, dans \companion{} et dans
  \desgraupes{}.

\item[\normalfont\pai{makeidx}] fournit des commandes pour réaliser
  un index.\\
  Décrite dans ce document, section~\ref{sec:indexing}, dans
  \companion{} et dans \desgraupes{}.

\item[\normalfont\pai{syntonly}] analyse un document sans le
  formater. Utile pour une vérification rapide de la syntaxe.\\
  Décrite dans \texttt{syntonly.pdf} et dans \companion{}.

\item[\normalfont\pai{inputenc}] permet de spécifier le codage des
  caractères utilisé dans le fichier source, parmi ASCII, ISO Latin-1,
  ISO Latin-2, 437/850 IBM code pages,  Apple Macintosh, Next,
  ANSI-Windows ou un codage défini par l'utilisateur.\\
  Décrite dans \texttt{inputenc.pdf}.
\end{description}
\end{lined}
\end{table}


\subsection{Styles de page}

\LaTeX{} propose trois combinaisons d'\wi{en-tête}s et de \wi{pieds de
page}, appelées styles de page et définies par le paramètre \emph{style} de la
commande :
\index{style de page!plain@\texttt{plain}}\index{plain@\texttt{plain}}
\index{style de page!headings@\texttt{headings}}
\index{headings@\texttt{headings}}
\index{style de page!empty@\texttt{empty}}\index{empty@\texttt{empty}}
\begin{lscommand}
\ci{pagestyle}\verb|{|\emph{style}\verb|}|
\end{lscommand}
Le tableau~\ref{pagestyle}
donne la liste des styles prédéfinis.

\begin{table}[!hbp]
\caption{Les styles de page de \LaTeX} \label{pagestyle}
\begin{lined}{\textwidth}
\begin{description}

\item[\normalfont\texttt{plain}] imprime le numéro de page au milieu
du pied de page. C'est le style par défaut.

\item[\normalfont\texttt{headings}] imprime le titre du chapitre
courant et le numéro de page dans l'en-tête de chaque page et laisse le
pied de page vide. C'est à peu près le style utilisé dans ce document.

\item[\normalfont\texttt{empty}] laisse l'en-tête et le pied de page
vides.
\end{description}
\end{lined}
\end{table}

On peut changer le style de la page en cours grâce à la commande
\begin{lscommand}
\ci{thispagestyle}\verb|{|\emph{style}\verb|}|
\end{lscommand}

Au chapitre~\ref{chap:spec}, page~\pageref{sec:fancyhdr}, vous apprendrez
comment créer vos propres en-têtes et pieds de pages.


\section{Les fichiers manipulés}

L'utilisateur de \LaTeX{} est amené à cotoyer un grand nombre de
fichiers aux \wi{suffixe}s \index{extension de fichier} variés et
probablement mystérieux. Comme chaque suffixe renseigne sur
le \wi{type de fichier} dont il s'agit, il est utile d'en connaître la
signification, voici les suffixes les plus courants. Si vous pensez
qu'il en manque, n'hésitez pas à nous le signaler :

\begin{description}
\item[\eei{.tex}] fichier source \TeX{} ou \LaTeX{}, qui peut être
  compilé avec les commandes \texttt{latex} ou \texttt{pdflatex} ;
\item[\eei{.sty}] fichier contenant des commandes, que l'on charge dans le
  préambule d'un document grâce à une commande \ci{usepackage} ;
\item[\eei{.dtx}] fichier contenant du code \LaTeX{} (commandes) documenté,
  le lancement de \LaTeX{} sur un fichier \texttt{.dtx} en extrait la
  documentation.
\item[\eei{.ins}] fichier permettant d'installer le contenu du
  fichier~\texttt{.dtx} de même nom. Une extension \LaTeX{}
  téléchargée de l'Internet est composée d'un fichier \texttt{.dtx} et
  d'un \texttt{.ins}. Exécuter \LaTeX{} sur le fichier \texttt{.ins}
  pour extraire les fichiers à installer du \texttt{.dtx}.
\item[\eei{.cls}] désigne un fichier de \emph{classe} contenant la
  description d'un type de document, chargé par la commande
  \ci{documentclass};
\item[\eei{.fd}] fichier contenant des définitions pour les polices de
  caractères ;
\end{description}

Les fichiers suivants sont produits par \LaTeX{} à partir du fichier
source (de suffixe~\texttt{.tex}) :

\begin{description}
\item [\eei{.pdf}] votre document compilé, résultat principal d'une
  compilation par la commande \texttt{pdflatex} ;
\item[\eei{.dvi}] signifie \emph{DeVice Independent}, c'est le fichier
  résultat d'une compilation par la commande historique \texttt{latex}. Il peut
  être visualisé avec un logiciel approprié, converti en
  \textsc{PostScript} (par \texttt{dvips} par exemple) ou en PDF. Si
  vous avez utilisé \hologo{pdfLaTeX} pour la compilation ce fichier
  ne devrait pas apparaître ;
\item[\eei{.log}] fichier contenant le compte-rendu détaillé de la
  compilation du fichier source (avec les messages d'erreur
  éventuels) ;
\item[\eei{.toc}] contient le matériel nécessaire à la production de
  la table des matières, si celle-ci a été demandée. Ce fichier sera
  lu à la prochaine exécution de \LaTeX{} ;
\item[\eei{.lof}] contient la liste numérotée des figures, si elle a
  été demandée ;
\item[\eei{.lot}] contient la liste numérotée des tableaux, si elle a
  été demandée ;
\item[\eei{.aux}] contient diverses informations utiles à \LaTeX, en
  particulier ce qui est nécessaire au fonctionnement des références
  croisées. Le fichier \texttt{.aux} produit lors d'une exécution de
  \LaTeX{} est lu lors de l'exécution suivante ;
\item[\eei{.idx}] fichier produit seulement si un index est demandé,
  il doit être traité par \texttt{makeindex} (voir
  section~\ref{sec:indexing} page~\pageref{sec:indexing}). \LaTeX{} y
  stocke tous les mots qui iront en index ;
\item[\eei{.ind}] fichier produit par \texttt{makeindex} à partir
  du~\texttt{.idx}, il contient l'index prêt à être inclus dans le
  document ;
\item[\eei{.ilg}] fichier contenant le compte-rendu du travail de
  \texttt{makeindex}.
\end{description}


% Package Info pointer
%
%



%
% Add Info on page-numbering, ...
% \pagenumbering

\section{Gros documents}

Lorsque l'on travaille sur de gros documents, il peut être
pratique de couper le fichier source en plusieurs morceaux. \LaTeX{} a
deux commandes qui vous permettent de faire cela.

\begin{lscommand}
\ci{include}\verb|{|\emph{fichier}\verb|}|
\end{lscommand}
Vous pouvez utiliser cette commande dans le corps de votre document
pour insérer le contenu d'un autre fichier source. \LaTeX{} ajoute
automatiquement le suffixe \texttt{.tex} au nom spécifié. Remarquez que
\LaTeX{} va sauter une page pour traiter le contenu de
\emph{fichier}\texttt{.tex}.

La seconde commande peut être utilisée dans le préambule. Elle permet
de dire à \LaTeX{} de n'inclure que certains des fichiers désignés par
les commandes \verb|\include|.
\begin{lscommand}
\ci{includeonly}\verb|{|\emph{fichier}\verb|,|\emph{fichier}%
\verb|,|\ldots\verb|}|
\end{lscommand}
Après avoir rencontré cette commande dans le préambule d'un document,
seules les commandes \ci{include} dont les fichiers sont cités en
paramètre de la commande \ci{includeonly} seront exécutées.

La commande \ci{include} saute une page avant de commencer le
formatage du texte inclus. Ceci est utile lorsqu'on utilise
\ci{includeonly}, parce qu'ainsi les sauts de pages ne bougeront pas,
même si certains morceaux ne sont pas inclus. Parfois ce comportement
n'est pas souhaitable. Dans ce cas, vous pouvez utiliser la commande :
\begin{lscommand}
\ci{input}\verb|{|\emph{fichier}\verb|}|
\end{lscommand}
\noindent qui insère simplement le fichier indiqué sans aucun traitement
sophistiqué.

\enlargethispage{\baselineskip}

Il est possible de demander à \LaTeX{} de simplement vérifier la syntaxe d'un
document, sans produire de fichier~\texttt{.pdf} pour gagner du temps,
en  utilisant l'extension~\texttt{syntonly} :
\begin{verbatim}
\usepackage{syntonly}
\syntaxonly
\end{verbatim}
La vérification terminée, il suffit de mettre ces deux lignes
(ou simplement la seconde) en commentaire en plaçant un~\texttt{\%}
en tête de ligne.

% Cette remarque est-elle utile ?
% Attention : certaines extensions redéfinissent parfois certains
% caractères spéciaux\footnote{Par exemple le caractère \enquote{souligné} :
% \texttt{\_}.}. Ils ne peuvent plus alors être utilisés dans les
% \emph{noms de fichiers}.

\endinput

%

% Local Variables:
% TeX-master: "lshort2e"
% mode: latex
% mode: flyspell
% End:
