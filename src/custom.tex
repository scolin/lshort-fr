%%%%%%%%%%%%%%%%%%%%%%%%%%%%%%%%%%%%%%%%%%%%%%%%%%%%%%%%%%%%%%%%%
% Contents: Customising LaTeX output
% $Id$
%%%%%%%%%%%%%%%%%%%%%%%%%%%%%%%%%%%%%%%%%%%%%%%%%%%%%%%%%%%%%%%%%

% Pour les informations de licence, voir contrib.tex.
% See contrib.tex for license information.

\chapter{Personnalisation de \LaTeX}

\begin{intro}
Les documents produits avec les commandes que vous avez apprises
jusqu'ici sont de très bonne qualité aux yeux d'un vaste public. Même s'ils
manquent de fantaisie, ils obéissent à toutes les règles de l'art de
la typographie, ce qui les rend agréables à lire.

Mais il y a des situations où \LaTeX{} ne propose pas de commande ou
d'environnement adapté à vos besoins, ou bien où le résultat obtenu par
une commande existante peut ne pas répondre  à votre attente.

Dans ce chapitre, vous allez avoir un aperçu sur la manière
d'enrichir les commandes de \LaTeX{} et de modifier la présentation
par défaut.
\end{intro}

\section{Vos propres commandes, environnements et extensions}

Vous avez sûrement constaté que toutes les commandes décrites dans cet
ouvrage sont présentées dans un cadre et sont référencées dans l'index
qui se trouve à la fin. Au lieu d'utiliser à chaque fois l'ensemble
des commandes \LaTeX{} nécessaires, nous avons créé une extension dans
laquelle nous avons défini de nouvelles commandes et de nouveaux
environnement adaptés à cet usage. Ainsi nous pouvons simplement
écrire :

\begin{example}
\begin{lscommand}
\ci{dum}
\end{lscommand}
\end{example}

Dans cet exemple, nous utilisons à la fois un nouvel environnement
appelé \ei{lscommand} qui est responsable du tracé du cadre et une
nouvelle commande appelée \ci{ci} qui compose le nom de la commande et
qui produit l'entrée correspondante dans l'index. Vous pouvez le
vérifier en cherchant la commande \ci{dum} dans l'index à la fin de ce
document ; vous
y trouverez une entrée pointant vers chaque page où la commande dum
est mentionnée.

Si nous décidons un jour que nous ne souhaitons plus voir de cadre
autour du nom des commandes, il nous suffira de modifier la définition
de l'environnement \ei{lscommand} pour définir un nouveau style.  C'est
bien plus simple (et efficace) que de parcourir tout le document pour
remplacer une à une toutes les commandes qui tracent les cadres.

% Dans le premier chapitre, nous avons vu que \LaTeX{} avait besoin de
% connaître la structure logique d'un document pour pouvoir le formater
% correctement. Ceci est une idée intéressante, mais en pratique on
% atteint assez vite ses limites quand on ne trouve pas dans \LaTeX{} la
% commande spécialisée pour exprimer ce que l'on voudrait.
%
% Une solution serait de combiner plusieurs commandes \LaTeX{} pour
% obtenir le résultat désiré. Ponctuellement, cela ne pose pas de
% problème ; mais si cela se reproduit, répéter chaque fois la même
% combinaison peut devenir long et fastidieux. De plus, si vous souhaitez
% plus tard modifier la présentation, il faudra aller modifier chaque
% combinaison de commandes tout au long du texte.
%
% Pour résoudre ce problème, \LaTeX{} vous permet de définir vos propres
% commandes et vos propres environnements.

\subsection{Nouvelles commandes}

Pour ajouter de nouvelles commandes, utilisez la commande :
\begin{lscommand}
\ci{newcommand}\verb|{|%
       \emph{nom}\verb|}[|\emph{num}\verb|]{|\emph{définition}\verb|}|
\end{lscommand}
\noindent Cette commande prend principalement deux
arguments : le \emph{nom} de la commande à créer et sa
\emph{définition}. L'argument \emph{num} entre crochets est
optionnel. Il indique le nombre de paramètres  qu'utilisera la
nouvelle commande (jusqu'à 9).

Les deux exemples ci-dessous vous aiderons à saisir le principe.
Le premier exemple définit une nouvelle commande appelée \ci{ucil}
qui est une abréviation de \enquote{une courte introduction à
\LaTeXe}. Une telle commande pourrait être utile si vous aviez à
citer de nombreuses fois le titre de ce livre.

\begin{example}
\newcommand{\ucil}
    {Une courte (?)
     introduction à \LaTeXe}
% dans le document :
Voici \enquote{\ucil}\dots
\end{example}

L'exemple suivant montre comment définir une nouvelle commande qui
prend deux arguments. La balise \verb|#1| est remplacée par le premier
paramètre, \verb|#2| par le deuxième, et ainsi de suite.

\begin{example}
\newcommand{\uxil}[2]
    {Une \emph{#1}
     introduction à #2}
% dans le document :
\begin{itemize}
  \item \uxil{courte}{\LaTeXe}
  \item \uxil{longue}{Word}
\end{itemize}
\end{example}

\LaTeX{} ne vous permet pas de créer une nouvelle commande si celle-ci
existe déjà. Si vous voulez explicitement remplacer une commande
existante, utilisez \ci{renewcommand}. Elle utilise la même syntaxe
que \verb|\newcommand|.

Dans certains cas, vous pouvez avoir besoin de
\ci{providecommand}. Cette commande fonctionne comme
\verb|\newcommand|, mais si la nouvelle commande est déjà définie,
\LaTeXe{} ignore  la nouvelle définition.

\LaTeX{} supprime en général les espaces qui suivent une commande
(voir page~\pageref{whitespace}).

\subsection{Nouveaux environnements}

De manière analogue à la commande \verb|\newcommand|, il est possible
de définir de nouveaux environnements.  La commande
\ci{newenvironment} se présente de la manière suivante :

\begin{lscommand}
\ci{newenvironment}\verb|{|%
       \emph{nom}\verb|}[|\emph{num}\verb|]{|%
       \emph{avant}\verb|}{|\emph{après}\verb|}|
\end{lscommand}

Il est ici aussi possible d'utiliser
\ci{newenvironment} avec un paramètre optionnel. Le contenu de
l'argument \emph{avant} est exécuté avant que le contenu de
l'environnement ne soit traité. Le contenu de l'argument \emph{après}
est traité lorsque l'on rencontre la commande
\verb|\end{|\emph{nom}\verb|}|.

L'exemple ci-dessous illustre l'utilisation de \verb|\newenvironment|.

\begin{example}
\newenvironment{king}
 {\rule{1ex}{1ex}%
      \hspace{\stretch{1}}}
 {\hspace{\stretch{1}}%
      \rule{1ex}{1ex}}

\begin{king}
Mes chers sujets, \dots
\end{king}
\end{example}

L'argument \emph{num} est utilisé de la même façon que pour la
commande \verb|\newcommand|. \LaTeX{} vous empêche de redéfinir un
environnement qui existe déjà. Si jamais vous vouliez redéfinir un
environnement existant, utilisez \ci{renewenvironment} qui utilise la
même syntaxe que \verb|\newenvironment|.

Les commandes utilisées dans l'exemple ci-dessus seront présentées
plus loin. Pour la commande \ci{rule}, voir page~\pageref{sec:rule},
pour \ci{stretch}, voir page~\pageref{cmd:stretch} enfin, pour plus
d'informations sur \ci{hspace}, voir page~\pageref{sec:hspace}.

\subsection{Espaces surnuméraires}

La création d'un nouvel environnement est souvent accompagnée du
problème récurrent d'espaces surnuméraires qui peuvent avoir des
effets indésirables. Par exemple lorsque vous voulez créer un
environnement de titre qui supprime sa propre indentation ainsi que
celle du paragraphe qui le suit immédiatement. La commande
\ci{ignorespaces} dans le bloc de début de l'environnement lui fera
ignorer tout espace situé après l'exécution du bloc de début. Le bloc
de fin est plus délicat car un processus spécifique se déroule à la
fin d'un environnement. Avec la commande \ci{ignorespacesafterend}
\LaTeX{} ajoutera un \ci{ignorespaces} qui ignorera les espaces après
que le processus spécifique se soit déroulé.

\begin{example}
\newenvironment{simple}%
 {\noindent}%
 {\par\noindent}

\begin{simple}
Voyez l'espace\\à gauche.
\end{simple}
Ici\\aussi.
\end{example}

\begin{example}
\newenvironment{correct}%
 {\noindent\ignorespaces}%
 {\par\noindent%
   \ignorespacesafterend}

\begin{correct}
Plus d'espace\\à gauche.
\end{correct}
Ici\\non plus.
\end{example}

\subsection{\LaTeX{} en ligne de commande}

Si vous travaillez sur un système de type Unix, il est probable que
vous utilisiez des Makefiles pour compiler vos projets \LaTeX{}. Dans
ce contexte il pourrait être intéressant de pouvoir produire des
versions sensiblement différentes du même document par le simple
changement de paramètres de ligne de commande lors de l'invocation de
\LaTeX{}. Si vous ajoutez la structure suivante à votre document :

\begin{verbatim}
\usepackage{ifthen}
\ifthenelse{\equal{\noiretblanc}{true}}{
  % mode noir et blanc; faire ceci..
}{
  % mode couleur; faire cela..
}
\end{verbatim}

Alors vous pouvez invoquer \LaTeX{} de la manière suivante:
\begin{verbatim}
latex '\newcommand{\noiretblanc}{true}\input{document.tex}'
\end{verbatim}

D'abord la commande \verb|\noiretblanc| est définie, puis le document
est effectivement lu. En modifiant \verb|\noiretblanc| à
\texttt{false} ce serait la version couleur du document qui serait
produite.

\subsection{Votre propre extension}

Si vous définissez beaucoup de nouveaux environnements et de nouvelles
commandes, le préambule de votre document va se rallonger
dangereusement. Il peut alors devenir intéressant de créer une
extension contenant toutes ces nouvelles définitions. Avec
la commande \ci{usepackage} vous pourrez rendre disponible votre
extension dans votre document.

\begin{figure}[!htbp]
\begin{lined}{\textwidth}
\begin{verbatim}
% Exemple d'extension par Tobias Oetiker
\ProvidesPackage{demopack}
\newcommand{\ucil}{Une courte (?) introduction
                   à \LaTeXe}
\newcommand{\uxil}[1]{Une \emph{#1}
                       introduction à \LaTeXe}
\newenvironment{king}{\begin{quote}}{\end{quote}}
\end{verbatim}
\end{lined}
\caption{Exemple d'extension} \label{package}
\end{figure}

Écrire une extension consiste principalement à copier le préambule de
votre document dans un fichier à part, dont le nom se termine par
\texttt{.sty}. Il y a une commande spécifique à utiliser sur la
première ligne de votre extension~:

\begin{lscommand}
\ci{ProvidesPackage}\verb|{|\emph{nom de l'extension}\verb|}|
\end{lscommand}

\noindent  \ci{ProvidesPackage} indique à \LaTeXe{} le nom de
l'extension afin notamment de lui permettre de produire des messages
d'erreur significatifs. La figure~\ref{package} montre un exemple
d'extension simple qui reprend les commandes définies dans les
exemples précédents.

\section{Polices et tailles des caractères}
\label{sec:fontsize}
\index{police}
\index{taille!des polices}

\subsection{Commandes de changement de police}
\LaTeX{} choisit la police de caractères et sa taille en fonction de la
structure logique du document (sections, notes de bas de
page, etc. ). Dans certains cas, on voudrait pouvoir changer la taille
de la police à la main. Pour cela, utilisez les commandes listées dans
les tableaux~\ref{fonts} et~\ref{sizes}. La taille exacte de chaque
police est un choix qui dépend de la classe de document et de ses
options. La table~\ref{tab:pointsizes} donne les tailles absolues en
points pour les commandes présentes dans les classes de document
standard.

\begin{example}
{\small Les romains
petits et \textbf{gras}
régnèrent sur}
{\Large toute la belle
et grande \textit{Italie}.}
\end{example}

%MPG: ajouté la note juste pour remplir la page...
Une caractéristique importante de \LaTeXe{} est que les différents
attributs d'une police sont indépendants\footnote{Ou presque. Par exemple,
  comme l'italique et les petites capitales relèvent d'un même attribut pour
  \LaTeX, il lui est impossible de concevoir des petites capitales italiques.
  \NdT}. Cela signifie que vous
pouvez exécuter des commandes de changement de taille ou même de
changement de police tout en conservant l'attribut gras ou
italique.

% Cela peut paraître évident à quelqu'un qui  débute avec
% \LaTeXe{}, mais ça ne l'est certainement pas pour quelqu'un qui était
% habitué à \LaTeX{}~2.09.

En mode \emph{mathématique}, vous pouvez utiliser les commandes de
changement de police pour quitter provisoirement le mode mathématique
et saisir du texte normal. Pour changer les attributs de la police en
mode mathématique, il existe un jeu de commandes
spéciales. Reportez-vous au tableau~\ref{mathfonts}.


\begin{table}[!bp]
\caption{Polices} \label{fonts}
\begin{lined}{\textwidth}
%
% Alan suggested not to tell about the other form of the command
% e.g. \verb|\sffamily| or \verb|\bfseries|. This seems a good thing to me.
%
\begin{tabular}{@{}r@{ }l@{\qquad}r@{ }l@{}}
\fni{textrm}\verb|{...}|        &       \textrm{\wi{romain}}&
\fni{texttt}\verb|{...}|        &       \texttt{machine à écrire}\\
\fni{textsf}\verb|{...}|        &       \textsf{\wi{sans serif}}\\[6pt]
\fni{textmd}\verb|{...}|        &       \textmd{moyen}&
\fni{textbf}\verb|{...}|        &       \textbf{\wi{gras}}\\[6pt]
\fni{textup}\verb|{...}|        &       \textup{\wi{droit}}&
\fni{textit}\verb|{...}|        &       \textit{\wi{italique}}\\
\fni{textsl}\verb|{...}|        &       \textsl{\wi{penché}}&
\fni{textsc}\verb|{...}|        &       \textsc{\wi{Petites Capitales}}\\[6pt]
\ci{emph}\verb|{...}|           &       \emph{en évidence} &
\fni{textnormal}\verb|{...}|    &       \textnormal{par défaut}
\end{tabular}

\bigskip
\end{lined}
\end{table}


\begin{table}[!bp]
\index{taille!prédéfinies}
\caption{Tailles des polices} \label{sizes}
\begin{lined}{12cm}
\begin{tabular}{@{}ll}
\fni{tiny}      & \tiny        minuscule \\
\fni{scriptsize}   & \scriptsize  très petit\\
\fni{footnotesize} & \footnotesize  assez petit \\
\fni{small}        &  \small            petit \\
\fni{normalsize}   &  \normalsize  normal  \\
\fni{large}        &  \large       grand
\end{tabular}%
\qquad\begin{tabular}{ll@{}}
\fni{Large}        &  \Large       plus grand \\[5pt]
\fni{LARGE}        &  \LARGE       très grand \\[5pt]
\fni{huge}         &  \huge        énorme \\[5pt]
\fni{Huge}         &  \Huge        géant
\end{tabular}

\bigskip
\end{lined}
\end{table}

\begin{table}[!tbp]
\caption{Tailles en points dans les classes standard}\label{tab:pointsizes}
\label{tab:sizes}
\begin{lined}{12cm}
\begin{tabular}{lrrr}
\multicolumn{1}{c}{taille} &
\multicolumn{1}{c}{10pt (défaut) } &
           \multicolumn{1}{c}{option 11pt}  &
           \multicolumn{1}{c}{option 12pt}\\
\verb|\tiny|       & 5pt  & 6pt & 6pt\\
\verb|\scriptsize| & 7pt  & 8pt & 8pt\\
\verb|\footnotesize| & 8pt & 9pt & 10pt \\
\verb|\small|        & 9pt & 10pt & 11pt \\
\verb|\normalsize| & 10pt & 11pt & 12pt \\
\verb|\large|      & 12pt & 12pt & 14pt \\
\verb|\Large|      & 14pt & 14pt & 17pt \\
\verb|\LARGE|      & 17pt & 17pt & 20pt\\
\verb|\huge|       & 20pt & 20pt & 25pt\\
\verb|\Huge|       & 25pt & 25pt & 25pt\\
\end{tabular}

\bigskip
\end{lined}
\end{table}

\begin{table}[!htbp]
\caption{Polices mathématiques} \label{mathfonts}
\begin{lined}{0.8\textwidth}
\begin{tabular}{@{}ll@{}}
\fni{mathrm}\verb|{...}|&     $\mathrm{Police\ romaine}$\\
\fni{mathbf}\verb|{...}|&     $\mathbf{Police\ grasse}$\\
\fni{mathsf}\verb|{...}|&     $\mathsf{Police\ sans\ serif}$\\
\fni{mathtt}\verb|{...}|&     $\mathtt{Police\ typewriter}$\\
\fni{mathit}\verb|{...}|&     $\mathit{Police\ italique}$\\
\fni{mathcal}\verb|{...}|&    $\mathcal{Police\ cursive}$\\
\fni{mathnormal}\verb|{...}|& $\mathnormal{Police\ normale}$\\
\end{tabular}

% \begin{tabular}{@{}lll@{}}
% \textit{Commande}&\textit{Exemple}&    \textit{Résultat}\\[6pt]
% \fni{mathcal}\verb|{...}|&    \verb|$\mathcal{B}=c$|&     $\mathcal{B}=c$\\
% \fni{mathrm}\verb|{...}|&     \verb|$\mathrm{K}_2$|&      $\mathrm{K}_2$\\
% \fni{mathbf}\verb|{...}|&     \verb|$\sum x=\mathbf{v}$|& $\sum x=\mathbf{v}$\\
% \fni{mathsf}\verb|{...}|&     \verb|$\mathsf{G\times R}$|&        $\mathsf{G\times R}$\\
% \fni{mathtt}\verb|{...}|&     \verb|$\mathtt{L}(b,c)$|&   $\mathtt{L}(b,c)$\\
% \fni{mathnormal}\verb|{...}|& \verb|$\mathnormal{R_{19}}| &\\
% & \verb|    \neq R_{19}$|& $\mathnormal{R_{19}}\neq R_{19}$\\
% \fni{mathit}\verb|{...}|&     \verb|$\mathit{ffi}\neq ffi$|& $\mathit{ffi}\neq ffi$
% \end{tabular}

\bigskip
\end{lined}
\end{table}

En complément des commandes de changement de taille, les
\wi{accolade}s jouent un rôle essentiel. Elles sont utilisées pour
former des \emph{\wi{groupe}s} qui limitent la
portée de la plupart des commandes de \LaTeX{}.

\begin{example}
Il aime les  {\LARGE grands et
{\small les petits}
caractères}.
\end{example}

Les commandes de changement de taille modifient également
l'interligne, mais seulement si le paragraphe se termine dans la
portée de la commande de changement de taille. L'accolade fermante
\verb|}| ne
doit donc pas être placée trop tôt. Remarquez la position de la
commande \verb|\par| dans les deux exemples suivants
\footnote{\ci{par} est équivalent à une ligne vide} :

\begin{example}
{\Large Ne lisez pas ceci !
 Ce n'est pas vrai !
 Croyez-moi !\par}
\end{example}
\begin{example}
{\Large Ce n'est pas vrai. Mais
n'oubliez pas que je suis un
menteur.}\par
\end{example}

Si vous voulez utiliser une commande de modification de la taille pour
tout un paragraphe ou même plus, vous pouvez utiliser la syntaxe des
environnements à la place des commandes.

%MPG: ajout de \raggedright:
\begin{example}
\begin{Large} \raggedright
Ceci n'est pas vrai
Mais, qu'est-ce qui l'est
de nos jours\dots\par
\end{Large}
\end{example}

\noindent Cela vous évitera d'avoir à compter les accolades
fermantes.

\subsection{Attention danger}

Il est dangereux d'utiliser de telles commandes de changement
explicite de police tout au long de vos documents, en effet ces
commandes vont à l'encontre de la philosophie de \LaTeX{} qui est de
séparer les aspects logiques et visuels d'un document. Cela signifie
que si vous voulez utiliser en plusieurs endroits la même commande de
changement de style afin de mettre en valeur un type particulier
\vadjust{\pagebreak[3]}%MPG: fighting underfull \vbox'es
d'information, vous devriez utiliser \ci{newcommand} pour définir une
nouvelle commande en insérant ainsi la commande de changement de style
dans une enveloppe logique.

\begin{example}
% dans le préambule ou dans
% une extension :
\newcommand{\danger}[1]{%
 \textbf{#1}}
% dans le document :
Défense d'\danger{entrer}.
Cette pièce contient des
\danger{machines} d'origine
inconnue.
\end{example}

Cette approche sera précieuse si vous décidez plus tard
d'utiliser une autre représentation typographique du danger que
\verb|\textbf|. Elle évitera d'avoir à rechercher et remplacer une à
une toutes les occurrences de \verb|\textbf| correspondant à la notion
de danger.

-Remarquez la différence entre demander à \LaTeX{} de \emph{mettre en
-valeur} un mot et lui demander de changer de \emph{police de
-caractères}. La commande \ci{emph} prend en compte le contexte alors
que les commandes de police sont absolues.

\begin{example}
\textit{Vous pouvez aussi
  \emph{mettre en valeur}\\
  du texte en italique,}
\textsf{ou dans une police
  \emph{sans-serif},}
\texttt{ou dans une police
  \emph{machine à écrire}.}
\end{example}

\subsection{Un conseil}

Pour conclure cette promenade au pays des commandes de changement de
police, voici un (mauvais) conseil :
\begin{quote}
  \underline{\textbf{N'oubliez pas\Huge!}} \textit{Plus}
  \textsf{V\textbf{\LARGE O}\texttt{u}\textsl{s}} \Huge utilisez
  \tiny de polices \footnotesize \textbf{dans} un \small \texttt{document}
  \large \textit{Plus} \normalsize il \textsc{devient}
  \textsl{\textsf{lisible} et bien prés\large e\Large n\LARGE t\huge é}.
\end{quote}


\section{Espacement}

\subsection{Entre les lignes}

\index{interligne} Pour utiliser un interligne plus grand pour un
document, vous pouvez utiliser la commande
\begin{lscommand}
\ci{linespread}\verb|{|\emph{facteur}\verb|}|
\end{lscommand}
\noindent dans le préambule de votre document. Utilisez
\verb|\linespread{1.3}| pour un interligne \enquote{un et demi} et
\verb|\linespread{1.6}| pour un \enquote{double}
interligne. L'interligne par défaut est~1.
\index{double interligne}

%Ce changement a souvent des effets indésirables dans les
%notes de bas de page, les tableaux, etc.  C'est assez compliqué à
%gérer, mais l'extension \pai{setspace} est prévue à cet effet.

%SC: hmmm, good translation for drastic ?
Notez que l'effet de la commande \ci{linespread} est drastique et
inapproprié pour la publication. Si vous avez une bonne raison de
changer l'interligne, vous pouvez utiliser la commande :
\pagebreak[3]%MPG: avoid underfull \vbox
\begin{lscommand}
\verb|\setlength{\baselineskip}{1.5\baselineskip}|
\end{lscommand}

\begin{example}
{\setlength{\baselineskip}%
           {1.5\baselineskip}
Ce paragraphe est formatté avec
un interligne fixé à 1,5 de ce
qu'il était avant. Remarquez la
commande par à la fin du
paragraphe.\par}

Ce paragraphe a un objet précis
et montre qu'après l'accolade
fermante tout redevient normal.
\end{example}

\subsection{Mise en page d'un paragraphe}\label{parsp}

Il y a deux paramètres qui jouent sur l'apparence d'un paragraphe. En
insérant une définition telle que :
\begin{code}
\ci{setlength}\verb|{|\ci{parindent}\verb|}{0pt}| \\
\verb|\setlength{|\ci{parskip}\verb|}{1ex plus 0.5ex minus 0.2ex}|
\end{code}
dans le préambule, vous supprimez le retrait des débuts de paragraphe
(1\iere{} définition) et vous augmentez l'espace entre deux paragraphes
(2\ieme{} définition).

Les arguments \emph{plus} et \emph{minus} de la deuxième définition
indiquent à \TeX{} de quelle taille il est autorisé à étendre et
rétrécir l'espace entre paragraphes, si cela lui est nécessaire pour
faire tenir un paragraphe dans une même page.

Attention, la deuxième définition a également une influence sur la
table des matières : ses lignes deviennent également plus
espacées. Pour éviter cela, vous pouvez déplacer ces commandes du
préambule vers le corps du document, après la commande
\ci{tableofcontents} (ou bien ne pas les utiliser du tout, car la
typographie professionnelle préfère utiliser l'indentation plutôt que
l'espacement pour séparer les paragraphes).

Pour indenter un paragraphe qui ne l'est pas, utilisez la commande :
\begin{lscommand}
\ci{indent}
\end{lscommand}
\noindent au début du paragraphe\footnote{Pour indenter
 systématiquement le premier
 paragraphe après le titre d'une section, utilisez\footnotemark l'extension
 \pai{indentfirst} de l'ensemble \package{tools}.}. Bien sûr cela ne
 marche que si \verb|\parindent| n'est pas nul.
 \footnotetext{Ou bien, si vous écrivez en français, ne faites rien :
   \pai{babel} s'en est chargé pour vous ! \NdT}

Pour créer un paragraphe sans indentation, utilisez :
\begin{lscommand}
\ci{noindent}
\end{lscommand}
\noindent
en tête du paragraphe.

L'option \pai{francais} de l'extension \pai{babel} modifie ici aussi
les règles par défaut de
\LaTeX{} pour s'adapter aux règles françaises.

Il est possible de commencer un paragraphe par une lettrine en utilisant
l'extension \pai{lettrine}%
\footnote{\texttt{CTAN:/macros/latex/contrib/supported/lettrine/}} :
\begin{lscommand}
\ci{lettrine}\verb|[|\emph{options}\verb|]{|\emph{lettrine}\verb|}|\verb|{|\emph{texte}\verb|}|
\end{lscommand}
La lettrine de la page~\pageref{lettrine} s'obtient par la commande :
\begin{code}
\verb|\lettrine{C}{e document}|
\end{code}

\subsection{Espacement horizontal}
\label{sec:hspace}

\LaTeX{} détermine l'espacement entre les mots et les phrases
automatiquement. Pour ajouter de l'espacement horizontal, utilisez :
\index{espacement!horizontal}
\begin{lscommand}
\ci{hspace}\verb|{|\emph{longueur}\verb|}|
\end{lscommand}
Si une telle espace doit être conservée, même lorsqu'elle tombe en début ou en
fin de ligne, utilisez \verb|\hspace*|. Dans le cas le plus simple,
\emph{longueur} est simplement un nombre suivi d'une unité. Les unités
les plus importantes sont listées dans le tableau~\ref{units}.
\index{unités}\index{dimensions}

\begin{example}
Ceci\hspace{1.5cm}est une espace
de 1.5 cm.
\end{example}
\suppressfloats
\begin{table}[tbp]
\caption{Unités \TeX} \label{units}\index{unités}
\begin{lined}{9.5cm}
\begin{tabular}{@{}ll@{}}
\texttt{mm} & millimètre \quad \demowidth{1mm} \\
\texttt{cm} & centimètre = 10~mm  \quad \demowidth{1cm}                     \\
\texttt{in} & pouce\footnote{\emph{Inch} en anglais. \NdT.}
	      $=$ 25,4~mm \quad \demowidth{1in}                    \\
\texttt{pt} & point $\approx 1/72$~pouce $\approx 0,35$~mm  \quad\demowidth{1pt}\\
\texttt{em} & largeur d'un  ``M'' dans la police courante \quad \demowidth{1em}\\
\texttt{ex} & hauteur d'un ``x'' dans la police courante \quad \demowidth{1ex}
\end{tabular}

\bigskip
\end{lined}
\end{table}

La commande :
\begin{lscommand}
\ci{stretch}\verb|{|\emph{n}\verb|}|
\end{lscommand}
\label{cmd:stretch}
\noindent produit une espace élastique. Elle s'étend jusqu'à ce que tout
l'espace libre sur la ligne soit occupé. Si plusieurs commandes
\verb|\hspace{\stretch{|\emph{n}\verb|}}| sont exécutées sur la même
ligne, les espaces s'étendent proportionnellement à leurs facteurs
d'élasticité \emph{n} respectifs.


\begin{example}
x\hspace{\stretch{1}}%
x\hspace{\stretch{3}}x
\end{example}

Lors que l'espacement horizontal est utilisé en conjonction avec du
texte, il est préférable de faire que l'espace soit ajusté en fonction
de la taille de la police courante. Ceci peut être réalisé en
utilisant les unités relatives \texttt{em} et \texttt{ex}:

\begin{example}
{\Large{}plus gr\hspace{1em}and}\\
{\tiny{}minuscu\hspace{1em}le}
\end{example}

\subsection{Espacement vertical}

L'espacement vertical entre les paragraphes, sections,
sous-sections\dots{} est déterminé automatiquement par \LaTeX{}.
En cas de besoin, de l'espace supplémentaire
\emph{entre deux paragraphes} peut être inséré avec la commande :
\begin{lscommand}
\ci{vspace}\verb|{|\emph{longueur}\verb|}|
\end{lscommand}

Cette commande doit normalement être utilisée entre deux lignes
vides. Si l'espacement doit être conservé en haut ou en bas d'une
page, utilisez la version étoilée, i.e. \verb|\vspace*|.
\index{espacement!vertical}
\index{vertical!espacement}

La commande \ci{stretch} en association avec \verb|\pagebreak| permet
d'imprimer du texte sur la dernière ligne d'une page ou de centrer
verticalement du texte sur une page.
\begin{code}
\begin{verbatim}
Du texte \dots

\vspace{\stretch{1}}
Ceci sera imprimé sur la dernière ligne.\pagebreak
\end{verbatim}
\end{code}

De l'espace supplémentaire entre deux lignes du \emph{même} paragraphe
ou à l'intérieur d'une table peut être obtenu par la commande :
\begin{lscommand}
\ci{\bs}\verb|[|\emph{longueur}\verb|]|
\end{lscommand}

Les commandes \ci{bigskip} et \ci{smallskip} permettent de créer des
espacements verticaux prédéfinis sans se préoccuper des dimensions
exactes.

\section{Disposition d'une page}

\begin{figure}[!hp]
\begin{center}
\makeatletter\@mylayout\makeatother
\end{center}
\vspace*{1.8cm}
\caption[Paramètres de disposition de page pour ce document]{Paramètres de disposition de page pour ce document. Utilisez l'extension \pai{layout} pour afficher la disposition de page de votre document}
\label{fig:layout}
\cih{footskip}
\cih{headheight}
\cih{headsep}
\cih{marginparpush}
\cih{marginparsep}
\cih{marginparwidth}
\cih{oddsidemargin}
\cih{paperheight}
\cih{paperwidth}
\cih{textheight}
\cih{textwidth}
\cih{topmargin}
\end{figure}
\index{disposition d'une page}
\LaTeXe{} permet d'indiquer la taille du papier en paramètre de  la commande
\verb|\documentclass|. Il définit ensuite automatiquement les
\wi{marges} les mieux adaptées. Parfois, on peut ne pas être satisfait par les
valeurs prédéfinies et vouloir les modifier.
%no idea why this is needed here ...
\thispagestyle{fancyplain}
La figure~\ref{fig:layout} montre tous les paramètres qui peuvent être
modifiés. Cette figure a été réalisée avec l'extension \pai{layout} de
l'ensemble tools
\footnote{\CTANref|pkg/tools|}.

\textbf{Attendez !}\dots{} avant de vous lancer dans \enquote{élargissons un peu
ce texte}, prenez deux secondes pour réfléchir. Comme souvent avec
\LaTeX{}, il y a de bonnes raisons pour disposer les pages de cette
façon.

Sans doute, comparé avec une page standard produite avec MS Word, une page de
\LaTeX{} à l'air horriblement étroite. Mais regardez votre livre
préféré\footnote{Un vrai livre, imprimé par un grand éditeur\dots} et
comptez le nombre de caractères sur une ligne normale. Vous verrez
qu'il n'y a guère plus de soixante-six caractères par
ligne. L'expérience montre qu'un texte devient moins lisible si le
nombre de caractères par ligne dépasse cette valeur, cela parce qu'il
devient plus difficile pour les yeux de passer de la fin d'une ligne
au début de la ligne suivante. Ceci explique aussi que les journaux
utilisent plusieurs colonnes.

Ainsi, si vous élargissez le corps du texte, ayez conscience que vous
le rendez aussi moins lisible. Ceci dit, si vous tenez à modifier les
paramètres qui contrôlent la
disposition d'une page, voici comment procéder :

\LaTeX{} dispose de deux commandes pour modifier ces paramètres. Elles
sont généralement utilisées dans le préambule.

La première commande affecte une valeur fixe  au paramètre spécifié :
\begin{lscommand}
\ci{setlength}\verb|{|\emph{paramètre}\verb|}{|\emph{longueur}\verb|}|
\end{lscommand}

La deuxième commande ajoute une longueur à ce paramètre :
\begin{lscommand}
\ci{addtolength}\verb|{|\emph{paramètre}\verb|}{|\emph{longueur}\verb|}|
\end{lscommand}

La deuxième commande est en pratique plus utile que \ci{setlength},
parce qu'elle fonctionne relativement à la taille
par défaut. Pour ajouter un centimètre à la largeur du texte, nous
utiliserions les commandes suivantes dans le préambule :
\begin{code}
\verb|\addtolength{\hoffset}{-0.5cm}|\\
\verb|\addtolength{\textwidth}{1cm}|
\end{code}

Dans ce contexte, il peut être intéressant d'utiliser l'extension
\pai{calc}. Elle permet d'utiliser des expressions algébriques
traditionnelles en argument de \verb|\setlength| ainsi que partout où
l'on utilise des valeurs numériques comme arguments de macros.

\section{Jouons un peu avec les dimensions}

Autant que possible nous évitons d'utiliser des dimensions absolues
dans des documents \LaTeX{}. Nous essayons plutôt de les définir
relativement aux dimensions d'autres éléments de la page. La largeur
d'une figure sera ainsi \verb|\textwidth| afin de lui faire occuper
toute la largeur de la page.

Les trois commandes suivantes permettent de déterminer la largeur, la
hauteur et la profondeur d'une chaîne de caractères.

\begin{lscommand}
\ci{settoheight}\verb|{|\emph{variable}\verb|}{|\emph{texte}\verb|}|\\
\ci{settodepth}\verb|{|\emph{variable}\verb|}{|\emph{texte}\verb|}|\\
\ci{settowidth}\verb|{|\emph{variable}\verb|}{|\emph{texte}\verb|}|
\end{lscommand}

\bigbreak
L'exemple ci-dessous montre une utilisation possible de ces
commandes :

\begin{example}
\flushleft
\newenvironment{vardesc}[1]{%
  \settowidth{\parindent}{#1\ }
  \makebox[0pt][r]{#1\ }}{}
\begin{displaymath}
a^2+b^2=c^2
\end{displaymath}

\begin{vardesc}{Où :}$a$,
$b$ sont les cotés adjacents à
l'angle droit d'un triangle
rectangle,\par
$c$ est l'hypothénuse du
triangle,\par
$d$ n'est pas utilisé ici.
\'Etonnant non ?
\end{vardesc}
\end{example}


\section{Boîtes}
\index{boîte}

\LaTeX{} construit ses pages en empilant des boîtes. Au commencement
chaque caractère est une petite boîte qui est ensuite collée à
d'autres boîtes-caractères pour former un mot. Ceux-ci sont alors
assemblés à d'autres mots, avec une colle spéciale qui est élastique
pour permettre de comprimer ou d'étirer des séries de mots afin de
remplir exactement une ligne sur la page.

Reconnaissons qu'il s'agit d'une description simpliste de ce
qui se passe réellement, mais le fait est là : \TeX{} travaille avec
des boîtes et de la colle. Les caractères ne sont pas les seuls à
pouvoir former des boîtes. Virtuellement tout peut être mis dans des
boîtes, y compris d'autres boîtes. Chaque boîte est ensuite traitée
par \LaTeX{} comme s'il s'agissait d'un simple caractère.

Dans de précédents chapitres vous avez déjà rencontré quelques
boîtes, même si nous ne l'avons pas signalé. L'environnement
\verb|tabular| et la commande \verb|\includegraphics|, par exemple,
produisent tous les deux des boîtes. Cela signifie que vous pouvez
facilement aligner deux illustrations ou deux tables côte à côte. Il
suffit de s'assurer que la somme de leurs largeurs ne dépasse pas la
largeur du texte.

Il est aussi possible d'emballer un paragraphe dans une boîte  :
\begin{lscommand}
\ci{parbox}\verb|[|\emph{pos}\verb|]{|\emph{largeur}\verb|}{|\emph{texte}\verb|}|
\end{lscommand}

\noindent on peut également utiliser un environnement :
\begin{lscommand}
\verb|\begin{|\ei{minipage}\verb|}[|\emph{pos}\verb|]{|\emph{largeur}\verb|}| texte
\verb|\end{|\ei{minipage}\verb|}|
\end{lscommand}

\noindent le paramètre \texttt{pos} peut être l'une des lettres
\texttt{c}, \texttt{t} ou \texttt{b} pour contrôler l'alignement
vertical de la boîte par rapport à la base du texte
précédent. \texttt{largeur} est une dimension indiquant la largeur de la
boîte. La différence majeure entre \ei{minipage} et \ci{parbox} est
qu'il est possible d'utiliser quasiment n'importe quelle commande ou
environnement dans \ei{minipage}, alors que ce n'est pas le cas\footnote{En
  fait, ce sont essentiellement les environnements et commandes de type
  \emph{verbatim} qui posent problème. \NdT} avec
\ci{parbox}.

Alors que \ci{parbox} englobe tout un paragraphe en réalisant coupure
des lignes et tout le reste, il existe également une catégorie de
commandes de gestion des boîtes qui ne travaillent que sur des
éléments alignés horizontalement. L'une d'elles nous est déjà
connue : il s'agit de \ci{mbox}. Celle-ci combine simplement une série de
boîtes pour en former une nouvelle; elle peut être utilisée pour
empêcher \LaTeX{} de couper une ligne entre deux mots. Puisqu'il est
possible de placer des boîtes dans d'autres boîtes, ces constructeurs de
boîtes horizontales sont extrêmement flexibles.

\begin{lscommand}
\ci{makebox}\verb|[|\emph{largeur}\verb|][|\emph{pos}\verb|]{|\emph{texte}\verb|}|
\end{lscommand}

\noindent Le paramètre \texttt{largeur} définit la largeur de la boîte vue de
l'extérieur\footnote{Cela signifie qu'elle peut être plus petite que
la largeur du contenu de la boîte. Dans un cas extrême on peut même
régler la largeur à \texttt{0pt}; ainsi le texte dans la boîte
sera placé sans influencer les boîtes adjacentes.}.  En plus des
expressions exprimant une longueur vous pouvez également utiliser
\ci{width}, \ci{height}, \ci{depth} et \ci{totalheight} à l'intérieur
du paramètre \texttt{largeur}. Leurs valeurs sont obtenues à partir
des dimensions réelles du \texttt{texte}. Le paramètre \texttt{pos}
est une lettre parmi \texttt{c} (\textbf{c}enter) pour centrer le
texte, \texttt{l} (flush\textbf{l}eft) pour l'aligner à gauche,
\texttt{r} (flush\textbf{r}ight) pour l'aligner à droite, ou
\texttt{s} (\textbf{s}pread) pour le répartir horizontalement dans la
boîte.

La commande \ci{framebox} fonctionne de la même façon que
\verb|\makebox|, mais en plus elle dessine un cadre autour du texte.

L'exemple suivant vous montre quelques choses que l'on peut faire avec
les commandes \ci{makebox} et \ci{framebox} :

\typeout{Next message about under-full at line 4 is OK!}%MPG
\begin{example}
\makebox[\textwidth]{%
    c e n t r \'e}\par
\makebox[\textwidth][s]{%
é t i r é}\par
\framebox[1.1\width]{Whoua
    le cadre !} \par
\framebox[0.8\width][r]{Rat\'e,
    je suis trop large} \par
\framebox[1cm][l]{c'est aussi
    mon cas.}
Pouvez-vous lire ceci ?
\end{example}
\typeout{Previous message about under-full at line 4 was OK!}%MPG

Maintenant que nous savons contrôler l'alignement horizontal, la suite
logique est de voir comment gérer l'alignement vertical\footnote{Le
contrôle total est obtenu en contrôlant en même temps l'alignement
horizontal et l'alignement vertical.}. Pas de problème avec
\LaTeX{}. La commande :

\begin{lscommand}
\ci{raisebox}\verb|{|\emph{élévation}\verb|}[|\emph{profondeur}\verb|][|\emph{hauteur}\verb|]{|\emph{texte}\verb|}|
\end{lscommand}

\noindent permet de définir les propriétés verticales d'une boîte. La
\emph{profondeur} correspond à une extension \emph{sous} la ligne de
base du texte, la \emph{hauteur} à une extension \emph{au-dessus} de
cette ligne. Vous pouvez utiliser \ci{width}, \ci{height}, \ci{depth}
et \ci{totalheight} dans les trois premiers paramètres afin d'agir en
fonction de la taille du texte contenu dans la boîte.

\begin{example}
\raisebox{0pt}[0pt][0pt]{\Large%
\textbf{Aaaa\raisebox{-0.3ex}{a}%
\raisebox{-0.7ex}{aa}%
\raisebox{-1.2ex}{r}%
\raisebox{-2.2ex}{g}%
\raisebox{-4.5ex}{h}}}
cria-t-il, mais la ligne suivante
ne remarqua pas qu'une chose
horrible lui était arrivée.
\end{example}

\section{Filets}
\label{sec:rule}

\index{filet}

%%% XXX Je ne sais pas quel est le terme technique exact pour 'strut'.

Quelques pages plus haut vous avez peut-être remarqué la commande :

\begin{lscommand}
\ci{rule}\verb|[|\emph{élévation}\verb|]{|\emph{largeur}\verb|}{|\emph{hauteur}\verb|}|
\end{lscommand}

\noindent En utilisation normale, elle produit une simple boîte
noire.

\begin{example}
\rule{3mm}{.1pt}%
\rule[-1mm]{5mm}{1cm}%
\rule{3mm}{.1pt}%
\rule[1mm]{1cm}{5mm}%
\rule{3mm}{.1pt}
\end{example}

\index{horizontal!filet}
\noindent C'est utile pour produire des lignes horizontales et
verticales. La ligne horizontale sur la page de titre par exemple a
été tracée à l'aide d'une commande \ci{rule}, par exemple.

\bigskip
{\flushright Fin.\par}

\endinput

%

% Local Variables:
% TeX-master: "lshort2e"
% mode: latex
% mode: flyspell
% End:
